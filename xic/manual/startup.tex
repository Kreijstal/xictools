% -----------------------------------------------------------------------------
% Xic Manual
% (C) Copyright 2013, Whiteley Research Inc., Sunnyvale CA
% $Id: startup.tex,v 1.25 2017/03/22 23:39:49 stevew Exp $
% -----------------------------------------------------------------------------

% -----------------------------------------------------------------------------
\chapter{{\Xic} Configuration and Startup}

% xic:graphics 110613
\section{Graphics Support and Requirements}

Starting with Generation 4, {\Xic} and all other Whiteley Research
products use the GTK-2 graphical user interface toolkit exclusively
(see {\vt http://www.gtk.org} for more information).  The toolkit must
be provided by the user's installation.  This is almost certain to be
the case under Linux, as GTK is a default part of virtually all Linux
distributions.  In FreeBSD, it will have to be installed from the
ports or packages collection, if it wasn't automatically installed
from the distribution DVD.  For Windows, a complete GTK-2 installation
package is available from Whiteley Research, which is a
self-extracting {\vt .exe} version of the zip distribution provided
from {\vt www.gtk.org}.  One of these packages must now be
installed on a Windows host.  Similarly, GTK-2 must be installed on
the Mac.  This makes use of the X-window system, which also must be
installed and running on the Mac.  Although there is a version of
GTK-2 that uses the native Quartz screen interface instead of
X-windows, it has very serious flaws and is unusable at present.

The Win32 graphical interface previously used under Windows is now
retired, as is the GTK-1 interface used in some earlier releases. 
Thus, all present releases will have precisely the same graphical
interface, which will greatly simplify documentation, maintenance, and
future development.

The GTK-2 toolkit handles the interface to the screen, window manager,
keyboard, and mouse.  It is not likely that there will be serious
incompatibilities.

% -----------------------------------------------------------------------------
% macosx 093017
\section{Apple OS X Notes}

In OS X, lurking beneath the pretty graphics is a complete Unix
operating system based on FreeBSD, including support for X-windows. 
Most of the open-source software developed for Linux/FreeBSD has been
ported to OS X, so for the Unix fans (like me), the operating system
can look like Unix with a great graphical interface that also runs
Macintosh applications.  Overall, OS X is a very impressive
desktop/laptop operating system.

All Generation 4 OS X distributions require OS X 10.10 (Yosemite) or
later.

The Whiteley Research programs presently require that the X-windows
server be running, as X is used for graphics.  Unlike in some earlier
OS X releases, it is not installed automatically in Yosemite.  The
Apple-sponsored X-server is called "XQuartz" and is available for
download from the project web site (google ``XQuartz download'').

Operation and behavior should be identical to the Unix/Linux versions
of the programs.  These are 64-bit binaries.

\subsection{Installation}

The package distribution files are ``flat'' Apple package files.  The
installation procedure is pretty much the same as under Unix/Linux,
using the {\vt wr\_install} script from a terminal window.  Due to the
X-Windows support, graphical operation from remote systems is possible
if running a local X-server program.

There are two differences from the Linux releases.
\begin{enumerate}
\item{The programs use the GTK-2 graphical interface toolkit for
display, which is supplied in a separate ``{\vt gtk2-bundle-x11}''
distribution file.  As in Windows, the bundle supplies all of the
libraries and supporting files needed, as they are not native to OS X. 
The user will need to download and install the bundle package.  Once
installed, it should rarely if ever need to be upgraded.  The bundle
can be installed with the {\vt wr\_install} script, or with Apple's
installer, the same as the program distribution files.}

\item{The program installation location can not be changed.  The
programs will be installed under {\vt /usr/local/xictools}, and the
bundle is installed under {\vt /usr/local/gtk2-bundle-x11}.}
\end{enumerate}

The user must have root permission to install the programs.  If a root
password has been defined, the user can use ``{\vt su}'' to become
root as in Unix/Linux.  Otherwise, there is a ``{\vt sudo}'' command
that can be issued which provides temporary root privileges to certain
pre-designated users.  From a command line, one can use {\vt sudo} to
execute commands that require root privileges.  The user is asked for
their password, and the command is executed if the password is
accepted.

The {\vt sudo} command is built into the {\vt wr\_install} script, so
that users should not need to become root explicitly to perform
updates.  To edit configuration files in the installation, root access
is needed.
%\ifxtlserv
%The {\vt sudo} command can be used to start the license server, if
%running the server is necessary.
%\fi

\subsection{Un-Installation}

To uninstall, the {\vt wr\_uninstall} script can be used.  This takes
care of file removal and updating the system package database.

\subsection{Running the Applications}

It is assumed that the user will be initiating the applications from a
terminal window, as under Unix/Linux.  The directories containing the
program binaries ({\vt /usr/local/xictools/bin}) and the bundle
programs ({\vt /usr/local/gtk2-bundle-x11/bin}) should be in the
shell search path.  Then, all executables will be found by name. 
Operation is the same as under Unix/Linux.

The X-Windows server must be running for successful execution of the
programs that use graphics.  In OS X Mavericks the Apple "XQuartz"
X-server will start automatically when needed (remember that this must
be downloaded, as it is not installed automatically in Mavericks),
however this causes the initial program startup time to be rather
long.

\subsection{MacBook Keyboard Mapping Issues}

The Darwin64 releases work great on a MacBook Pro, however there are
some keyboard mapping issues.  Keys which have normal significance to
{\Xic}, such as {\kb Home}, {\kb Page Up} and {\kb Page Down}, and the
numeric keypad plus and minus keys are nowhere to be found.  Yet, all
functionality is present, but maps to alternative key combinations. 
One can run the {\cb Key Map} command in the {\cb Attributes Menu} if
another mapping is needed.

The table below describes the default mapping.

\begin{tabular}{|l|p{5in}|} \hline
\bf MacBook Pro &  \bf Normal keyboard\\ \hline\hline
\kb Delete & {\kb Backspace}\newline
  The key labeled ``{\kb Delete}'' actually sends a backspace
  character.\\ \hline
\kb fn-Delete & {\kb Delete}\newline
  Press the {\kb fn} key with {\kb Delete} to get a real delete
  character.\\ \hline
\kb fn-Left & {\kb Home}\newline
  Press the {\kb fn} key with the left arrow for the center-full-view
  function in {\Xic}.\\ \hline
\kb fn-Up & {\kb Page Up}\newline
  Press the {\kb fn} key with up arrow to get a page up code, used in
  {\Xic} for displaying DRC errors.\\ \hline
\kb fn-Down & {\kb Page Down}\newline
  Press the {\kb fn} key with down arrow to get a page down code, used
  in {\Xic} for displaying DRC errors.\\ \hline\hline
\multicolumn{2}{|c|}
  {These mappings are set in Apple releases only.}\\ \hline\hline
\kb fn-Enter & {\kb KeyPad Enter}\newline
  Press the {\kb fn} key with {\kb Enter} to get the numeric keypad
  enter code.  This is mapped to the zoom-in action, as for {\kb
  Numeric Plus}.\\ \hline
\kb fn-Right & {\kb End}\newline
  Press the {\kb fn} key with the right arrow to get the end key code. 
  This is mapped to the zoom-out action, as for {\kb Numeric Minus}.\\
  \hline
\end{tabular}

Note that if you use a ``normal'' keyboard with your Apple computer,
the two new mappings will be in addition to the normmal mappings.

The ``secondary press'' mentioned in Apple documents is button 3 (the
right mouse button on a three-button mouse).  You should probably
change the track-pad settings in the {\cb Preferences} to enable this. 
There is no button 2 (middle mouse button).  You'll have to live
without it, or get a three-button pointing device.

\subsection{The Alt Key Issue}

The MacBook Pro and probably other Mac machines lack a compatible {\kb
Alt} key.  This {\kb Alt} key is used in {\Xic} as a menu accelerator,
and for a button-press modifier.  It is reasonable to live without it,
but there is a fix.

The following fix works on my MacBook Pro.  Create a file in your home
directory named ``{\vt .Xmodmap}'' containing the following two lines:

\begin{quote}\vt
keycode 66 = Alt\_L\\
add mod1 = Alt\_L
\end{quote}

This will map the left ``option'' button to {\kb Alt}.  The right
option button will still do the normal Mac thing, i.e., send alternate
character keycodes.

However, this depends on the left option key returning scan code 66,
which may not be true on different hardware.  The {\vt xev} program
can be used to find the actual scan code.

This will apply to all X applications, and the mapping will be
recorded when the X server starts.  You can also give the command
\begin{quote}
\vt xmodmap -e "keycode 66 = Alt\_L" -e "add mod1 = Alt\_L"
\end{quote}
which will re-map the keys for the current X session.


% -----------------------------------------------------------------------------
% mswindows 102817
\section{Microsoft Windows Notes}

This section contains notes relevant to the Microsoft Windows
release of the {\XicTools}.

\subsection{Installation and Setup}

%\ifxtlserv
%In order to license a Windows host, two pieces of information are
%needed:
%\begin{enumerate}
%\item{The machine's host name}
%\item{The Windows Product ID}
%\end{enumerate}
%
%The recommended way to retrieve this information is to download the
%{\vt licinfo.exe} program from the {\vt wrcad.com} web site.  When
%run, this program generates a file named ``{\vt tLicenseInfo}'' which
%should be emailed to Whiteley Research.  A pop-up window displays the
%information, and indicates success or failure.
%\fi

The distributions are available from the {\vt wrcad.com} web site,
along with the distributions for Linux and OS X.  The distributions
come in self-extracting {\vt .exe} files.  Simply run the files to do
the installation.  The programs can later be uninstalled, either from
the {\cb Control Panel} or by clicking the {\cb Uninstall} icons in
the {\cb XicTools} program group in the {\cb Start} menu.

The same process can be used to install updated releases -- it is not
necessary to uninstall first.  A more convenient way to keep current
is to use the updating feature of the help system (see \ref{update}).

WARNING\\
The programs use an entry in the Windows Registry to find their
startup files, etc.  This entry is created by the installer program. 
Thus
\begin{enumerate}
\item{The correct way to move an existing installation to another
location is to uninstall the program, and reinstall to the new
location using the standard distribution file.  If you just move the
files to a new location, the Registry won't be updated and the program
won't run correctly.}

\item{You can not simply copy files from another machine when creating
a new installation.  The files must be installed through the
distribution files, or the Registry entry won't be set.}
\end{enumerate}

The Registry entry used (by the {\vt inno} installer program) is
(for example)
\begin{quote}\vt
HKLM$\backslash$Software$\backslash$Microsoft$\backslash$Windows$\backslash$CurrentVersion$\backslash$Uninstall$\backslash$Xic-4is1
\end{quote}

The Windows {\vt regedit} utility can be used to repair the Registry
if necessary.  The Windows {\vt reg} utility can also be used to query
and modify the Registry from the command line.

The programs are installed by default under {\vt
C:{$\backslash$}usr{$\backslash$}local}, which can be specified to the
program installer.  The installation directories will be created if
necessary.  All of the programs will install under a directory named
``{\vt xictools}'' under the prefix, (for example, {\Xic} files would
be installed by default in {\vt
C:$\backslash$usr$\backslash$local$\backslash$xictools$\backslash$xic}). 
The structure of the tree is exactly that as under Linux, which
simplifies compatibility.  It is recommended that the default
installation location be used, if possible.

The Generation 4 programs all use the GTK-2 toolkit for the user
interface.  The DLLs and other support files for this toolkit are in
the {\vt gtk2-bundle} distribution file.  This must be installed, by
executing the file, in the same manner as the program distribution
files.  You will only need to do this once, or at least very
infrequently.  The default location is under {\vt
C:$\backslash$usr$\backslash$local$\backslash$gtk2-bundle}.

By default, the actual binary executable is run from a script ({\vt
.bat}) file, which is installed in the same directory as the
executables, which by default is {\vt
C:$\backslash$usr$\backslash$local$\backslash$xictools$\backslash$bin}. 
The script takes care of properly specifying the path to the DLLs
provided by the gtk2-bundle.

A program group named {\vt XicTools} is created in the {\vt Start}
menu (or equivalent), from which the programs can be started.  The
programs can also be started from a command line, in either a Windows
{\cb Command Prompt} window or a Cygwin shell window.  One will need
to type the full path to the bat file (e.g., type ``{\vt
C:$\backslash$usr$\backslash$local$\backslash$xictools$\backslash$bin$\backslash$xic}''
to start {\Xic}).  There are two ways to avoid having to type the
whole path:

\begin{enumerate}
\item{Add the directory to your search path.  This is the {\vt PATH}
variable in the environment.  This can be set in your {\cb Command
Prompt}window by giving a command like
\begin{quote}
 \vt PATH=\%PATH\%;c:$\backslash$usr$\backslash$local$\backslash$xictools$\backslash$bin
\end{quote}
or the {\vt PATH} can be set from the {\cb Control Panel} (the
procedure is described below for Windows 8).}

\item{The {\vt bat} files can be copied from the installation location
into a directory that is already in the search path, or to the current
directory.}
\end{enumerate}

WARNING\\
In early alpha test releases, it was suggested that the {\vt
gtk2-bundle/bin} be added to the system {\vt PATH}.  This is a very
bad idea, since this may have an adverse effect on other programs. 
Unless you really know what you are doing, the {\vt gtk2-bundle/bin}
directory should never appear in a global search path.

\subsection{General Notes}

The {\XicTools} for Windows are supported on Windows XP and later. 
The programs retain the ``look and feel'' of the Unix/Linux versions
as much as possible, given the constraints of the Windows operating
system.

Starting with Generation 4, the programs use the GTK-2 graphical
interface toolkit, as used by the other releases.  The native Win32
interface is gone.  The GTK-2 libraries are supplied in a separate
installation module.  Installation is mandatory, but the libraries are
quite static so will not require much attention after the initial
installation.

Most basic features are available under Windows.  Some of the more
advanced features are not.

\begin{itemize}
\item{There is presently no support for the Tcl/Tk or Python script 
language plug-ins.
%\ifoa
There is also no support for the OpenAccess plug-in.
%\fi
}

\item{There is no provision for remote running of the programs as
with the X window system in Linux.}

\item{Under Unix/Linux, when the program crashes (of course, a very
rare occurrence!), the {\vt gdb} debugger is called to generate a
stack trace, which is emailed to Whiteley Research for analysis. 
Since it is rare to find {\vt gdb} on a Windows system, an alternative
is built in.  This produces a file named {\it progname\/}.{\vt
stackdump}, which is emailed (if possible) to Whiteley Research.}

\item{Windows does not provide a reliable interface for internet mail,
so the email clients and crash-dump report in the {\XicTools} may not
work.  The mail in {\XicTools} works by passing the message to a
Windows interface called ``MAPI'', which in turn relies on another
installed program to actually send the mail.

To get this working in Windows 8, I had to download and install
something called ``live mail'' from Microsoft, which eventually
worked.  This app supports MAPI, apparently the Windows 8 Mail app
does not(?).  The Windows 8 app also does not work with POP3 servers,
solidifying my disrespect.}
\end{itemize}

The ``environment variables'' mentioned in the {\Xic}/{\WRspice}
documentation are available, and can be set in a {\cb Command Prompt}
window with the ``{\vt set}'' command before starting the programs, or
from the {\cb System} entry in the {\cb Control Panel} (or wherever
this capability lives in your version of Windows).  Only the latter
method works if the programs are started from an icon or menu.

Directory path names used by the programs can use either `/' or
`$\backslash$' as the directory separator character, interchangeably. 
The path can also contain a drive specifier.

The path variables used by {\Xic} that contain lists of directory
paths must use either a space or `;' (semicolon) as a separator. 
Under Unix, the separation characters are space and `:' (colon).

The text files used by the programs can have either DOS or Unix line
termination.  Text files produced by the programs under Windows will
use the DOS format.

Under Windows, where the concept of a ``home directory'' is somewhat
tenuous, the programs will look for environment variables,
particularly {\et HOME}, and if found interpret the value as a path to
the home directory.  This is true when programs look for startup
files.  When the program is started from an icon or shortcut, and the
start directory is not explicitly set in the icon properties (it
defaults to {\vt C:/}), the current directory will be the home
directory, rather than {\vt C:/}.

Those used to a Unix environment are encouraged to download and
install the Cygwin tools.  These include most of your favorite Unix
commands, plus a complete compiler toolchain for application
development.  In particular, the bash shell is quite useful, as it
provides a ``DOS box'' that responds to Unix shell commands, and from
which one can execute shell scripts.  The tools can be downloaded as
individual modules.

If it is needed and does not exist, {\Xic} and {\WRspice} will create
a {\vt {$\backslash$}tmp} directory on the current drive.  This will
contain temporary files, used by the programs.  These should be
removed automatically when the programs terminate, but if not the
files can be safely deleted if {\Xic} and {\WRspice} are not running.

\subsection{Setting Environment Variables}

By running Cygwin, the setting of environment variables and similar
becomes very familiar to a Linux user.  In particular, running {\Xic}
from a Cygwin bash-shell window emulates pretty well the Linux
experience.  This is a recommended approach for those familiar with
Unix/Linux.

Otherwise, environment variables can be set manually in a {\cb Command
Prompt} window from which the programs are run.  The {\vt bat} files
can be modified and ``{\vt set}'' lines added, as an option to avoid
manual setting of variables that should always be in force.  Another
option is to set the system default environment variables.  Be aware
that all other programs will see the variables.  Setting the system
environment variables is probably something to avoid if possible.  If
you insist, here is the procedure for Windows 8.1.  Other supported
Windows releases are probably not horribly different.

\begin{enumerate}
\item{Go to the infamous {\cb Start} page, click on the circle with
down arrow icon near the bottom-left corner.  This shifts to the {\cb
Apps} page.}

\item{Find the {\cb Control Panel}, it is listed on the {\cb Apps}
page under {\cb Windwos System} You can use the search tool if
necessary.  Eventually, you'll find the icon, then click it to bring
up the {\cb Control Panel}.}

\item{Click {\cb System and Security}.  The display will change to a
new set of choices.}

\item{Click {\cb System}.}

\item{Click {\cb Advanced system settings} along the left.  This
brings up a {\cb System Properties} window.}

\item{Click the {\cb Environment Variables} button near the bottom of
the {\cb Advanced} page (this page should be shown initially).}
\end{enumerate}

There are a couple of things one may want to do here, as examples.

\begin{enumerate}
\item{Add the {\XicTools} {\vt bin} directory  to the system search path.\\\\
Scroll the lower {\cb System variables} window to find the {\vt Path}
entry.  Click on this to select it.  Click the {\cb Edit...} button
below, which brings up a text entry window.  In the {\cb Variable
value} window, scroll all the way to the right, and add, for example
(use the actual paths if different on your system)
\begin{quote}\vt
;c:$\backslash$usr$\backslash$local$\backslash$xictools$\backslash$bin
\end{quote}
Check the spelling, and make sure there is no white space, and that
the character before the `{\vt c}' is a semicolon, and the character
that follows the `{\vt c}' is a colon.  Then click the {\cb OK}
button.}

\item{Add a {\vt HOME} variable to define a ``home directory''.\\\\
Press the {\cb New...} button below the UPPER listing window (not the
one you just used).  This brings up a text entry as we saw before. 
Enter {\vt HOME} for {\cb Variable name}, For {\cb Variable value},
enter a path to some directory which you want to be your ``working''
directory, where {\Xic} and {\WRspice} will look for startup files,
etc.  Enter the full path to this directory.  Check spelling, Click
{\cb OK}.}
\end{enumerate}

Finally, click the {\cb OK} button at the bottom of the window, we're
done.

When a program is started from an icon, an icon property specifies the
directory where the program logically starts from.  This is the
apparent current directory seen by the user when running the program. 
By default, this is usually sonething like ``{\vt C:$\backslash$}'',
which is not a good choice.  The user should have a directory
dedicated for this, and the following procedure can be used to cause
the programs started from an icon to start in this directory.

\begin{enumerate}
\item{Go to the {\cb Start} page, click on the circle with down arrow
icon near the bottom-left corner.  This shifts to the {\cb Apps} page.}

\item{Find the {\cb XicTools} program group.  There should be entries
for the programs that you have installed.}

\item{For each program:\\
\begin{enumerate}
  \item{Click on the program icon with the RIGHT mouse button.  An
  icon banner along the bottom of the screen will appear.}

  \item{Click on {\cb Open File Location}.  This brings up a listing
  showing the {\cb XicTools} programs.}

  \item{Above the list, find the {\cb Properties} icon and click it. 
  This brings up a multi-page {\cb Properties} pop-up.}

  \item{In the {\cb Shortcut} page, change the entry in the {\cb Start
  in} entry area to a full directory path to the directory where the
  program should start.  This might be the same directory that you
  used for the {\vt HOME} environment variable.}

  \item{Then click the {\cb Apply} button, and click {\cb OK} if there
  is a confirmation pop-up.}
\end{enumerate}
}
\end{enumerate}

This applies to the icon in the {\cb Apps} page.  Other icons can be
set similarly.


% -----------------------------------------------------------------------------
% xic:cmdline 093017
\section{Command Line Options}

\index{command line}
The following syntax applies when {\Xic} is invoked from the command
line.  Arguments not recognized as options are expected to be files
containing layout information in supported formats.  The first such file
(if any) will be loaded into the editor.  Subsequent files can be loaded
sequentially with the {\cb Open} command.

\vspace*{5mm}
\begin{quote}\vt \rr
xic [-F {\it filetool\_args\/}] | [
  [-B{\it batch\_opt\/} | -S[{\it port\/}]
  [-C | -C1] [-E] [-G{\it geometry\_spec\/}]
  [-H{\it directory\_path}]
  \hspace*{1em}[-K{\it password}] [-L{\it serverhost\/}[:{\it port\/}]]
  [-R{\it prefix\_path\/}]
  [-T[{\it extension\/}]]
  [{\it toolkit\_options\/}]
  \hspace*{1em}[{\it filename} ...] ]
\end{quote}

{\Xic} will accept command line options common to applications
designed around the GTK user interface toolkit.  In addition, there
are a few command line options used exclusively by {\Xic}.  Options
are keyed by a hyphen `{\vt -}', and can not be grouped.  Above, the square
brackets indicate that the specification is optional (which applies to
all arguments), and the `{\vt |}' symbol is a logical ``OR'' operator
indicating that one may specify one of the surrounding forms.

\begin{description}
\index{command line!options!-B}
\index{-B option}
\index{batch mode!start in}
\item{\et -B}{\it batch\_opt\/}\\
{\Xic} supports a batch mode of operation, where {\Xic} will run a
script or perform certain commands without graphics.  The form for
this option is one of
\begin{quote}\vt
  -B{\it scriptfile\/}[,{\it args\/}...]\\
  -B-{\it command}[{\vt @}{\it arguments\/}]
\end{quote}
Batch mode will be described in \ref{batchmode}.
\end{description}

\index{colormap options}
The {\et -C} and {\et -C1} options apply only to ``pseudo-color''
displays.  These are displays with ``8-bits'' or ``256 colors'', found
on older workstations.  By default, {\Xic} uses a large percentage of
the system colormap.  If there are insufficient colormap entries
available, {\Xic} will create its own virtual colormap, which is
loaded when an {\Xic} window has the keyboard focus.  A problem is
that some X terminals and emulators apparently do not support virtual
colormaps, or do so improperly.  Also, the use of a virtual colormap
can be annoying.  For these reasons, options have been provided to
limit colormap usage, and avoid creation of a virtual colormap.

\begin{description}
\index{command line!options!-C}
\index{-C option}
\item{\et -C}\\
This option applies only in pseudo-color visual modes.  The {\et -C}
option, if given, will prevent {\Xic} from allocating private colors
from the system colormap.  Instead, it will use cells shared with
other applications.  The colormap usage can be dramatically reduced by
this option.  The cost is 1) the colors may not be quite ``right'' if
the colormap is already heavily used by other applications, 2) there
is no blinking, 3) the colors can not be changed, and 4) highlighting
may be difficult to see, as for the {\vt -C1} option.  A second copy
of {\Xic} running with the same technology file as the first will use
no additional colormap space.  A virtual colormap is never produced if
the {\et -C} option is given.  This option is recommended primarily
for users who want to run multiple copies of {\Xic} without the
virtual colormap.

\index{dual plane colorcells}
\index{-C1 option}
\item{\et -C1}\\
This option applies only in pseudo-color visual modes.  The {\et -C1}
option similarly saves colormap space by directing {\Xic} to allocate
single-plane cells.  By default, and if sufficient colormap space is
available, {\Xic} will allocate ``dual-plane'' color cells for the
layer rendering colors.  These cells contain two pixel values, one
representing the color, and one which is white.  The white pixel is
addressed during highlighting, and having one white pixel per layer
ensures that the exclusive-or drawing mode always produces white
highlighting.

Single-plane color cells use half the colormap space of dual plane
cells.  However, the exclusive-or highlighting is only guaranteed to
be white over the background, and the highlighting can take any color
over the layers.  This can sometimes be difficult to see.

\index{command line!options!-E}
\index{-E option}
\index{electrical mode!start in}
\item{\et -E}\\
The {\et -E} option signals {\Xic} to start in electrical mode.  The
default is to start in physical mode.

\index{command line!options!-F}
\index{-F option}
\index{filetool option}
\item{\et -F}\\
This option must be the first given, and arguments that follow must be
appropriate for the {\FileTool} utility (see Appendix \ref{filetool}). 
The program will behave as the command-line {\FileTool} program, which
can perform various manipulations and diagnostics on layout files.

If the {\vt xic}, {\vt xicii}, or {\vt xiv} binary executable files
(or Windows {\vt .exe} equivalents) are copied or linked under the
name ``{\vt filetool}'' (``{\vt filetool.exe}'' under Windows), the
new program will behave as a {\FileTool} when invoked.

\index{command line!options!-G}
\index{-G option}
\item{\et -G}{\it geometry\_spec}\\
The {\it geometry\_spec} is an X-style window geometry specification,
which allows the main window size and position to be specified.  There
is no space between {\vt -G} and the specification.  The command line
specification will override the {\et XIC\_GEOMETRY} variable.  The
format of the {\it geometry\_spec} is described with the environment
variable.

\index{command line!options!-H}
\index{-H option}
\item{\et -H}{\it directory\_path}\\
Giving this option will cause {\Xic} to start in {\it directory\_path}
as the current working directory.  Note that there is no space between
the ``{\vt -H}'' and the directory path.

\index{command line!options!-K}
\index{-K option}
\item{\et -K}{\it password}\\
The password used to enable use of encrypted scripts can be given to
{\Xic} on the command line with this option.  Note that there is no
space between the ``{\vt -K}'' and the password.  As the password can
contain almost any character, if the password contains characters
which could be misinterpreted by the shell, the password should be
quoted, e.g., {\vt -K'}{\it password\/}{\vt '}.

If no password is given to {\Xic} with the {\vt -K} option, a default
password is effective.  The default password has a key that is
compiled into the executable file, which can be changed with the {\vt
wrsetpass} utility.  The ``factory'' default password is
\begin{quote}
{\cb Default password}: {\vt qwerty}
\end{quote}
The password set with the {\vt -K} option overrides the default
password.  The password can also be set with the {\vt SetKey} script
function.

If the {\vt .xicinit} or {\vt .xicstart} file, or the function library
file, or a script run from batch mode is encrypted, the encryption
password must be given to {\Xic} with the {\vt -K} option, or be the
default password.  As the password can be changed with the {\vt
SetKey} script function, {\cb User Menu} scripts can in principle use
different passwords, which must be set before the script is executed.

%\ifxtlserv
%\index{command line!options!-L}
%\index{-L option}
%\item{\et -L}{\it serverhost}[:{\it port\/}]\\
%This supplies the host name of the machine running the license server,
%and optionally specifies the port number.  Note that there is no space
%after {\vt -L}.  If given, this will override the server host supplied
%by other means.
%
%Below is the logic hierarchy for setting the license server host, each
%method will override those listed lower.  See the documentation for
%the {\vt xtlserv} (license server) program for more information.
%\begin{quote}
%-L{\it serverhost}[:{\it port\/}]\\
%{\et XTLSERVER} in environment\\
%{\vt license.host} file\\
%{\vt xtlserver} in {\vt /etc/hosts}\\
%name of local machine
%\end{quote}
%\fi

\index{command line!options!-R}
\index{-R option}
\item{\et -R}{\it prefix\_path}\\
If given, the {\it prefix\_path} internally replaces ``{\vt
/usr/local}'' when {\Xic} composes directory paths to search for
startup files.  This will override the value of the {\et XT\_PREFIX}
environment variable.  This is one method of specifying to {\Xic} the
startup file location, if the distribution was installed in a
non-default location.  Under Windows, the installation location is
saved in the registry and is available to {\Xic}, so {\Xic} should be
able to find its startup files without this option.

\index{command line!options!-S}
\index{-S option}
\index{server mode!start in}
\item{\et -S}[{\it port\/}]\\
If the {\et -S} option is given, {\Xic} will run in server mode.  In
this mode, {\Xic} runs in the background as a daemon process, serving
requests through a communications port.  This mode will be described
in \ref{servermode}.  The option can be immediately followed (no
space) by a port number to use for connections.

\index{command line!options!-T}
\index{-T option}
\index{technology file!extension}
\index{xic\_tech file}
\item{\et -T}[{\it extension\/}]\\
The {\et -T}{\it extension\/} option is used to designate a particular
technology file, which is a file used by {\Xic} to initialize itself
to a particular manufacturing process and set of user preferences. 
The technology file has a name of the form {\vt xic\_tech} or {\vt
xic\_tech.}{\it extension\/}, the base name is always ``xic\_tech'',
but there may be an arbitrary extension (characters other than `.'
following `.').  If no {\et -T} option is given, then the {\vt
xic\_tech} file is used.  Otherwise, the {\it extension\/} given in
the option will signal {\Xic} to use the technology file with the same
extension.  Note that it is allowable to start {\Xic} without any
technology file, which is the effect of giving just the {\et -T}
without any extension.  Note that there must not be any space between
the {\et T} and the extension.
\end{description}

The graphical interface accepts the following options.  These options
are not processed by {\Xic}, but are intercepted by the graphics
subsystem and affect the interface to the X-window system.  The
multiple forms are equivalent.

\index{command line!options|(}
\index{command line!options!-d}
\index{command line!options!-display}
\index{command line!options!\symbol{45}\symbol{45}display}
\index{X display!specification}
\index{environment!DISPLAY}
\begin{description}
\item\parbox[b]{4in}{
{\et -d} {\it dispname}\\
{\et -display} {\it dispname}\\
{\et \symbol{45}\symbol{45}display} {\it dispname}}\\
This option specifies the name of the X display to use.  The {\it
dispname} is in the form
\begin{quote}
 [{\it host\/}]:{\it server\/}[.{\it screen\/}]\\
\end{quote}
The {\it host\/} is the host name of the
physical display, {\it server\/} specifies the display server number,
and {\it screen\/} specifies the screen number.  Either or both of the
{\it host\/} and {\it screen\/} elements to the display specification
can be omitted.  If {\it host\/} is omitted, the local display is
assumed.  If {\it screen\/} is omitted, screen 0 is assumed (and the
period is unnecessary).  The colon and (display) {\it server\/} are
necessary in all cases.  If no display is specified on the command
line, the display is set to the value of the environment variable {\et
DISPLAY}.

\index{command line!options!-name}
\index{command line!options!\symbol{45}\symbol{45}name}
\item\parbox[b]{4in}{
{\et -name} {\it string}\\
{\et \symbol{45}\symbol{45}name} {\it string}}\\
This option provides an alternative name to the application, as known
to the X window system.  The application name is used by X to apply
resource specifications.

\index{command line!options!\symbol{45}\symbol{45}class}
\item{\et \symbol{45}\symbol{45}class} {\it string}\\
This option provides an alternative class name to the application, as
known to the X window system.  The application class name is used by X
to apply resource specifications.

\index{command line!options!-synchronous}
\index{command line!options!\symbol{45}\symbol{45}sync}
\item\parbox[b]{4in}{
{\et -synchronous}\\
{\et \symbol{45}\symbol{45}sync}}\\
This option indicates that requests to the X server should be sent
synchronously, instead of asynchronously.  Since the X system normally
buffers requests to the server, errors do not necessarily get reported
immediately after they occur.  This option turns off the buffering so
that the application can be debugged more easily.  It should never be
used with a working program.

\index{command line!options!\symbol{45}\symbol{45}no-xshm}
\index{MIT-SHM extension}
\item{\et \symbol{45}\symbol{45}no-xshm}\\
In releases running under the X-Window system (Unix/Linux), {\Xic}
will use the MIT-SHM shared memory extension if the X server supports
this extension, and the server is running on the local machine.  This
allows image data to be transferred to the X server via shared memory,
which is faster than the normal X socket interface.  Screen updates
may be faster as a result.

Giving the option {\et \symbol{45}\symbol{45}no-xshm} on the command
line will prevent use of this extension, if for some reason this is
necessary.

\index{command line!options!\symbol{45}\symbol{45}v}
\item{\et \symbol{45}\symbol{45}v}\\
If this option is given, {\Xic} will print a string containing three
tokens and exit.  The tokens are
\begin{quote}
{\it version osname arch}       
\end{quote}
for example ``{\vt 4.3.1 LinuxRHEL7 x86\_64}''.

\index{command line!options!\symbol{45}\symbol{45}vv}
\item{\et \symbol{45}\symbol{45}vv}\\
If this option is given, {\Xic} will print a CVS-style tag string and
exit.  The format is, for exmple, ``{\vt xic-4-3-1}''.

\index{command line!options!\symbol{45}\symbol{45}vb}
\item{\et \symbol{45}\symbol{45}vb}\\
If this option is given, {\Xic} will print the build date and exit. 
\end{description}

Any words found in the command line that are not recognized as options
will be interpreted as files to load into {\Xic} for editing.  The
files will be loaded in order of their appearance, with the first file
loaded at startup, and the other files loaded in response to an {\cb
Open} command.
\index{command line!options|)}


% -----------------------------------------------------------------------------
% xic:env 112019
\section{{\Xic} Environment Variables}
\label{xicenv}
\index{environment}
Environment variables are keyword/value pairs that are made available
to an application by the command shell or operating system.  The value
of an environment variable is a text string, which may be empty. 
Environment variables can be set by the user to control various
defaults in {\Xic}.

\subsection{Unix/Linux}

Environment variables are maintained by the user's command shell.  It
is often convenient to set environment variables in a shell startup
file such as {\vt .cshrc} or {\vt .login} for the C-shell or {\vt
.profile} for the Bourne shell.  These files reside in the user's home
directory.  See the manual page for your shell for more information.

For the C-shell, the command that sets an environment variable is
\begin{quote}
{\vt setenv} {\it variable\_name} [{\it value\/}]
\end{quote}
For example,
\begin{quote}
{\vt setenv XT\_DUMMY "hello world!"}
\end{quote}
Note that if the value contains white space, it should be quoted. 
Note also that it is not necessary to have a value, in which case the
variable acts as a boolean (set or not set).

In the C-shell, one can use {\vt setenv} without arguments, or {\vt
printenv}, to list all of the environment variables currently set.

For a modern Bourne-type shell, such as {\vt bash}, the corresponding
command is
\begin{quote}
{\vt export} {\it variable\_name\/}[={\it value\/}]
\end{quote}
In this type of shell one can list the variables currently set by
giving the {\vt set} command with no arguments.

\subsection{Microsoft Windows}

Under Windows, environment variables can be set in a DOS box with the
{\vt set} command before starting the program from the command line,
or in the {\vt AUTOEXEC.BAT} file, or from the {\cb System} entry in
the {\cb Control Panel}.  Only the latter two methods work if the
programs are started from an icon.  If using a Cygwin bash-box,
environment variables can be set in the startup file as under Unix.

\subsection{{\XicTools} Environment Variables}

The following environment variables are used by all {\XicTools}
programs.

\begin{description}
\index{environment!CYGWIN\_BIN}
\item{\et CYGWIN\_BIN}\\
This variable applies only when running under Microsoft Windows, and
Cygwin is installed.  Cygwin is Linux-like environment and tool set
which is a very useful adjunct to Windows.  In particular, it provides
a bash shell with standard Linux commands, and an X server, among many
installable features.

{\XicTools} programs will in some cases, such as when popping up a
shell window, look for a Cygwin program.  If the Cygwin program
binaries ({\vt .exe} files) are located in {\vt /bin} or {\vt
/cygwin/bin} on the current disk drive, they will be found
automatically.  Otherwise, this variable can be set to the Windows
path, including a drive letter if necessary, to the directory
containing the Cygwin binaries.  This is not necessarily the path one
perceives from within Cygwin, since the {\XicTools} programs do not
know about the Cygwin mount points or symbolic links.  The path is the
one that would be seen from a DOS box, with forward or reverse slash
directory separators.

%\ifxtlserv
%\index{environment!XT\_AUTH\_MODE}
%\item{\et XT\_AUTH\_MODE}\\
%By default, Unix/Linux versions of {\Xic} and {\WRspice} use
%authorization provided by an external license server, possibly hosted
%on a different machine.  On the other hand, the Windows versions use
%built-in local authentication.  Both the external license server and
%the programs not using the license server make use of a file named
%``{\vt LICENSE}'' provided by Whiteley Research, Inc., which provides
%authorization to run on the host computer.
%
%Whether to use a license server or local authentication can be set
%with the environment variable {\et XT\_AUTH\_MODE}.  This variable has
%meaning if set to one of the keywords ``{\vt Server}'' or ``{\vt
%Local}''.  If set to ``{\vt Server}'' on a Windows machine, {\Xic} and
%{\WRspice} can by authenticated from a license server.  If set to
%``{\vt Local}'' on a Linux/OS X system, {\Xic} and {\WRspice}
%will be self-validating.
%
%{\Xic} and/or {\WRspice} users on a single licensed workstation may
%prefer to set the environment variable in their shell startup file and
%not use the external license server.
%
%When the programs look for the LICENSE file in ``{\vt Local}'' mode,
%if the file is not found in the startup or license directories, the
%programs will look in the home and current directories, in that order,
%unless {\et XT\_LICENSE\_PATH} is also set.
%
%\index{environment!XT\_LICENSE\_PATH}
%\index{XT\_LICENSE\_PATH environment variable}
%\item{\et XT\_LICENSE\_PATH}\\
%When using local validation (i.e., not using the license server) {\et
%XT\_LICENSE\_PATH} can be set to the full path to the license file. 
%Only this file will be used -- the regular search is suppressed.
%
%\index{environment!XTLSERVER}
%\index{XTLSERVER environment variable}
%\item{\et XTLSERVER}\\
%This provides the host name of the host running the license server
%needed to validate the application.  It is in a format understandable
%to the local name server.  The host name can optionally be suffixed by
%``{\vt :}{\it port\/}'', where {\it port} is the port number in use by
%the server.  There should be no space around the colon when using this
%form.
%\fi

\index{environment!XT\_PREFIX}
\index{XT\_PREFIX environment variable}
\item{\et XT\_PREFIX}\\
All of the {\XicTools} programs respond to the {\et XT\_PREFIX}
environment variable.  When the tools are installed in a non-standard
location, i.e., other than {\vt /usr/local}, this can be set to the
directory prefix which effectively replaces ``{\vt /usr/local}'', and
the programs will be able to access the installation library files
without further directives.  The {\Xic} {\et -R} command line option
can also be used for this purpose.  This should not be needed under
Windows, as the Registry provides the default paths.

\index{environment!XT\_HOMEDIR}
\index{XT\_HOMEDIR environment variable}
\item{\et XT\_HOMEDIR}\\
This should be set to the Windows path to the user's MSYS2 home
directory.  This can be done from the {\vt bash\_profile} file by
adding a line
\begin{quote}
{\vt export XT_HOMEDIR=c:/msys64/home/}{\ityourlogin}
\end{quote}
Setting this will allow {\Xic} and {\WRspice} to find files in the
user's MSYS2 home directory, even though the programs are Windows
native and don't know the MSYS2 paths.

\index{environment!XTNETDEBUG}
\index{XTNETDEBUG environment variable}
\item{\et XTNETDEBUG}\\
If the variable {\et XTNETDEBUG} is defined, {\Xic} and {\WRspice}
will echo interprocess messages sent and received to the console.  In
server mode, {\Xic} will not go into the background, but will remain
in the foreground, printing status messages while servicing requests.

\index{memory management}
\index{environment!XT\_LOCAL\_MALLOC}
\index{XT\_LOCAL\_MALLOC environment variable}
Linux and FreeBSD releases can use an included local memory allocation
package.  In earlier {\Xic} releases, this allocator, rather than the
allocator provided by the operating system, was used by default.  In
32-bit releases, the local allocator was often able to allocate more
memory than the allocators provided by the operating system.  It also
provided custom error reporting and statistics.

This feature is now disabled, as in modern operating systems there is
dubious benefit, and it can produce stability problems in some cases. 
However, if this variable is set in the environment when {\Xic} is
started, the local allocator will be used.  The interested user is
encouraged to experiment.

\index{environment!XT\_SYSTEM\_MALLOC}
\index{XT\_SYSTEM\_MALLOC environment variable}
\item{\et XT\_SYSTEM\_MALLOC}\\
This variable was once used to disable the internal local memory
allocator, which in earlier releases was enabled by default. 
Currently, this variable is ignored.

\index{environment!XT\_GUI\_COMPACT}
\index{XT\_GUI\_COMPACT environment variable}
\item{\et XT\_GUI\_COMPACT}\\
When set, no extra space is allowed around pushbutton contents in the
graphical interface.  Such space can cause side menu button images to
be truncated on low-resolution displays if the theme in use imposes
too much space.  Setting this variable is a quick fix for this
problem, though one could also change the theme.
\end{description}

\subsection{{\Xic} Environment Variables}

The following paragraphs describe the environment variables which are
relevant to {\Xic}.

\begin{description}
\index{environment!FORCE\_XICII}
\index{FORCE\_XICII environment variable}
\item{\et FORCE\_XICII}\\
If this variable is set when {\Xic} starts, the program will run as
{\XicII}.  {\XicII} was a reduced feature set (layout editor only)
version of {\Xic} available at lower cost.  Operating in this mode may
simplify things for some users.  One can create an ``{\vt xicii}''
program with the following shell script:

\begin{quote}\vt
\#! /bin/sh\newline\newline
FORCE\_XICII=1 xic \$*
\end{quote}

\index{environment!FORCE\_XIV}
\index{FORCE\_XIV environment variable}
\item{\et FORCE\_XIV}\\
If this variable is set when {\Xic} starts, the program will run as
{\Xiv}.  {\Xiv} was a reduced feature set (layout viewer only) version
of {\Xic} available at lower cost.  Operating in this mode may
simplify things for some users.  One can create an ``{\vt xiv}''
program with the following shell script:

\begin{quote}\vt
\#! /bin/sh\newline\newline
FORCE\_XIV=1 xic \$*
\end{quote}

\index{environment!XIC\_HOME}
\index{XIC\_HOME environment variable}
\item{\et XIC\_HOME}\\
This environment variable applies only to the {\Xic} program.  If
found in the environment when {\Xic} starts, it is expected to contain
a path to the {\Xic} installation area or equivalent, which defaults
to ``{\vt /usr/local/xictools/xic}''.  This overrides {\et XT\_PREFIX}
if that environment variable is also found.
 
There is an important subtlety when using this variable.  Although it
allows {\Xic} to find its startup files anywhere, only the directory
structure implied by {\et XT\_PREFIX}, that is, for {\Xic},
\begin{quote} \vt
\$XT\_PREFIX/xictools/xic
\end{quote}
is compatible with the program installation script.  The variable is
perhaps useful for pointing {\Xic} toward a secondary set of startup
files, perhaps heavily customized by the user, which may reside in
an arbitrary location.

\index{environment!XIC\_GEOMETRY}
\index{XIC\_GEOMETRY environment variable}
\item{\et XIC\_GEOMETRY}\\
This can be set to an X-style geometry string, to specify the 
default size and position of the {\Xic} main window.

If the geometry has been specified, {\Xic} will use it to position and
size the main window (if the window manager permits this).  The
geometry specification, used to define window size and position, is a
string in the form
\begin{quote}
{\it width\/}{\vt x}{\it height\/}$+${\it xoff\/}$+${\it yoff}
\end{quote}
where {\it width\/}, {\it height\/}, {\it xoff\/}, and {\it yoff} are
numbers representing screen pixels.  The ``{\vt x}'' or ``{\vt X}''
between the {\it width} and {\it height} is literal.  A plus sign
`$+$' or minus sign `$-$' must appear ahead of {\it xoff} and {\it
yoff}.

\begin{description}
\item{$+${\it xoff}}\\
The left edge of the window is to be placed {\it xoff} pixels in from
the left edge of the screen.

\item{$-${\it xoff}}\\
The right edge of the window is to be placed {\it xoff} pixels in from
the right edge of the screen.

\item{$+${\it yoff}}\\
The top edge of the window is to be {\it yoff} pixels below the top
edge of the screen.

\item{$-${\it yoff}}\\
The bottom edge of the window is to be {\it yoff} pixels above the
bottom edge of the screen.
\end{description}

\index{tech directory}
\index{environment!XIC\_TECH\_DIR}
\index{XIC\_TECH\_DIR environment variable}
\item{\et XIC\_TECH\_DIR}\\
The value is a path to a directory.  If given, the directory is
searched for the technology file, if not found in the current
directory, and before other locations are checked.

\index{temp directory}
\index{environment!XIC\_TMP\_DIR}
\index{XIC\_TMP\_DIR environment variable}
\item{{\et XIC\_TMP\_DIR}, {\et TMPDIR}}\\
By default, {\Xic} uses the directory {\vt /tmp} for temporary files. 
In some installations, this directory may be too small to accommodate
the large files needed by {\Xic}, for example when producing hard copy
plots.  An alternative directory for temporary files can be specified
with the {\et XIC\_TMP\_DIR} environment variable (which has
precedence) or with the {\et TMPDIR} variable, which is a Unix
standard.  One of these should be set to a path to a directory to use
for temporary files, if necessary.

\index{log files}
\index{environment!XIC\_LOGDIR}
\index{XIC\_LOGDIR environment variable}
\item{\et XIC\_LOGDIR}\\
The variable {\et XIC\_LOGDIR} can be set to a path to a directory
which will be used to store certain log files produced while {\Xic} is
running.  The location used for the log files is the first defined of
{\et XIC\_LOGDIR}, {\et XIC\_TMP\_DIR}, {\et TMPDIR}, or {\vt /tmp} if
none of these variables is defined.  The log files are removed on
normal exit.

\index{menu!right side placement}
\index{environment!XIC\_MENU\_RIGHT}
\index{XIC\_MENU\_RIGHT environment variable}
\item{\et XIC\_MENU\_RIGHT}\\
If the variable {\et XIC\_MENU\_RIGHT} is defined in the environment,
{\Xic} will place the side menu and layer table to the right of the
main window.  The default to to place the menu at the left.

\index{menu!buttons on top}
\index{environment!XIC\_HORIZ\_BUTTONS}
\index{XIC\_HORIZ\_BUTTONS environment variable}
\item{\et XIC\_HORIZ\_BUTTONS}\\
If this variable is set in the environment when {\Xic} starts, the
buttons in the side menu will be arrayed horizontally across the top
of the main window instead.

\index{environment!XIC\_PLUGIN\_DBG}
\index{XIC\_PLUGIN\_DBG environment variable}
\item{\et XIC\_PLUGIN\_DBG}\\
If this variable is set in the environment when {\Xic} starts, error
messages concerning plug-in loading will be printed in the console
window.  Without this set, {\Xic} will simply silently not load a
plug-in if an error occurs.  These diagnostic messages can help
identify why the plug-in is not being loaded, and are instrumental in
tracking down problems when the user expects success.

\index{environment!XIC\_START\_DIR}
\index{XIC\_START\_DIR environment variable}
\item{{\et XIC\_START\_DIR}, {\et HOME}}\\
Under Windows, the user's ``home'' directory is determined by looking
at environment variables.  The first one found to be set is assumed to
contain a path to the user's home directory.  First, {\et
XIC\_START\_DIR} is checked.  This is {\Xic} (and family) specific, and
would be set previously by the user.  If not found, {\et HOME} is
checked.  This can be set by environments such as Cygwin, when
starting in a Cygwin shell window, or may be set by other environments
or by the user.  If not found, the {\et HOMEDIR} and {\et HOMEPATH}
variables, if both are found, are concatenated to yield the home
directory path.  In the unlikely event that these are not set, the
{\et USERPROFILE} variable is checked, and if all else fails, ``{\vt
C:$\backslash$}'' is assumed.  The {\et HOMEDIR}/{\et HOMEPATH} and
{\et USERPROFILE} variables are set by Windows, at least in recent
Windows versions.

Under other operating systems, the home directory is well-defined
and is obtained from operating system calls.

Under Windows, if {\Xic} finds itself in the {\vt C:$\backslash$}
directory on startup, it will change the working directory to the home
directory.  This is the default when starting from the Windows {\cb
Start Menu} or otherwise from an icon, unless the icon property is
changed.

\index{exit command}
\index{environment!XIC\_EXIT\_CMD}
\index{XIC\_EXIT\_CMD environment variable}
\item{\et XIC\_EXIT\_CMD}\\
If the environment variable {\et XIC\_EXIT\_CMD} is set to a command string,
that command will be executed when {\Xic} exits.  If the command string
contains spaces, the command should be quoted.  For example, using
\begin{quote}
{\vt setenv XIC\_EXIT\_CMD "/usr/games/fortune -o"}
\end{quote}
may print a rude limerick on some installations.  This feature may
have less frivolous uses, however.

\index{environment!XIC\_SYM\_PATH}
\index{XIC\_SYM\_PATH environment variable}
\index{environment!XIC\_LIB\_PATH}
\index{XIC\_LIB\_PATH environment variable}
\index{environment!XIC\_HLP\_PATH}
\index{XIC\_HLP\_PATH environment variable}
\index{environment!XIC\_SCR\_PATH}
\index{XIC\_SCR\_PATH environment variable}
\item{{\et XIC\_SYM\_PATH}, {\et XIC\_LIB\_PATH}, {\et XIC\_HLP\_PATH},
  {\et XIC\_SCR\_PATH}}\\
There are four additional environment variables used to specify
locations where {\Xic} is to look for certain types of files.  These
variables are {\et XIC\_SYM\_PATH}, {\et XIC\_LIB\_PATH}, {\et
XIC\_HLP\_PATH}, and {\et XIC\_SCR\_PATH}.  These variables are
described in the next section.

The internal default values for the paths assume that the installation
location is the standard place under {\vt /usr/local}, or if the {\et
XT\_PREFIX} variable is set, that value will be taken instead of
``{\vt /usr/local}''.

\index{environment!XIC\_DOCS\_DIR}
\index{XIC\_DOCS\_DIR environment variable}
\item{\et XIC\_DOCS\_DIR}\\
The environment variable {\et XIC\_DOCS\_DIR} can be set to an
alternate location for the archive of release notes.  This location is
searched in the {\cb Release Notes} command in the {\cb Help Menu}. 
The default location is {\vt /usr/local/xictools/xic/docs}, or,
if {\et XT\_PREFIX} is set, its value will replace {\vt /usr/local}.

%\ifoa
\index{environment!XIC\_OASO\_PATH}
\index{XIC\_OASO\_PATH environment variable}
\item{\et XIC\_OASO\_PATH}\\
Plugins are normally found in the {\vt plugins} directory in the
installation area, which by default is
\begin{quote}
\vt /usr/local/xictools/xic/plugins
\end{quote}
This variable can be set to the full path to the OpenAccess plug-in,
which {\Xic} will attempt to load on program startup instead of
looking in the default location.
%\fi

\index{environment!XIC\_PYSO\_PATH}
\index{XIC\_PYSO\_PATH environment variable}
\item{\et XIC\_PYSO\_PATH}\\
Plugins are normally found in the {\vt plugins} directory in the
installation area, which by default is
\begin{quote}
\vt /usr/local/xictools/xic/plugins
\end{quote}
This variable can be set to the full path to the Python plug-in,
which {\Xic} will attempt to load on program startup instead of
looking in the default location.

\index{environment!XIC\_TCLSO\_PATH}
\index{XIC\_TCLSO\_PATH environment variable}
\item{\et XIC\_TCLSO\_PATH}\\
Plugins are normally found in the {\vt plugins} directory in the
installation area, which by default is
\begin{quote}
\vt /usr/local/xictools/xic/plugins
\end{quote}
This variable can be set to the full path to the TclTk or Tcl-only
plug-in, which {\Xic} will attempt to load on program startup instead
of looking in the default location.

\index{environment!XIC\_LIBRARY\_PATH}
\index{XIC\_LIBRARY\_PATH environment variable}
\item{\et XIC\_LIBRARY\_PATH}\\
This applies to Linux and OS X only.  If set, the value will be
prepended to the {\vt LD\_LIBRARY\_PATH} in the {\Xic} wrapper script. 
This can be used to point to installed libraries needed for plugins,
%\ifoa
for example the OpenAccess libraries,
%\fi
without having to set {\vt LD\_LIBRARY\_PATH} in the environment. 

\index{environment!XICNOMAIL}
\index{XICNOMAIL environment variable}
\item{\et XICNOMAIL}\\
If the variable {\et XICNOMAIL} is set, no mail will be sent during a
crash.  If a fatal error is encountered, a file named ``{\vt gdbout}''
is created in the current directory, which contains a stack backtrace
from the stack frame of the error.  Despite the name, the file is
generated internally on all platforms, and no longer makes use of the
{\vt gdb} program.
 
By default, this file will be emailed to Whiteley Research for
analysis.  However, the emailing can be suppressed by setting this
variable in the environment.  The {\vt gdbout} file is produced in any
case, and would be very useful to Whiteley Research for fixing program
bugs.

\index{environment!XTNOMAIL}
\index{XTNOMAIL environment variable}
\item{\et XTNOMAIL}\\
This has the same effect as {\et XICNOMAIL} but also prevents email
from the {\WRspice}> program.

\index{environment!SPICE\_HOST}
\index{SPICE\_HOST environment variable}
\index{environment!SPICE\_EXEC\_DIR}
\index{SPICE\_EXEC\_DIR environment variable}
\index{environment!SPICE\_EXEC\_NAME}
\index{SPICE\_EXEC\_NAME environment variable}
\item{{\et SPICE\_HOST}, {\et SPICE\_EXEC\_DIR}, {\et SPICE\_EXEC\_NAME}}\\
When connecting to SPICE in the {\cb run} command, the {\et
SPICE\_HOST} variable is used to set the name of a remote SPICE host
which provides SPICE service.  The name can optionally be followed by
a colon and a port number, if a non-default port is used by the SPICE
server.  The {\et SPICE\_EXEC\_DIR} environment variable provides the
directory which contains the {\vt wrspice} executable, which may need
to be identified to {\Xic} if it is other than {\vt /usr/local/bin}. 
The {\et SPICE\_EXEC\_NAME} environment variable can be used to
provide an alternate name for the {\vt wrspice} executable, if it has
been changed.  The default is, of course, ``{\vt wrspice}''.  Each or
these environment variables can be overridden by a corresponding
internal variable, which can be set with the {\cb !set} command.

\index{environment!IMSAVE\_PATH}
\index{IMSAVE\_PATH environment variable}
\item{\et IMSAVE\_PATH}\\
The printing interface includes a driver for generating image files in
various formats.  A few formats are handled internally, however vastly
more are available through other software that may be available on the
system.  The driver can usually locate these programs by looking in
standard places, however, if the programs exist but can't be located,
this variable can be set to a colon-separated list of directories to
search for the executables.  This applies to Unix/Linux/OS X only. 
See the description of the {\bf Image} print driver in \ref{hcformat}
for more information.
\end{description}


% -----------------------------------------------------------------------------
% xicpaths 031611
\section{{\Xic} Search Paths}
\label{searchpaths}
\index{paths}
\index{search paths}

There are four search paths used by {\Xic}.  Search paths are lists of
directories, which are searched in left-to-right order for files of a
particular type.  In addition to search paths, {\Xic} provides a
``redirect file'' mechanism for finding files, which supplements the
search path.  If a specific file is being sought, the first file with
matching name is used.  The format used for search path strings can be
one of two forms:

\begin{description}
\item{{\bf Unix-shell style:}\hspace{5mm}
  {\vt ( directory{\it 1} directory{\it 2} ...  directory{\it N} )}}\\
The tokens are separated by white space.  If white space appears in a
directory entry, that entry should be single or double quoted.  The
entire path should be enclosed in parentheses.  Space between the
parentheses and directory names is optional.

Examples:
\begin{quote}
{\vt ( ~. )}\\
{\vt ( /usr/local/bin "/Program Files/xic/stuff" \symbol{126}/work )}
\end{quote}

This format is the same in Windows and Unix releases, however in
Windows, back and forward slashes are equivalent, and the drive
specifier can appear in the entries.

\item{{\bf Traditional search path:}\hspace{5mm}
  {\vt directory{\it 1\/}:directory{\it 2\/}:...:directory{\it N}}}\\
The entries are separated by a special character, which is a colon
`{\vt :}' in Unix/Linux, and a semicolon `{\vt ;}' in Windows.  There
should be no white space that is not part of a file/directory name. 
An entry should be single or double quoted if it contains the
separation character.  In the examples here, a colon is used, which in
Windows must be converted to a semicolon.  The separation character is
optional at the front or end of the path, unless it is needed to
delimit white space that is part of an entry.

Examples:
\begin{quote}
{\vt .}\\
{\vt /usr/local/bin:/Program Files/xic/stuff:\symbol{126}/work}
\end{quote}
\end{description}

In earlier {\Xic} releases, parsing was fairly loose, and in
particular hybrids of the two formats would be accepted.  This is not
true in the present release, due to support for white space in path
entries.  The format used in a path string must be consistent.

The following special symbols are recognized in entries:\\
\begin{tabular}{|l|p{3.5in}|} \hline
\vt . & The current directory\\ \hline
\vt .. & The parent directory of the current directory\\ \hline
\vt \symbol{126}  & The user's home directory (Unix) or the content of the
    {\et HOME} environment variable (Windows)\\ \hline
\vt \symbol{126}joe & The home directory of user {\vt joe} (Unix only, no
   substitution in Windows)\\ \hline
\end{tabular}

\index{cell data path}
\index{paths!cell data}
The four paths are the design data path, the library path, the help
path, and the script path.  The design data path is used to locate
design data files, consisting of native cell, archive, and library
files.  The library path is used to locate the technology file, device
and model libraries, and various other configuration files.  The help
path contains files for the help system, and the script path contains
executable scripts and libraries which appear as commands in the {\cb
User Menu}.

These paths can be set in the technology file, the {\vt .xicinit} or
{\vt .xicstart} initialization files, or by use of environment
variables, or with the {\cb !set} command.  A specification in the
{\vt .xicinit} will override specification in the environment, which
is in turn superseded by a specification in the technology file, and
the {\vt .xicstart} file supersedes the technology file.  Once {\Xic}
is running, the {\cb !set} command can be used to set or examine the
search paths.  Similar commands exist in the script interpreter
interface function library.

\index{design data path!updating}
\index{paths!design data!updating}
In addition, the design data path is augmented with any path preceding
a native cell file to open in the {\cb Open} command.  By default, the
path is added to the beginning of the present design data path.  For
example, suppose a design hierarchy exists in the directory {\vt
/usr/work}.  If the user enters {\vt /usr/work/maincell} in response
to the prompt which appears after pressing the {\cb Open} button, then
the file {\et maincell} is opened for editing, and the directory {\vt
/usr/work} is added to the front of the design data path.  Once the
design data path is updated, the cells in that path can be accessed by
their base file name only.  The treatment of any path which is given
with a native cell to open in the {\cb Open} command can be altered
with the {\et NoReadExclusive} and {\et AddToBack} variables.

The use of paths facilitates user customization of {\Xic}, particularly
when the directories used in the system installation are not writable
by the user.  By installing a different search path, the user can
augment or substitute for the system default files and libraries.

Below are the environment variable names and internal defaults:
\index{environment!XIC\_SYM\_PATH}
\index{environment!XIC\_HLP\_PATH}
\index{environment!XIC\_LIB\_PATH}
\index{environment!XIC\_SCR\_PATH}

\begin{description}
\item{Design Data Path}\\
\begin{tabular}{ll}
variable: & {\et Path}\\
environment: & {\et XIC\_SYM\_PATH}\\
default: & {\vt ( ~. )}
\end{tabular}
\item{Library Path}\\
\begin{tabular}{ll}
variable: & {\et LibPath}\\
environment: & {\et XIC\_LIB\_PATH}\\
default: & {\vt ( ~. /usr/local/xictools/xic/startup )}
\end{tabular}
\item{HelpPath}\\
\begin{tabular}{ll}
variable: & {\et HlpPath}\\
environment: & {\et XIC\_HLP\_PATH}\\
default: & {\vt ( /usr/local/xictools/xic/help )}
\end{tabular}
\item{ScriptPath}\\
\begin{tabular}{ll}
variable: & {\et ScriptPath}\\
environment: & {\et XIC\_SCR\_PATH}\\
default: & {\vt ( /usr/local/xictools/xic/scripts )}
\end{tabular}
\end{description}

If the {\et XT\_PREFIX} environment variable is set, its value will be
taken instead of ``{\vt /usr/local}'' in the defaults.

The ``variable'' field in the table above provides the name of the
variable, which can be altered with the {\cb !set} command to set the
path.  Unlike other variables, these are always defined and cannot be
unset.  The same name is also used as a keyword in the technology
file.

Files containing cell data, whether {\Xic} native, GDSII, or some
other format, are expected to be found in a directory along the design
data search path.  The first file found matching the name requested is
opened.  Normally, it is desirable to include the current directory
`.' in the design data path, otherwise files located in the current
directory will not be found.

\index{library path}
\index{paths!library}
The technology file, {\vt device.lib} file, {\vt model.lib} file and
other model files are found along the library path.

The search behavior of the library path is slightly different from the
other paths, in that an attempt is made to open a file in the current
directory before looking through the search directories.  Thus, the
current directory `.' is always logically at the head of the library
path.  There is no problem if `.' is also explicitly defined in the
path.  A consequence is that startup files that exist in the current
directory will {\it always} have precedence over files located in
other directories.

\index{help path}
\index{help mode! path}
\index{paths!help}
Each directory in the help path is expected to contain help database
files.  These files use names with an extension ``{\vt .hlp}''.  The
directories may also contain graphics files used by the help system. 
Changing this path allows the user to provide their own help files for
the custom functions (scripts) which appear in the {\cb User Menu},
for example, or to add information topics, such as about local design
rules, to the database.

\index{script path}
\index{paths!script}
The scripts and related files are found along the script path.  Only
files which have the extension ``{\vt .scr}'' are taken as scripts. 
The directories in this path may also contain script menus, with
extension ``{\vt .scm}'', and files named ``{\vt library}'' which
contain subroutines used by other scripts.  Whenever the script path
is changed, a {\cb rehash} is performed, i.e., the {\cb User Menu} is
rebuilt.


% -----------------------------------------------------------------------------
% xt\_redirect 031611
\section{Redirect Files}
\index{redirect files}
\index{xt\_redirect file}
Redirect files are an adjunct to the search path mechanism used by
{\Xic} for finding files.  Redirect files are files created by the
user, that tell {\Xic} about additional locations to search for input
files.

Redirect files {\bf must} be named ``{\vt xt\_redirect}'', and are
text files with the following format and properties:

\begin{itemize}
\item{Lines that start with `{\vt \#}' or contain only white space
  are ignored.}
\item{Each line otherwise contains one or more directory paths,
  separated by white space.  If a directory path contains white
  space or other special characters, it should be double-quoted
  (i.e., as {\vt "..."}).}
\item{Multiple directories can be provided on a single line, or in
  different lines.}
\item{Paths that are not rooted are taken as relative to the
  directory containing the redirect file.}
\item{Paths that do not point to an existing directory are silently
  ignored.}
\end{itemize}

When searching a directory, the directories found in a redirect file
are also searched, in order, after the current directory.  The search
is recursive, so that arbitrarily deep hierarchies can be searched via
the redirect file mechanism.

With redirect files, only the top directory of a hierarchy needs to be
included in the search path (or given explicitly).  This can be very
convenient for organizing collections of native cell files, for
example.

The {\cb Path Files Listing} panel from the {\cb File Menu} will list
files found through the redirect files on separate pages for each
redirected directory, just as for the directories contained in the
search path.


% -----------------------------------------------------------------------------
% initfiles 102817
\section{Initialization Files}
\index{initialization files}

When {\Xic} is started, a number of files are read.  This section
describes these files, and the order of access.  None or these files
is required to exist.

Prior releases of {\Xic} could be configured to check for the
availability of program updates on startup.  There was also provision
for display of a message if one was ``broadcast'' from the Whiteley
Research web site.  This latter feature was never used, and neither
feature is currently supported in {\Xic}.  Thus, there is no longer a
network access attempt on program startup, which may save time.

Program updates are handled in the help system (see \ref{update}), for
all of the {\XicTools} packages.  Either the help system built into
{\Xic} and {\WRspice}, or the stand-alone {\it mozy} program can be
used to check for, download, and install updates.  Giving the keyword
``{\vt :xt\_pkgs}'' will display a page that provides update
information and download/install buttons.

If a new {\Xic} release is run for the first time, the release notes
will appear in a pop-up window, as if the {\cb Notes} button in the
{\cb Help} menu was pressed.  There is a file in the user's {\vt
.wr\_cache} directory named {\vt xic\_current\_release} that contains a
release number.  If, when {\Xic} starts, this file is missing or the
release number is not current, {\Xic} will show the release notes and
update the file.  If the release numbers match, there is no action.

On installation, a default configuration is provided for {\Xic}.  The
user will need to reconfigure {\Xic} for their requirements.  This
reconfiguration is accomplished primarily by editing a custom
technology file, which {\Xic} reads on startup, and also by possibly
setting some of the environment variables before starting {\Xic}. 
These variables can be set in the user's shell startup file, as
appropriate for the user's operating system. 

The default technology file, plus several other files needed, are
placed in a system-wide location on installation, usually {\vt
/usr/local/xictools/xic/startup}, which is included in library
path.  This directory is typically set to be read-only, thus the user
must establish an alternative location in their own directory tree for
customized startup files, and add this to the library path to the left
of or instead of the default location.  The default technology file
provided with {\Xic} is for generic MOSIS scalable CMOS.

\begin{description}
%\ifxtlserv
%\index{license.host file}
%\item{{\vt license.host} file}\\
%When using a license server on a remote machine, it is necessary to
%provide the name of this machine or {\Xic} can not run.  One way to do
%this is to create a {\vt license.host} file in the startup directory,
%e.g. {\vt /usr/local/xictools/xic/startup}.  The {\vt
%license.host} file consists of a single line of text, giving the host
%name of the license server machine.  The host name can optionally be
%suffixed with ``{\vt :}{\it port\/}'', where {\it port} is the port
%number in use by the license server.  This is required if for some
%reason the license server is not running on the default port.
%\fi

\item{X resource file}\\
As the program starts and the graphics is initialized, the X window
system may access various files for resource resetting.  See the X
documentation for details.  The attribute (non-layer) colors used in
{\Xic} can be set through the resource mechanism (see
\ref{resourcefile}), but one must take care that these are not reset
in the technology file.

\index{.xicinit file}
\item{{\vt .xicinit} file}\\
Next, an ``{\vt .xicinit}'' initialization script, if present, will be
read and executed.  The user may create this file, it is not present
by default.  The initialization script uses exactly the same format as
other script files, as are normally found along the script search
path.  The script can set user preferences or otherwise modify {\Xic}. 
Since this file is read before other files, it can be used to set the
search paths used to find other startup files, in particular the
technology file.  The base name for the script is ``{\vt .xicinit}'',
and the same extension as the technology files can be present.

If, for example, {\Xic} is started with an extension ``{\vt .ext}''
({\et -Text} given on the command line), {\Xic} will look for files
{\vt ./.xicinit.ext} and {\vt \$HOME/.xicinit.ext}, then {\vt
./.xicinit} and {\vt \$HOME/.xicinit}, in that order, where ``{\vt
\$HOME}'' indicates the user's home directory.  The first file found
will be executed.  If {\Xic} is started without a technology file
extension, only the script files without an extension will be
executed.

\index{technology file}
\item{Technology file}\\
If a technology file is being used, {\Xic} will read the file at this
point, before reading the user's script and macro files (below).

The technology file contains all of the information {\Xic} needs for
physical and electrical layout, extraction, and design rule checking,
plus information on hard copy support, printer commands, and the like. 
It also provides values for a number of presentation attributes
including the colors used on-screen.

The {\cb Save Tech} button in the {\cb Attributes Menu} creates an
updated copy of the technology file in the current directory.  Most of
the changes to an existing technology file can be performed from
within {\Xic}, though some text editing may be required on occasion.

\index{.xicstart file}
\item{{\vt .xicstart} file}\\
Next, an initialization script, if present, will be read and executed. 
This file can be created by the user, is is not present by default. 
The initialization script uses exactly the same format as other script
files, as are normally found along the script search path.  The script
can set user preferences or otherwise modify {\Xic}, and, unlike the
similar ``{\vt .xicinit}'' file, performs these commands after the
technology file has been read.  The base name for the script is ``{\vt
.xicstart}'', and the same extension as the technology files can be
present.

If, for example, {\Xic} is started with an extension ``{\vt .ext}''
({\vt -Text} given on the command line), {\Xic} will look for the
files {\vt ./.xicstart.ext} and {\vt \$HOME/.xicstart.ext}, and then
{\vt ./.xicstart} and {\vt \$HOME/.xicstart}, in that order, where
``{\vt \$HOME}'' indicates the user's home directory.  The first file
found will be executed.  If {\Xic} is started without a technology
file extension, only the script files without an extension will be
executed.

\index{xic\_stipples file}
\item{{\vt xic\_stipples} file}\\
The {\vt xic\_stipples} file is read, which initializes the default
fill pattern registers in the fill pattern editor in the {\cb
Attributes Menu}.  Like the device and model libraries, the technology
file, font files, etc., the library search path is used to locate this
file.  A default stipple file is provided, and new files can be
obtained from the {\cb Dump Defs} button in the {\cb Fill Pattern
Editor}.

\index{.xicmacros file}
\item{{\vt .xicmacros} file}\\
Next, {\Xic} will attempt to read a file with the base name ``{\vt
.xicmacros}'', and the same extension as the technology files can be
present.  This file does not exist by default, but is created if the
user defines macro definitions which are mapped to key presses, as
generated by the {\cb Key Map} command in the {\cb Attributes Menu}. 
The {\vt .xicmacros} file is rarely if ever directly edited by the
user.

If, for example, {\Xic} is started with an extension ``{\vt .ext}''
({\et -Text} given on the command line), {\Xic} will look for files
{\vt ./.xicmacros.ext} and {\vt \$HOME/.xicmacros.ext}, then {\vt
./.xicmacros} and {\vt \$HOME/.xicmacros}, in that order, where ``{\vt
\$HOME}'' indicates the user's home directory.  The first file found
will be read.  If {\Xic} is started without a technology file
extension, only the script files without an extension will be read. 

\index{xic\_font file}
\item{{\vt .xic\_font} file}\\
If a file named ``xic\_font'' is found in the library search path, the
file is read to obtain the text font used for on-screen label text. 
This file is created by the user from the {\cb Dump Vector Font}
button in the {\cb Font Selection} panel, and is subsequently editing
to the user's requirements.  The default font is hard-coded
internally.

\index{xic\_logofont file}
\item{{\vt .xic\_logofont} file}\\
If a file named ``xic\_logofont'' is found in the library search path,
the file is read to obtain the text font used for the {\cb logo}
(physical text) command.  This file is created by the user from the
{\cb Dump Vector Font} button in the {\cb Logo Font Setup} panel, and
is subsequently editing to the user's requirements.  The default font
is hard-coded internally.

\index{xic\_mesg file}
\item{{\vt xic\_mesg} file}\\
This is a text file providing the legal disclaimer.  It once supplied
text for the {\cb About} window, but is no longer used for that
purpose.

\item{Device Libraries}\\
As needed, {\Xic} will also read the device library ({\vt device.lib})
file, search and map the device models and help files, and open the
first command line file for editing.  The device library file supplies
the device templates used in electrical mode.  The model files provide
SPICE models used for generating SPICE output.  These files are read
the first time access is required.  Defaults are provided for these
files, but the user will very likely need custom device and model
library files.
\end{description}


% -----------------------------------------------------------------------------
% logfiles 051115
\section{Log Files and Error Reporting}

There are several methods by which error and warning messages are
presented to the user.  In many commands, particularly those that use
input from the prompt line, the prompt line is used to print messages
informing the user of incorrect input, and general command status. 
These messages are intended to direct the user toward correct usage of
the command.

More serious errors are reported in a pop-up window.  There are two
types of messages:  those that are logged, and those that aren't.  If
a message is logged, it is assigned a unique sequence number, and is
saved in the {\vt xic\_error.log} file discussed below.

The same pop-up window is generally used for both types of message. 
Most error and warning messages are logged.  A few messages are
unlogged, these generally report an immediate command failure due to
some condition such as lack of a current cell, or something such as a
help keyword not found message which is probably not worth logging.

The text window presenting an unlogged message will contain only that
message.  One of the disadvantages of unlogged messages is if several
are emitted, only the most recent is shown in the window, the others
are lost.  This is unlikely to happen in current {\Xic} releases.

The text window will display the sequence number and text of an
emitted logged message, and some number (currently hard-coded as 20)
of the previously emitted messages.  One can scroll through the list
to find previously emitted messages, which unlike in the unlogged case
still exist.

The error message window contains two buttons in addition to the
{\cb Dismiss} button.

\begin{description}
\item{\cb Save Text}\\
This allows the user to save the text shown in the pop-up to a file. 
This may be useful for documenting errors seen for bug reporting, and
for other purposes.

\item{\cb Show Error Log}\\
This button will bring up a file browser window loaded with the {\vt
xic\_error.log} file.  This allows the user to browse all errors, in
sequence.  This can be used to revisit old errors that have scrolled
off the end of the list in the pop-up error window.
\end{description}

\subsection{Log Files}
\index{logfiles}
\label{logfiles}

While {\Xic} is running, various log files are produced.  These files
contain a record of operations and errors, which may be useful for
debugging purposes.  Ordinarily, though, many of the log files are
rarely used, and these files are stored in a temporary directory which
is removed when {\Xic} exits normally.  Other log files, such as DRC
error reports, are saved in the current directory and are not removed
on exit.

Below is a listing of the log files that are saved in a temporary
directory.  The files in this directory can be browsed from within
{\Xic} with the {\cb Log Files} button in the {\cb Help Menu}.  In
addition, a button in the error pop-up allows the {\vt xic\_error.log}
to be viewed.

The {\cb Logging Options} panel from the {\cb Logging} button in the
{\cb Help Menu} selects whether or not certain operations are logged,
such as those done during extraction.  This will optionally produce
additional log files not listed below.

\begin{description}
\item{\vt xic\_run.log}\\
\index{xic\_run.log file}
This file contains a listing of key press/release and mouse button
press/release events, in a format which can be understood as script
instructions.  Although presently this feature in incomplete, the
instructions can be used to ``play back'' the current session by
executing the log file as a script.  The file is limited in size to
about 100Kb, at which point the file is given a ``{\vt .0}'' extension
and a new file is started.  If {\Xic} should ever crash or otherwise
misbehave, the current {\vt xic\_run.log} should be included with the
bug report sent to Whiteley Research.  This will greatly help in
tracking down the problem.
    
\item{\vt xic\_error.log}\\
\index{xic\_error.log file}
This file contains a list of error messages generated during the
session.  The previous 20 errors are displayed in the error pop-up
window in {\Xic}, but the {\vt xic\_error.log} file retains a complete
record.  This file may also be of use in diagnosing problems within
{\Xic}, and should be included with the bug report if it contains an
entry relevant to the problem.

\item{\vt xic\_mem\_errors.log}\\
\index{xic\_mem\_errors.log file}
This file, used under Unix/Linux only, is generated or appended to if
memory corruption is detected.  If this file exists when {\Xic} exits,
it will be emailed to Whiteley Research (by default).  However, if
either {\et XICNOMAIL} or {\et XTNOMAIL} is set in the environment,
the file will instead be moved to the current directory, and a message
will be printed requesting that the user mail it to Whiteley Research. 
Memory corruption should never occur, and this file contains stack
trace information that will help identify the problem.

\item\parbox[b]{4in}{\vt
read\_cgx.log\\
read\_cif.log\\
read\_gds.log\\
read\_oas.log\\
read\_native.log}\\
These files contain messages emitted when a file is read into {\Xic}
for editing.  The file name generated depends on the type of file read.

\item\parbox[b]{4in}{\vt
write\_cgx.log\\
write\_cif.log\\
write\_gds.log\\
write\_oas.log\\
write\_native.log}\\
These files contain messages emitted when a file is written to disk.
The file name generated depends on the type of file written.

\item\parbox[b]{4in}{\vt
convert\_cgx.log\\
convert\_cif.log\\
convert\_gds.log\\
convert\_oas.log\\
convert\_native.log}\\
These files contain messages emitted when a file is converted directly
to another format through the commands in the {\cb Convert Menu}.
\end{description}

The size of the log files that grow progressively as {\Xic} is running
are size-limited to about 100Kb.  If the file exceeds this size, the
file is moved to the same name with a ``{\vt .0}'' extension, and the
original log file is reopened.  Thus, a maximum of 200Kb per log of
information is retained.

The environment variable {\et XIC\_LOGDIR} can be set to an existing
directory that will be used to store the log files.  The log files
will be placed in a directory
\begin{quote}
{\it logdir}/{\vt xic}.{\it pid}
\end{quote}
where {\it logdir} is the first defined of the environment variables
{\et XIC\_LOGDIR}, {\et XIC\_TMP\_DIR}, {\et TMPDIR}, or defaults to
``{\et /tmp}''.  The {\it pid} is the process id of the {\Xic}
process.  This directory is created when {\Xic} starts, and is deleted
when {\Xic} terminates normally.  If {\Xic} terminates abnormally, the
log files will still be around for inspection.  If a user needs to
look at a log file after running {\Xic}, the file must be copied to
another location before exiting {\Xic}.  The {\cb !logfiles} command
can be used to read logfiles from within {\Xic}.

This mechanism lets multiple copies of {\Xic} run on the same machine
from any directory, and minimizes the pollution of the file system and
in particular the current directory with a lot of generally unused log
files.

\subsection{Abnormal Termination Logging}
\index{panic}
\index{crash}
If {\Xic} experiences an internal memory referencing error, {\Xic}
will terminate.  Such occurrences should be rare to nonexistent,
however this is the ideal and generally not the reality.  During a
``panic'', the following will happen:

\begin{itemize}
\item{A subdirectory will be created in the current directory, with
the name ``{\vt panic.}{\it pid\/}'', where {\it pid} is the process
id number of the running program.}

\item{All cells in memory that have the modified flag set will be
written into this directory.  The files will be in the original file
format.  Cells created in {\Xic} and never saved will be saved in
native format.  Although it can not be guaranteed that these files are
not corrupted by whatever error occurred, generally they are clean and
accurately reflect unsaved work.  After a thorough check, they can be
copied back to the original file name.}

\item{A file named ``{\vt xic\_panic.log}'' is created in the current
directory.  This contains the log messages emitted while the modified
cells are being dumped, and other information.}

\item{The log files that are normally removed after normal exit are
retained.  The location of the log files is given in the {\vt
xic\_panic.log} file.}

\item{Unless either of the environment variables {\et XICNOMAIL} or
{\et XTNOMAIL} is set, a stack trace is emailed to Whiteley Research,
which will be analyzed to resolve the cause of the fault, and if
possible the problem will be fixed in the next {\Xic} release.  The
file that is emailed is named ``{\vt gdbout}''.  The file will be
created in the current directory.}
\end{itemize}

% -----------------------------------------------------------------------------
% plugins 012813
\section{Plug-Ins}
\index{plug-ins}
A ``plug-in'' is a software library that is read into a running
program, that provides additional features or capability.  Within
{\Xic}, plug-ins provide optional support for
%\ifoa
OpenAccess, and
%\fi
the Python and Tck/Tk languages.  The plug-in provides an interface to
external libraries that may or not be present on the user's computer. 
If the needed libraries are present, the plug-in will be loaded into
{\Xic} on program startup, and a message, such as
\begin{quote}
``{\vt Using Tcl/Tk (tcltk.so)}''
\end{quote}
will appear in the console among the text generated on program
startup.  If the needed libraries are not found, the plug-in is not
loaded, but {\Xic} will run normally except that the plug-in's
features will be absent.

At present, plug-ins are supported on all platforms except for
Microsoft Windows.  Windows does not provide the type of shared
library technology needed for plug-ins.  Although a similar capability
could be instituted, there are many substantial issues and it is not
clear if it is worth the development effort.

Plug-ins are distributed as shared library code, and are found in the
{\vt plugins} sub-directory in the distribution area, i.e.,
\begin{quote}
{\it prefix\/}{\vt /xictools/xic/plugins}
\end{quote}
The plug-in files are version-specific, and will work {\cb only} with
the program from the same distribution file.  Of course, {\Xic} needs
to be able to find its startup files for the plug-ins to be available. 
If {\Xic} is not installed in the standard location, the {\et
XT\_PREFIX} environment variable should be set to enable {\Xic} to
find its startup files.

Normally, if a plug-in is not loaded, there is no message.  If,
however, the {\et XIC\_PLUGIN\_DBG} environment variable is set,
diagnostic messages will be printed.  These can help identify why the
plug-in is not being loaded, and are instrumental in tracking down
problems when the user expects success.

Lack of success loading a plug-in and generally due to the inability
of the plug-in code to find the shared libraries needed on the host
computer.  Unless the library is ``standard'' on the system, which may
be true of Python, then it will be necessary to use the {\et
LD\_LIBRARY\_PATH} environment variable to specify where to look for
the libraries.  The libraries much match the address size (32 or 64
bit) of the running {\Xic} program.

%\ifoa
% -----------------------------------------------------------------------------
% oaplugin 040216
\section{OpenAccess Support}
\label{oaplugin}
\index{OpenAccess}

{\bf This interface is presently not available under Microsoft Windows.}

OpenAccess is an open-source database for CAD/EDA data.  It is used
by Cadence Virtuoso 6.{\it x\/}, and by many other tools.  It provides
a commonality among tools from different vendors, and is intended to
facilitate seamless integration of tools from different vendors into a
process flow.  OpenAccess is distributed by Si2 ({\vt www.si2.org}). 
Source code and binary distributions are available for a number of
operating systems, to registered users and coalition members.

{\Xic} can connect to an OpenAccess (OA) database through a plug-in,
which is supplied with all {\Xic} releases for other than Microsoft
Windows, on which there is no OA support.  Since there is no default
location for OA, the user must set the {\vt XIC\_LIBRARY\_PATH} or the
{\vt LD\_LIBRARY\_PATH} variable to include the library location in
the search path during program loading.  This is most conveniently
done in the user's shell startup script.

Presently, the plug-in is compiled with the OpenAccess releases below. 
The plug-in can work with libraries from newer OpenAccess releases,
but not with libraries from an older release.
 
\begin{tabular}{ll}
\bf Distribution & \bf OpenAccess Release\\
LinuxRHEL5 (32 and 64 bit) & 22.04.054\\
all others & 22.04.064\\
\end{tabular}
 
These are known to be compatible will OpenAccess from Ciranova PyCell
Studio release 4.7.1 and later, and Cadence Virtuoso 6.1.4 and later. 
This includes Virtuoso 6.1.7 which jumps to OpenAccess API level 5. 
The existing plugin can handle API levels 4 and 5.

There are at least three ways to obtain OpenAccess.
\begin{enumerate}
\item{From a Cadence Virtuoso installation on the same computer.}
\item{From a Ciranova (now Synopsys) PyCell Studio portable-pcell
  installation on the same computer.}
\item{From an OpenAccess distribution from the Si2 organization.}
\end{enumerate}

The OpenAccess plug-in is provided with OS X releases, though
(regrettably) Ciranova and Cadence do not support the platform.  Those
inclined can get OpenAccess from Si2 and build it on OS X, using
patches available from the Whiteley Research web site.  Unlike
Windows, OS X is inherently a first-class CAD platform, too bad the
big guys don't support it.  The patch also ports OpenAccess to
FreeBSD, and improves support for recent Linux releases.

Probably, the main interest in using OA is for limited compatibility
with Cadence Virtuoso.  There are two levels here.  The first level is
compatibility with the OA system.  This is basically complete, as any
{\Xic} design can be saved to and read from OA without data loss or
change.  The second level is compatibility with the conventions and
methods used in the Virtuoso product, much of which is proprietary or
undocumented.  This is a much tougher nut to crack.  Presently, there
is fairly reasonable capability of taking Virtuoso designs into
{\Xic}, but the reverse is not true.  Presently, physical (layout
view) data from {\Xic} can be read by Virtuoso and should appear
correct, however there is no netlist information or connection to a
schematic.  It is as if the layout view was read from a GDSII file. 
Schematic and schematic symbol views from {\Xic} can not be read as
anything but garbage by Virtuoso.  There are plans for a data
translation stage in the future to possibly adapt {\Xic} schematics to
Virtuoso format.

When the OpenAccess plug-in is loaded, there are several changes to
{\Xic}.

\begin{enumerate}
\item{There is an {\cb OpenAccess Libs} entry added to the {\cb File
Menu}.  Pressing this will bring up the {\cb OpenAccess Libraries}
panel, which provides access to the existing OpenAccess design data.}

\item{A number of ``bang'' commands (text-mode commands that start
with '!') are made available.  These commands are typed into the
prompt line to start.  Much of the functionality of these commands is
also available graphically in the panel.
\begin{quote}
{\cb !oaversion}\\
{\cb !oanewlib}\\
{\cb !oabrand}\\
{\cb !oatech}\\
{\cb !oasave}\\
{\cb !oaload}\\
{\cb !oadelete}
\end{quote}
}
\end{enumerate}

In addition, the standard commands for reading and writing design data
become operable with OpenAccess data.  When specifying a cell, one
provides two words:  the OpenAccess library name and the cell name.

It is not possible to write to an OA library unless the library has
been ``branded'' by {\Xic}.  By default, libraries created in {\Xic}
are writable from {\Xic}, libraries created by other tools are not. 
The read-only status from {\Xic} of any library can be set from the
{\cb OpenAccess Libraries} panel, or with the {\cb !oabrand} command.

\subsection{Representing {\Xic} Cells in OpenAccess}

When an {\Xic} cell is saved in OpenAccess, up to three views may be
created.  The user has specified a library name where the views will
be saved, and of course the cell name.  Some write commands allow the
user to save a cell under a different name.

If the cell contains physical data, this will be saved in a view named
``{\vt layout}'' of OpenAccess view type ``{\vt maskLayout}''.  If the
cell contains electrical data, the schematic will be saved in a view
named ``{\vt schematic}'' of view type ``{\vt schematic}''.  If a
symbolic representation has been defined, this will be saved in a view
named ``{\vt symbol}'' of OpenAccess view type ``{\vt
schematicSymbol}''.  This latter view can only exist, as part of an
{\Xic} cell representation, if a schematic view also exists.  Reading
or writing an {\Xic} cell will involving translating each of these
views that exist.

This group of properties applies to the OpenAccess interface.

\begin{description}
\index{properties!stdvia}
\index{stdvia property}
\item{\et stdvia} property, number 7160\\
This property is applied to standard via sub-masters and instances,
and is used by the translator to convert OpenAccess standard vias to
{\Xic} standard vias, and the reverse.  The property is used in {\Xic}
to identify and specify standard via instances and sub-masters.  The
format of the property string is described in \ref{stdviaprp}.

\index{properties!oa\_cstmvia}
\index{oa\_cstmvia property}
\item{\et oa\_cstmvia} property, number 7161\\
This property is applied by the OpenAccess reader to master cells that
represent a custom via.  In {\Xic}, vias are cells, they have no
unique type as in OpenAccess.  The string format consists of the cell
identifier followed by parameter specifications.  The cell identifier
has the form
\begin{quote}
$<${\it libname\/}$><${\it cellname\/}$><${\it viewname\/}$>$
\end{quote}
This is followed by a space-separated parameter specification string
in the same format as the {\et pc\_params} property.  A custom via
master is basically a pcell sub-master.

When written back to OpenAccess, cells with this property will be
ignored.  A sub-master for the custom via will be created within
OpenAccess when needed.

\index{properties!oa\_orig}
\index{oa\_orig property}
\item{\et oa\_orig} property, number 7183\\
This property is used transiently when loading OpenAccess cell data
into {\Xic}.  If is applied to cells, and removed when reading
completes.  If an instance is read before the corresponding cell
definition, a dummy {\Xic} cell descriptor is created and given this
property.  The property string contains the library and cell names,
separated by a forward slash (`{\vt /}') character.  Using this
information, the cell is read later.
\end{description}
%\fi

% -----------------------------------------------------------------------------
% pyplugin 032017
\section{Python Support}
\label{pyplugin}
\index{Python}

{\bf This interface is presently not available under Microsoft Windows.}

The Python ({\vt www.python.org}) scripting language is a powerful,
versatile language enjoying much popularity.  In particular, it has
become the language of choice for writing portable parameterized
cells, as used in the PyCell Studio project from Ciranova, Inc.  (now
Synopsys).
%\ifoa
This download provides the essentials for creating portable pcells,
using the Python language, and OpenAccess.  Whiteley Research strongly
favors this approach, and will integrate Ciranova standards as tightly
as possible.
%\fi

Python is made available, when Python-2.6 or newer is found on the
user's computer, via the Python plug-in.  Red Hat Enterprise Linux 6
and 7 provide a compatible native Python.  Presently, only Python
release 2.6 is supported on Red Hat Enterprise Linux 5, so
installation of an updated package is required on that operating
system.

%\ifoa
The Ciranova PCell Studio provides Python 2.6, as well as OpenAccess. 
If using Ciranova, the Ciranova-supplied Python should be used.
%\fi

\begin{description}
\item{Red Hat Enterprise Linux 6,7}\\
This supplies a native Python-2.6/2.7, which will work with the
plug-in without any configuration.
%\ifoa
Unfortunately, this is not compatible with the Python-2.6 provided by
Ciranova.  {\Xic} can use either one.  The Python-2.6 provided by
Ciranova was built with different setup flags for handling UTF8 text
than the stock Python-2.6.
%\fi

\item{Red Hat Enterprise Linux 5}\\
The operating system provides Python-2.4, which is not supported.  The
Ciranova PyCell Studio provides Python-2.6, which is one source for a
compatible Python.  Another is to install the {\vt python26} extension
package.  Using the {\cb Package Manager} or {\vt yum}, install
\begin{quote}
{\vt epel-release-5-4.noarch}.
\end{quote}
This will add additional repositories.  Then, in the {\vt epel}
repository, find and install a release like ``{\vt
python26-2.6.8-2.el5.x86\_64}''.
\end{description}

To use a non-default Python such as that supplied by Ciranova in the
PyCell Studio, one will need to set the {\et LD\_LIBRARY\_PATH}
variable to include the alternative shared library location.
%\ifoa
This will happen automatically if Ciranova's setup procedure is
followed before starting {\Xic} (see \ref{ciranova}).
%\fi

Failure to load the Python plug-in is by default silent.  If the
environment variable {\vt XIC\_PLUGIN\_DBG} is set, diagnostics and
error messages will be printed in the console when attempting to load
plug-ins at program startup.

When the Python plug-in is loaded, {\Xic} is able to execute Python
scripts.  This includes stand-alone scripts, and scripts that are used
in parameterized cells.  Note that Ciranova PyCells, which are also
Python-based, are supported via OpenAccess, and are independent of
Python support in {\Xic}.  However, future plans are to support
PyCells natively in {\Xic}.  {\Xic} is presently able to support the
Ciranova protocols for stretch handles and abutment natively.

{\bf This information is preliminary, and may change.}

The entire library of native script functions are callable from
Python.  However, at this point many of the more complicated data
types found in the native function library are unsupported.  There are
two ways to call a native function from Python:

\begin{quote}
{\vt xic.}{\it native\_func\/}{\vt (}{\it args\/}, ...{\vt )}\\
{\vt xic.eval}{\vt ("}{\it native\_func\/}{\vt ",} {\it args\/}, {\vt ...)}
\end{quote}

The choice of style is up to the user, the first is probably slightly
more efficient and is recommended.

The Python script must include some initialization lines in order to
use the {\Xic} interface.  As a simple example, the script below will
draw two boxes in the current cell, using the current layer.

\begin{quote}\vt
import xic\\
import xicerr\\
import sys\\
sys.stderr = xicerr\\
xic.Box(2.0, 2.0, 6.0, 7.0)\\
xic.eval("Box", 1.0, 1.0, 5.0, 6.0)\\
xic.Commit()\\
\end{quote}

The first line is mandatory for using any native script functions.  It
loads the {\Xic} interface module.

The next three lines redirect Python error messages to the {\Xic}
error reporting system.  These are optional, if not included Python
messages will be printed on the console window.

The final three lines call functions from the native script library. 
The first two of these lines illustrate calling the {\vt Box} function
using the two syntax styles.  The final line calls the {\vt Commit}
function, which registers the change with the undo system, among other
things.

The first four lines are implicitly added during pcell evaluation,
thus no not have to be included in a Python pcell script (see
\ref{pcells}).

Presently, datatypes translate in the following manner.  If an
un-handled data type is encountered, the script will terminate with a
fatal error.

\begin{tabular}{|l|p{4in}|} \hline
\bf {\Xic} type & Python type\\ \hline
string & String.\\ \hline
scalar & Float.\\ \hline
array & List of float.\\ \hline
zlist & List of ``{\vt zlist}'' followed by lists of
  six integers (LL, LR, YL, UL, UR, YU in internal units).\\ \hline
handle & A list containing ``{\vt xic\_handle}'',
  followed by the handle integer value.  For stringlist handles only,
  the strings follow.\\ \hline
\end{tabular}

When these forms are passed back to {\Xic} functions, they are
reverted to the {\Xic} data type.  Note that handles can be passed
through Python, but except for stringlist handles they are useless in
Python at present.

When the Python plug-in is loaded, the {\cb !py} command is available. 
This command will execute a script file containing Python commands,
the path to which is given as the argument.  Also, the following
script functions are available:

\begin{tabular}{ll}
{\vt RunPython} & Run a Python script.\\
{\vt RunPythonModFunc} & Execute a Python module function.\\
{\vt ResetPython} & Reset the Python interpreter.\\
\end{tabular}


% -----------------------------------------------------------------------------
% tclplugin 022713
\section{Tcl/Tk Support}
\label{tcltk}
\index{Tcl}
\index{TK}

{\bf This interface is presently not available under Microsoft Windows.}

{\Xic} provides a plug-in interface to Tcl/Tk.  Tcl (Tool control
language) is a popular open source scripting language, and Tk is a
graphical package addition.  The language syntax is provided in
documentation supplied with Tcl/Tk, and is described in several books.

Since this capability is dynamically loaded, {\Xic} can use this
capability if it has been installed, but does not require the
installation.  Support is provided for Tcl, with and without Tk.

\index{LD\_LIBRARY\_PATH environment variable}
If Tcl/Tk have been installed via a standard distribution file on the
system, which is common for Linux, the plug-in should be able to
locate the shared libraries automatically.  If the installation is
non-standard, the user may need to inform the system dynamic linker of
the shared library location.  This is generally accomplished by
setting the {\et LD\_LIBRARY\_PATH} variable in the environment,
before running {\Xic}.  This would normally be done in the user's
shell startup file.

There are two text-mode commands that can be used to run a Tcl/Tk
script.

\begin{description}
\item{\cb !tcl}\\
This command will exist only if the Tcl language support plug-in is
loaded, which will occur on program startup if the Tcl shared
libraries are found.  The script should contain only Tcl commands, not
Tk.

\item{\cb !tk}\\
This command will exist only if the Tcl and Tk language support
plug-in is loaded, which will occur on program startup if both Tcl and
Tk shared libraries are found.  The script may contain any combination
of Tcl and Tk commands.
\end{description}

In either case, the first argument is a path to a file containing the
script body.  Additional arguments are taken as arguments to the
script.  The script will be executed as if by the {\et wish} shell
supplied with Tcl/Tk.

The startup file, which can be used to set defaults, is named ``{\vt
.xic-wishrc}'' in the user's home directory.  The contents is
analogous to the {\vt .wishrc} file normally used with Tcl/Tk.  The
user must create this file if needed.

All of the {\Xic} script functions are exported to Tcl/Tk and can be
called by name from a Tcl/Tl script.  However, only the basic data
types are supported.  There is also a function named ``{\vt xic}''
which can be used in the following manner:
\begin{quote}
{\vt xic} {\it function arguments...}
\end{quote}

The function {\vt xic} is a Tcl function which loads the interface
function or user-defined function given in the first argument (a
string).  User defined functions can be accessed if they are already
known to {\Xic}, i.e., they were defined in a library file or were
defined in a previously-run {\Xic} script.  The arguments to the
function follow, and should match the arguments expected by the
function.  This form must be used when executing a user-defined
function.

The variable type of an argument is inferred as follows:
\begin{itemize}
\item{A single-token numeric value without leading or trailing
characters not part of the number is taken as a scalar.}
\item{A token of the form \&{\it arrayname}() is taken as an array.}
\item{Anything else is taken as a string.}
\end{itemize}

To explicitly coerce a numeric token into a string, backslash escaped
double quotes should be used to delimit the token.  For example, {\vt
$\backslash$"1.234$\backslash$"} is taken as a string.  The backslash
prevents {\et tcl} from removing the double quotes before passing the
token.

Arrays passed to interface functions must use ``0'', ``1'', etc.  as
indices, and are ordered accordingly (in {\et tcl}, array indices can
be any text token and have no natural order).  The ``0'' element (at
least) must be set before the array can be passed to a function.  If
the array is dynamically expanded, new {\et tcl} elements will be
created.  The initial size of the array is implied by the largest
contiguous index assigned.  Thus, for example, if the interface
function requires an array of size 4, the following {\et tcl} code
could be used

\begin{quote}\vt
set array(0) 0\\
set array(1) 0\\
set array(2) 0\\
set array(3) 0\\
xic {\it Function} \&array()
\end{quote}

When the function returns, the array values will be updated.  Only
one-dimensional arrays are available.

There is an additional special {\et tcl} function which has been added.
\begin{quote}
{\vt xwin} {\it win\_name}
\end{quote}
This function returns the X window id of the {\et tk} window given as
a widget path in {\it win\_name}.  This is used to obtain the window
id of a {\et tk} window to be used for {\Xic} graphics through the
{\vt GRopen} interface.  A suggested way to use a {\et tk} window for
exported drawing from {\Xic} is given in the example below.  The {\vt
xwin} procedure is used to obtain the window id.  This window should
be configured with `{\vt -background ""}' which allows redraws to be
handled through a procedure bound to the window with the {\vt bind}
command which responds to expose events.  Otherwise, expose events
will cause the window to be redrawn in gray {\it after} the event
handler is called.  A pixmap is used to store the image for redraws.

Example

\begin{verbatim}

# This is the window used for drawing by Xic.
# Note the '-background ""' directive.  This
# is necessary for proper redrawing after expose
# events.
frame .f -width 8c -height 8c -background ""
pack .f

set win_id [xwin .f]
set ghandle [xic GRopen ":0" $win_id]
# The win_id is the X id of the drawing window,
# the ghandle is the handle value returned from
# Xic upon opening graphics on this window.

set size(0) 0
set size(1) 0
set size(2) 0
set size(3) 0
xic GetWindowView 0 &size()
# The size array contains the displayed area of the
# cell in the main Xic window, in order L, B, R, T

xic GRdraw $ghandle $size(0) $size(1) $size(2) $size(3)
# This draws the Xic view into the Tk window

xic GRupdate $ghandle
# Due to the way Tk (and X) works, unless GRupdate is
# called after drawing, the drawing won't be visible.
# The operations are stuck in a cache somewhere waiting.
# GRupdate flushes the operations.

set dsize(0) 0
set dsize(1) 0
xic GRgetDrawableSize $ghandle $win_id &dsize()
# The dsize array contains the size in pixels of the
# Tk drawing area.

set pixm [xic GRcreatePixmap $ghandle $dsize(0) $dsize(1)]
xic GRcopyDrawable $ghandle $pixm $win_id 0 0 $dsize(0) $dsize(1) 0 0
xic GRupdate $ghandle
# We have created a pixmap of the same size and depth as
# the drawing area, and copied the drawing area into it.
# This will be used to redraw the drawing area after an
# expose event.

bind .f <Expose> {
    # This sets up a handler for expose events.  Expose
    # events are received when a previously obscured part
    # of the window is uncovered.  The pixmap is copied
    # into the Tk window.
    xic GRcopyDrawable $ghandle $win_id $pixm 0 0 $dsize(0) $dsize(1) 0 0
    xic GRupdate $ghandle
}
\end{verbatim}

The {\vt TextCmd} script function can be used to launch a {\et
tcl/tk} script.  At present, {\et tcl/tk} scripts are not recognized
in the script path, but one can use a native language wrapper to
include {\et tck/tk} scripts in the {\cb User Menu}.

The following native script functions can also be used to run Tcl/Tk
scripts, or perform other related manipulations related to the Tcl/Tk
interpreter.

\begin{tabular}{ll}
{\vt RunTcl} & Run a Tcl or Tk script.\\
{\vt ResetTcl} & Reset the Tcl/Tl interpreter.\\
\end{tabular}

