
\chapter{Utility Programs}

% spMain.hlp:utilities 012609

The {\WRspice} distribution provides a few supplemental utility
and accessory programs.

%S-----------------------------------------------------------------------------
\section{The {\vt csvtoraw} Utility: CSV to Rawfile Conversion}
\index{csvtoraw program}

% spMain.hlp:csvtpraw 091524

This program will convert a space/comma separated file to a rawfile.

Usage:  {\vt csvtoraw} [{\it filename/\}]\\
Output goes to the standard output channel.  Input is taken from the
file if a name was given, otherwise from standard input.

A ``csv'' (comma-separated values) file is assumed to have a form as
described below.
\begin{itemize}
\item{Any lines that start with white space or a comment character
ahead of the header line are ignored.  Comment characters are {\vt *#!}.}
\item{The header line contains a number of words, space and/or comma
separated, these are the vector names.  If they contain a comment
character or comma the word should be double-quoted.}
\item{Lines that follow are ignored if the first non-space character is
a comment character, comma, or the line is blank.}
\item{Otherwise the lines should contain the same number of numbers as
words in the header line.  Any number format as used in {\WRspice}
is fine, numbers are separated by spaces and/or commas.}
\item{The leftmost logical column will be taken as the scale vector.}
\end{itemize}

%S-----------------------------------------------------------------------------
\section{The {\vt mmjco} Utility: Tunnel Junction Model Calculator}
\index{mmjco program}
\label{mmjco}

% spMain.hlp:mmjco 090222

{\WRspice} contains an internal tunnel junction model (TJM) of a
Josephson junction.  The model requires pre-prepared tables of ``fit''
parameters, which avoid performing lengthly calculation at run time. 
The {\vt mmjco} program generates the needed files.

In general, the user may not need to interact directly with {\vt
mmjco}, as {\WRspice} will call {\vt mmjco} when needed to create new
fit files, and store these under a directory named {\vt .mmjco} in the
user's home directory.  Once a fit file has been created for a
particular parameter set, it will be reused in future {\WRspice}
sessions when needed,

The core functionality of {\vt mmjco} is derived from the MiTMoJCo
project by D.  R.  Gulevich (found on GitHub.com), specifically the
{\vt mitmojco.py} environment that provides an interface for creating
tunnel junction amplitude (TCA) and fitting tables.  This has been
implemented in {\vt mmjco} as two C++ classes, one for creating tunnel
junction amplitudes, which basically evaluates the Werthamer model as
formulated for MiTMoJCo by Gulevich, the second to compress the
amplitudes into a compact representation.  This representation is used
by the TJM (JJ level=3) device model in {\WRspice}.  The method is
that of Odintsov, Semenov, and Zorin (See below for a list of
references).

The {\vt mmjco} program has additional features.
\begin{itemize}
\item{Sweep tables can be produced for a range of temperatures,
allowing temperature sweeps to be performed in simulation without
having to create a fit file at each temperature.}
\item{There is a built-in BCS computation of energy gap given
temperature, superconducting transition temperature, and Debye
temperature of the junction materials.}
\item{Tunnel current amplitude tables are generated in ``rawfile''
format, so can be plotted within {\WRspice} or Synopsys
{\it WaveView}.}
\item{The fit and sweep file formats are compatible with the TJM
developed for Synopsys PrimeSim-HSPICE under the IARPA SuperTools
program.}
\end{itemize}

\subsection{Running {\vt mmjco}}

Running {\vt mmjco} enters a command-line processing loop, the user
responds to the ``{\vt mmjco> }'' prompt with commands from the list
below.  Each command can be followed by options as indicated. 
Additionally, there are ``{\vt cdf}'' and ``{\vt swp}'' modes where
{\vt mmjco} will create TCA and fit files or temperature sweep files
according to the arguments, and exit.  This is mostly to support
{\WRspice}, which uses this mode to create new models on-the-fly.

In this document, text in square brackets ([...]) is optional.  A
vertical pipe character ( $|$ ) indicates that either the text to the
right or left is acceptable, i.e., it separates options.

\subsubsection{Command Line Operations}

These modes are used by the TJM device model in {\WRspice}.

Command:  {\vt mmjco cdf} {\it arguments}

If the first argument to the {\vt mmjco} executable is {\vt cdf}, a
TCA file and corresponding fit file are created, and {\vt mmjco} exits
immediately in this case.  Arguments following {\vt cdf} are the same
arguments that can follow the {\vt cd} and {\vt cf} interactive
commands.

Command:  {\vt mmjco swp -fs} {\it sweepfile temperature}

Similarly, {\vt swf} will create a possibly-interpolated fit file from
an existing sweep file, for the temperature provided.

\subsubsection{Interactive {\vt mmjco} Commands}

The following are the interactive commands which can be entered after
starting {\vt mmjco} with no arguments.

\begin{description}
\item{{\vt cd}[{\vt ata}]  [{\vt -t} {\it temp\/}] 
      [{\vt -d}$|${\vt -d1}$|${\vt -d2} {\it delta\/}]
      [{\vt -s} {\it smooth\/}] [{\vt -x} {\it nx\/}]
      [{\vt -f} {\it filename\/}] [{\vt -r}$|${\vt -rr}$|${\vt -rd}]\\
Create TCA data, save internally and to a file.  See below for
an explanation of the options.

Tunnel current amplitude and smoothing options:\\
\begin{tabular}{ll}
\vt -t & \parbox[t]{4.5in}{The assumed temperature follows, in Kelvin.  Default
 4.2.}\\
\vt -d & \parbox[t]{4.5in}{This will set both d1 and d2, the pair breaking energy
 in milli-electron volts, of the two superconducting banks.  The default
 is 1.4 mev.}\\
\vt -d1,-d2 & \parbox[t]{4.5in}{Like {\vt -d}, but apply to only one of the banks.
 The final occurrence of {\vt d},{\vt d1},{\vt d2} will have precedence.}\\
\vt -s & \parbox[t]{4.5in}{This provides the smoothing parameter as used in
 MiTMoJCo.  the accepted range is 0.0 -- 0.099.  The default is 0.008.
 If less than 0.001, 0 is assumed.  When 0, no smoothing is done and raw
 BCS tunnel amplitudes are generated.}\\
\vt -x & \parbox[t]{4.5in}{The number of points used to create the tunnel current
 amplitudes. The range of sweep of voltage normalized to the gap voltage
 ({\vt d1+d2}) extends from 0.001 through 2.0.  The default point
 count is 500.}\\
\vt -f & \parbox[t]{4.5in}{A name for the TCA amplitude file.  If not given, a
 default is used, described below.}\\
\vt -r & Output file is a complex-valued rawfile.\\
\vt -rr & Output file is a real-valued rawfile.\\
\vt -rd & \parbox[t]{4.5in}{Output file simple data file.  If none of {\vt
 -r},{\vt -rr},{\vt -rd} options is set, the format will be {\vt -rr}
 if the program was built for {\XicTools}, {\vt -rd} otherwise.}\\
\end{tabular}
}

\item{{\vt cf}[{\vt it}] [{\vt -n} {\it terms\/}]
      [{\vt -h} {\it thr\/}] [{\vt -ff} {\it filename\/}]\\
Create fit parameters for TCA data currently in memory from {\vt cd}
or {\vt ld} commands.  This is saved internally and to a file.  See
below for an explanation of the options.

Fitting table options:\\
\begin{tabular}{ll}
\vt -n & \parbox[t]{4.5in}{The size of the table, defaults to 8.  Larger tables
 are more accurate but take more time to generate and process.  A maximum
 of 20 is enforced.}\\
\vt -h & \parbox[t]{4.5in}{The ratio of the absolute to relative tolerances,
 used in compression, the default is 0.2.}\\
\vt -ff & \parbox[t]{4.5in}{A name for the fitting parameter table.  If not
 given, a default is used, described below.}\\
\end{tabular}
}

\item{{\vt cm}[{\vt odel}] [{\vt -h} {\it thr\/}]
  [{\vt -fm} [{\it filename\/}]] [{\vt -r}$|${\vt -rr}$|${\vt -rd}]\\
Create a model for TCA data using fitting parameters currently in
memory, compute the residual, and optionally save to a file.  If {\vt
-fm} is given without a filename, a file name will be generated
internally.  If {\vt -fm} is not given, the model will not be saved to
a file, but used only to compute the residual.  The printed residual
number is an indication of the fit quality, smaller values indicate
better matching.

If one of {\vt -r}, {\vt -rr}, {\vt -rd} is given when a TCA file is
being generated, it overrides the default type for file to produce.
\begin{quote}
\begin{tabular}{ll}
\vt -r & complex-valued rawfile\\
\vt -rr & real-valued rawfile\\
\vt -rd & simple data file\\
\end{tabular}
\end{quote}
The default file type is {\vt -rr} when built for {\XicTools}, or the
simple data file format otherwise.

The model files are saved in the current directory (unless a path is
given explicitly).
}

\item{{\vt cs}[{\vt weep}] {\it Tstrt Tend} [{\it Tdelta\/}] {\it arguments}\\
Create a temperature sweep file, which involves creating sequential
records of fit parameters for temperatures starting with {\it Tstrt}
and ending at at or near {\it Tend}, spaced in temperature by {\it
Tdelta}.  These are real numbers in Kelvin.  If the third number {\it
Tdelta} does not appear, 0.1K is assumed.  The {\it arguments} are
those that can be given to the {\vt cd} or {\vt cf} commands.}

\item{{\vt ct}[{\vt ab}] {\it T1 T2} [{\it ... TN\/}] {\it arguments}\\
Create a temperature table file.  The first two temperatures {\it T1}
and {\it T2} are required, and these can be followed by an arbitrary
number of additional temperatures.  The temperatures are real numbers
in Kelvin.  The {\it arguments} are those that can be given to the
{\vt cd} or {\vt cf} commands.  The file will contain fit parameters
for each temperature given.}

\item{{\vt d}[{\vt ir}] {\it directory\_path}\\
Give a path to a directory where all amplitude and fit files will be
stored and loaded from, if a rooted path is not given with the file
names.  When built for {\XicTools}, the default location is a
subdirectory ``{\vt .mmjco}'' in the user's home directory, otherwise
the current directory is assumed.}

\item{{\vt g}[{\vt ap}] [{\vt -tc} {\it Tc\/}] [{\vt -td} {\it Td\/}]
  [{\it T1 T2...\/}]\\
Compute and print the superconducting energy gap at temperatures.  The
{\vt -tc} specifies the superconducting transition temperature, and
{\vt -td} specifies the Debye temperature, both Kelvin.  If not given,
the defaults are for Niobium:  Tc = 9.26K and Td = 276K.  There can be
zero to 10 real number arguments representing temperatures in Kelvin. 
If zero, print the gap for temperatures from 0 to Tc at 0.1K
increments.  If two numbers, print the gap for the smaller to the
larger in 0.1K increments.  Otherwise, print the gap at the given
temperatures.}

\item{{\vt ld}[{\vt ata}] {\it filename}\\
Load the internal TCA data register from a TCA data file whose name
must be given.  This understands all supported file formats.}

\item{{\vt lf}[{\vt it}] {\it filename}\\
Load the internal fit parameters register from a fit parameter file
given.}

\item{{\vt ls}[{\vt weep}] {\vt -fs} {\it filename temp}\\
Load the internal fit register from a sweep file, interpolating to
temperature {\it temp}.}

\item{{\vt h}[{\vt elp}] $|$ {\vt v}[{\vt ersion}] $|$ ?\\
Print help and the running mmjco release number.}

\item{{\vt q}[{\vt uit}] $|$ {\vt e}[{\vt xit}]\\
Exit {\vt mmjco}.}
\end{description}

\subsubsection{File Name Encoding}

Running {\vt mmjco} can generate three types of files:  a file
containing values of the tunnel current amplitudes (TCA files), a file
containing the fitting parameters used to represent the tunnel current
amplitues in a compact form, and files that contain collections of
fitting parameter sets, as for a temperature sweep.  The ``model''
files use the same format as the TCA files.  By default, these files
are given a name that encodes the various parameter values used in
creation.

Tunnel current amplitude files, including model files, are by default
given an internally-generated file name in the forms below.
\begin{quote}\vt
tcaTTTTTTdddddDDDDDssPPPP.data\\
tcaTTTTTTdddddDDDDDssPPPP.raw
\end{quote}

The ``{\vt tca}'' and ``{\vt .data}''/``{\vt .raw}'' and similar are
literal.  All fields are fixed width and zero padded.  Real numbers
are converted to integers by multiplying by a scale factor and
rounding to integer.

\begin{description}
\item{{\vt TTTTTT}\\
A six-digit integer equal to 1e4*{\it temperature\/}.}
\item{{\vt ddddd}\\
A 5-digit integer equal to 1e7*{\it d1\/}, where {\it d1} is the
 side 1 gap potential in volts.}
\item{{\vt DDDDD}\\
A 5-digit integer equal to 1e7*{\it d2\/}, where {\it d2} is the
 side 2 gap potential in volts.}
\item{{\vt ss}\\
A two digit integer equal to 1e3*{\it sf\/}, where {\it sf} is the
 smoothing parameter.}
\item{{\vt PPPP}\\
The number of points used ({\vt -x} option).}
\end{description}

In {\WRspice}, a rawfile ({\vt .raw} extension) can be loaded with the
{\et load} command, and the amplitudes can then be plotted with ``{\vt
plot all}''.

The fitting parameter file name has the following form.
\begin{quote}\vt
tcaTTTTTTdddddDDDDDssPPPP-nnHHH.fit
\end{quote}

The first part, ahead of the `-', is as described above.  Following
the hyphen:
\begin{description}
\item{{\vt nn}\\
The two-digit fitting table size.}
\item{{\vt HHH}\\
A three digit integer equal to 1e3*{\it thr\/}, where {\it thr} is
the compression threshold.}
\end{description}

Sweep file names have the following form.
\begin{quote}\vt
tswNNNttttttTTTTTTssPPPP-nnHHH.swp
\end{quote}

The part following the hyphen is the same as for the fit file.
\begin{description}
\item{{\vt NNN}\\
A three digit integer giving the number of records in the file.}
\item{{\vt tttttt}\\
A six digit integer equal to 1e4*{\it Tstart\/}, where
{\it Tstart} is the starting temperature in Kelvin.}
\item{{\vt TTTTTT}\\
A six digit integer equal to 1e4*{\it Tdelta\/}, where
{\it Tdelta} is the temperature spacing.}
\item{{\vt ss}\\
A two digit integer equal to 1e3*{\it sf}, where {\it sf} is the
 smoothing parameter.}
\item{{\vt PPPP}\\
The number of points used ({\vt -x} option).}
\end{description}

The fit parameter table format is similar to that used by MiTMoJCo,
identical if the header is ignored.  Note that the table values do not
match those found in the MiTMoJCo distribution.  It seems that these
values are not unique, and the various programs can converge to
different sets.  It was found that the original and present versions
of {\vt mitmojco.py} gave different values, and neither matched the
values provided in the amplitudes folder.

\subsection{File Formats}


\subsubsection{TCA file formats}

Tunnel current amplitude (TCA) tables are created by the {\vt cd} and
{\vt cm} commands.  They consist of real and imaginary parts of the
pair and quasiparticle amplitudes, on a scale of normalized potential
across the structure, where the unit value is the sum of the gap
energies of the two electrodes.  The scale extends from 0 to 2 in
these units.

There actually are three formats available, selectable with options to
the {\vt cd} and {\vt cm} commands.

\begin{description}
\item{option: {\vt -rd}}\\
The file name is described in the previous section, with a suffix
``{\vt .data}''.  This is a generic numerical format, consisting of a
comment line and six columns of numbers.  The number of data lines is
the value given with the ``{\vt -x}'' option to the {\vt cd} command,
defaulting to 500.  The example below illustrates the format.
\end{description}

\begin{verbatim}
#    X            Jpair_real   Jpair_imag   Jqp_real     Jqp_imag
0    1.00000e-03  7.50623e-01  2.85275e-04  -7.50625e-01 3.37356e-04
1    5.00601e-03  7.51136e-01  1.36358e-03  -7.51123e-01 1.62410e-03
2    9.01202e-03  7.51643e-01  2.27465e-03  -7.51594e-01 2.74301e-03
3    1.30180e-02  7.52144e-01  3.03372e-03  -7.52040e-01 3.70900e-03
...
498  1.99599e+00  4.22402e-01  -5.29166e-01 1.52235e-02  1.87539e+00
499  2.00000e+00  4.21420e-01  -5.28564e-01 1.51180e-02  1.87962e+00
\end{verbatim}

The other two options emit the data using the SPICE ``rawfile''
format.  This is a format developed for plot data in Berkeley Spice3,
which is supported by most plotting programs, including Synopsys
WaveView and the {\cb load} function of {\WRspice}.  The only
difference is that one format ouptuts complex numbers for two
variables (pair and quasiparticle amplitudes), while the other format
outputs real values for four variables (the real and imaginary parts
of the amplitudes).

\begin{description}
\item{option: {\vt -r}}\\
The file name is described in the previous section, with a suffix
``{\vt .raw}''.  Output is in rawfile format using complex numbers.

\item{option: {\vt -rr}}\\
The file name is described in the previous section, with a suffix
``{\vt .raw}''.  Output is in rawfile format using real numbers.
\end{description}

The rawfile format description can be found in \ref{rawfilefmt}.

\subsubsection{Fit file format}

A fit file contains a compacted digest of a TCA table, as prescribed
by the Odintsov, Semenov and Zorin (OSZ) algorithm.  These can be
generated with the {\vt cf} command.  Fit files can be used as input
to simulators that contain a compatible tunnel junction model (TJM). 
Presently, {\WRspice} and Synopsys HSPICE can use these files.

An example fit file is shown below.

\begin{verbatim}
tcafit  4.2000e+00  1.3696e-03  1.3696e-03 0.008 500  8  0.200 3.9580e-2
-5.55136e+00, 7.35249e-02, 1.28573e+00, 1.08362e+01,-1.58776e-01, 2.87279e+01
-1.24623e-02, 1.00032e+00, 5.84894e-03,-4.50828e-04, 5.81876e-03,-2.11396e-04
-3.83312e-02, 1.00112e+00, 2.10779e-02,-3.09528e-04, 2.13882e-02, 1.13197e-03
-1.18261e-01, 9.98252e-01, 6.29481e-02, 1.95061e-03, 6.04875e-02, 1.68472e-02
-5.41049e-02, 7.52572e-07,-2.72420e-04, 5.67468e+00, 5.43002e-04, 2.37802e+00
-9.94491e-01, 6.50936e-01, 7.80341e-01, 1.14849e-01,-2.36003e-01, 9.49693e-01
-3.42835e-01, 9.60836e-01, 1.75970e-01, 1.81214e-02, 1.47672e-01, 1.40112e-01
-2.80018e-01, 6.49039e-03, 6.26863e-03, 1.81423e-01, 9.20013e-03, 3.23103e-02
\end{verbatim}

The first line is a header, the first word of which is ``{\vt tcafit}''.
The numbers that follow in this line are:

\begin{itemize}
\item{The temperature in Kelvin (4.2000e+00).}
\item{The left electrode pair-breaking energy in ev (1.3696e-03).}
\item{The right electrode pair-breaking energy in ev (1.3696e-03).}
\item{The smoothing parameter value used to create the TCA table, {\vt
-s} option in the {\vt cd} command for example (0.008).}
\item{The number of scale points used in the TCA table, {\vt -x}
option in the {\vt cd} command for example (500).}
\item{The number of terms used in the fit table, {\vt -n} option in
the {\vt cf} command (8).}
\item{he value of the threshold parameter used when generating the fit
parameter table, {\vt -h} option of the {\vt cf} command (0.200).}
\item{Normalized quasiparticle current at x=0.8, used to estimate the
sub-gap conductance (3.9580e-2).}
\end{itemize}

Following the header line are six columns of real numbers.  The number
of rows is equal to the ``terms'', which is the {\vt -n} option to the
{\vt cf} command.  The columns are the OSZ parameters P.real, P.imag,
A.real, A.imag, B.real, B.imag.

\subsubsection{Sweep file format}

Temperature sweep files are concatenations of fit records as described
above for a temperature range.  These allow rapid temperature modeling
through interpolation in supporting simulators ({\WRspice} and
Synopsys HSPICE).  Sweep files are created with the {\vt cs} command.

Below is an example temperature sweep file.

\begin{verbatim}
tsweep 91 0.1000 0.1000 0.008 500 8 0.200
tcafit  1.0000e-01  1.4086e-03  1.4086e-03  1.3185e-02
-8.51331e+00, 1.15164e-01, 1.29900e+00, 1.11105e+01,-4.05570e-01, 5.65607e+01
-1.06700e-02, 1.00009e+00, 3.58651e-03,-2.25268e-04, 3.57013e-03,-5.66586e-05
-2.47980e-02, 1.00072e+00, 1.14692e-02,-5.49747e-04, 1.15840e-02,-1.07382e-04
-6.31470e-02, 1.00196e+00, 2.92347e-02,-1.70767e-03, 2.93324e-02, 2.09767e-03
-1.86179e+00, 9.23541e-01, 7.52628e-01,-8.71131e-02,-4.37397e-01, 2.11473e+00
-1.56178e-01, 1.00378e+00, 6.81933e-02,-6.76474e-03, 7.04144e-02, 1.46763e-02
-3.74403e-01, 1.01258e+00, 1.53951e-01,-3.71976e-02, 1.73747e-01, 7.86682e-02
-8.73847e-01, 1.06711e+00, 2.56553e-01,-1.53780e-01, 4.86150e-01, 3.82557e-01
tcafit  2.0000e-01  1.4086e-03  1.4086e-03  1.3185e-02
-8.51331e+00, 1.15164e-01, 1.29900e+00, 1.11105e+01,-4.05570e-01, 5.65607e+01
-1.06700e-02, 1.00009e+00, 3.58651e-03,-2.25268e-04, 3.57013e-03,-5.66586e-05
...
\end{verbatim}

The first line is a file header starting with the word ``{\vt
tsweep}''.  The numbers that follow on this line are:

\begin{itemize}
\item{The number of fit records contained in this file (91).}
\item{The lowest temperature K used for fit parameters in the file,
this will be used in the first fit record (0.1000).}
\item{The temperature delta K used in the sweep file (0.1000).}
\item{The smoothing parameter, {\vt -s} option, used to create all TCA
tables (0.008).}
\item{The number of scale points, {\vt -x} option, used to create all
TCA tables (500).}
\item{The number of terms used for each fit table, {\vt -n} option
(8).}
\item{The value of the threshold parameter used in each fit table
(0.200).}
\end{itemize}

Following this header, fit records are concatenated.  These are
similar to the format described above, the only difference is that the
header line is simplified to omit redundant information.  The fit
record header contains the following value following the word ``{\vt
tcafit}''.

\begin{itemize}
\item{The temperature in Kelvin (4.2000e+00).}
\item{The left electrode pair-breaking energy in ev (1.3696e-03).}
\item{The right electrode pair-breaking energy in ev (1.3696e-03).}
\item{Normalized quasiparticle current at x=0.8, used to estimate
the sub-gap conductance (3.9580e-2).}
\end{itemize}


\subsection{References}

Background references from the MiTMoJCo project.

Tunnel current calculation:\\
A. I. Larkin and Yu. N. Ovchinnikov, Sov. Phys. JETP 24, 1035 (1967).\\
D. R. Gulevich, V. P. Koshelets, and F. V. Kusmartsev, Phys. Rev. B 96,
 024515 (2017).\\
A. B. Zorin, I. O. Kulik, K. K. Likharev, and J. R. Schrieffer, Sov. J. 
 Low Temp. Phys. 5, 537 (1979).\\
D. R. Gulevich, V. P. Koshelets, and F. V. Kusmartsev, Phys.  Rev. B 96,
 024515 (2017).\\
D. R. Gulevich, V. P. Koshelets, F. V. Kusmartsev, arXiv:1709.04052 (2017).\\
D. R. Gulevich, L. V. Filippenko, V. P. Koshelets, arXiv:1809.01642 (2018).\\

Compression:\\
A. A. Odintsov, V. K. Semenov and A. B. Zorin, IEEE Trans. Magn. 23, 763
 (1987).\\
D. R. Gulevich, V. P. Koshelets, and F. V. Kusmartsev, Phys. Rev. B 96,
  024515 (2017).

%S-----------------------------------------------------------------------------
\section{The {\vt multidec} Utility: Coupled Lossy Transmission Lines}
\index{multidec program}

% spMain.hlp:multidec 012609

The standalone program {\vt multidec} produces a subcircuit for
multiconductor lossy transmission lines in terms of uncoupled (single)
simple lossy lines.  This decomposition is valid only if the following
hold:

\begin{enumerate}
\item{The electrical parameters (R, G, Cs, Cm, Ls, Lm) of all wires are
 identical and independent of frequency.}
\item{Each line is coupled only to its (maximum 2) nearest neighbors.}
\end{enumerate}

The subcircuit is sent to the standard output and is intended to be
included in an input file.

The command-line options for {\vt multidec} are as follows:
\begin{quote}
{\vt -l}$<$self-inductance Ls$>$\\
{\vt -c}$<$self-capacitance Cs$>$\\
{\vt -r}$<$series-resistance R$>$\\
{\vt -g}$<$parallel-conductance G$>$\\
{\vt -k}$<$coeff-of-inductive-coupling K$>$\\
{\vt -x}$<$mutual-capacitance Cm$>$\\
{\vt -o}$<$subckt-name$>$\\
{\vt -n}$<$number-of-conductors$>$\\
{\vt -L}$<$length$>$
\end{quote}

The inductive coupling coefficient K is the ratio of Lm
to Ls. Values for -l, -c, -o, -n and -L must be specified.

Example:
\begin{quote}
{\vt multidec -n4 -l9e-9 -c20e-12 -r5.3 -x5e-12 -k0.7 -otest -L5.4}
\end{quote}

This utility was written by J.S.  Roychowdhury for use with the lossy
transmission line model \cite{ltra}.


%S-----------------------------------------------------------------------------
\section{The {\vt printtoraw} Utility: Print to Rawfile Conversion}
\index{printtoraw program}

% spMain.hlp:printtoraw 012609

The {\vt printtoraw} program is a stand-alone utility provided with
the {\WRspice} distribution.  This converts the data in files produced
by the {\et print} command using output redirection into the rawfile
format, which can be plotted.  This works only for print files in the
standard columnar form.

\begin{quote}
Usage: {\vt printtoraw} [{\it printfile\/}]
\end{quote}

The argument, if given, is assumed to be a path to a file that was
produced by the {\WRspice} {\et print} command through redirection. 
If no argument is given, the standard input is read.  The data are
converted to rawfile format and dumped to the standard output.

Example:
\begin{quote}
{\vt wrspice> run}\\
{\vt wrspice> print v(1) v(2) v(3) > myfile}\\
{\vt wrspice> quit}\\
\\
{\vt bash> printtoraw myfile > myfile.raw}\\
\\
{\vt wrspice> load myfile.raw}\\
{\vt wrspice> plot all}
\end{quote}


%S-----------------------------------------------------------------------------
\section{The {\vt proc2mod} Utility: BSIM1 Model Generation}
\index{proc2mod program}

% spMain.hlp:proc2mod 012609

This utility, provided with SPICE3, produces a set of BSIM1 models
from process-dependent data provided in a ``process'' file.  An
example process ({\vt .pro}) file is provided with the {\WRspice}
examples.  This utility was written by J.  Pierret \cite{pierret},
and the reference presumably provides more information.


%S-----------------------------------------------------------------------------
\section{The {\vt wrspiced} Daemon: Remote SPICE Controller}
\label{wrspiced}
\index{wrspiced program}

% spMain.hlp:wrspiced 021811

{\WRspice} can be accessed and run from a remote system for
asynchronous simulation runs, for assistance in computationally
intensive tasks such as Monte Carlo analysis, and as a simulator for
the {\Xic} graphical editor.  This is made possible through a daemon
(background) process which controls {\WRspice} on the remote machine. 
The daemon has the executable name ``{\vt wrspiced}'', and should be
running on the remote machine.  This can be initiated in the system
startup procedure, or manually.  Generally, any user can start {\vt
wrspiced}, but only one daemon can be running on the host computer.

The {\vt wrspiced} program is part of the {\WRspice} distribution, and
is installed in the same directory as the {\vt wrspice} executable. 
The daemon manages the queue of submitted jobs and responses, and
maintains the communications port.  The {\vt wrspiced} daemon will
establish itself on a port, and wait for client messages.

%SU-------------------------------------
\subsection{SPICE Server Configuration}

There is little or no configuration required to run {\vt wrspiced},
but there are a few basic prerequisites.  Our assumption is that
{\WRspice} is installed on a network-reachable remote computer (the
``SPICE server''), and we wish to submit jobs to this {\WRspice} from
within {\Xic}, or from within {\WRspice} running on local computers
(the ``clients'').

The SPICE server must have {\WRspice} installed, and {\WRspice} must
be licensed to run on the server.
As a prerequisite, {\WRspice} should operate on the
SPICE server host in the normal way.

Historically, {\vt wrspiced} has used the service name ``{\vt spice}''
and port number 3004.  Releases 3.2.8 and later use the service name
``{\vt wrspice}'' instead of ``{\vt spice}'', and use port number 6114
by default.  The port 6114 is registered with IANA for this service.

The system services database is represented by the contents of the
file {\vt /etc/services} in simple installations.  If using NIS, then
the system will get its services information from elsewhere.  A system
administrator can add service names and port assignments to this
database.  The {\vt wrspiced} program does not require this.

%SU-------------------------------------
\subsection{Starting the Daemon}

The {\vt wrspiced} program command line has the following form:

\begin{quote}
{\vt wrspiced} [{\vt -fg}] [{\vt -l} {\it logfile\/}]
  [{\vt -p} {\it program\/}] [{\vt -m} {\it maxjobs\/}] [{\vt -t} {\it port\/}]
\end{quote}

There are five optional arguments.
\begin{description}
\item{\vt -fg}\\
If given, the {\vt wrspiced} program will remain in the foreground
(i.e., not become a ``daemon''), but will service requests normally. 
This may be useful for debugging purposes.

\item{{\vt -l} {\it logfile}}\\
The {\it logfile} is a path to a file that will receive status
messages from {\vt wrspiced}.  The default is the value of the {\et
SPICE\_DAEMONLOG} environment variable if set when the program is
started, or {\vt /tmp/wrspiced.log}.

\item{{\vt -p} {\it program}}\\
This specifies the {\WRspice} program to run, in case for some reason
the {\vt wrspice} binary has been renamed, or {\vt wrspice} is not in
the expected location.  This overrides the values of the {\et
SPICE\_PATH} and {\et SPICE\_EXEC\_DIR} environment variables, which
can also be used to set the path to the binary.  The default is ``{\vt
/usr/local/xictools/bin/wrspice}''.

\item{{\vt -m} {\it maxjobs}}\\
This sets the maximum number of jobs that the server will allow to be
running at the same time.  The default is 5.

\item{{\vt -t} {\it port}}\\
This sets the port to be used by the daemon, and overrides any port
set in the services database.  Clients must use the same port number
to connect to the SPICE server.
\end{description}

The daemon is started by simply typing the command.  If a machine is
to operate continuously as a SPICE server, it is recommended that the
{\vt wrspiced} daemon be started in the system initialization scripts. 
The daemon will run until explicitly killed by a signal, or the
machine is halted.  When the {\vt wrspiced} process terminates, any
{\WRspice} processes under management will also be killed.  The daemon
can be terminated, by the process owner, by giving the command ``{\vt
ps aux | grep wrspiced}'' and noting the process id (pid) number of
the running {\vt wrspiced} process, and then issuing ``{\vt kill} {\it
pid\/}'' using this pid number.

It may be necessary to become root before starting {\vt wrspiced}, as
on some systems connection to the port will otherwise be refused due
to permission requirements.  Starting by root is also required if the
log file is to be written to a directory such as {\vt /var/log} that
requires root permission for writing.

%SU-------------------------------------
\subsection{Client Configuration}

The port number used by the client must be the same as that used for
the server.  As for the server, if not supplied the port number will
be resolved if possible in the services database (e.g., the {\vt
/etc/services} file), and will revert to a default if not found.

In {\Xic} and {\WRspice}, the port number to use can be specified with
the host name, by appending the number following a colon, i.e.,
\begin{quote}
{\it hostname\/}[:{\it port\/}]
\end{quote}

A {\WRspice} server can receive jobs from {\Xic}, and from {\WRspice}
({\cb rspice} command).  Both programs have means by which the SPICE
server can be specified from within the program.  One means common to
both programs is through use of the {\et SPICE\_HOST} environment
variable.  The variable should be set to the host name of the SPICE
server, as resolvable by the client, followed by the optional colon
and port number.  When set, {\Xic} will by default use this server for
SPICE jobs initiated with the {\cb Run} button in the side menu, and
{\WRspice} will use this host in the {\et rspice} command.  In a
situation where the SPICE server provides the only SPICE available,
the {\et SPICE\_HOST} variable should be set in the user's shell
startup script.  In {\WRspice} the {\et rhost} shell variable and the
{\cb rhost} command can also be used to specify the remote host, and
these override any value set in the environment.

Note:  In {\Xic}, when {\WRspice} connects, a message is printed in
the terminal window similar to
\begin{quote}
{\vt Stream established to wrspice://chaucer, port 4573.}
\end{quote}
The ``port'' in this case is {\it not} the {\vt wrspiced} port
discussed above, but is a transient port created for the process.

