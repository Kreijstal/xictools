\chapter{{\WRspice} Commands}
\label{commands}

% spCommands.hlp:commands 083122

When a line is entered, it is interpreted in one of several ways. 

\begin{enumerate}
\item{An alias}\\
First, it may be an alias, in which case the line is replaced with the
result after alias substitution, and the line is re-parsed.

\item{A codeblock}\\
Second, it may be the name of a codeblock, which is a user-defined
command obtained from a script file, in which case the codeblock is
executed. 

\item{A command}\\
Third, it may be a pre-defined command, in which case it is executed. 

\item{An assignment, implivit {\cb let}}\\
Fourth, it may be an assignment statement, which consists of a vector
name, an `{\vt =}' symbol, and an expression, in which case it is
executed as if it were preceded by the word ``{\vt let}''.

\item{A circuit filename, implicit {\cb source}}\\
Fifth, it may be the name of a circuit file, in which case it is
loaded as if with a {\cb source} command, or it may be the name of a
command script -- {\WRspice} searches the current {\et sourcepath}
(search path) for the file and executes it when it is found.  The
effect of this is identical to the effect of ``{\vt source} {\it
file}'', except that the variables {\et argc} and {\et argv} are set.

\item{An opeerating system command}\\
Sixth, it may be a command known to the hosting operating system, in
which case if the variable {\et unixcom} is set, it is executed as
though it were typed to the operating system shell.

\item{An expression list}\\
Finally, if the command line can be recognized as a list of
expressions, the {\cb print} command is invoked on the line.
\end{enumerate}


The following table lists all built-in commands understood by
{\WRspice}.
\newpage

\begin{longtable}{|l|l|}\hline
\multicolumn{2}{|c|}{\bf Control Structures}\\ \hline
\cb cdump & Dump control structures for debugging\\ \hline
\hline
\multicolumn{2}{|c|}{\bf String Comparison and Global Return Value}\\ \hline
\cb strcmp & Compare strings\\ \hline
\cb strcicmp & Compare strings, case insensitive\\ \hline
\cb strprefix & Check if string is prefix of another\\ \hline
\cb strciprefix & Check if string is prefix of another, case insensitive\\ \hline
\cb retval & Set the global return value\\ \hline
\hline
\multicolumn{2}{|c|}{\bf User Interface Setup Commands}\\ \hline
\cb mapkey & Create keyboard mapping\\ \hline
\cb setcase & Check/set case sensitivity for name classes\\ \hline
\cb setfont & Set graphical interface fonts\\ \hline
\cb setrdb & Set X resources\\ \hline
\cb tbupdate & Save tool window configuration\\ \hline
\cb wrupdate & Download/install program updates\\ \hline
\hline
\multicolumn{2}{|c|}{\bf Shell Commands}\\ \hline
\cb alias & Create alias\\ \hline
\cb cd & Change directory\\ \hline
\cb echo & Print string\\ \hline
\cb echof & Print string to file\\ \hline
\cb history & Print command history\\ \hline
\cb pause & Pause script execution\\ \hline
\cb pwd & Print the current working dirsctory\\ \hline
\cb rehash & Update command database\\ \hline
\cb set & Set a variable\\ \hline
\cb shell & Execute operating system commands\\ \hline
\cb shift & Shift argument list\\ \hline
\cb unalias & Destroy alias\\ \hline
\cb unset & Unset a variable\\ \hline
\cb usrset & Print list of internally used variables\\ \hline
\hline
\multicolumn{2}{|c|}{\bf Input and Output Commands}\\ \hline
\cb codeblock & Manipulate codeblocks\\ \hline
\cb dumpnodes & Print node voltages and branch currents\\ \hline
\cb edit & Edit text file\\ \hline
\cb listing & List current circuit\\ \hline
\cb load & Read plot data from file\\ \hline
\cb print & Print vectors\\ \hline
\cb printf & Print vectors to logging file\\ \hline
\cb return & Return from script immediately, possibly with a value\\ \hline
\cb sced & Bring up {\Xic} schematic editor\\ \hline
\cb source & Read circuit or script input file\\ \hline
\cb write & Write data to rawfile\\ \hline
\cb xeditor & Edit text file\\ \hline
\hline
\multicolumn{2}{|c|}{\bf Simulation Commands}\\ \hline
\cb ac & Perform ac analysis\\ \hline
\cb alter & Change circuit parameter\\ \hline
\cb alterf & Dump alter list to Monte Carlo output file\\ \hline
\cb aspice & Initiate asynchronous run\\ \hline
\cb cache & Manipulate subcircuit/model cache\\ \hline
\cb check & Initiate range analysis\\ \hline
\cb dc & Initiate dc analysis\\ \hline
\cb delete & Delete watchpoint\\ \hline
\cb destroy & Delete plot\\ \hline
\cb devcnt & Print device counts\\ \hline
\cb devload & Load device module\\ \hline
\cb devls & List available devices\\ \hline
\cb devmod & Change device model levels\\ \hline
\cb disto & Initiate distortion analysis\\ \hline
\cb dump & Print circuit matrix\\ \hline
\cb findlower & Find lower edge of operating range\\ \hline
\cb findrange & Find edges of operating range\\ \hline
\cb findupper & Find upper edge of operating range\\ \hline
\cb free & Delete circuits and/or plots\\ \hline
\cb jobs & Check asynchronous jobs\\ \hline
\cb loop & Alias for sweep command\\ \hline
\cb mctrial & Run a Monte Carlo trial\\ \hline
\cb measure & Set up a measurement\\ \hline
\cb noise & Initiate noise analysis\\ \hline
\cb op & Compute operating point\\ \hline
\cb pz & Initiate pole-zero analysis\\ \hline
\cb reset & Reset simulator\\ \hline
\cb resume & Resume run in progress\\ \hline
\cb rhost & Identify remote SPICE host\\ \hline
\cb rspice & Initiate remote SPICE run\\ \hline
\cb run & Initiate simulation\\ \hline
\cb save & List vectors to save during run\\ \hline
\cb sens & Initiate sensitivity analysis\\ \hline
\cb setcirc & Set current circuit\\ \hline
\cb show & List parameters\\ \hline
\cb state & Print circuit state\\ \hline
\cb status & Print trace status\\ \hline
\cb step & Advance simulator\\ \hline
\cb stop & Specify stop condition\\ \hline
\cb sweep & Perform analysis over parameter range\\ \hline
\cb tf & Initiate transfer function analysis\\ \hline
\cb trace & Set trace\\ \hline
\cb tran & Initiate transient analysis\\ \hline
\cb vastep & Advance Verilog simulator\\ \hline
\cb where & Print nonconvergence information\\ \hline
\hline
\multicolumn{2}{|c|}{\bf Data Manipulation Commands}\\ \hline
\cb compose & Create vector\\ \hline
\cb cross & Vector cross operation\\ \hline
\cb define & Define a macro function\\ \hline
\cb deftype & Define a data type\\ \hline
\cb diff & Compare plots and vectors\\ \hline
\cb display & Print vector list\\ \hline
\cb fourier & Perform spectral analysis\\ \hline
\cb let & Create or assign vectors\\ \hline
\cb linearize & Linearize vector data\\ \hline
\cb pick & Create vector from reduced data\\ \hline
\cb seed & Seed random number generator\\ \hline
\cb setdim & Set current plot dimensions\\ \hline
\cb setplot & Set current plot\\ \hline
\cb setscale & Assign scale to vector\\ \hline
\cb settype & Assign type to vector\\ \hline
\cb spec & Perform spectral analysis\\ \hline
\cb undefine & Undefine macro function\\ \hline
\cb unlet & Undefine vector\\ \hline
\hline
\multicolumn{2}{|c|}{\bf Graphical Output Commands}\\ \hline
\cb asciiplot & Generate line printer plot\\ \hline
\cb combine & Combine plots\\ \hline
\cb hardcopy & Send plot to printer\\ \hline
\cb iplot & Plot during simulation\\ \hline
\cb mplot & Plot range analysis output\\ \hline
\cb plot & Plot simulation results\\ \hline
\cb plotwin & Pop down and destroy plot windows\\ \hline
\cb xgraph & Plot simulation results using {\vt xgraph}\\ \hline
\hline
\multicolumn{2}{|c|}{\bf Miscellaneous Commands}\\ \hline
\cb bug & Submit bug report\\ \hline
\cb help & Enter help system\\ \hline
\cb helpreset & Clear help system cache\\ \hline
\cb qhelp & Print command summaries\\ \hline
\cb quit & Exit program\\ \hline
\cb rusage & Print resource usage statistics\\ \hline
\cb stats & Print resource usage statistics\\ \hline
\cb version & Print program version\\ \hline
\end{longtable}


%S-----------------------------------------------------------------------------
\section{Control Structures}
\index{control structures}

% spCommands.hlp:controlcmds 012209

Control structures operate on expressions involving vectors,
constants, and ({\vt \$}-substituted) shell variables.  A non-zero
result (of any element, if the length is greater than 1) indicates
``true''.  The following control structures are available:

Although control structures are most commonly used in command scripts,
they are also allowed from the command line.  While a block is active,
the command prompt changes to one or more ``$>$'' characters, the
number of which represents the current depth into the control
commands.  As with a UNIX shell, control structures can be used from
the command line to repeat one or more commands.

\begin{description}
\item{{\bf repeat} block}\\
\index{repeat block}

\begin{quote}
{\vt {\bf repeat} [{\it number\/}]}\\
\quad {\it statement}\\
\quad {...}\\
{\bf\vt end}
\end{quote}
Execute the statements in the block defined by the {\et repeat} line
and the corresponding {\et end} statement {\it number\/} times, or
indefinitely if no {\it number} is given.  The {\it number} must be a
constant, or a shell variable reference that evaluates to a constant,
which may be a vector reference in the {\vt \$\&} form.  A vector name
is not valid.

\item{{\bf while} block}\\
\index{while block}

\begin{quote}
{\bf\vt while} {\it condition}\\
\quad {\it statement}\\
\quad {...}\\
{\bf\vt end}
\end{quote}
The {\et while} line, together with a matching {\et end} statement,
defines a block of commands that are executed while the {\it
condition} remains true.  The {\it condition} is an expression which
is considered true if it evaluates to a nonzero value, or if a vector,
any component is nonzero.  The test is performed at the top of the
loop, so that if the {\it condition} is initially false, the
statements are not executed.

\item{{\bf dowhile} block}\\
\index{dowhile block}

\begin{quote}
{\bf\vt dowhile} {\it condition}\\
\quad {\it statement}\\
\quad {...}\\
{\bf\vt end}
\end{quote}
The {\et dowhile} line, together with a matching {\et end} statement,
defines a block of commands that are executed while the {\it
condition} remains true.  The {\it condition} is an expression which
is considered true if it evaluates to a nonzero value, or if a vector,
any component is nonzero.  Unlike the {\et while} statement, the test
is performed at the bottom of the loop -- so that the loop executes at
least once.

\item{{\bf foreach} block}\\
\index{foreach block}

\begin{quote}
{\bf\vt foreach} {\it var value\/} ...\\
\quad {\it statement}\\
\quad {...}\\
{\bf\vt end}
\end{quote}
The {\et foreach} statement opens a block which will be executed once
for each {\it value} given.  Each time through, the {\it var} will be
set to successive {\it value\/}s.  After the loop is exited it will
have the last value that was assigned to it.  The {\it var} can be
accessed in the loop with the {\vt \$var} notation, i.e., it should be
treated as a shell variable, not a vector.  This is set to each {\it
value} as a text item.

\item{{\bf if} block}\\
\index{if block}

\begin{quote}
{\bf\vt if} {\it condition}\\
\quad {\it statement}\\
\quad {...}\\
{\bf\vt else}\\
\quad {\it statement}\\
\quad {...}\\
{\bf\vt end}
\end{quote}
If the {\it condition\/} is non-zero then the first set of statements is
executed, otherwise the second set.  The {\et else} and the second set of
statements may be omitted.

\item{{\bf label} statement}\\
\index{label statement}

\begin{quote}
{\bf\vt label} {\it labelname}
\end{quote}
This defines a label which can be used as an argument to a {\et goto}
statememt.

\item{{\bf goto} statement}\\
\index{goto statement}

\begin{quote}
{\bf\vt goto} {\it label}
\end{quote}
If there is a {\et label} statement defining the {\it label} in the
block or an enclosing block, control is transfered there.  If the {\et
goto} is used outside of a block, the label must appear ahead of the
{\et goto} (i.e., a forward {\et goto} may occur only within a block). 
There is a {\et begin} macro pre-defined as ``{\vt if 1}'' which may
be used if forward label references are required outside of a block
construct.

\item{{\bf continue} statement}\\
\index{continue statement}

\begin{quote}
{\bf\vt continue [{\it number}]}
\end{quote}
If there is a {\et while}, {\et dowhile}, {\et foreach} or {\et
repeat} block enclosing this statement, the next iteration begins
immediately and control passes to the top of the block.  Otherwise an
error results.  If a {\it number} is given, that many surrounding
blocks are continued.  If there are not that many blocks, an error
results.

\item{{\bf break} statement}\\
\index{break statement}

\begin{quote}
{\bf\vt break [{\it number}]}
\end{quote}
If there is a {\et while}, {\et dowhile}, {\et foreach}, or {\et
repeat} block enclosing this statement, control passes out of the
block.  Otherwise an error results.  If a {\it number} is given, that
many surrounding blocks are exited.  If there are not that many
blocks, an error results.

\item{{\bf end} statement}\\
\index{end statement}

\begin{quote}
{\bf\vt end}
\end{quote}
This statement terminates a block.  It is an error for an {\et end} to
appear without a matching {\et if}, {\et while}, {\et dowhile}, {\et
foreach}, or {\et repeat} statement.  The keywords {\et endif}, {\et
endwhile}, {\et enddowhile}, {\et endforeach}, and {\et endrepeat} are
internally aliased to {\et end}.
\end{description}

Control structures may be nested.  When a block is entered and the
input is from the keyboard, the prompt becomes a number of $>$'s
equalling the depth of blocks the user has entered.  The current
control structures may be examined with the debugging command {\cb
cdump}.

\newcommand{\spcmd}[1]{The {\cb #1} Command}

%SU-------------------------------------
\subsection{\spcmd{cdump}}
\index{cdump command}

% spCommands.hlp:cdump 012709

The {\cb cdump} command prints out the contents of the currently
active control structures.  The command takes no arguments.  It is
intended primarily for debugging.


%S-----------------------------------------------------------------------------
\section{String Comparison and Global Return Value}
\index{string comparison}
\index{global return value}

% spCommands.hlp:strcmds 030519

These commands are used for string comparison, and for setting the
global return value.  The global return value is an internal global
variable that can be set and queried from any script (with the {\vt
\$?} construct).  This can be used to pass numeric data from a script,
but one must take care that the value is not overwritten before use,
as its scope is global.  The string comparison functions return their
comparison result in the global return value.  There is no native
string data type in the scripting language, and the commands here
provide basic string support.

\begin{tabular}{|l|l|}\hline
\cb strcmp & Compare strings\\ \hline
\cb strcicmp & Compare strings, case insensitive\\ \hline
\cb strprefix & Check if string is prefix of another\\ \hline
\cb strciprefix & Check if string is prefix of another, case insensitive\\ \hline
\cb retval & Set the global return value\\ \hline
\end{tabular}

%SU-------------------------------------
\subsection{\spcmd{strcmp}}
\index{strcmp command}

% spCommands.hlp:strcmp 080816

The {\cb strcmp} command is used for string comparison in control
structures.
\begin{quote}\vt
strcmp [{\it varname\/}] {\it string1 string2}
\end{quote}
This supports the original Spice3 {\cb strcmp} which returns is value
in a given variable, and the {\WRspice} convention where the
comparison value is returned in the global return value (accessible
with ``{\vt \$?}'').

In either case, the comparison value is a number that is less than,
equal to, or greater than zero according to whether {\it string1} is
lexically before, equal to, or after {\it string2\/}.

If three arguments are given, the first argument is taken as the name
of a variable which is set to the comparison value.  This convention
is supported for backwards compatibility, of this function only. 
Otherwise, the global return value will be set to this value.  The
other arguments are literal strings.

Example
\begin{quote}\vt
.control\\
set str1="abcd efgh"\\
set str2="bbcd efgh"\\
strcmp "\$str1" "\$str2"\\
if (\$? < 0)\\
\hspace*{2em}echo "\$str1" ahead of "\$str2"\\    
else\\
if (\$? = 0)\\
\hspace*{2em}echo strings are the same\\    
else\\
\hspace*{2em}echo "\$str1" after "\$str2"\\    
end\\
end\\
.endc
\end{quote}

%SU-------------------------------------
\subsection{\spcmd{strcicmp}}
\index{strcicmp command}

% spCommands.hlp:strcicmp 080816

The {\cb strcicmp} command is used for string comparison in control
structures.
\begin{quote}\vt
strcicmp {\it string1 string2}
\end{quote}
The {\cb strcicmp} command is similar to {\cb strcmp}, however the
comparison result is case-insensitive, and the Spice3 return
convention is not supported.  The global return value (accessible as
``{\vt \$?}'') is set to the comparison value.  The comparison value
is a number that is less than, equal to, or greater than zero
according to whether {\it string1} is lexically before, equal to, or
after {\it string2}.  The two arguments are literal strings.

%SU-------------------------------------
\subsection{\spcmd{strprefix}}
\index{strprefix command}

% spCommands.hlp:strprefix 080816

The {\cb strprefix} command will set the global return value to one if
{\it string1} is a prefix of {\it string2\/}, or zero if not.
\begin{quote}\vt
strprefix {\it string1 string2}
\end{quote}

%SU-------------------------------------
\subsection{\spcmd{strciprefix}}
\index{strciprefix command}

% spCommands.hlp:strciprefix 080816

The {\cb strciprefix} command will set the global return value to one
if {\it string1} is a case-insensitive prefix of {\it string2\/}, or
zero if not.
\begin{quote}\vt
strciprefix {\it string1 string2}
\end{quote}

%SU-------------------------------------
\subsection{\spcmd{retval}}
\index{retval command}

% spCommands.hlp:retval 030519

The {\cb retval} command will set the global return value to the numeric
value given.
\begin{quote}\vt
retval {\it value}
\end{quote}
This can be used to pass a value back from a script.  The value is
initialized to zero whenever a script is executed, so that zero is the
default return value.  The global return value is a global value
available in any script and the command prompt line, and is accessed
with the special variable name {\vt \$?}.  The global return value is
set by this function and the string comparison functions, and
optionally by the {\cb return} function.


%S-----------------------------------------------------------------------------
\section{User Interface Setup Commands}
\index{user interface setup commands}

% spCommands.hlp:uicmds 102817

These commands perform setup and control of aspects of the user
interface, both graphical and non-graphical.

\begin{tabular}{|l|l|}\hline
\multicolumn{2}{|c|}{Uset Interface Setup Commands}\\ \hline
\cb mapkey & Create keyboard mapping\\ \hline
\cb setcase & Check/set case sensitivity for name classes\\ \hline
\cb setfont & Set graphical interface fonts\\ \hline
\cb setrdb & Set X resources\\ \hline
\cb tbupdate & Save tool window configuration\\ \hline
\cb wrupdate & Download/install program updates\\ \hline
\end{tabular}

%SU-------------------------------------
\subsection{\spcmd{mapkey}}
\index{mapkey command}

% spCommands.hlp:mapkey 011909

The {\cb mapkey} command provides limited keyboard mapping support.
\begin{quote}\vt
mapkey [ -r [{\it filename\/}] | -w [{\it filename\/}] |
 {\it keyname data} ]
\end{quote}
Only the keys that are used for command line editing are mappable. 
This is to account for ``strange'' terminals that may not send the
expected data when a key is pressed.

The following keys can be mapped:
\begin{quote}\kb
Ctrl-A\\
Ctrl-D\\
Ctrl-E\\
Ctrl-K\\
Ctrl-U\\
Ctrl-V\\
Tab\\
Backspace\\
Delete\\
LeftArrow\\
RightArrow\\
UpArrow\\
DownArrow
\end{quote}
Of these, the arrow keys and {\kb Delete} are most likely to need
remapping.

If no argument is given, the user is prompted to press each of these
keys, and the internal map is updated.  After doing this, the keys
should have their expected effect when pressed while entering a
{\WRspice} command.

If ``{\vt -w} [{\it filename\/}]'' is given, the present internal
map will be saved in the named file, or ``{\vt wrs\_keymap}'' in the
current directory if no {\it filename} is given.

If ``{\vt -r} [{\it filename\/}]'' is given, the file will be read
as a key mapping file, and the internal map will be updated.  The
{\it filename}, if not given, defaults to ``{\vt wrs\_keymap}''.  If
no path is given, it will be found in the current directory or the
startup directory.

If ``{\it keyname data...}'' is given, a single key in the internal
map can be updated.  The format is the same as the entries in the
mapping file, i.e., one of the names above, followed by one or more
hex bytes of data.  The bytes represent the stream sent when the named
key is pressed, and will henceforth be interpreted as the pressing of
that key.  The bytes should be in hex format, and the first byte of a
multi-byte sequence must be the {\kb Escape} character (1b).

Example (from real life):

After installing the latest X-window system, suppose one finds that,
when running {\WRspice} in an {\vt xterm} window, the {\kb Delete} key
no longer deletes the character under the cursor in {\WRspice}, but
instead injects some gibberish.  There are three ways to fix this. 
The first two are specific to the {\vt xterm} program, and instruct
the {\vt xterm} to send the ASCII Del character when {\kb Delete} is
pressed, rather than use the new default which is to send the VT-100
``delete character'' string.  The third method is to map this string
into the delete function in {\WRspice}.

\begin{enumerate}
\item{From the main {\vt xterm} menu, find and click on the ``{\cb
Delete is DEL}'' entry.  Usually, holding the {\kb Ctrl} key and
clicking in the {\vt xterm} with button 1 displays this menu.}

\item{Create a file named ``{\vt XTerm}'' in your home directory,
containing the line
\begin{quote}
{\vt *deleteIsDEL: true}
\end{quote}}

\item{In {\WRspice}, type ``{\vt mapkey}'' and follow the prompts. 
You can save the new map, and add a line to a {\vt .wrspiceinit}
startup file to read it when {\WRspice} starts.}
\end{enumerate}


%SU-------------------------------------
\subsection{\spcmd{setcase}}
\index{setcase command}

% spCommands.hlp:setcase 091511

Syntax: {\vt setcase} [{\it flags\/}]

This command sets or reports the case sensitivity of various name
classes in {\WRspice}.  These classes are:

\begin{quote}
Function names.\\
User-defined function names.\\
Vector names.\\
.PARAM names.\\
Codeblock names.\\
Node and device names.\\
\end{quote}

The {\it flags} is a word consisting of letters, each letter
corresponds to a class from the list above.  If lower-case, the class
will be case-sensitive.  If upper-case, the class will be
case-insensitive.

The letters are {\vt f}, {\vt u}, {\vt v}, {\vt p}, {\vt c}, and {\vt
n} corresponding to the classes listed above.  By default, all
{\WRspice} identifiers are case-insensitive, which corresponds to the
string ``{\vt FUVPCN}''.  Letters can appear in any order, and
unrecognized characters are ignored.  Not all letters need be
included, only those seen will be used.

If given an argument string as described above, and called from a
startup file, the case sensitivities will be set.  This can {\bf not}
be done from the {\WRspice} prompt.  Case sensitivity can also be set
from the command line by using the {\vt -c} option.

If no argument, a report of the case sensitivity status is printed. 
This can be done from the {\WRspice} prompt.


%SU-------------------------------------
\subsection{\spcmd{setfont}}
\index{setfont command}

% spToolbar.hlp:setfont 030518

Syntax: {\vt setfont} {\it font\_num} {\it font\_specifier}

This command can be used to set the fonts employed in the graphical
interface.  Although this can be given at a prompt, it is intended to
be invoked in a startup script.

The first argument is an integer 1--6 (1--4 on Windows) which
designates the font category.  The index corresponds to the entries in
the drop-down menu of font categories found in the Font Selection
panel.

The rest of the line is a font description string.  This varies
between graphics types.

\begin{description}
\item{Unix/Linux}\\
For GTK1 releases, the name is the X Logical Font Descriptor name for
a font available on the user's system, or an alias.  For GTK2
releases, the name is a Pango font description name.  There is a very
modest attempt to interpret a specification of the wrong type.
  
\item{Windows}
The name is in one or the following formats:
 \begin{description}
 \item{New standard ({\WRspice} release 2.3.58 and later)}\\
 {\it face\_name} {\it pixel\_height}
 \begin{quote}
 Example: {\vt Lucida Console 12}
 \end{quote}

 \item{Old standard (deprecated)}\\
 ({\it pixel\_height\/}){\it face\_name}
 \begin{quote}
 Example: {\vt (12)Lucida Console}
 \end{quote}
 \end{description}

 The {\it face\_name} is the name of a font family installed on the
 system, and the {\it pixel\_height} is the on-screen size.
\end{description}

You will probably never need to use the {\cb setfont} command
directly.  All settable fonts are saved in the {\vt .wrspiceinit}
startup file when the {\cb Update Tools} menu command in the {\cb
File} menu is given, or the {\cb tbupdate} command is invoked.

%SU-------------------------------------
\subsection{\spcmd{setrdb}}
\index{setrdb command}

% spCommands.hlp:setrdb 012209

The {\cb setrdb} command adds resources to the X resource database.
\begin{quote}\vt
setrdb {\it resource\/}: {\it value}
\end{quote}
The user interface toolset currently used to implement the {\WRspice}
user interface is the GTK toolkit ({\vt www.gtk.org}) which does not
use the X resource mechanism.

{\WRspice} presently only recognizes resource strings which set the
plotting colors for the {\cb plot} command.  The names of these
resources are ``{\vt color0}'' through ``{\vt color19}'', which
correspond directly to the shell variables of the same name, and to
the colors listed in the {\cb Colors} tool of the {\cb Tools} menu of
the {\cb Tool Control} window.  To set a color using the {\cb setrdb}
command, one can use forms like
\begin{quote}
``{\vt setrdb *color2:  pink}''
\end{quote}

%SU-------------------------------------
\subsection{\spcmd{tbupdate}}
\index{tbupdate command}

% spToolbar.hlp:tbupdate 042118

This command will update the user's {\vt .wrspiceinit} file in the
home directory to reflect the current tool setup.
\begin{quote}\vt
tbupdate
\end{quote}
The window arrangement should be the same the next time the user
starts {\WRspice}.  This command is also performed when the user
presses the {\cb Update Tools} button in the {\cb File} menu of the
{\cb Tool Control} window.

%SU-------------------------------------
\subsection{\spcmd{wrupdate}}
\index{wrupdate command}

% spCommands.hlp:wrupdate 042118

This command can be used to check for, download, and install updates
to the program.
\begin{quote}
{\vt wrupdate}
\end{quote}
This command is equivalent to giving the special keyword ``{\vt
:xt\_pkgs}'' to the help system, which brings up the {\XicTools}
package management page (see \ref{update}).  The page lists installed
and available packages for each of the {\XicTools} programs for the
current operating system, and provides buttons to download and install
the packages.

Unlike in earlier {\WRspice} releases, there is no provision for
automatic checking for updates, so this command or equivalent should
be run periodically to check for updated packages.  The computer must
have http access to the internet for successful use of this
functionality.


%S-----------------------------------------------------------------------------
\section{Shell Commands}
\index{shell commands}

% spCommands.hlp:shellcmds 012709

The commands listed below are built into the {\WRspice} shell, or
control shell operation. 
    
\begin{tabular}{|l|l|}\hline
\multicolumn{2}{|c|}{Shell Commands}\\ \hline
\cb alias & Create alias\\ \hline
\cb cd & Change directory\\ \hline
\cb echo & Print string\\ \hline
\cb echof & Print string to file\\ \hline
\cb history & Print command history\\ \hline
\cb pause & Pause script execution\\ \hline
\cb pwd & Print the current working dirsctory\\ \hline
\cb rehash & Update command database\\ \hline
\cb set & Set a variable\\ \hline
\cb shell & Execute operating system commands\\ \hline
\cb shift & Shift argument list\\ \hline
\cb unalias & Destroy alias\\ \hline
\cb unset & Unset a variable\\ \hline
\cb usrset & Print list of internally used variables\\ \hline
\end{tabular}

%SU-------------------------------------
\subsection{\spcmd{alias}}
\index{alias command}

% spCommands.hlp:alias 011909

The {\cb alias} command is used to create aliases, as in the C-shell.
\begin{quote}\vt
alias [{\it word\/}] [{\it text\/}]
\end{quote}
The {\cb alias} command causes {\it word\/} to be aliased to {\it
text}.  Whenever a command line beginning with {\it word} is typed,
{\it text} is substituted.  Arguments are either appended to the end,
or substituted in if history characters are present in the text.  With
no argument, a list of the current aliases is displayed.

In the body of the alias text, any strings of the form {\vt !:}{\it
number} are replaced with the {\it number\/}'th argument of the actual
command line.  Note that when the alias is defined with the {\cb
alias} command, these strings must be quoted to prevent history
substitution from replacing the {\vt !}'s before the alias command can
get to them.  Thus the command
\begin{quote}\vt
alias foo echo '!:2' '!:1'
\end{quote}
causes ``{\vt foo bar baz"}'' to be replaced with ``{\vt echo baz
bar}''.  Other {\vt !} modifiers as described in the section on
history substitution may also be used, always referring to the actual
command line arguments given.  If a command line starts with a
backslash `{\vt \symbol{92}}' any alias substitution is inhibited.

%SU-------------------------------------
\subsection{\spcmd{cd}}
\index{cd command}

% spCommands.hlp:cd 011909

The {\cb cd} command is used to change the current working directory.
\begin{quote}\vt
cd [{\it directory\/}]
\end{quote}
The command will change the current working directory to {\it
directory\/}, or to the user's home directory if none is given.

%SU-------------------------------------
\subsection{\spcmd{echo}}
\index{echo command}

% spCommands.hlp:echo 011909

The {\cb echo} command will print its arguments on the standard output.
\begin{quote}\vt
echo [-n][{\it stuff} ...]
\end{quote}
If the {\vt -n} option is given, then the arguments are echoed without
a trailing newline.

%SU-------------------------------------
\subsection{\spcmd{echof}}
\index{echof command}

% spCommands.hlp:echof 012709

This command is only available from the control scripts which are
active during Monte Carlo or operating range analysis. 

The {\cb echof} command is used in the same manner as the {\cb echo}
command, however the text is directed to the output file being
generated as the analysis is run.  If the file is not open, there is
no action.  This command can be used in the scripts to insert text,
such as the Monte Carlo trial values, into the output file.

%SU-------------------------------------
\subsection{\spcmd{history}}
\index{history command}

% spCommands.hlp:historycomm 011909

The {\cb history} command prints the last commands executed.
\begin{quote}\vt
history [-r] [{\it number\/}]
\end{quote}
The command will print out the last {\it number} commands typed by the
user, or all the commands saved if {\it number} is not given.  The
number of commands saved is determined by the value of the {\et
history} variable.  If the {\vt -r} flag is given, the list is printed
in reverse order.

%SU-------------------------------------
\subsection{\spcmd{pause}}
\index{pause command}

% spCommands.hlp:pause 011909

The {\cb pause} command is used in scripts to cause the executing
script to wait for a keypress.  The function takes no arguments, and
the keypress is discarded.

%SU-------------------------------------
\subsection{\spcmd{pwd}}
\index{pwd command}

% spCommands.hlp:pwd 091714

Print the current working directory.

%SU-------------------------------------
\subsection{\spcmd{rehash}}
\index{rehash command}

% spCommands.hlp:rehash 011909

The {\cb rehash} command rebuilds the command list from the files
found along the user's executable file search path.  The command
will recalculate the internal hash tables used when looking up
operating system commands, and make all operating system commands in
the user's {\et PATH} available for command completion.  This
command takes no arguments, and has effect only when the {\et
unixcom} variable is set.

%SU-------------------------------------
\subsection{\spcmd{set}}
\index{set command}
\label{setcmd}

% spCommands.hlp:set 041611

The {\cb set} command allows the user to examine and set shell
variables.  It is also possible to assign vectors with the {\cb set}
command.
\begin{quote}\vt
set [{\it varname} [= {\it value}] ...]
\end{quote}
In addition, shell variables are set which correspond to definitions
supplied on the {\vt .options} line of the current circuit, and there
are additional shell variables which are set automatically in accord
with the current plot.  The shell variables that are currently active
can be listed with the {\cb set} command given without arguments, and
are also listed within the {\cb Variables} window brought up from the
{\cb Tools} menu of the {\cb Tool Control} window.  In these listings,
a `{\vt +}' symbol is prepended to variables defined from a {\vt
.options} line in the current circuit, and a `{\vt *}' symbol is
prepended to those variables defined for the current plot.  These
variable definitions will change as the current circuit and current
plot change.  Some variables are read-only and may not be changed by
the user, though this is not indicated in the listing.

Before a simulation starts, the options from the {\vt .options} line
of the current circuit are merged with any of the same name that have
been set using the shell.  The result of the merge is that options
that are booleans will be set if set in either case, and those that
take values will assume the value set through the shell if conflicting
definitions are given.  The merge will be suppressed if the shell
variable {\et noshellopts} is set {\it from the shell}, in which case
the only options used will be those from the {\vt .options} line, and
those that are redefined using the {\cb set} command will be ignored.

Above, the {\it varname} is the name of the shell variable to set, and
{\it value}, if present, is a single token to be assigned. 
Multiple variables can be assigned with a single {\cb set} command. 
If {\it value} is missing (along with the `{\vt =}'), then {\it
varname} is of boolean type and always taken as ``true'' when set. 
If {\it value} is a pure number not double quoted, then {\it
varname} will reference that number.  Otherwise, {\it varname} will
reference {\it value} as a character string, unless {\it value} is
a list.  A list is a space-separated list of tokens in space-separated
parentheses, as in
\begin{quote}\vt
set mylist = ( abc def 1.2 xxdone )
\end{quote}
which sets the variable {\vt mylist} to the list of four tokens.  The
{\cb unset} command can be used to delete a variable.

The value of a variable {\it word\/} may be inserted into a command by
writing {\vt \${\it word}}.  If a variable is set to a list of values
that are enclosed in parentheses (which must be separated from their
values by white space), the value of the variable is the list.

The set command can also be used to assign values to vectors (vectors
are described in \ref{vectors}).  The syntax in this case is
 
\begin{quote}
{\vt set \&}{\it vector} {\vt =} {\it value}
\end{quote}
 
which is equivalent to
 
\begin{quote}
{\vt let} {\it vector} {\vt =} {\it value}
\end{quote}
 
When entering this form from the {\WRspice} command line, the `{\vt
\&}' character must be hidden from the shell, perhaps most
conveniently be preceding it with a backslash.  The {\it value} must
be numeric, and a value must be given, unlike for a variable which can
be set as a boolean.

There are a number of variables with internal meaning to {\WRspice},
and in fact this is the mechanism by which most {\WRspice} defaults
are specified.  Several of the other buttons in the {\cb Tools} menu,
including {\cb Commands}, {\cb Debug}, {\cb Plot Opts}, {\cb Shell},
and {\cb Sim Opts} bring up panels from which these special variables
can be modified.

The predefined variables which have meaning to {\WRspice} (see
\ref{variables}) can be listed with the {\cb usrset} command.  In
general, variables set in the {\vt .options} line are available for
expansion in {\vt \$}{\it varname} references, but do not otherwise
affect the functionality of the shell.

%SU-------------------------------------
\subsection{\spcmd{shell}}
\index{shell comand}

% spCommands.hlp:shell 011909

The {\cb shell} command will pass its arguments to the operating system
shell.
\begin{quote}\vt
shell [{\it command\/}]
\end{quote}
The command will fork a shell if no {\it command\/} is given, or
execute the arguments as a command to the operating system.

%SU-------------------------------------
\subsection{\spcmd{shift}}
\index{shift command}

% spCommands.hlp:shift 011909

The {\cb shift} command facilitates handling of list variables in
shell scripts.
\begin{quote}\vt
shift [{\it varname\/}] [{\it number\/}]
\end{quote}
If {\it varname} is the name of a list variable, it is shifted to
the left by {\it number} elements, i.e., the {\it number} leftmost
elements are removed. The default {\it varname} is {\vt argv}, and
the default {\it number} is 1.

%SU-------------------------------------
\subsection{\spcmd{unalias}}
\index{unalias command}

% spCommands.hlp:unalias 011909

The {\cb unalias} command is used to remove aliases previously set with
the {\cb alias} command.
\begin{quote}\vt
unalias [{\it word} ...]
\end{quote}
The command removes any aliases associated with each of the {\it
word\/}s.  The argument may be ``{\vt *}'', in which case all aliases
are deleted.

%SU-------------------------------------
\subsection{\spcmd{unset}}
\index{unset command}

% spCommands.hlp:unset 011909

The {\cb unset} command will remove the definitions of shell
variables, previously defined with the {\cb set} command, passed as
arguments.
\begin{quote}\vt
unset [{\it varname} ...]
\end{quote}
All of the named variables are unset (undefined).  The argument may be
``{\vt *}'', in which case all variables are unset (although this is
usually not something that one would want to do).

%SU-------------------------------------
\subsection{\spcmd{usrset}}
\index{usrset command}

% spCommands.hlp:usrset 011909

The {\cb usrset} command prints a (long) list of all of the variables
used internally by {\WRspice} which can be set with the {\cb set}
command.
\begin{quote}\vt
usrset [-c][-d][-p][-sh][-si] [{\it keyword} ...]
\end{quote}
{\WRspice} provides a substantial number of internal switches and
variables which can be configured with the {\cb set} command.  The
{\cb usrset} command prints a listing and brief description of each of
the variables with internal significance to {\WRspice}.  If no
arguments are given, all of the variables which control {\WRspice}
will be printed.  The options print sets of keywords associated with
certain functions, which are in turn associated with a particular
panel accessible from the {\cb Tool Control} window.
\begin{quote}
\begin{tabular}{|l|l|l|}\hline
Option & Toolbar Button & Description\\ \hline\hline
\vt -c & \cb Commands & Variables which control {\WRspice} commands\\ \hline
\vt -d & \cb Debug & Debugging variables\\ \hline
\vt -p & \cb Plot Opts & Variables which control plotting\\ \hline
\vt -sh & \cb Shell & Variables which control the shell\\ \hline
\vt -si & \cb Sim Opts & Simulation control and SPICE options\\ \hline
\end{tabular}
\end{quote}
Other arguments are taken as variable names, which will result in
a description of that variable being printed.


%S-----------------------------------------------------------------------------
\section{Input and Output Commands}

% spCommands.hlp:iocmds 011721

These commands manage input to {\WRspice}, or allow {\WRspice}
output to be saved in files.

\begin{tabular}{|l|l|}\hline
\multicolumn{2}{|c|}{Input and Output Commands}\\ \hline
\cb codeblock & Manipulate codeblocks\\ \hline
\cb dumpnodes & Print node voltages and branch currents\\ \hline
\cb edit & Edit text file\\ \hline
\cb listing & List current circuit\\ \hline
\cb load & Read plot data from file\\ \hline
\cb print & Print vectors\\ \hline
\cb printf & Print vectors to logging file\\ \hline
\cb return & Return from script immediately, possibly with a value\\ \hline
\cb sced & Bring up {\Xic} schematic editor\\ \hline
\cb source & Read circuit or script input file\\ \hline
\cb write & Write data to rawfile\\ \hline
\cb xeditor & Edit text file\\ \hline
\end{tabular}

%SU-------------------------------------
\subsection{\spcmd{codeblock}}
\index{codeblock command}
\label{codeblock}

% spCommands.hlp:codeblock 032320

The {\cb codeblock} command manipulates codeblocks.
\begin{quote}\vt
codeblock [{\it -options\/}] [{\it filename\/}]
\end{quote}
A codeblock is a stored executable structure derived from a script
file.  Being internal representations, codeblocks execute more
efficiently than script files.  A codeblock generally has the same
name as the script file from which it was derived.

Option characters, which may be grouped or given as separate tokens,
following a '--' character, are listed below.
\begin{quote}
\begin{tabular}{|l|l|}\hline
{\vt p} & print the text of a block (synonym {\vt t})\\ \hline
{\vt d} & delete the block (synonym {\vt f})\\ \hline
{\vt a} & add a block\\ \hline
{\vt b} & bind the block to the ``controls'' of the current circuit\\ \hline
{\vt be} & bind the block to the ``execs'' of the current circuit\\ \hline
{\vt c} & list bound codeblocks of the current circuit\\ \hline
\end{tabular}
\end{quote}

If no {\it filename} is given, and neither of the bind options is
given, all of the blocks in the internal list are listed by name, and
their commands are printed if {\vt p} is given, and the blocks are
deleted if {\vt d} is given.  In the latter case, the current circuit
codeblock references become empty.

If no {\it filename} is given and one of the bind options is given,
the respective bound codeblock reference in the current circuit is
removed.  Only one of {\vt b} or {\vt be} can be given.

In either case, if {\vt c} is given, the bound codeblocks in the
current circuit are listed, after other operations.  The {\vt a}
option is ignored if no {\it filename} is given.

The bound codeblocks for the current circuit are also listed in the
{\cb listing} command.

Otherwise, when a name is given, the named file/block is acted on.
If no option is given, the add option is assumed.  Added blocks
overwrite existing blocks of the same name.  The options all apply
if given, and the operations are performed in the order
\begin{quote}
{\vt p} (if {\vt a} not given)\\
{\vt d}\\
{\vt a}\\
{\vt p} (if {\vt a} given)\\
{\vt b} or {\vt be}\\
{\vt c}
\end{quote}

When a command is entered in response to a prompt or in a script (or
another codeblock), the blocks are checked first, then the {\WRspice}
internal commands, then scripts, then vectors (for the implicit {\cb
let} in {\it vector} = {\it something\/}) and finally operating system
commands if {\et unixcom} is set.

Thus, once a codeblock has been added, it can be executed by simply
entering its name, as if it were a shell command.  If a name conflicts
with an internal command or script, the codeblock has precedence.

A codeblock can be ``bound'' to the current circuit with the {\vt b}
and {\vt be} options.  If {\vt be}, the block is bound as an ``exec''
codeblock, and if {\vt b} is given, the block is bound as a
``control'' codeblock.  Each circuit has one of each type, which are
by default derived from the {\vt .exec} and {\vt .control} statements
from the circuit file.  Binding an external codeblock overrides the
blocks obtained from the file.  If no {\it filename} was given, the
existing binding is deleted from the current circuit, according to
whether the {\vt b} or {\vt be} was given.  Separate calls are
required to unbind both blocks.

Note:  Bound codeblocks are parameter expanded, named codeblocks
are not.  In a named codeblock, parameters are available through
the {\vt @}{\it parmname} (special vector) syntax.

Operating range and Monte Carlo analysis can make use of ``bound''
codeblocks.  In both types of analysis, the ``controls'' codeblock
execution sets a variable indicating whether the circuit simulated
properly according to user specified criteria.  When a margin analysis
file is input, the lines between {\vt .control} and {\vt .endc} become
the default controls codeblock.  Similarly, the lines between {\vt
.exec} and {\vt .endc} become the default exec codeblock.  A bound
codeblock will always supersede the default codeblock.

%SU-------------------------------------
\subsection{\spcmd{dumpnodes}}
\index{dumpnodes command}

% spCommands.hlp:dumpnodes 092611

\begin{quote}\vt
dumpnodes
\end{quote}

This command prints, on the standard output, a table of the most
recently computed node voltages (and branch currents) for the current
circuit.

%SU-------------------------------------
\subsection{\spcmd{edit}}
\index{edit command}

% spCommands.hlp:edit 093015

The {\cb edit} command allows the text of an input file to be edited.
\begin{quote}\vt
edit [-n][-r] [{\it filename\/}]
\end{quote}
The command will bring up a text editor loaded with the named file. 
If no file name is given, the file associated with the current circuit
will be edited.  If no file is associated with the current circuit,
the current circuit will be printed into a temporary file which is
opened for editing.  If no circuits are present, an empty file is
opened for editing.  Pressing the {\cb Text Editor} button in the {\cb
Edit} menu of the {\cb Tool Control} window is equivalent to giving
the {\cb edit} command without arguments.

It should be noted that one can also provide input to {\WRspice}> from
an arbitrary text editor by ``saving'' the file to the active fifo
file (see \ref{fifo}) that {\WRspice} creates in the user's home
directory.  This is a special file the contains a port into
{\WRspice}, whereby data written to the fifo appear in {\WRspice} as
if sourced from a regular file (if {\WRspice} is busy, the fifo write
will block until {\WRspice} is ready).

The editor used is named by the {\et editor} variable, the {\et
SPICE\_EDITOR} environment variable, or the {\et EDITOR} environment
variable, in that order.  If none of these is set, or the first one
found is set to ``{\vt xeditor}'', the internal editor is used, if
graphics is available.  If graphics is not available and no editor is
specified, {\WRspice} will attempt to use the ``{\vt vi}'' editor. 
The internal editor has the advantage of asynchronous deck sources
with the edit window displayed at all times, through the {\cb Source}
button in the editor's {\cb Options} menu.  The {\cb xeditor} command
is similar to the {\cb edit} command, but will always call the
internal editor.  See \ref{xeditor} for a description of the internal
editor.

If an external editor is used, if graphics is available the default
action is to start the editor in a new {\vt xterm} window.  This can
be suppressed if the {\et noeditwin} variable is set.  This variable
should be set if the external editor creates its own window to avoid
the unneeded {\vt xterm}.  It can also be set for an editor such as
{\vt vi}, in which case the editing will take place in the same window
used to interact with {\WRspice}.

The {\vt -r} and {\vt -n} options are available only when the internal
editor is {\it not} being used, and the editor is a text-mode editor
such as {\vt vi} and {\et noeditwin} is set so that editing takes
place in the console controlling {\WRspice}.  If this is the case,
after quitting the editor, the file will be sourced automatically if
the text was saved.  The {\vt -n} (no source) option prevents this,
and should be given if the editor is used to browse files that are not
SPICE input files.  The {\vt -r} (reuse) option will reuse the
existing circuit for the automatic source, rather than creating a new
one.  This saves memory, but prevents revisiting earlier revisions of
the circuit.  If the internal editor, or any editor that creates its
own window is used, {\WRspice} will pop up the editor and resume
command prompting.  There is no automatic source in this case.

%SU-------------------------------------
\subsection{\spcmd{listing}}
\index{listing command}

% spCommands.hlp:listing 022514

The {\cb listing} command is used to generate a listing of the current
circuit.
\begin{quote}\vt
listing [l[ogical]] [p[hysical]] [d[eck]] [e[xpand]] [n[ocontinue]]
\end{quote}
The command will print a listing of the current circuit to the
standard output.  The arguments control the format of the listing.  A
{\vt logical} listing is one in which comments are removed and
continuation lines are appended to the end of the continued line.  A
{\vt physical} listing is one in which comments and continuation lines
are preserved.  A {\vt deck} listing is a {\vt physical} listing
without line numbers, so as to be acceptable to the circuit parser ---
it recreates the input file verbatim.  The last option, {\vt expand},
is orthogonal to the previous three --- it requests that the circuit
be printed after subcircuit expansion.  Note that only in an expanded
listing are error messages associated with particular lines visible. 
When using {\vt deck} and {\vt expand}, by default long lines are
broken into continuation lines.  If the {\vt nocontinue} option is
also given, this will not be done.  This option is ignored in other
cases.

If no argument is given, {\vt logical} is understood.

%SU-------------------------------------
\subsection{\spcmd{load}}
\index{load command}

% spCommands.hlp:load 060214

The {\cb load} command loads data from the files given.
\begin{quote}\vt
load [{\it filename\/}] [{\vt -p} {\it printfile\/}]
 [{\vt -c}{\it N\/}[{\vt +}[{\it M\/}]] [{\it datafile\/}] [...]
\end{quote}

Several file formats are supported, as is discussed below.

The file data will be converted into internal plot structures
containing vectors available for printing, plotting, and other
manipulation just as if the analysis had been run.  The last plot read
becomes the current plot.  Data files can also be loaded from the {\cb
Load} button in the {\cb Files} menu of the {\cb Tool Control} window. 
A file name given without a path prefix is searched for in the source
path.

The {\cb load} command is internet aware, i.e., if a given filename
has an {\vt http://} or {\vt ftp://} prefix, the file will be
downloaded from the internet and loaded.  The file is transferred as a
temporary file, so if a permanent local copy is desired, the {\cb
write} command should be used to save a file to disk.

ASCII and binary rawfiles, and Common Simulation Data Format (CSDF)
files can be listed without options.  These formats are auto-detected
and the file data will be processed appropriately.  The rawfile format
is the native format used in {\WRspice} and Berkeley SPICE3.  CSDF is
one of the formats used by HSPICE, and post-processing tools such as
Synopsys WaveView.

In HSPICE, ``{\vt .options csdf=1}'' and ``{\vt .options post=csdf}''
will produce CSDF files.  These files can be loaded into {\WRspice}
for display and other purposes with the {\cb load} command.

In {\WRspice} rawfiles or CSDF files can be produced by the {\cb Save
Plot} button in {\cb plot} windows, the {\cb write} and {\cb run}
commands, and may be generated in batch mode.

If no argument is given, {\WRspice} will attempt to load a file with a
default name.  The default name is the value of the {\et rawfile}
variable if set, or the argument to the {\vt -r} command line option
if one was given, or ``{\vt rawspice.raw}''.

If the option flag {\vt -p} appears before a file name, the file that
follows is assumed to be a file produced with the {\WRspice} {\cb
print} command.  This works for the default columnar print format
only.  The format is common to other SPICE programs.  This can be
useful on occasion, but the print format lacks to expressiveness of
the plot data file formats.

The {\vt -c} option will allow parsing of general columnar numerical
data, and is useful for extracting data from output from other
programs, or report text files.  The option has several forms.

\begin{description}
\item{\vt -c}{\it N}\\
{\it N} is an integer greater than 0, representing the number of
numerical columns.  A plot with {\it N} vectors will be created,
with names ``{\vt column\_0}'', ``{\vt column\_1}'', etc.  The
{\vt column\_0} vector will be taken as the scale vector.  The file is
read, amd all lines that start with {\it N} space or comma-separated
numbers will contribute to the vectors.  Any additional text on the
line following the numbers is ignored.  Lines that don't provide
{\it N} numbers are also ignored.

\item{\vt -c}{\it N\/}{\vt +}\\
As above, but lines must provide exactly {\it N} numbers or will
be ignored.  Parsing of a line stops if a token is read that is
not a number, so that any numbers following a non-number in the
line will always be ignored.

\item{-c}{\it N\/}{\vt +}{\it M}\\
This assumes that there are {\it N} columns of numbers in a logical
block, followed by a logical block containing {\it M} columns of
numbers.  We assume that there are {\it N} + {\it M} vectors, and the
lines have been broken to avoid being too long, as is done in the
SPICE printing if the number of columns to be printed would exceed the
page width.  However, it is required that {\it M} be less than {\it
N}, and only one ``wrap'' can be accommodated.  If for some reason the
{\it M} vectors end up being a different length than the {\it N}
vectors, they will be truncated or zero-padded so that all vectors
will have the same length.
\end{description}

When reading columnar or print data, the scale vector is checked for
cyclicity, and the plot dimensions will be set if found.  Only
two-dimensional vectors are produced, higher dimensions can not be
determined.

%SU-------------------------------------
\subsection{\spcmd{print}}
\index{print command}
\label{print}

% spCommands.hlp:print 083122

The {\cb print} command is used to print vector data on-screen or
to a file using output redirection.
\begin{quote}\vt
print [/{\it format\/}] [col | line] {\it expr\/} [...]
\end{quote}
The command prints the values of the given expressions to the standard
output.

If command line input can be recognized as an expression list, the
print command will be invoked implicitly.  In this case, the line
cannot contain directives or a format string, This saves a bit of
typing when using the {\WRspice} command line as a calculator, for
example.

The default is to use exponential format for all values, with the
number of digits given by the {\et numdgt} variable.  However this,
and some other presentation attributes, can be specified in the format
string, if given.  If given, the format string must be the first
argument, and the string must start with a '/' (forward slash)
character.  The syntax is further described below.

All vectors listed will be printed in the same format, except for the
scale vector, which is printed by default in the {\vt col} mode, which
is printed with the default notation.

If {\vt line} is specified, the value of each expression is printed on
one line (or more if needed).  If all expressions have a length of 1,
the default style is {\vt line}, otherwise {\vt col} is the default.

If {\vt col} is specified, the values are printed in columns.  This is
the default if any of the vectors are multi-valued.  This mode makes
use of the {\vt height} and {\vt width} variables to define the page
size.  By default, per-page formatting is applied, with page eject
characters between pages.  With column formatting, by default the
scale vector ({\et time}, {\et frequency}) will be shown in the first
column.  If there are more vectors that can be accommodated with the
page width, the print will be repeated, with a new set of columns
(other than the scale) until all variables have been printed.

If the expression is ``{\vt all}'', all of the vectors in the current
plot are printed.  If no arguments are given, the arguments to the
last given {\cb print} command are used.  If only the format argument
is given, the arguments from the last given {\cb print} command other
than the format are used, with the new format.

If the argument list contains a token consisting of a single period
(``.''), this is replaced with the vector list found in the first {\vt
.print} line from the input file with the same analysis type as the
current plot.  For example, if the input file contains
\begin{quote}\vt
    .tran .1u 10u\\
    .print tran v(1) v(2)\\
\end{quote}
then one can type ``{\vt run}'' followed by ``{\vt print .}'' to print
{\vt v(1)} and {\vt v(2)}.

The related syntax {\vt .@}{\it N} is also recognized, where {\it N}
is an integer representing the {\it N\/}'th matching {\vt .print} line. 
The count is 1-based, but {\it N\/}=0 is equivalent to {\it N\/}=1. 
The token is effectively replaced by the vector list from the
specified {\vt .print} line found in the circuit deck.

The print command is responsive to the following variables.

\begin{description}
\item{{\et width}, {\et height}}\\
These option variables set the page size (in characters and lines)
assumed for the output when directed to a flie or device.  If not set,
a standard A-size page is assumed.  When printing on-screen, the
actual screen or window size will be used.

\item{\et nopage}\\
This boolean option will suppress page breaks between pages when set. 
This is always true when printing to a screen.  Page breaks consist of
a form-feed character, which may be followed by a two-line page
header.
\end{description}

The following variables are all booleans, and apply only to column
mode of the {\cb print} command.

\begin{description}
\item{\et printautowidth}\\
When set, the window width or the setting of the {\et width} variable
is ignored, and a line width sufficient to include columns for all
variables being printed is used, if possible.  There is a hard limit
of 2048 characters in the lines.  Variables that don't fit are printed
subsequently, as in the case with {\et printautowidth} not set.

\item{\et printnoheader}\\
When set, don't print the top header, which consists of the plot
title, circuit name, data, and a line of ``{\vt -}'' characters (three
lines).  This is normally printed at the top of the first page of
output.

\item{\et printnoindex}\\
When set, don't print the vector indices, which are otherwise printed
in the leftmost column of each page.

\item{\et printnopageheader}\\
When set, don't print the page header.  The page header, which
consists of the variable names at the top of each column and a line of
``{\vt -}'' characters, is otherwise printed at the top of each page
of output.

\item{\et printnoscale}\\
When set, don't print the scale vector in the leftmost data column. 
This is otherwise done for each set of variables printed.  The Spice3
{\et noprintscale} variable is an alias, but deprecated.
\end{description}

The syntax of the format string to the {\cb print} command allows
overriding the states of the switches listed above while printing. 
The format string, if used, must be the first argument given to the
{\cb print} command, and must begin with a `{\vt /}' (forward slash)
character.  It contains no space, and is a sequence of the characters
and forms shown below, all of which are optional.

\begin{description}
\item{\it integer}\\
The {\it integer} is the number of figures to the right of the decimal
point to print.  If not given, the value of the {\et numdgt} variable
is used if set, otherwise a default of 6 is used.

\item{\vt f}\\
If `{\vt f}' is found in the string, data values will be printed using
a fixed-point format, rather than the default exponential format.
\end{description}

The remaining options apply/unapply the switches, whose defaults are
set by the {\et print...} variables described above.  The format
string always overrides the variables.

\begin{description}
\item{\vt -}\\
Negate the effect of options that follow.
\item{\vt +}\\
Don't negate effect of options that follow.  This is redundant
unless it follows `{\vt -}'.
\item{\vt a}\\
Take {\et printautowidth} as if set, or not set if negated.
\item{\vt b}\\
Take {\et nopage} as if set, or not set if negated.
\item{\vt h}\\
Take {\et printnoheader} as if set, or not set if negated.
\item{\vt i}\\
Take {\et printnoindex} as if set, or not set if negated.
\item{\vt p}\\
Take {\et printnopageheader} as if set, or not set if negated.
\item{\vt s}\\
Take {\et printnoscale} as if set, or not set if negated.
\item{\vt n}\\
Alias for ``{\vt abhips}''.
\end{description}

Examples
\begin{quote}\vt
print /3f+ahi-ps ...
\end{quote}
Print using a fixed three decimal place format, and as if {\et
printautowidth}, {\et printnoheader}, and {\et printnoindex} were set,
and {\et printnopageheader} and {\et printnoscale} were unset.
\begin{quote}\vt
print /n ...
\end{quote}
Print the vectors listed, and nothing but the vectors listed.  This is
useful when one wants to feed a simple list of numbers to another
application.
\begin{quote}\vt
print /n-s ...
\end{quote}
As above, but print the scale in the first column.  The `{\vt -}' can
be used as shown to undo individual implicit settings from `{\vt n}'.
\begin{quote}\vt
print /3f v(5)
\end{quote}
This prints v(5) to three decimal places in fixed-point notation.
\begin{quote}\vt
print /4f v(2) v(3) v(4) > myfile
\end{quote}
This prints the vectors to four decimal places in the file "myfile".
\begin{quote}\vt
print 2*v(2)+v(3) v(4)-v(1)
\end{quote}
This prints the computed quantities using the default format.

%SU-------------------------------------
\subsection{\spcmd{printf}}
\index{printf command}

% spCommands.hlp:printf 011721

The {\cb printf} command is equivalent to the {\cb print} command,
however output goes to the logging file for use in operating range and
Monte Carlo analysis.
\begin{quote}\vt
print [/{\it format\/}] [col | line] {\it expr\/} [...]
\end{quote}
This is used in codeblocks evaluated while those processes are active.

%SU-------------------------------------
\subsection{\spcmd{return}}
\index{return command}

% spCommands.hlp:return 030519

This will cause the currently executing script or codeblock to
terminate immediately and return to the caller.
\begin{quote}\vt
return [{\it expression\/}]
\end{quote}
If an expression follows, it will be evaluated and the global return
value will be set to the result.  The global return value is an
internal global variable that can be set and queried from any script
or the comand prompt as the special variable name {\vt \$?}.  The {\cb
retval} command is used to set the global return value without
immediately returning.


%SU-------------------------------------
\subsection{\spcmd{sced}}
\index{sced command}

% spCommands.hlp:sced 012109

The {\cb sced} command brings up the {\Xic} graphical editor (if
available) in electrical mode.
\begin{quote}\vt
sced [{\it filename} ...]
\end{quote}
This allows schematic capture, with most of the {\WRspice}
functionality directly available through the {\Xic} interface.  If the
{\Xic} graphical editor is not available for execution, this command
will exit with a message indicating that {\Xic} is not available. 
Otherwise, the {\cb sced} command will bring up the schematic capture
front-end with file {\it filename\/}, which must be an {\Xic} input
file ({\it not} a standard {\WRspice} circuit file!).  If the current
circuit originated from {\Xic}, that file will be loaded into {\Xic}
if no {\it filename} is given. 

When {\it Xic} saves a native-mode top-level cell containing a
schematic, the circuit SPICE listing is appended to the file. 
{\WRspice} is smart enough to ignore the geometric information in
these files and read only the circuit listing.

{\Xic} can also be started from the {\cb Xic} button in the {\cb Edit}
menu of the {\cb Tool Control} window.

%SU-------------------------------------
\subsection{\spcmd{source}}
\index{source command}

% spCommands.hlp:source 051516

The {\cb source} command is used to load circuit files and command
scripts.
\begin{quote}\vt
source [{\vt -r}] [{\vt -n}] [{\vt -c}] {\it file} [{\it file} ...]
\end{quote}
If more than one file name is given, the files will be concatenated
into a temporary file, which is read.  The command will read and
process circuit descriptions and command text from the file(s).  If
{\vt .newjob} lines are found within the files, the input will be
partitioned into two or more circuit decks, divided by the {\vt
.newjob} lines.  Each circuit deck is processed independently and in
sequence.

If a file does not have a path prefix, it is searched for in the
search path specified by the {\et sourcepath} variable.  If not in the
search path or current directory, a full path name must be given.

The {\cb source} command is internet aware, i.e., if a given filename
has an ``{\vt http://}'' or ``{\vt ftp://}'' prefix, the file will be
downloaded from the internet and sourced.  The file is transferred as
a temporary file, so if a permanent local copy is desired, the {\cb
edit} or {\cb listing} commands should be used to save the circuit
description to disk. 

When an input file or set of files is ``sourced'', the following steps
are performed for each circuit deck found.  The logic is rather
complex, and the following steps illustrate but perhaps oversimplify
the process.  In particular, the subcircuit/model cache substitution
is omitted here.

\begin{enumerate}
\item{The input is read into a ``deck'' in memory.  Line continuation
is applied.}

\item{In interactive mode, the title line from the circuit is printed
on-screen, unless the {\et noprtitle} variable is set, in which case
this printing is suppressed.  The variable can be set by checking the
box in the {\cb source} page of the {\cb Command Options} tool from
the {\cb Tools} menu.}

\item{The deck is scanned for {\vt .param} lines which are outside of
subcircuit definitions.  These are shell expanded, and used to
evaluate {\vt .if}, {\vt .elif} and similar lines.  Lines that are not
in scope are ignored.}

\item{Files referenced from {\vt .include} and {\vt .lib} lines are
resolved and read.  At each level, parameters are scanned again, so
that {\vt .if}, etc.  lines do the right thing at each level.}

\item{Verilog blocks, {\vt .exec} blocks, and {\vt .control} blocks
are moved out of the main deck into separate storage.}

\item{The {\vt .exec} lines, if any, are executed by the shell.}

\item{The {\vt .options} lines are extracted, shell expanded, and
evaluated.  During evaluation, the shell receives the assignment
definitions.}

\item{The remaining lines in the deck are shell expanded.}

\item{Subcircuit expansion is performed.  This takes care of parameter
expansion within subcircuit definition blocks.}

\item{The circuit (if any) is parsed, and added to the internal
circuits list.}

\item{The {\vt .control} lines, if any, and executed by the shell.}
\end{enumerate}

After a {\cb source}, the current circuit will be the last circuit
parsed.

There are three option flags available, which modify the behavior
outlined above.  These can be grouped or given as individual tokens,
following a `--' character.  Note that if a file name starts with
`--', it must be quoted with double-quote marks.  The options are
applied before files are read.

\begin{description}
\item{\vt r}\\
Reuse the current circuit.  The current circuit is destroyed before
the new circuit is created, which becomes the current circuit.  This
option is ignored if {\vt -n} is also given.

\item{\vt n}\\
Ignore any circuit definition lines in input.  Executable lines will
still be executed, but no new circuit will be produced.

\item{\vt c}\\
Ignore any {\vt .control} commands.  However, {\vt .exec} lines will
still be executed.

\item{{\vt n} and {\vt c}}\\
If both of the {\vt n} and {\vt c} options are given, all lines of
input except for the first ``title'' line are taken to be executable,
and are executed, as if for a startup file.
\end{description}

\subsubsection{Implicit Source}
\index{source, implicit}
\index{implicit source}

In many cases, the ``{\vt source}'' is optional.  If the name of an
existing file is given as a command, the {\cb source} is applied
implicitly, provided that the file name does not clash with a
{\WRspice} command.

\subsubsection{Input Format Notes}

The first line in the input file (after concatenation of multiple
input files), and the first line following a {\vt /newjob} line, is
considered a title line and is not parsed but kept as the name of the
circuit.  The exceptions to this rule are old format margin analysis
input files and {\Xic} files.

Command lines must be surrounded by the lines {\vt .exec} or {\vt
.control} and {\vt .endc} in the file, or prefixed by ``{\vt *@}'' or
``{\vt *\#}'' in order to be recognized as commands, except in startup
files where all lines but the title line are taken as executable. 
Commands found in {\vt .exec} blocks or {\vt *@} lines are executed
before the circuit is parsed, thus can set variables used in the
circuit.  Commands found in {\vt .control} blocks or {\vt *\#} lines
are executed after the circuit is parsed, so a control line of ``{\vt
ac ...}'' will work the same as the corresponding {\vt .ac} line, for
example.  Use of the ``comment'' control prefixes {\vt *@} and {\vt *\#}
makes it possible to embed commands in {\WRspice} input files
that will be ignored by earlier versions of SPICE.

Shell variables found in the circuit deck (but not in the
commands text) are evaluated during the source.  The {\cb reset}
command can be used to update these variables if they are later
changed by the shell after sourcing. 

%SU-------------------------------------
\subsection{\spcmd{write}}
\index{write command}
\label{writecmd}

% spCommands.hlp:write 022117

The {\cb write} command is used to save simulation data to a file.
\begin{quote}\vt
write [{\it file} [{\it expr} ...]]
\end{quote}
There are two data formats universally available, the ``rawfile''
format native to {\WRspice} and other simulators based on Berkeley
SPICE3, and the Common Simulation Data Format (CSDF).  The CSDF is one
of the formats generated by HSPICE, and is compatible with
post-processors such as Synopsys WaveView.

\index{PSF file format}
In the Red Hat 6 and 7 releases, a third output format is available: 
the Cadence PSF format.  This support is provided through third-party
libraries which support only the indicated operating systems.  Unlike
the other formats, PSF output can not be read back into {\WRspice}. 
This format is used by the waveform viewer component of the Cadence
Analog Design Environment (ADE) product.

PSF output consists of files created in a specified directory. 
Presently, output is available only for AC, DC, transient, and
operating-point analysis.  Only simple analysis is supported, no
chained DC or looping.

To specify PSF output, one gives a ``filename'', for example to the
{\cb write} command or the {\vt rawfile} variable, in the form

\begin{quote}
{\vt psf}[{\vt @}{\it path\/}]
\end{quote}

If this is simply ``{\vt psf}'', output goes to a directory named {\vt
psf} in the current directory.  Otherwise, the {\vt psf} keyword can
be followed by a `{\vt @}' character and a path to a directory, with
no white space around the {\vt @}.  Output will go to the indicated
directory.  In either case, the directory will be created if it
doesn't exist, but in the second case and parent directories must
currently exist, they won't be created.

\index{CSDF file format}
If the file name is given an extension from among those listed
below, CSDF output will be generated.  Otherwise, rawfile format
will be used.
\begin{quote}
{\vt .csdf}\\
{\vt .tr}{\it N}\\
{\vt .ac}{\it N}\\
{\vt .sw}{\it N}
\end{quote}
The {\it N} is an integer, and {\vt tr}, {\vt ac}, and {\vt sw}
correspond to transient, ac, and dc sweep results, respectively.  This
is the same convention as used by HSPICE when generating files for
post-processing.
 
If no {\it expr} is given, then all vectors in the current plot will
be written, the same as giving the word ``{\vt all}'' as an {\it
expr}.  If, in addition, no file name is given, a default name will be
used.  The default name is the value of the {\et rawfile} variable if
set, or the argument to the {\vt -r} command line option if one was
given, or ``{\vt rawspice.raw}''.
 
The command writes out the {\it exprs} to the {\it file}.  First,
vectors are grouped together by plots, and written out as such.  For
example, if the expression list contained three vectors from one plot
and two from another, then two plots will be written, one with three
vectors and one with two.  Additionally, if the scale for a vector
isn't present, it is automatically written out as well.

The default rawfile format is ASCII, but this may be changed with the
{\et filetype} variable or the {\et SPICE\_ASCIIRAWFILE} environment
variable.

If the {\et appendwrite} variable is set, the data will be appended to
an existing file.

Files that have been appended to, or have multiple plots, are
concatenations of data for a single plot.  This is expected and
perfectly legitimate for rawfiles, and for CSDF files used only by
{\WRspice}, but concatenated CSDF files may not be portable to other
applications.


%SU-------------------------------------
\subsection{\spcmd{xeditor}}
\index{xeditor command}
\label{xeditor}

% spCommands.hlp:xeditorcom 012209

The {\cb xeditor} command invokes a text editing window for editing
circuit and other text files.  It is available only when graphics is
enabled.
\begin{quote}\vt
xeditor [{\it file\/}]
\end{quote}
This is similar to the {\cb edit} command, however the internal editor
is always used.  The {\et editor} variable and the environment
variables used by the {\cb edit} command are ignored by the {\cb
xeditor} command.

The {\cb xeditor} command brings up a general-purpose text editor
window.  The same pop-up editor is invoked in read-only mode by the
{\cb Notes} button of the {\cb Help} menu in the {\cb Tool Control}
window for use as a file viewer.  In that mode, commands which modify
the text are not available.

See \ref{textedit} for more information about the text editor.


%S-----------------------------------------------------------------------------
\section{Simulation Control Commands}
\label{simcmds}

% spCommands.hlp:simcmds 032420

The commands described in this section initiate, control, and monitor
{\WRspice} simulations.  One can monitor the progress of a run in two
ways, in addition to the percentage complete that is printed in the
{\cb Tool Control} window.  First, the {\cb iplot} command can be used
to plot one or more variables as the simulation is progressing.  To
plot {\vt v(1)}, for example, one would type, before the run is
started, ``{\vt iplot v(1)}''.  During the run, {\vt v(1)} will be
plotted on screen, with the plot rescaled as necessary.  Second, one
can print variables.  For example, the {\cb trace} command can be
used, by typing ``{\vt trace time}'' before the run starts, to cause
the time value to be printed at each output point during transient
analysis.

\index{runops}
\label{runop}
The {\cb iplot} and {\cb trace} commands are examples of what are
called ``runops''.  Other runops include the {\cb stop} and {\cb
measure} commands.  A runop remains in effect until deleted with the
{\cb delete} command, and the runops in effect can be listed with the
{\cb status} command.  The runops can also be listed, deleted, or made
inactive with the {\cb Trace} tool from the {\cb Tools} menu in the
{\cb Tool Control} window.  All runops are available as commands,
which apply to any circuit while in force.  Some runops can be
specified from within the {\WRspice} input file, in which case the
runop applies when simulating that file only.  The table below lists
the runops that are presently available.

\begin{tabular}{|l|l|}\hline
\multicolumn{2}{|c|}{Runops}\\ \hline
\cb Command & \cb Input Keyword\\ \hline\hline
\cb save & \vt .save\\ \hline
\cb trace & \vt -\\ \hline
\cb iplot & \vt -\\ \hline
\cb measure & \vt .measure\\ \hline
\cb stop & \vt .stop\\ \hline
\end{tabular}

The run can be paused at any time by typing {\kb Ctrl-C} in the
controlling text window.  The run can be resumed with the {\cb resume}
command, or reset with the {\cb reset} command.

It is possible to transparently execute simulations on a remote
machine while in {\WRspice}, if the remote machine has a {\vt
wrspiced} daemon running.  It is also possible to run simulations
asynchronously on the present machine.  These jobs are not available
for use with the {\cb iplot} command, however.  The {\cb jobs} command
can be used to monitor their status.

Many of these commands operate on the ``current circuit'' which by
default is the last circuit entered into {\WRspice} explicitly with
the {\cb source} command, or implicitly by typing the file name.  The
{\cb setcirc} command can be used to change the current circuit.  The
{\cb Circuits} button in the {\cb Tools} menu also allows setting of
the current circuit.

When a circuit file is read, any references to shell variables are
expanded to their definitions.  Shell variables are referenced as {\vt
\$}{\it name}, where {\it name} has been set with the {\cb set}
command or in the {\vt .options} line.  This expansion occurs only
when the file is sourced, or the {\cb reset} command is given, so that
if the variable is changed, the circuit must be sourced or reset to
make the change evident in the circuit.  If a variable is set in the
shell and also in the {\vt .options} line, the value from the shell
is used.

\begin{tabular}{|l|l|}\hline
\multicolumn{2}{|c|}{Simulation Commands}\\ \hline
\cb ac & Perform ac analysis\\ \hline
\cb alter & Change circuit parameter\\ \hline
\cb alterf & Dump alter list to Monte Carlo output file\\ \hline
\cb aspice & Initiate asynchronous run\\ \hline
\cb cache & Manipulate subcircuit/model cache\\ \hline
\cb check & Initiate range analysis\\ \hline
\cb dc & Initiate dc analysis\\ \hline
\cb delete & Delete watchpoint\\ \hline
\cb destroy & Delete plot\\ \hline
\cb devcnt & Print device counts\\ \hline
\cb devload & Load device module\\ \hline
\cb devls & List available devices\\ \hline
\cb devmod & Change device model levels\\ \hline
\cb disto & Initiate distortion analysis\\ \hline
\cb dump & Print circuit matrix\\ \hline
\cb findlower & Find lower edge of operating range\\ \hline
\cb findrange & Find edges of operating range\\ \hline
\cb findupper & Find upper edge of operating range\\ \hline
\cb free & Delete circuits and/or plots\\ \hline
\cb jobs & Check asynchronous jobs\\ \hline
\cb loop & Alias for sweep command\\ \hline
\cb mctrial & Run a Monte Carlo trial\\ \hline
\cb measure & Set up a measurement\\ \hline
\cb noise & Initiate noise analysis\\ \hline
\cb op & Compute operating point\\ \hline
\cb pz & Initiate pole-zero analysis\\ \hline
\cb reset & Reset simulator\\ \hline
\cb resume & Resume run in progress\\ \hline
\cb rhost & Identify remote SPICE host\\ \hline
\cb rspice & Initiate remote SPICE run\\ \hline
\cb run & Initiate simulation\\ \hline
\cb save & List vectors to save during run\\ \hline
\cb sens & Initiate sensitivity analysis\\ \hline
\cb setcirc & Set current circuit\\ \hline
\cb show & List parameters\\ \hline
\cb state & Print circuit state\\ \hline
\cb status & Print trace status\\ \hline
\cb step & Advance simulator\\ \hline
\cb stop & Specify stop condition\\ \hline
\cb sweep & Perform analysis over parameter range\\ \hline
\cb tf & Initiate transfer function analysis\\ \hline
\cb trace & Set trace\\ \hline
\cb tran & Initiate transient analysis\\ \hline
\cb vastep & Advance Verilog simulator\\ \hline
\cb where & Print nonconvergence information\\ \hline
\end{tabular}

%SU-------------------------------------
\subsection{\spcmd{ac}}
\index{ac command}

% spCommands.hlp:ac 011909

The {\cb ac} command initiates an ac analysis of the current circuit.
\begin{quote}\vt
ac {\it ac\_args} [dc {\it dc\_args\/}]
\end{quote}
The {\it ac\_args} are the same as appear in a {\vt .ac} line (see
\ref{acline}).  If a dc sweep specification follows, the ac analysis
is performed at each point of the dc analysis (see \ref{dcline}).

%SU-------------------------------------
\subsection{\spcmd{alter}}
\index{alter command}

% spCommands.hlp:alter 041711

The {\cb alter} command allows circuit parameters to be changed for
the next simulation run.
\begin{quote}\vt
alter [{\it device\_list\/} , {\it param} [{\vt =}] {\it value\/}
  [{\it param} [{\vt =}]  {\it value\/} ... ]]
\end{quote}
The parameters will revert to original values on subsequent runs,
unless the {\cb alter} command is reissued.

If given without arguments, a list of previously entered alterations
of the current circuit, to be applied in the next analysis run, is
printed.  List entries may have come from previously given {\cb alter}
commands, or from assignments to the {\vt @}{\it device\/}{\vt [}{\it
param\/}{\vt ]} special vectors.

The {\it device\_list} is a list of one or more device or model names
found in the circuit.  The names are separated by white space, and the
list is terminated with a comma.  Following the comma is one or more
name/value pairs, optionally an equal sign can appear between the two
tokens.  The name is a device or model keyword, which should be
applicable to all of the names listed in the {\it device\_list}.  Note
that this probably means that the {\it device\_list} can contain
device names or models, but not both.  The device and model keywords
can be obtained from the {\cb show} command.

The {\cb alter} command can be issued multiple times, to set
parameters of devices or models which can't be intermixed according to
the rule above.

The {\it device\_list} can contain ``globbing'' (wild-card)
characters with similar behavior to globbing (global substitution) in
the {\WRspice} shell.  Briefly, `{\vt ?}' matches any character, '{\vt
*}' matches any set of characters or no characters, ``{\vt [abc]}''
matches the characters `{\vt a}', `{\vt b}', and `{\vt c}', and ``{\vt
a}\{{\vt bc,de}\}'' matches ``{\vt abc}'' and ``{\vt ade}''.

When the next simulation run of the current circuit is started, the
given parameters will be substituted.  Thus, the {\cb show} command,
if given before the next run, will not show the altered values.  The
internal set of altered values will be destroyed after the
substitutions.

Examples:
\begin{quote}
{\vt alter R2, resistance=50}\\
{\vt alter c\{1,2,3\}, capacitance 105p}
\end{quote}

%SU-------------------------------------
\subsection{\spcmd{alterf}}
\index{alterf command}

% spCommands.hlp:alterf 011721

This will dump the alter list to the output file, for use in Monte
Carlo analysis.
\begin{quote}\vt
alterf
\end{quote}
In this approach, the {\cb alter} command, or equivalently forms like
``{\vt let @}{\it device\/}{\vt [}{\it param\/}{\vt ]} {\vt =} {\it
trial\_value}'' are used to set trial values in the {\vt exec} block. 
Once set, this can be called to dump the values into the output file.

In 4.3.13, these are added to the logging file automatically, so
this command may be obsolete.

%SU-------------------------------------
\subsection{\spcmd{aspice}}
\index{aspice command}

% spCommands.hlp:aspice 011909

The {\cb aspice} command allows simulation jobs to be run in the
background on the present machine.
\begin{quote}\vt
aspice {\it infile} [{\it outfile\/}]
\end{quote}
This command will run a simulation asynchronously with
{\it infile} as an input circuit.  If {\it outfile} is given, the
output is saved in this file, otherwise a temporary file is used. 
After this command is issued, the job is started in the background,
and one may continue using {\WRspice} interactively.  When the job is
finished, the rawfile is loaded and becomes the current plot, and any
output generated is printed.  Specifically, {\WRspice} forks off a new
process with the standard input set to {\it infile}, and which writes
the standard output to {\it outfile}.  The forked program is expected
to create a rawfile with name given by a {\vt -r} command line option. 
The forked command is effectively ``{\vt {\it program} -S -r {\it
rawfile} <{\it infile} >{\it outfile}}, where {\it program} is the
{\et spicepath} variable (which defaults to calling {\WRspice}), {\it
rawfile} is a temporary file name, and {\it outfile} is the file
given, or a temporary file name.  Although the {\cb aspice} command is
designed for use with {\WRspice}, it may be used with other simulators
capable of emulating the {\WRspice} server mode protocol.  One may
specify the pathname of the program to be run with the {\et spicepath}
variable, or by setting an environment variable.

%SU-------------------------------------
\subsection{\spcmd{cache}}
\index{cache command}

% spCommands.hlp:cache 011909

This function provides a control interface to the subcircuit/model cache.
\begin{quote}\vt
cache [{\it keyword\/}] [{\it tagname\/}]
\end{quote}

The subcircuit/model cache contains representations of blocks of input
lines that were enclosed in {\vt .cache} and {\vt .endcache} lines. 
These representations are used instead of the actual lines of input,
reducing setup time.

The command can have the following forms, the first argument is a
keyword (or letter).  additional arguments are tag names (the names
that follow ``{\vt .cache}'' in SPICE input).

\begin{description}
\item{\vt cache h}[{\vt elp}]\\
Print command usage information.
\item{\vt cache l}[{\vt ist}]\\
Print a list of the tag names currently in the cache.  The
{\cb cache} command with no arguments does the same thing.
\item{\vt cache d}[{\vt ump}] [{\it tagname}...]\\
This will dump the lines saved in the cache, for each
{\it tagname} given, or for all names if no {\it tagname} is
given.  Presently, {\vt .param} lines are listed as comments; 
the actual parameters are in an internal representation and not
explicitly listed.
\item{\vt cache r}[{\vt emove}] {\it tagname} [{\it tagname} ...]\\
This will remove the cached data associated with each
{\it tagname} given.  The given names will no longer be in the cache.
\item{\vt cache c}[{\vt lear}]\\
This will clear all data from the cache.
\end{description}

%SU-------------------------------------
\subsection{\spcmd{check}}
\index{check command}
\label{checkcom}

% spCommands.hlp:check 032520

The {\cb check} command is used to initiate margin analysis.  Margin
analysis can consist of either a swept operating range analysis, or
a Monte Carlo analysis.
\begin{quote}
{\vt check} [{\vt -a}] [{\vt -b}] [{\vt -c}] [{\vt -m}] [{\vt -r}]
  [{\vt -f}] [{\vt -s}] [{\vt -k}] [{\vt -h}] [{\vt -v}]
  [[{\it pstr1\/}] {\it val1} {\it del1} {\it stp1\/}]
  [[{\it pstr2\/}] {\it val2} {\it del2} {\it stp2\/}]
  [{\it analysis\/}]
\end{quote}
See Chapter \ref{oprange} for a full description of operating range
and Monte Carlo analysis.  The current circuit is evaluated, and must
have an associated block of control statements which contain the
pass/fail script.  A second associated block of executable statements
contains initialization commands.  These blocks can be provided in the
circuit file, or be previously defined codeblocks bound to the
circuit.  Codeblocks are executable data structures described in
\ref{codeblock}.  Setting up the file in one of the formats described
described in Chapter \ref{oprange} will ensure that these blocks are
created and bound transparently, however it is possible to do this by
hand.

The option characters can be grouped following a single ``{\vt -}'',
or entered separately.

\begin{description}
\item{\vt -a}\\
If the {\vt -a} flag is given, operating range analysis is performed
at every point (all points mode).  Otherwise, the analysis attempts to
limit computation by identifying the contour containing the points of
operation.  This algorithm can be confused by operating ranges with
strange shapes, or which possess islands of fail points.  If the input
file contains a {\vt .checkall} line, then the {\vt -a} flag to the
{\cb check} command is redundant, all points will be checked in this
case.

\item{\vt -b}\\
If this is given, the analysis will be paused after setup and the {\cb
check} command will return.  This is the start for atomic Monte Carlo
(see \ref{atomic}); a script can call the {\cb mctrial} command
numerous times at this point, then ``{\vt check -c}'' to clean up and
end the analysis.

\item{\vt -c}\\
The {\vt -c} (clear) option will clear any margin analysis in progress
if the analysis has been paused, for example by pressing {\kb Ctrl-C},
or if in an atomic Monte Carlo script.  Return is immediate whether or
not there is a present analysis to clear.  Unlike in release 4.3.8 and
earlier, no new analysis is started, and other options are ignored.

A paused margin analysis is resumed if the {\cb check} command is
given which does not have the {\vt -c} option set, and any arguments
given in this case are ignored.  The {\cb resume} command will also
restart a paused margin analysis.

\item{\vt -m}\\
If the {\vt -m} option is given, Monte Carlo analysis is performed,
rather than operating range analysis.  This is the default if a {\vt
.monte} line appeared in the file; the {\vt -m} option is only
required if there is no {\vt .monte} line.  The {\vt -a} option is
ignored if {\vt -m} is given, as is {\vt .checkall}.  Monte Carlo
analysis files differ from operating range files only in the header
(or header codeblock).  During Monte Carlo analysis, the header block
is executed before every simulation so that variables can be updated. 
In operating range analysis variables are initialized by the header
block only once, at the start of analysis.

\item{\vt -r}\\
If the {\vt -r} (remote) option is given, remote servers will be
assigned simulation runs, allowing parallelism to increase analysis
speed.  The remote servers must have been specified through the {\cb
rhost} command, and each must have a {\vt wrspiced} server running. 
More information on remote asynchronous runs can ge found in
\ref{rhost} and \ref{rspice}.
\end{description}

Ordinarily, during operating range and Monte Carlo analysis, only the
current data point is retained.  The amount of data retained can be
altered with the {\vt -f}, {\vt -s}, and {\vt -k} options.  However,
if a {\vt .measure} line appears in the circuit deck, or the {\cb
iplot} runop is active, data will be retained internally so that the
{\vt .measure} or {\cb iplot} is operational.

\begin{description}
\item{\vt -f}\\
The {\vt -f} option will cause the data for the current trial to be
retained.  This is implied if any {\vt .measure} lines are present, or
if an {\cb iplot} is active.  The data are overwritten for each new
trial.  The data for the last trial are available after the analysis
is complete, or can be accessed for intermediate trials if the
analysis is paused.

\item{\vt -s}\\
The {\vt -s} option also causes retention of the data for the current
trial, but in addition will dump the data to a family of rawfiles,
similar to the {\vt segment} keyword of the {\vt .tran} line (though
this works with other than transient analysis).  The default file name
is the name of the range analysis output file, suffixed with ``{\vt
.s}{\it NN\/}'', where {\it NN} is 00, 01, etc.  Each trial generates
a new suffix in sequence.

\item{\vt -k}\\
With the {\vt -k} option, all data are retained, in a
multi-dimensional plot.  Note that this can be huge, so use of the
{\et maxdata} variable and {\vt .save} lines may be necessary.  One
can see the variations by plotting some or all of the dimensions of
the output.  Recall forms like {\vt v(1)[{\it N\/}]} refer to the {\it
N+1\/}'th trial, and {\vt v(1)[{\it N,M\/}]} includes the data for the
{\it N+1\/}'th to the {\it M+1\/}'th trials.  The {\et mplot} command
has a facility for displaying trial data in a simplified manner.

\item{\vt -h}\\
Finally, the {\vt -h} (help) option will simply print a brief summary
of the options to the {\cb check} command.

\item{\vt -v}\\
If {\vt -v} (verbose) is given, results and other messages are printed
on-screen as the analysis is performed, otherwise the analysis is
silent, except for any printing statements executed in the associated
command scripts.  The {\cb mplot} command can be used to follow
progress graphically.
\end{description}

If an {\cb iplot} is active, {\vt -f} (current trial data retention)
is implied.  The data will be plotted for each trial in the same
{\cb iplot}, erasing after each trial is complete.  If {\vt -k} is
given, all data will be plotted, without erasure.  Note that an
iplot doubles internal memory requirements.

The command line may include one or two range specifications.  In
operating range analysis, each specification consists of an optional
parameter specification string, followed by three numbers.  These
numbers will augment or override the {\vt checkVAL1}, {\vt checkDEL1},
{\vt checkSTP1}, {\vt checkVAL2}, {\vt checkDEL2}, and {\vt checkSTP2}
vectors that may be in effect.  The numbers are parsed in the order
shown, and all are actually optional.  A non-numeric token will
terminate a block, and missing values must be set from the vectors.

In Monte Carlo analysis, each block can contain only a single number,
which will override the {\vt checkSTP1} and {\vt checkSTP2} values (if
any), in that order.  These values are used to define how many Monte
Carlo trials to perform.

The optional {\it pstr1} and {\it pstr2} strings take the same format
and significance as in the {\cb sweep} command.  See the description
of that command for a description of the format.  If a parameter
specifier is given, the specified device parameters will be altered
directly, and the variables and vectors normally used to pass values
will {\bf not} be set.  This applies only to operating range analysis,
and the explicit parameter strings can only be applied from the {\cb
check} command line and not from the file.  If the analysis is two
dimensional, then both dimensions must have a parameter specification,
or neither dimension can have a parameter specification; the two
mechanisms can not be mixed.

The {\it analysis} to be performed is given, otherwise it is found in
the circuit deck.  In interactive mode, if no analysis is specified,
the user will be prompted for an analysis string.

During operating range analysis, a file is usually created and placed
in the current directory for output.  This file is named with the base
name of the input file, with an extension {\vt .d}{\it NN}, where {\it
NN} is replaced with {\vt 00}, {\vt 01}, etc.~ --- the first case
where the filename is unique.  If for some reason the input file name
is unknown, the basename ``check'' will be used.  Similarly, in Monte
Carlo analysis, a file named {\it basename\/}.{\vt m}{\it NN} is
generated.  In either case, the shell variable {\et mplot\_cur} is set
to the current output file name.  These files can be plotted on-screen
with the ``{\vt mplot [{\it filename\/}]}'' command.

The results from operating range/Monte Carlo analysis are hidden away
in the resulting plot structure.  The plot can be displayed by
entering ``{\vt mplot} {\it vec}'' where {\it vec} is any vector in
the plot.

When a {\vt .measure} is included in an iterative analysis, data are
saved as follows.  Before each iteration, the previous result vector
and its scale are saved to the end of a ``history'' vector and scale,
and are then deleted.  The result vector and scale are recreated when
the measurement is completed during the iteration.  Thus, at the end
of the analysis, for a measurement named ``example'', one would have
the following vectors:

\begin{tabular}{ll}
\et  example         & the result from the final trial\\
\et  example\_scale  & the measurement interval or point in the last trial\\
\et  example\_hist   & results from the prior trials\\
\et  example\_hist\_scale & intervals from the prior trials\\
\end{tabular}

Thus, during each trial, the result vector will have the same
properties as in a standard run.  It can be used in the {\vt .control}
block of a Monte Carlo or operating range file (recall that {\vt
\$?}{\it vector} can be used to query existence, and that if there is
no {\et checkPNTS} vector defined, the {\vt .control} block is called
once at the end of each trial).

In the current circuit, the parameters to be varied are usually
included as shell variables {\vt \$value1} and {\vt \$value2}.  These
are special hard-coded shell variables which contain the parameter
values during simulation.  Before the file is sourced (recall that
variable substitution occurs during the read-in), these variables can
be set with the {\cb set} command, and the file simulated just as any
other circuit.  Initially, the variables {\vt \$value1} and {\vt
\$value2} are set to zero.  The {\et value1} and {\et value2} names
can be changed to other names, and other mechanisms can be used to
supply trial values, as described in Chapter \ref{oprange}.

Briefly, operating range analysis works as follows.  The analysis
range and other parameters are specified by setting certain vectors in
the header script, or by hand.  The range is evaluated by rows
(varying {\et value1}) for each column ({\et value2}) point.  Columns
are then reevaluated if the terminating pass point was not previously
found.  For a row, starting at the left, points are evaluated until a
pass point is found.  The algorithm skips to the right, and evaluates
toward the left until a pass point is found.  This minimizes
simulation time, however strange operating ranges, such as those that
are reentrant or have islands, will not be reproduced correctly.  The
only fool-proof method is to evaluate every point, which will occur if
the {\vt -a} option is given, or the {\vt .checkall} line was given in
the input file.

The range of evaluation is set with {\it center\/}, {\it step\/}, and
{\it number\/} variables.  The {\it number\/} is the number of steps
to take above and below the {\it center\/}.  Thus, if {\it number\/}
is 1, the range is over the three points {\it center-step\/}, {\it
center\/}, and {\it center+step\/}.  One can set ranges for {\et
value1} and {\et value2}, or alternatively one can set {\et value2},
and the algorithm can determine the operating range for {\et value1}
at each {\et value2} point.  These values represent the parameter
variation range in operating range analysis, but serve only to
determine the number of trials in Monte Carlo analysis.

\index{check command!variables}
\index{check command!vectors}
There are a number of vectors with defined names which control
operating range and Monte Carlo analysis.  In addition, there are
relevant shell variables.  The {\cb check} command creates a plot
structure, which contains all of the special control vectors, plus
vectors for each circuit node and branch.  This plot becomes the
current plot after the analysis starts.  The special vectors which
have relevance to the operating range analysis are listed below.

\begin{description}
\item{{\et checkPNTS} (real, length $>=$ 1)}\\
These are the points of the scale variable (e.g., time in transient
analysis) at which the pass/fail test is applied.  If a fail is
encountered, the simulation is stopped and the next trial started. 
This vector is usually specified as an array, with the {\cb compose}
command, and is used in operating range and Monte Carlo analysis.  If
not specified, the evaluation is performed after the trial completes.

\item{{\et checkVAL1} (real, length 1)}\\
This is the initial central value of the first parameter to be varied
during operating range analysis.  It is not used in Monte Carlo
analysis.

\item{{\et checkDEL1} (real, length 1)}\\
The first central value will be incremented or decremented by this
value between trials in operating range analysis.  It is not used in
Monte carlo analysis.

\item{{\et checkSTP1} (integer, length 1)}\\
This is the number of trials above and below the central value.  In
Monte Carlo analysis, it partially specifies the number of simulation
runs to perform, and specifies one coordinate of the visual array used
to monitor progress (with the {\cb mplot} command).  In operating
range analysis, the default is zero.  In Monte Carlo analysis, the
default is 3.

\item{\et checkVAL2, checkDEL2, checkSTP2}\\
These are as above, but relate to the second parameter to be varied in
the circuit in operating range analysis.  In Monte Carlo analysis,
only {\et checkSTP2} is used, in a manner analogous to {\et
checkSTP1}.  The total number of simulations in Monte Carlo analysis
is (2*{\et checkSTP1} + 1)*(2*{\et checkSTP2} + 1), the same as would
be checked in operating range analysis.

\item{{\et checkFAIL}  (integer, length 1, value 0 or 1)}\\
This is the global pass/fail flag, which is set after each trial, 1
indicates failure.  This variable is used in both operating range and
Monte Carlo analysis.

\item{{\et checkINIT}  (integer, length 1, value 0 or 1)}\\
This is set to 1 by {\WRspice} before the initial execution of the
header block, before operating range or the first Monte Carlo trial. 
It is set to 0 otherwise.  Thus one can identify the first trial in
Monte Carlo analysis from within the header script.

\item{{\et opmin1, opmax1, opmin2, opmax2} (real, length $>=$ 1)}\\
The operating range analysis can be directed to find the operating
range extrema of the one parameter for each value of the other
parameter.  These vectors contain the values found.  They are not used
in Monte Carlo analysis.

\item{{\et value} (real, length variable)}\\
This vector can be used to pass trial values to the circuit, otherwise
shell variables are used.  This pertains to operating range and Monte
Carlo analysis.

\item{{\et checkN1, checkN2} (integer, length 1)}\\
These are the indices into the value array of the two parameters being
varied in operating range analysis.  The other entries are fixed. 
These vectors are not used if shell variables pass the trial values to
the circuit, and are not used in Monte Carlo analysis.
\end{description}

The shell variables are:

\begin{description}
\item{{\et checkiterate} (0-10)}\\
This is the binary search depth used in finding operating range
extrema.  This is not used in Monte Carlo analysis.

\item{\et value1, value2}\\
These variables are set to the current trial values to be used in the
circuit (parameters 1 and 2).  The {\WRspice} deck should reference
these variables (as {\vt \$value1} and {\vt \$value2}) as the
parameters to vary.  Alternatively, the value array can be used for
this purpose.  These variables can be used in Monte Carlo analysis. 
Additionally, these variables, and a variable named ``{\vt value}''
can be set to a string.  When done, the variable or vector named by
the string will take on the functionality of the assigned-to variable. 
For example, if {\vt set value1 = L1} is given, the variable {\vt L1}
is used to pass trial parameter 1 values to the circuit (references
are {\vt \$L1}).

Instead of using shell substitution and the {\vt value1}/{\vt value2}
variables to set varying circuit parameters, one can use an internal
parameter passing method which is probably more efficient.
 
The form, given before the analysis,
\begin{quote}
{\vt set value1="\%}{\it devicelist\/}{\vt ,}{\it paramlist\/}{\vt "}
\end{quote}
sets up a direct push into the named {\it parameters} of listed {\it
devices}, avoiding shell expansion and vectors.  Note that the list
must follow a magic `{\vt \%}' character, which tells the system to
use the {\it devlist\/},{\it paramlist} syntax, as used in the {\cb
sweep} command (see \ref{psetting}).  This is equivalent to giving
{\it pstr1\/}, {\it pstr2} on the command line.

The {\vt jjoprng2.cir} file in the examples illustrates use of this
syntax.

\end{description}

The {\et checkVAL1}, {\et checkDEL1}, etc.  vectors used must be
defined and properly initialized, either in the deck or directly from
the shell.

The shell variables {\et value1} and {\et value2} are set to the
current variable 1 and variable 2 values.  In addition, vector
variables can be set.  This is needed for scripts such as optimization
where the parameter to be varied is required to be under program
control.  If a vector named {\et value} exists, as does a vector named
{\et checkN1}, then the vector entry {\et value[{\et checkN1\/}]} is
set to {\vt \$value1} if {\et checkN1} is in the range of {\et value}. 
Similarly, if a vector {\et checkN2} exists, then the vector entry
{\et value[{\et checkN2\/}]} is set to {\vt \$value2}, if {\et
checkN2} is in the range of {\et value}.  Thus, instead of invoking
{\vt \$value1} and {\vt \$value2} in the {\WRspice} text, one can
instead invoke {\vt \$\&value[\$\&checkN1]}, {\vt
\$\&value[\$\&checkN2]}, where we have previously defined the vectors
{\et value}, {\et checkN1}, {\et checkN2}.  The file could have a
number of parameters set to {\vt \$\&value[0]}, {\vt \$\&value[1]},
...  .  If {\et checkN1} is set to 2, for example, {\vt \$\&value[2]}
would be varied, and the other values would be fixed at predefined
entries.  The name ``value'' can be redefined by setting a shell
variable named ``{\vt value}'' to the name of another vector.

If any of the shell variables {\et value1}, {\et value2}, or a {\it
shell} variable {\et value} are set to a string, then the shell
variable or vector named in the string will have the same function as
the assigned-to variable.  For example, if in the header one has
``{\vt set value1 = L1}'', then the variable reference {\vt \$L1}
would be used in the file to introduce variations, rather than {\vt
\$value1}.  Similarly, if we have issued ``{\vt set value = myvec}'',
the vector {\et myvec} would contain values to vary (using the pointer
vectors {\et checkN1} and {\et checkN2}), and a reference would have
the form {\vt \$\&myvec[\$\&checkN1]}.  Note that the alternate
variables are not automatically defined before the circuit is parsed,
so that they should be set to some value in the header.  The default
{\vt \$value1} and {\vt \$value2} are predefined to zero.

In Monte Carlo analysis, the header block is executed before each
simulation.  In the header block, shell variables and vectors may be
set for each new trial.  These variables and vectors can be used in
the SPICE text to modify circuit parameters.  The names of the
variables used, and whether to use vectors or variables, is up to the
user (variables are a little more efficient).  Monte Carlo analysis
does not use predefined names for parameter data.  Typically, the {\vt
gauss} function is used to specify a random value for the variables
in the header block.

One can keep track of the progress of the analysis in two ways. 
{\WRspice} will print the analysis point on the screen, plus indicate
whether the circuit failed or passed at the point, if the {\vt -v}
option was given to the {\cb check} command.  The {\cb echo} command can
be used in the codeblock to provide more information on-screen, which
is printed whether or not the {\vt -v} option was given.  The second
method uses the {\cb mplot} command, which graphically records the
pass/fail points.  In this mode, the relevant arguments to {\cb mplot}
are as follows.
\begin{description}
\item{\vt mplot -on}\\
This will cause subsequent operating range analysis results to be plotted
while the analysis is running.

\item{\vt mplot -off}\\
This will return to the default (no graphical output while simulating).
\end{description}

The analysis can search for the actual edge of the operating region
for each row and column.  These data are stored in vectors named {\et
opmin1}, {\et opmax1}, {\et opmin2}, and {\et opmax2} with length
equal to the number of points of the fixed variable.  For example,
{\vt opmin1[0]} will contain the minimum parameter 1 value when
parameter 2 is equal to {\it central2} - {\it delta2\/}*{\it steps2},
and {\vt opmin1[2*{\it steps2\/}]} will contain the minimum parameter
1 value when parameter 2 is {\it central2} + {\it delta2}*{\it
steps2}.

The binary search depth is controlled by a shell variable {\et
checkiterate}, with possible values of 0--10.  If set to 1--10, the
search is performed (setting to 0 skips the range finding).  Higher
values provide more accuracy but take more time.  If the search is
performed, a vector called {\et range} and its scale {\et r\_scale}
are also produced.  These contain the Y and X coordinates of the
operating range contour, which can be plotted with the command ``{\vt
plot range}''.

A typical approach is to first unset {\et checkiterate}, {\et
checkSTP1}, and {\et checkSTP2}.  The {\cb check} command is used to
run a single-point analysis, while changing the values of {\et value1}
and {\et value2} until a pass point is found.  After the pass point is
found, {\et checkiterate} can be set to a positive value, which will
yield the ranges for the two variables.  Then, the {\et checkSTP1} and
other variables can be set to cover this range with desired
granularity, and the analysis performed again.

When only one point is checked ({\vt checkSTP1 = checkSTP2 = 0}), no
output file is generated.  If {\et checkiterate} is nonzero and the
{\vt -a} option is given, and a vector is used to supply trial values,
the range of each entry in the vector is determined, and stored in the
{\et opmin1} and {\et opmax1} vectors.  A mask vector can be defined,
with the same length as the value vector and same name with the suffix
``{\vt \_mask}''.  Value entries corresponding to nonzero entries of
this vector do not have the range computed.  If the {\vt -a} flag is
not given, the range is found in the usual way.  The central value
must pass, or the range will not be computed.

See Chapter \ref{oprange} for more information on performing operating
range and Monte Carlo analysis, and the suggested file formats.

%SU-------------------------------------
\subsection{\spcmd{dc}}
\index{dc command}

% spCommands.hlp:dc 011909

The {\cb dc} command performs a swept dc analysis of the current
circuit.
\begin{quote}\vt
dc {\it .dc dc\_args}
\end{quote}
The {\it dc\_args} are the same as used in the {\vt .dc} line (see
\ref{dcline}).

%SU-------------------------------------
\subsection{\spcmd{delete}}
\index{delete command}

% spCommands.hlp:delete 022519

The {\cb delete} command is used to remove ``runops'' (traces or
breakpoints) from the runop list.
\begin{quote}\vt
delete [[in]active] [all | save | trace | iplot | measure | stop |
 {\it number\/}] ...]
\end{quote}
With no arguments, a list of existing runops is printed, and the
user is prompted for one to delete.  The {\cb status} command also
prints a list of runops.  Runopss can also be controlled with the
panel brought up with the {\cb Trace} button in the {\cb Tools}
menu.

If the {\vt inactive}/{\vt active} keyword is given, breakpoints
listed to the right but before another {\vt (in)active} keyword are
deleted only if they are inactive/active.  Otherwise, they are deleted
unconditionally.  If one of {\vt stop}, {\vt measure}, {\vt trace},
{\vt iplot}, or {\vt save} is given, runops of that type only are
deleted.  These keywords can appear in combination.

Each runop is assigned a unique number, which is available through
the {\cb status} command.  This number can also be entered on the
command line causing that runop to be deleted (if the activity
matches the {\vt inactive} keyword, if given).  A range of numbers
can be given, for example ``2-6''.  There must be no white space in
the range token.

Examples:
\begin{quote}
Delete all traces and iplots:\\
{\vt delete trace iplot}\\
\\
Delete all inactive runops:\\
{\vt delete inactive all}\\
\\
Delete all traces and inactive iplots:\\
{\vt delete traces inactive iplots}
\end{quote}

%SU-------------------------------------
\subsection{\spcmd{destroy}}
\index{destroy command}

% spCommands.hlp:destroy 062515

The {\cb destroy} command will delete plot structures.
\begin{quote}\vt
destroy [all] | [{\it plotname} ...]
\end{quote}
Giving this command will throw away the data in the named plots and
reclaim the storage space.  This can be necessary if a lot of large
simulations are being done.  {\WRspice} should warn the user if the
size of the program is approaching the maximum allowable size (within
about 90\%), but it is advisable to run the {\vt rusage space} command
occasionally if running out of space is a possibility.  If the
argument to {\cb destroy} is {\vt all}, all plots except the constants
plot will be thrown away.  It is not possible to destroy the constants
plot.  If no argument is given the current plot is destroyed.

%SU-------------------------------------
\subsection{\spcmd{devcnt}}
\index{devcnt command}

% spCommands.hlp:devcnt 080215

This command will print a table of instantiation counts of the
different device types found in the current circuit.
\begin{quote} \vt
devcnt [{\it model\_name} ...]
\end{quote}
These are the number of device structures used in the internal
representation of the circuit, after subcircuit expansion.

If no arguments appear, all devices found will be included. 
Otherwise, arguments are taken as model names (the leftmost element
printed in the output), which may include use of ``globbing''
characters `{\vt *}' and `{\vt ?}' and friends.  Briefly, `{\vt ?}'
matches any character, '{\vt *}' matches any set of characters or no
characters, ``{\vt [abc]}'' matches the characters `{\vt a}', `{\vt
b}', and `{\vt c}', and ``{\vt a}\{{\vt bc,de}\}'' matches ``{\vt
abc}'' and ``{\vt ade}''.  Matching is case-insensitive.

Note that every device has a model, which is created internally if not
given explicitly.  In particular, simple resistor, inductor, and
capacitor devices have default models named ``{\vt R}'', ``{\vt L}'',
and ``{\vt C}''.

The {\cb devcnt} table for all devices is also appended to the
standard output of batch jobs.

%SU-------------------------------------
\subsection{\spcmd{devload}}
\index{devload command}
\index{loadable device modules}
\label{devload}

% spCommands.hlp:devload 042814

This command will load a loadable device module into {\WRspice}.
\begin{quote}\vt
devload [{\it module\_path\/} | {\vt all}]
\end{quote}

{\WRspice} supports runtime-loadable device modules.  Once loaded, the
corresponding device is available during simulation runs, in the same
way as the internally-compiled devices in the device library.

This command can be used at any time to load a device module into
{\WRspice}.  If given without arguments, a list of the dynamically
loaded device modules currently in memory is printed.  Otherwise, the
single argument can be a path to a loadable device module file to be
loaded, or a path to a directory containing module files, all of which
will be loaded.

Once a module is loaded, it can not be unloaded.  The file can be
re-loaded, however, so if a module is modified and rebuilt, it can be
loaded again to update the running {\WRspice}.

On program startup, by default known loadable device modules are
loaded automatically.  Modules are known to {\WRspice} through the
following.

\begin{enumerate}
\item{If the {\vt modpath} variable is set to a list of directory
paths, modules are loaded from each directory in the list.  The {\vt
modpath} can be set from the {\vt .wrspiceinit} file.}

\item{If the {\vt modpath} variable is {\bf not} set, then modules are
loaded from the {\vt devices} subdirectory of the {\vt startup}
directory in the installation area (which is generally installed as\\
{\vt /usr/local/xictools/wrspice/startup/devices}).  Note that if the
user sets up a {\vt modpath}, this directory must be explicitly
included for these devices, which are supplied with the {\WRspice}
distribution, to be loaded.}
\end{enumerate}

If the boolean variable {\vt nomodload} is set in the {\vt
.wrspiceinit} file, then the module auto-loading is suppressed. 
Equivalently, giving the ``{\vt -mnone}'' command line option will
also suppress auto-loading, by actually setting the {\vt nomodload}
variable.  Auto-loading is also suppressed if the ``{\vt -m}'' command
line option is given, which is another method by which modules can be
loaded.

If, instead of a module path, the keyword ``{\vt all}'' is given to
the {\cb devload} command, all known modules as described above will
be loaded, the same as for the auto-load.  This will be done whether
or not {\vt nomodload} is set.

This gives the user flexibility in setting up devices in the {\vt
.wrspiceinit} file.  Normally, devices are auto-loaded after {\vt
.wrspiceinit} is processed, so that calls to the {\cb devmod} command
(for example) in {\vt .wrspiceinit} would likely fail.  However, one
can first call ``{\vt devload all}'' to auto-load the devices, and set
{\vt nomodload} to avoid the automatic loading.  Then, one can call
commands which require that devices be loaded.

The ``{\vt all}'' form may also be useful in scripts, in conjunction
with setting the {\vt modpath} to different values.

%SU-------------------------------------
\subsection{\spcmd{devls}}
\index{devls command}

% spCommands.hlp:devls 121011

This command lists currently available devices.
\begin{quote}\vt
devls [{\it key\/}[{\it minlev\/}[-{\it maxlev\/}]]] ...
\end{quote}

This commnd prints a listing of devices available for use in
simulation, from the built-in device library or loaded as modules at
run time.  With no argument, all available devices are listed.

Arguments take the form of a key letter, optionally followed by an
integer, or two integers separated by a hyphen to indicate a range. 
This will print only devices that match the key letter, and have model
levels that match the integer or integer range given.  Any number of
these arguments can be given.

Example:  {\vt devls c r1 m30-40}\\

This will print all devices keyed by `{\vt c}' (capacitors), all
devices keyed by `{\vt r}' (resistors) with model level 1, and devices
keyed by `{\vt m}' (mos) with model levels 30--40 inclusive.

%SU-------------------------------------
\subsection{\spcmd{devmod}}
\index{devmod command}

% spCommands.hlp:devmod 121611

It is possible to program the model levels associated with devices
in {\WRspice}.
\begin{quote}\vt
devmod {\it index} [{\it level} ...]
\end{quote}

This allows the user to set up model levels for compatibility with
another simulator, or to directly use simulation files where the model
level is different from that initially assigned in {\WRspice}.  The
effect is similar to the {\vt .mosmap} input directive, but applies to
all device types.

All devices have built-in levels, which are the defaults.  This
command allows levels to be changed in the currently running
{\WRspice}.  The change occurs in memory only so is not persistent
across different {\WRspice} sessions.  However, the command can be
used in a startup script to perform the changes each time {\WRspice}
is invoked.

The first argument to {\cb devmod} is a mandatory device index.  This
is an integer that corresponds to an internal index for the device. 
These are the numbers that appear in the listing from the {\cb devls}
command.

If there are no other arguments, the device is simply listed, in the
same format as the entries from {\cb devls}.

Any following arguments are taken as model levels.  Each level is an
integer in the range 1--255, and up to eight levels can be given.  The
device will be called for any of the level numbers listed.

After pressing {\kb Enter}, the device entry is printed with the new
model levels.  The entire device list is checked, and if there are
clashes from the new model level, a warning is issued.  If two similar
devices have the same model number, the device with the lowest index
will always be selected for that value.

There are a few devices that have levels that can not be changed. 
These are built-in models, such as MOS and TRA, where the model code
is designed to handle several built-in levels (such as MOS levels 1--3
and 6).  Attempting to change these levels will fail.

Model level 1 is somewhat special in that it is the default when no
model level is given in SPICE input for a device.  Level 0 is reserved
for internal use and can not be assigned.  The largest possible model
level is 255 in WRspice.

%SU-------------------------------------
\subsection{\spcmd{disto}}
\index{disto command}

% spCommands.hlp:disto 011909

The {\cb disto} command will initiate distortion analysis of the
current circuit.
\begin{quote}\vt
disto {\it disto\_args} [dc {\it dc\_args\/}]
\end{quote}
The {\it disto\_args} are the same as appear in a {\vt .disto} line
(see \ref{distoline}).  If a dc sweep specification follows, the
distortion analysis is performed at each point of the dc analysis
(see \ref{dcline}).

%SU-------------------------------------
\subsection{\spcmd{dump}}
\index{dump command}

% spCommands.hlp:dump 072018

The {\cb dump} command sends a print of the internal matrix data
structure last used by the simulator for the current circuit to the
standard output.  It is used for program debugging, amd may also be
useful for analyzing convergence problems.

\begin{quote}
{\vt dump} [{\vt -r}] [{\vt -c}] [{\vt -t}] [{\vt -f} {\it filename\/}]
\end{quote}

The command takes the following optional arguments.
\begin{description} 
\item{\vt -r}\\
Print the reordred matrix, the default is to print the matrix as it
exists before internal reordering is performed to optimize stability.

\item{\vt -c}\\
Print in compact form, showing only which elements are nonzero (marked
with `x') and zero (marked with `.').

\item{-t}\\
Terse format, do not print header information.

\item{\vt -f} {\it filename}\\
Print output in the given file.
\end{description}

%SU-------------------------------------
\subsection{\spcmd{findlower}}
\index{findlower command}

% spCommands.hlp:findlower 032120

This command can be used to find the lower operating limit of one or
two parameters in the circuit.  See the {\cb findrange} command for a
complete description.

\begin{quote}
{\vt findlower} {\it findrange\_args}
\end{quote}

%SU-------------------------------------
\subsection{\spcmd{findrange}}
\index{findrange command}

% spCommands.hlp:findrange 032120

The command, and its associated commands {\cb findlower} and {\cb
findupper}, can be used to find the operating margins of one or two
circuit parameters.

\begin{quote}
{\vt findrange} [{\vt -n1} {\it name1\/}] [{\vt -n2} {\it name2\/}]
[[{\it pstr1\/}] {\it val1\/}] [[{\it pstr2\/}] {\it val2\/}]
\end{quote}

This utilizes the infrastructure developed for Operating Range
analysis in \ref{oprange}, but can be used in scripts for finer
control of the process.  The depth used in the binary search can be
given in the {\et checkiterate} variable as for standard range
analysis, or defaults to 6 if not set.

These commands can be running only when a range analysis has been
initiated with the {\cb check} command (see \ref{checkcom}), generally
by giving the ``{\vt -b}'' option.  Any number of the {\cb findrange}
commands or the variants can be given, as well as other commands such
as {\cb mctrial}.  When finished the {\cb check} command should be
given with the ``{\vt -c}'' option to terminate the mode and free
internal memory.

A usage example can be found in the examples: {\vt JJexamples/nor\_op.cir}.

By default, the lower and upper range values will be saved in vectors
named {\vt opmin1}, {\vt opmin2}, {\vt opmax1}, and {\vt opmax2},
which are created if necessary.  If given following ``{\vt -n1}'' or
``{\vt -n2}'' respectively, {\it name1} and {\it name2} tokens will
serve as a base for new names that replace vector names {\vt opmin1},
{\vt opmin2}, {\vt opmax1}, {\vt opmax2} for range results.  For
example,
\begin{quote}
\vt -n1 foo
\end{quote}
will save output in vectors named {\vt foo\_min}, {\vt foo\_max}.

There is a subtlety in the syntax:  a double-quoted name, e.g.,
{\vt -n1 "}{\it pname\/}{\vt "}, is accepted and the quotes will
be stripped before use.  The quotes prevent parameter substitution, so
this allows use of a name that has also been defined as a parameter
(with {\vt .param} directive or otherwise).

Each of the three functions can take parameter definitions and range
parameters, in the same syntax as supplied to the {\cb check} and {\cb
sweep} commands, however only the starting parameter value is needed. 
The simulation must run correctly at the starting value.  The command
line may include one or two specifications.  Each specification
consists of an optional parameter specification string, followed by
the starting value.  The numbers will override the {\vt checkVAL1},
and {\vt checkVAL2} vectors that may be in effect.

The optional {\it pstr1} and {\it pstr2} strings take the same format
and significance as in the {\cb sweep} command.  See the description
of that command for a description of the format.  If a parameter
specifier is given, the specified device parameters will be altered
directly, and the variables and vectors normally used to pass values
will {\bf not} be set.  If two parameters are being set, either both
must be set using the syntax above, or neither, the two methods can't
be mixed.

%SU-------------------------------------
\subsection{\spcmd{findupper}}
\index{findupper command}

% spCommands.hlp:findupper 032120

This command can be used to find the upper operating limit of one or
two parameters in the circuit.  See the {\cb findrange} command for a
complete description.

\begin{quote}
{\vt findupper} {\it findrange\_args}
\end{quote}

%SU-------------------------------------
\subsection{\spcmd{free}}
\index{free command}

% spCommands.hlp:free 011909

The {\cb free} command is used to free memory used by circuit and plot
structures.
\begin{quote}\vt
free [c[ircuit]] [p[lot]] [a[ll]] [y[es]]
\end{quote}
This command releases the memory used to store plot and circuit
structures for reuse by {\WRspice}.  The virtual memory space used by
plots in particular can grow quite large.  If {\cb free} is given
without an argument, the user is queried as to whether to delete the
current plot and circuit structures (independently).  If the argument
{\vt all} is given, the user is queried as to whether to delete all
plot and circuit structures.  If the argument {\vt circuit} is given,
only circuits will be acted on.  Similarly, if the argument {\vt plot}
is given, only plots will be acted on.  If neither {\vt circuit} or
{\vt plot} is given, both circuits and plots will be acted on.  If the
argument {\vt yes} is given, the user prompting is skipped, and the
action performed.  Only the first letter of the keywords is needed. 
Plots can also be freed from the panel brought up by the {\cb Plots}
button in the {\cb Tools} menu, and circuits can be freed from the
panel brought up with the {\cb Circuits} button.  The {\cb destroy}
command can also be used to free plots.

%SU-------------------------------------
\subsection{\spcmd{jobs}}
\index{jobs command}

% spCommands.hlp:jobs 012209

The {\cb jobs} command produces a report on the asynchronous
{\WRspice} jobs currently running.  Asynchronous jobs can be started
with the {\cb aspice} command locally, or on a remote system with the
{\cb rspice} command.  {\WRspice} checks to see if the jobs are
finished every time a command is executed.  If a job is finished, then
the data are loaded and become available.  This command takes no
arguments.

%SU-------------------------------------
\subsection{\spcmd{mctrial}}
\index{mctrial command}

% spCommands.hlp:mctrial 032520

This is a command to run a single trial for use when performing
script-driven Monte Carlo analysis.
\begin{quote}\vt
mctrial
\end{quote}

This can run only when a Monte Carlo analysis mode has been initiated
with the {\cb check} command, generally by giving the ``{\vt -b}'' and
``{\vt -m}'' options.  Any number of the {\cb mctrial} commands can be
given, as well as other commands such as {\cb findrange}.  When
finished the {\cb check} command should be given with the ``{\vt -c}''
option to terminate the mode and free internal memory.

A usage example can be found in the examples:  {\vt
JJexamples/nor\_mc.cir}.

%SU-------------------------------------
\subsection{\spcmd{measure}}
\index{measure command}

% spCommands.hlp:measure 071722

The {\cb measure} command allows one to set up a runop (see
\ref{runop}) which will identify a measurement point or interval, and
evaluate an expression at that point, or call a number of measurement
primitives that apply during the interval, such as rise time or pulse
width.

\begin{quote}
measure {\it analysis} {\it resultname} {\it point} | {\it interval}
[{\it measurements\/}] [{\it postcmds\/}]\\
measure {\it analysis} {\it resultname} param={\it expression}
[{\it postcmds\/}]
\end{quote}

The command will apply, if active, when any circuit is being run. 
There is also a {\vt .measure} input syntax element which will set up
the measurement on that circuit only.  Both use the identical syntax
described below.  The syntax is based on the {\vt .measure} statement
in HSPICE.

\begin{description}
\item{\it analysis}\\
This specifies the type of analysis during which the measurement will
be active.  Exactly one of the following keywords should appear in
this field:  {\vt tran}, {\vt ac}, {\vt dc}.

\item{\it resultname}\\
This field specifies a name for the measurement.  The name should be
unique among the measurements in the circuit, and among vectors in
scope during simulation.  The name should start with an alphabetic
character and contain no white space or other special characters.
\end{description}

A vector with this name will be added to the current plot, if the
measurement is successful.  Vector names found in {\vt .measure} lines
are added to the internal save list, guaranteeing that the necessary
data will be available when needed, whether or not the vector has been
mentioned in a {\vt .save} line.

\subsubsection{Point and Interval Specification}
\index{measurement interval}

The field that follows the {\it resultname} contains a description of
the conditions which initiate a measurement.  There are three basic
types:  a point specification, an interval specification, and a
post-measurement specificantion.

The {\it interval} begins with the ``trigger'' and ends with the
``target''.  Measurement will apply during this interval.  If no
target is given, the trigger sets the {\vt point}, where measurement
will be performed.  The trigger and target are independently specified
as follows:

\begin{description}
\item{\it point}\\
\begin{quote}
[{\vt trig}] {\it pointlist}
\end{quote}
This consists of the keyword {\vt trig} (which is optional) followed
by a point specification list.  The keyword ``{\vt from}'' is
equivalent to ``{\vt trig}''.

\item{\it interval}\\
\begin{quote}
[{\vt trig}] {\it pointlist} {\vt targ} {\it pointlist}
\end{quote}
An interval contains a second point specification initiated with the
mandatory keyword {\vt targ}.  The keyword ``{\vt to}'' is equivalent
to ``{\vt targ}''.

\item{post-measurement}\\
\begin{quote}
{\vt param=}{\it expression}
\end{quote}
Measurements in this form will be performed when all {\it point} and
{\it interval} measurements are complete.  After all {\it point} and
{\it interval} measurements have been performed, the {\it expression}
will be evaluated and the result saved in {\it resultname}.  The {\it
expression} can reference other measurement results in addition to the
usual vectors and functions provided by the system.  These measurement
lines are evaluated in the order found in the input.

\item{\it pointlist}\\
\begin{quote}
{\it pointspec} [{\it pointspec\/}] ...
\end{quote}
The point is specified with a list of {\it pointspec} specifications,
and the event is registered on the first occasion when all {\it
pointspec} elements are true, i.e., the conjunction is true.

\item{\it pointspec}\\
{\it keyword} {\it expression1} [{\vt =}][{\vt val=}] [{\it expression2\/}]
[{\vt cross=}{\it crosses\/}] [{\vt rise=}{\it rises\/}]
[{\vt fall=}{\it falls\/}]\newline
 [{\vt minx=}{\it min\_delta}] [{\vt td=}{\it delay\/}]

\end{description}

The {\it pointspec} begins with one of the following kewords:  {\vt
before}, {\vt at}, {\vt after}, {\vt when}.  The {\vt at} keyword
strobes, meaning that the event is triggered only if the conjunctions
(other {\it pointspecs}) in the list are true at the specified event. 
The {\vt after} and {\vt when} keywords are equivalent, but varied use
can give a natural language feel to the conjuction list.  They are not
strobing, meaning that the conjunctions can become true anytime at or
after the specified event.  The {\vt before} keyword negates logic: 
the {\it pointspec} is true before the specified event.  This can be
useful as an element in the conjunction list.

Once a {\it pointspec} becomes triggered, it remains triggered for the
remainder of the simulation run.  Once triggered, a {\vt before} {\it
pointspec} will evaluate false, preventing the overall list from
triggering.  Otherwise, the overall list triggers when each {\it
pointspec} is true.  Similarly, an {\vt at} clause that did not have
all conjunctions true at its event time will thereafter always be
false.

Following the keyword are one or two general expressions.  There can
be an optional equal sign (``{\vt =}'') or a ``{\vt val}'' keyword
``{\vt val=}'' between the expressions.

An {\it expression} in this context can be:
\begin{itemize}
\item{A number or constant expression.  This is taken as the
triggering point, meaning that the event occurs during simulation when
the scale variable is equal to or exceeds the value.}

\item{An integer enclosed in square brackets.  This is interpreted as
an output index, which increments whenever data would be written out
from the running simulation.  This is most useful when the printing
increment is constant.  The event triggers when the output index
equals the integer given.}

\item{The {\it resultname} of a measure in the circuit.  The event
occurs when the referenced measurement is performed.}

\item{A general expression consisting of constants, vector names, and
circuit variables.  Frequently this will be simply a vector name
corresponding to a node voltage or branch current in the circuit.}
\end{itemize}

In {\WRspice}, an expression token is the minimum text required for a
syntactically complete expression, and may include white space. 
Single quotes or parentheses can be used to delimit expressions in the
{\it pointspec}, if nexessary.  The normal single-quote expression
expansion and substitution is suppressed in this context.

If only one {\it expression} is given, and it is not a constant
expression or a measure name, the event is triggered at the first time
the expression becomes logically true, meaning that the absolute value
is one or larger.  This corresponds to logical true produced by
comparison and other logical operations in {\WRspice}.  For example,
the expression ``{\vt v(5) > 0.25}'' returns 0 if false and 1 if true.

It may be a bit confusing but the form {\it expr1\/}{\vt =}{\it expr2}
is interpreted as two expressions, but the same form with any
relational operator other than {\vt =} is taken as a single expression
with a binary result.  Either the symbol or the keyword equivalent can
be used.  The relational operators available are listed below.

\begin{tabular}{|l|l|}\hline
{\vt eq}   or   $=$  & equal to\\ \hline
{\vt ne}   or   $<>$ & not equal to\\ \hline
{\vt gt}   or   $>$  & greater than\\ \hline
{\vt lt}   or   $<$  & less than\\ \hline
{\vt ge}   or   $>=$ & greater than or equal to\\ \hline
{\vt le}   or   $<=$ & less than or equal to\\ \hline
\end{tabular}

If two expressions are given, neither can be a measure result name. 
We are implicitly comparing the values of the two expressions, finding
points where the two expressions are equal.  By default, the first
time the values of the two expressions cross will trigger the event. 
The following keywords can be assigned an integer value to trigger at
the indicated point.

\begin{description}
\item{{\vt cross=}{\it crosses}}\\
Integer {\it crosses} is the number of crossings.

\item{{\vt rise=}{\it rises}}\\
Integer {\it rises} is the number of times that {\it expression1}
rises above {\it expression2}.

\item{{\vt fall=}{\it falls}}\\
Integer {\it falls} is the number of times that {\it expression1}
falls below {\it expression2}.

\item{{\vt minx=}{\it min\_delta}}\\
Real {\it min\_delta} is the minimum time before a following crossing
event will be recognized, used to suppress spurious crossings from
noise or ringing.
\end{description}

\begin{description}
\item{{\vt td=}{\it delay}}\\
If two expressions are given, the {\it delay} is the amount of scale
value (e.g., time in transient analysis) before starting to look for
crossing events.

If one expression is given, and the expression is not constant or a
measure result name, the {\it delay} is the amount of the scale value
to wait before checking to see if the expression evaluates true.  If
the expression is a measurement name, than the delay is added to the
measurement time of the referenced measurement.

There is a special case, where no expressions are given, only a {\vt
td=}{\it delay} value.  This can be a second or subsequent {\it
pointspec} in the {\it pointlist}.  This will trigger at the time of
the previous {\it pointspec} in the list (to the left) delayed by {\it
delay}.

\item{{\vt ts=}{\it delay}}\\
This is similar to {\vt td}, however it is strobing.  In the two
expression case, in addition to having the effect of {\vt td}, it will
convert {\vt when} and {\vt after} clauses to work as {\vt at},
requiring conjunctions to be true at the time of the event.  It simply
acts as {\vt td} for {\vt at} and {\vt before}.

In the single expression case, it requires that the expression and
any conjunctions be true at the value given for {\vt ts}.
\end{description}

Examples:\\
{\vt at v(2)=0.5 rise=3 td=0.2nS after td=0.1nS}\\
\hspace*{1em}Trigger 0.1nS after the third rising edge of v(2) after
0.2nS crosses 0.5V.\\
{\vt when v(2)<v(1) before v(2)<v(3)}\\
\hspace*{1em}Trigger the first time that {\vt v(2) < v(1)} if and only
if {\vt v(2) < v(3)} has never been true.

\subsubsection{Syntax Compatibility}

The present syntax supported by the {\vt .measure} command in
{\WRspice} is a super set of the previous syntax cases, which
    are shown below.  These should all work in the present system.

Form 1:
\begin{quote}\vt
    trig|targ at={\it value}
\end{quote}

Form 1 is straightforward; the interval starts ({\vt trig}) or ends
({\vt targ}) at {\it value}.  {\it Value} must be within the
simulation range of the scale variable (e.g., time in transient
analysis).

The same effect can be achieved with:
\begin{quote}\vt
    from={\it value} to={\it value}
\end{quote}

Form 2:
\begin{quote}\vt
    trig|targ {\it variable} [val=]{\it value}
        [td={\it delay}] [cross={\it crosses}]
        [rise={\it rises}] [fall={\it falls}] [minx={\it min\_delta}]
\end{quote}

Form 2 allows the interval boundaries to be referenced to times when a
variable crosses a threshold.  The {\it variable} can be any vector
whose value is available during simulation.  The {\it value} is a
constant which is used to measure crossing events.  The {\vt val=}
which precedes the {\it value} is optional.  At least one of the {\vt
rise/fall/cross} fields should be set.  Their values are integers
which represent the variable crossing the threshold a number of times. 
The {\vt rise} indicates the variable rising through the threshold,
{\vt fall} indicates the variable decreasing from above to below the
threshold, and {\vt cross} indicates {\vt rises + falls}.  If given,
the {\vt minx} value sets the minimum time delta between the crossing
events, those that occur too soon are ignored.  This can be used to
suppress false triggering from ringing or noise.  The interval
boundary is set when the specified number of transitions is reached.

If the delay is specified, transition counting starts after the
specified delay.

\spexampo{trig v(2) 2.5 td=0.1ns rise=2}

This indicates that the interval begins at the second time {\vt v(2)} rises
above 2.5V after 0.1ns.

Form 3:
\begin{quote}\vt
    trig|targ when {\it expr1\/}={\it expr2} [td={\it delay}]
        [cross={\it crosses}] [rise={\it rises}] [fall={\it falls}]
        [minx={\vt min\_delta}]
\end{quote}

The third form is similar to the second form, except that crossings
are defined when {\it expr1 = expr2}.  These are expressions, which
must be enclosed in parentheses if they contain white space or commas. 
A rise is defined as {\it expr1} going from less than to greater than
{\it expr2}.

\subsubsection{Measurements}
\index{measurement types}

One should be aware that measurements are performed using data saved
in the plot structure as a simulation progresses.  The accuracy of the
results is directly affected by the density of saved points.  In
transient analysis, one may wish to use internal time point data by
setting the {\vt steptype=nousertp} option.  This avoids the
interpolation to tranient time increments which may reduce accuracy if
the increment is too coarse.

The following measurements are available when an interval has been
specified.

\begin{description}
\item{{\vt find} {\it expr}}\\
Evaluate the difference:  {\it expr} at target minus {\it expr} at
trigger.
\item{{\vt min} {\it expr}}\\
Find the minimum value of {\it expr\/}.
\item{{\vt max} {\it expr}}\\
Find the maximum value of {\it expr\/}.
\item{{\vt pp} {\it expr}}\\
Find the (maximum - minimum) value of {\it expr\/}.
\item{{\vt avg} {\it expr}}\\
Compute the average of {\it expr\/}.
\item{{\vt rms} {\it expr}}\\
Compute the rms value of {\it expr\/}.
\item{{\vt pw} {\it expr}}\\
This will measure the full-width half-maximum of a pulse contained in
the interval.  The baseline is taken as the initial or final value
with the smallest difference from the peak value.  The algorithm will
measure the larger of a negative going or positive going pulse.
\item{{\vt rt} {\it expr}}\\
This will measure the 10-90 percent rise or fall time of the edge
contained in the interval.  The reference start and final values are
the values at the ends of the interval.
\end{description}

These functions are also available in general expressions outside of
the {\cb measure} command:  {\cb mmin}, {\cb mmax}, {\cb mpp}, {\cb
mavg}, {\cb mrms}, {\cb mpw}, {\cb mrft}.  Each of these functions
takes three arguments:  ({\it vector\/}, {\it scaleval1\/}, {\it
scaleval2\/}).  The two scale values frame the area of measurement. 
These must be chosen to isolate the feature of interest for
rise/fall/width measurement.  If not in range of the {\it vector}
scale, the {\it vector} scale endpoints are assumed.

When a point has been specified, the only measurement form
available is

\begin{description}
\item{{\vt find} {\it expr}}\\
Evaluate {\it expr} at point.
\end{description}

A {\vt .measure} statement can contain any number of measurements,
including no measurements.  If no measurement is specified, the vector
produced contains only zeros, however the scale vector contains the
start and stop values, which may be the only result needed.  The
created vector, which is added to the current plot, will be of length
equal to the number of measurements, with the results placed in the
vector in order.

The measurement scale point(s) in {\vt .measure} statements are also
saved in a vector, which is the scale for the result vector.  If the
measurement name is ``{\vt result}'', then the scale vector is named
``{\vt result\_scale}''.  The scale contains one or two values,
depending on whether it is a point or interval measurement.

\subsubsection{Post-Measurement Commands}
\index{post-measurement commands}

There are a few commands which can be performed after measurement,
which will run whether or not any measurements are actually made.

\begin{description}
\item{{\vt print}, {\vt print\_terse}}\\
By default, nothing is printed on-screen for a {\vt .measure} line
during interactive simulation.  If the keyword {\vt print} appears in
the {\vt .measure} line, the results will be printed on the standard
output.  A more concise format can be obtained from the alternative
keyword {\vt print\_terse}.  The result vectors are created in all
cases.

\item{\vt stop}\\
If the keyword {\vt stop} appears in a {\vt .measure} line, the
analysis will be paused when {\it all} measurements are complete. 
Thus if the deck contains several {\vt .measure} lines and {\vt stop}
is given in at least one, the analysis will pause when all of the
measurements are complete, not just the one containing {\vt stop}. 
The analysis can then be resumed with the {\cb resume} command, or
reset with the {\cb reset} command.

\item{{\vt exec} {\it command}}\\
Execute the {\WRspice} shell command found in {\it command\/}, which
should be double-quoted if it contains white space.  Note that
multiple commands can be given, separated by semicolon ({\vt ';'})
characters.  This will be run before a script is called (see below) so
can be used to pass information to the script.  The command will be
executed once only, after measurements if any.

\item{{\vt call} {\it script}}\\
After the measurement (if any) is performed and any command string is
executed, the named script will be called.  The script can be a normal
script file or codeblock.  The special names ``{\vt .exec}'', ``{\vt
.control}'', and ``{\vt .postrun}'' call the exec, control, or postrun
bound codeblocks of the running circuit, if they exist.

The script can be used for additional processing or testing of
whatever sort.  If the script returns 1, the current simulation will
pause immediately (no waiting for other measures) however a calling
analysis, such as Monto Carlo, will continue.  If 2 is returned, this
indicates a fatal global error and any calling analysis will be
stopped too.  Any other return value allows the run to continue
normally.
\end{description}

When a {\vt .measure} is included in an iterative analysis (Monte
Carlo, loop, etc.), data are saved as follows.  Before each iteration,
the previous result vector and its scale are saved to the end of a
``history'' vector and scale, and are then deleted.  The result vector
and scale are recreated when the measurement is completed during the
iteration.  Thus, at the end of the analysis, for a measurement named
``example'', one would have the following vectors:

\begin{tabular}{|l|l|}\hline
\vt example & the result from the final trial\\ \hline
\vt example\_scale & the measurement interval or point in the last
 trial\\ \hline
\vt example\_hist & results from the prior trials\\ \hline
\vt example\_hist\_scale & intervals from the prior trials\\ \hline
\end{tabular}

Thus, during each trial, the result vector will have the same
properties as in a standard run.  It can be used in the {\vt .control}
block of a Monte Carlo or operating range file (recall that {\vt
\$?}{\it vector} can be used to query existence, and that if there is
no {\vt checkPNTS} vector defined, the {\vt .control} block is called
once at the end of each trial).

\index{measurements, chained}
Multiple {\vt .measure} lines can be ``chained'' in the following
manner.  The vector name following the {\vt from}, {\vt to}, {\vt
trig}, or {\vt targ} keywords can be the name of another measure.  In
this case, the effective start time is the measure time of the
referenced measure.  The measure time is the end of the interval or
the measure point.  The {\vt td}, {\vt rise}, and other keywords can
be used in the referencing measure.  The {\vt td} will be added to the
imported time, and the other keywords operate in the normal way.  If
there are no keywords other than {\vt td} specified, the time is the
delay time plus the imported time.

Example:
\begin{quote}\vt
    .measure tran t1 trig v(5) val=.4m rise=3\\
    .measure tran t2 trig v(5) val=.4m rise=4\\
    .measure tran pw trig t1 td=20p targ t2 td=20p pw v(5) max v(5)\\
\end{quote}

In this case, the measures {\vt t1} and {\vt t2} ``frame'' a period
of an (assumed) repeating signal v(5).  Note that no actual
measurement is performed for these lines.  Their purpose is to be
referenced in the third line, which takes as its interval the {\vt
t1}--{\vt t2} interval delayed by 20 pS, and measures the pulse
width and peak value.

\subsubsection{Referencing Results in Sources}
\index{measurements, source reference}

It is possible to reference {\vt .measure} results in sources.  The
referencing token has the same form as a circuit variable, with an
optional index, i.e.
\begin{quote}\vt
    @{\it result\/}[{\it index\/}]
\end{quote}
where the {\it index\/}, if used, is an integer that references a
specific component of the result (0-based).  The value is always zero
for timepoints before the measurement has been performed, and a
constant value afterward.

Example:
\begin{quote}\vt
    .measure tran peak from=50n to=150n max v(5)\\
    .measure tran stuff trig v(4) val=4.5 rise=1 targ v(4) val=4.5 fall=2\\
    + min v(4) max v(4) pp v(4) avg v(4) rms v(4) print\\
    vxx 1 0 @peak\\
    vyy 2 0 @stuff[2]\\
\end{quote}

In this example, during transient analysis, {\vt vxx} is zero until
150 nS, where the measurement takes place, at which point it jumps to
the value measured.  Likewise, {\vt vyy} is zero until the
measurement, at which point it jumps to the third component (``{\vt pp
v(4)}'') result.  The resulting voltages can be used elsewhere in the
circuit.  Note that we have two implementations of a behavioral peak
detector.


%SU-------------------------------------
\subsection{\spcmd{noise}}
\index{noise command}

% spCommands.hlp:noise 011909

The {\cb noise} command initiates a small-signal noise analysis of the
current circuit.
\begin{quote}\vt
noise {\it noise\_args} [dc {\it dc\_args\/}]
\end{quote}
The {\it noise\_args} are the same as appear in a {\vt .noise} line
(see \ref{noiseline}).  If a dc sweep specification follows, the noise
analysis is performed at each point of the dc analysis (see
\ref{dcline}).

%SU-------------------------------------
\subsection{\spcmd{op}}
\index{op command}

% spCommands.hlp:op 011909

The {\cb op} command will initiate dc operating point analysis of the
current circuit (see \ref{opline}).  The command takes no arguments.

%SU-------------------------------------
\subsection{\spcmd{pz}}
\index{pz command}

% spCommands.hlp:pz 011909

The {\cb pz} command will initiate pole-zero analysis on the current
circuit.
\begin{quote}\vt
pz {\it pz\_args}
\end{quote}
The {\it pz\_args} are the same as appear in a {\vt .pz} line (see
\ref{pzline}).

%SU-------------------------------------
\subsection{\spcmd{reset}}
\index{reset command}

% spCommands.hlp:reset 011909

The {\cb reset} command will reinitialize the current circuit.
\begin{quote}\vt
reset [-c]
\end{quote}
The command will throw out any intermediate data in the circuit (e.g,
after a breakpoint or user pause with {\kb Ctrl-C}) and re-parse the
deck.  Any standard analysis in progress will be cleared, however the
state of any margin analysis (started with the {\cb check} command),
or loop analysis (started with the {\cb loop} command) is retained by
default.  However, if the {\vt -c} option is given, these too are
cleared.  Thus, the {\cb reset} command can be used to update the
shell variables in a deck with or without affecting the status of a
margin or loop analysis in progress.

%SU-------------------------------------
\subsection{\spcmd{resume}}
\index{resume command}

% spCommands.hlp:resume 011909

The {\cb resume} command will resume an analysis in progress.  The
simulation can be stopped by typing an interrupt ({\kb Ctrl-C}) or
with the {\cb stop} command.  If no analysis is currently in
progress, the effect is the same as the {\cb run} command.  Each
circuit can have one each of a standard analysis, a loop analysis
(started with the {\cb loop} command), and a margin analysis (from
the {\cb check} command) in progress.  The {\cb resume} command will
resume standard analysis, margin analysis, and loop analysis in that
precedence.  Paused margin and loop analysis can also be restarted
with the {\cb check} and {\cb loop} commands.  These commands, and
the {\cb reset} command, can be used to clear stopped analyses.  The
{\cb resume} command takes no arguments.

%SU-------------------------------------
\subsection{\spcmd{rhost}}
\index{rhost command}
\label{rhost}

% spCommands.hlp:rhostcom 012411

The {\cb rhost} command allows addition of host names which are
available for remote {\WRspice} runs.
\begin{quote}\vt
rhost [-a][-d] [{\it hostname\/}]
\end{quote}
This command allows manipulation of a list of host names which are
available for remote {\WRspice} runs with the {\cb rspice} command. 
If no arguments are given, the list of hosts is printed.  The {\vt -a}
and {\vt -d} options allow a host name to be added to or deleted from
the list, respectively.  The default is {\vt -a}.  Hosts are added to
the list if they have been specified in the environment or with the
{\et rhost} variable, and a job has been submitted.  The {\it
hostname} can optionally be suffixed with a colon followed by the port
number to use to communicate with the {\vt wrspiced} daemon.  If not
given, the port number is obtained from the operating system for
``wrspice/tcp'', or 6114 if this is not defined.  Port number 6114 is
registered with IANA for this service.

%SU-------------------------------------
\subsection{\spcmd{rspice}}
\index{rspice command}
\label{rspice}

% spCommands.hlp:rspice 011909

The {\cb rspice} command is used to initiate simulation runs on a
remote machine.
\begin{quote}\vt
rspice {\it inputfile\/}\\
{\rm or}\\
rspice [-h {\it host\/}][-p {\it program\/}][-f {\it inputfile\/}]
[{\it analysis\/}]
\end{quote}
This command initiates a remove {\WRspice} job, using the {\it
inputfile} as input, or the current circuit if no {\it inputfile} is
given.  If the {\vt -h} option is not used, the default host can be
specified in the environment before {\WRspice} is started with the
{\et SPICE\_HOST} environment variable, or with the {\et rhost}
variable.  In addition, a list of hosts which are nominally available
for remote runs can be generated with the {\et rhost} command.  The
default host used is the host known to {\WRspice} that has the fewest
active submissions, or which appears last on the list (hosts are added
to the front of the list).  If the {\vt -p} option is not used,
{\WRspice} will use the program found in the {\et rprogram} variable,
and if not set will use the same {\it program} as the {\cb aspice}
command.  If the {\vt -f} option is not used, the current circuit is
submitted, otherwise {\it inputfile} is submitted.  If there is no
{\it analysis} specification, there must be an analysis specified in
{\it inputfile}.  If the current circuit is being submitted, there
must be an {\it analysis} specification given on the command line.

Once the job is submitted, {\WRspice} returns to interactive mode. 
When the job is complete, the standard output of the job, if any, is
printed, and the rawfile generated becomes the current plot.

Remote runs can only be performed on machines which have the {\vt
wrspiced} daemon operating, and have permission to execute the target
program.

%SU-------------------------------------
\subsection{\spcmd{run}}
\index{run command}

% spCommands.hlp:rspice 082909

The {\cb run} command initiates the analysis found in the deck associated
with the current circuit.
\begin{quote}\vt
run [{\it file\/}]
\end{quote}
The command will run the simulation as specified in the input file. 
If there were any of the analysis specification lines ({\vt .dc}, {\vt
.tran}, etc.~) they are executed.  The output is put in {\it
file\/} if it was given, in addition to being available
interactively.

There are two file formats available, the native ``rawfile'' format,
and the Common Simulation Data Format (CSDF) used by HSPICE.  See the
description of the {\cb write} command (\ref{writecmd}) for
information on format selection.

If a standard analysis run is in progress and halted with the {\cb
stop} command or by pressing {\kb Ctrl-C}, the {\cb run} command will
resume that run.  This applies only to standard analyses, and not
margin analysis or loop analysis.  Standard analyses started with the
analysis commands {\cb tran}, {\cb dc}, etc.~, will always start a new
analysis, after clearing any paused standard analysis.

%SU-------------------------------------
\subsection{\spcmd{save}}
\index{save command}

% spCommands.hlp:save 011909

The {\cb save} command can be used to save a particular set of outputs
from a simulation run.
\begin{quote}\vt
save [all] [{\it nodename\/} ...]
\end{quote}
The command will save a set of outputs, the rest will be discarded. 
If a node has been mentioned in a {\cb save} command, it will appear
in the working plot after a run has completed, or in the rawfile if
{\WRspice} is run in batch mode (in this case, the command can be
given in the input file as {\vt .save} ...).  If a node is traced or
used in an {\cb iplot} it will also be saved.  If no save commands are
given, all nodes will be saved.  The {\cb save} can be deleted with
the {\cb delete} command, or from the panel brought up by the {\cb
Trace} button in the {\cb Tools} menu.

If a {\cb save} command is given at the prompt in interactive mode, it
is placed in a global list, and activity will persist until deleted
(with the {\cb delete} command).  If the command is given in a file,
the command will be added to a list for the current circuit, and will
apply only to that circuit.  Thus, for example, a {\WRspice} file can
contain lines like
\begin{quote}
{\vt *\# save v(1) ...}
\end{quote}
and the action will be performed as that circuit is run, but the
``{\vt save v(1)} ...'' directive will not apply to other circuits.

One can save ``special'' variables, i.e., device/circuit parameters
that begin with `{\vt @}'.  If a device parameter is a list type, only
a single component can be saved.  The single component can be
specified with an integer, or with a vector name that evaluates to an
integer.  For example, the initial condition values for a Josephson
junction can be accessed as a list, say for a junction named ``b1'',
one can specify
\begin{quote}
{\vt @b1[ic,0]} or {\vt @b1[ic][0]}
\end{quote}
which are equivalent, and each the same as {\vt @b1[vj]}, the initial
voltage.  Similarly,
\begin{quote}
{\vt @b1[ic,1]} or {\vt @b1[ic][1]}
\end{quote}
are equivalent, each being the same as {\vt @b1[phi]}, the initial
phase.

One can also have
\begin{quote}
{\vt let val = 1}\qquad\qquad (this vector is defined somewhere)\\
{\vt @b1[ic,val]} or {\vt @b1[ic][val]}
\end{quote}
Thus, commands like
\begin{quote}
{\vt save @b1[ic,0]}
\end{quote}
or equivalently
\begin{quote}
{\vt save @b1[ic][0]}
\end{quote}
are accepted.  Note that ``{\vt save @b1[ic]}'' is the same as ``{\vt
save @b1[ic,0]}''.  The ``0'' can be an integer, or a vector name that
evaluates to an integer.

%SU-------------------------------------
\subsection{\spcmd{sens}}
\index{sens command}

% spCommands.hlp:sens 011909

The {\cb sens} command initiates sensitivity analysis on the current
circuit.
\begin{quote}\vt
sens {\it sens\_args\/} [dc {\it dc\_args\/}]
\end{quote}
The {\it sens\_args} are the same as appear in a {\vt .sens} line
(see \ref{sensline}).  If a dc sweep specification follows, the
sensitivity analysis is performed at each point of the dc analysis
(see \ref{dcline}).

%SU-------------------------------------
\subsection{\spcmd{setcirc}}
\index{setcirc command}

% spCommands.hlp:setcirc 011909

The {\cb setcirc} command will set the ``current circuit'' assumed
by {\WRspice}.
\begin{quote}\vt
setcirc [{\it circuit\_name\/}]
\end{quote}
The current circuit is the one that is used by the analysis commands. 
When a circuit is loaded with the {\cb source} command it becomes the
current circuit.  If no arguments are given, a list of circuits is
printed, and the user is requested to choose one.  The current circuit
can also be selected from the panel brought up by the {\cb Circuits}
button in the {\cb Tools} menu.
 
%SU-------------------------------------
\subsection{\spcmd{show}}
\index{show command}

% spCommands.hlp:show 062515

The {\cb show} command is used to display information about devices,
models, and internal statistics.
\begin{quote}\vt
show [-r|-o|-d|-n {\it nodename}|-m|-D[M]|-M] [{\it args\/}]
  [, {\it parmlist\/}]
\end{quote}
If {\vt -r} is given, system resource values are printed.  The {\it
args} are resource keywords as in the {\cb rusage} and {\cb
stats} commands, and there is no {\it parmlist}.  If there are no {\it
args}, only total time and space usage are printed.

If {\vt -o} is given, {\WRspice} option values are printed.  These
values are obtained from the {\vt .options} line of the current
circuit, or have been set with the {\cb set} command.  If no {\it
args} are given, the default is {\vt all}.  There is no {\it
parmlist}.

If {\vt -d} is given, or if no option is given, device parameters are
printed.  The {\it args} are device names, and the {\it parmlist},
which is separated from the device list by a comma, consists of device
parameter keywords.  The parameters are expected to apply to each
device in the list.  Both lists can contain ``globbing'' (wild-card)
characters with similar behavior to globbing (global substitution) in
the {\WRspice} shell.  Briefly, `{\vt ?}' matches any character, '{\vt
*}' matches any set of characters or no characters, ``{\vt [abc]}''
matches the characters `{\vt a}', `{\vt b}', and `{\vt c}', and ``{\vt
a}\{{\vt bc,de}\}'' matches ``{\vt abc}'' and ``{\vt ade}''.  Either
the device {\it args} or the {\it parmlist} can be ``{\vt all}'',
and the default is ``{\vt all, all}'' (``{\vt all}'' is
equivalent to `{\vt *}').  Either the device {\it args} or the {\it
parmlist} can be ``{\vt all}'', and the default is ``{\vt all, all}''. 
If the {\it parmlist} is the keyword ``{\vt none}'', then no
parameters are listed, only the devices with their resolved model
names.  This can be useful for determining which model is actually
used for a MOS device, if L/W model selection is being used.  The
command ``{\vt show -d m*,none}'' will display the name of the model
used for each MOS device.

If {\vt -n} is given, followed by the name of a circuit node, the
output is in the same form as for {\vt -d} however only devices
connected to the named node are displayed.

If {\vt -m} is given, model parameters are printed.  The {\it args}
are model names, and the {\it parmlist} is the list of model
parameters to print.  Wild-carding is accepted in both lists.  The
default is {\vt all, all}.  The parameters are expected to apply to
each model in the list.  See the entries for the various devices and
models for the parameter names, or type the {\cb show} command without
a parameter list to see the current values for all available
parameters for the devices or models mentioned.

Spaces around the ``,'' are optional, as is the ``,'' itself if no
parameters are given.  If no argument is given to the show command,
all parameters of all devices in the current circuit will be
displayed.

The {\vt -D} and {\vt -M} options are similar, but keywords and
descriptions from the internal models are listed, and no values are
shown.  It is not necessary to have a circuit loaded, as it is with
{\vt -d} and {\vt -m}.  The {\it args} are single characters which key
the devices in {\WRspice}, such as `{\vt c}' for capacitors, `{\vt q}'
for bipolar transistors, etc.~.  For devices with a {\vt level} model
parameter such as MOSFETs, an integer indicating the model level can
follow the key argument, without any space.

If these options are given with no argument, the device or model
info is printed for each device or model (both for ``{\vt show
-DM}'') found in the device library.  If an argument is given, only
the matching device or model will be be shown, but all of the
parameters will be listed in addition.  The {\vt -D} option lists
the instance parameters, and {\vt -M} the model parameters, and {\vt
-DM} will list both.  In the listing, the letters `RO' indicate a
read-only parameter, which is a computed quantity not set in the
instance or model lines.  The letters `NR' indicate a parameter that
can't be read, i.e., it is input-only.  Recall that device
parameters can be accessed as vectors with the {\vt @}{\it
devname\/}[{\it param\/}] construct.  There is no {\it parmlist} for
the {\vt -D} and {\vt -M} options.

For example, to print the resistance of all resistors in the current
circuit, enter
\begin{quote}\vt
show -d r*, resistance
\end{quote}
The {\vt -d} above is optional, being the default.  To print the {\vt
cbs} and {\vt cbd} parameters of mosfets {\vt m1}--{\vt m4}
\begin{quote}\vt
show m[1-4], c{\rm \{}bd,bs{\rm \}}
\end{quote}
To print the current value of the relative tolerance option, enter
\begin{quote}\vt
show -o reltol
\end{quote}
Entering
\begin{quote}\vt
show -DM q m5
\end{quote}
will list the instance and model parameters of bipolar transistors and
level 5 (BSIM2) MOSFETs.

%SU-------------------------------------
\subsection{\spcmd{state}}
\index{state command}

% spCommands.hlp:state 011909

The {\cb state} command will print the name and a summary of the state
of the current circuit.  The command takes no arguments.

%SU-------------------------------------
\subsection{\spcmd{status}}
\index{status command}

% spCommands.hlp:status 022519

The {\cb status} command is used to print a list of the ``runops''
currently in force.  The command will print out a summary of all the
{\cb save}, {\cb iplot}, {\cb trace}, {\cb measure}, and {\cb stop},
commands that are active.  Each runop is assigned a unique number,
which can be supplied to the {\cb delete} command to remove the runop. 
The runop list can also be manipulated from the panel brought up with
the {\cb Trace} button in the {\cb Tools} menu.  The command takes no
arguments.

%SU-------------------------------------
\subsection{\spcmd{step}}
\index{step command}

% spCommands.hlp:step 011909

The {\cb step} command allows single-stepping through a transient
simulation.
\begin{quote}\vt
step [{\it number\/}]
\end{quote}
The command will simulate through the number of user output points
given, or one, if no number is given.

%SU-------------------------------------
\subsection{\spcmd{stop}}
\index{stop command}

% spCommands.hlp:stop 032220

The {\cb stop} command will add a stop point to the runop list.
\begin{quote}
{\vt stop} {\it analysis} {\it point} [{\it postcmds\/}]
\end{quote}
When a condition is true, simulation will stop, but can be resumed,
after clearing the stop point, with the {\cb resume} command.  The
stop points can be cleared with the {\cb delete} command, and listed
with the {\cb status} command.  The panel brought up by the {\cb
Trace} button in the {\cb Tools} menu can also be used to manipulate
stop points.

The {\cb stop} command is a ``runop'' similar to the {\cb measure}
command, and is in fact implemented internally from the same
components.  Analogous to {\cb measure}, there is also a {\vt .stop}
input syntax element which uses the same syntax, which will be in
force when simulating the circuit containing the line.  When entered
on the command line, the stop is in force for all circuits, while the
runop is active.

Note that this is a different implementation of the {\cb stop} command
than found in {\WRspice}-4.3.8 and earlier, which was based on the
Berkeley Spice3 implementation.  Although similar, the present syntax
is a little different, and the command has more features and options. 

\begin{description}
\item{\it analysis}\\
This specifies the type of analysis during which the break condition
will be active.  Exactly one of the following keywords should appear
in this field:  {\vt tran}, {\vt ac}, {\vt dc}.  Note that this did
not appear in the {\cb stop} syntax used in {\WRspice}-4.3.8 and
earlier.

\item{\it point}\\
This is precisely the same {\it point} specification as is used in the
{\cb measure} command.  Please refer to that section for a description
of the syntax.  Note that this should cover the pre-4.3.9 syntax,
however an analysis point index is now an integer enclosed is square
brackets, numeric values are now assumed to be scale values (such as
time) otherwise.

\item{\vt repeat} {\it delta}\\
The {\it delta} is a real number scale extent.  After the {\it point}
trigger, the actions are repeated on every multipole of {\it delta}
that follows, as long as a {\vt call} script (see below) returns 1.
\end{description}

There are a couple of optional ``{\it postcmd}'' operations which can
be performed when the stop is triggered, but before simulation ends.

\begin{description}
\item{\vt exec} {\it command}\\
Execute the {\WRspice} shell command found in {\it command\/}, which
should be double-quoted if it contains white space.  Note that
multiple commands can be given, separated by semicolon ({\vt `;'})
characters.  This will be run before a script is called (see below) so
can be used to pass information to the script.  The {\it command} will
be executed only once, and only if the {\it point} condition is
reached.

\item{\vt call} {\it script}\\
After the {\it point} is reached and any command string is executed,
the named script will be called.  The script can be a normal script
file or codeblock.  The special names ``{\vt .exec}'', ``{\vt
.control}'', and ``{\vt .postrun}'' call the exec, control, or postrun
bound codeblocks of the running circuit, if they exist.

The script can be used for additional processing or testing of
whatever sort.  If the script returns 1, the current simulation will
{\bf not} stop, and will continue as if the stop condition never
occurred.  If the script returns 2, a global error is indicated and
the present analysis is terminated.  If any other value is returned,
or there is no explicit return, the analysis will stop as normal, and
can be resumed.

\item{\vt silent}\\
Normally when the stop is activated a message is printed.  If the {\vt
silent} keyword is given, no message will be printed.  Suppressing the
message may be desirable when the stop is being used to terminate
failed Monte Carlo trials, for example, where message output simply
clutters the screen.
\end{description}

If a {\cb stop} command is given at the prompt in interactive mode, it
is placed in the global runop list, and activity will persist until
deleted (with the {\cb delete} command).  If the command is given in a
file, the command will be added to a list for the current circuit, and
will apply only to that circuit.  Thus, for example, a {\WRspice} file
can contain lines like
\begin{quote}\vt
*\# stop tran when ...
\end{quote}
and the action will be performed as that circuit is run, but the
``{\vt stop tran when ...  }'' directive will not apply to other
circuits.  This is the same effect as a {\vt .stop} line.


%SU-------------------------------------
\subsection{\spcmd{sweep}}
\index{sweep command}
\index{loop command}

% spCommands.hlp:sweep 032520

The {\cb sweep} command, which for historical compatibility is also
available as ``{\cb loop}'', is used to perform a simulation analysis
over a range of conditions.

\begin{quote}\vt
sweep [-c] [[[{\it pstr1\/}] {\it min1} {\it max1\/} [{\it step1\/}]]
 [[{\it pstr2\/}] {\it min2\/} [{\it max2\/} [{\it step2\/}]]]
 [{\it analysis\/}]]
\end{quote}
The command works something like a chained dc sweep, running an
analysis over a one or two-dimensional range of parameter values.  The
resulting plot will be multi-dimensional.

There are two fundamental ways in which parameter values can be passed
to the circuit.  In the command, the {\it pstr1} and {\it pstr2} are
text tokens which specify the device parameters to vary, in a format
to be described.  In a two-dimensional sweep, both {\it pstr1} and
{\it pstr2} must be given, or neither can be given.  The two different
value-setting mechanisms can not be mixed.

The specified analysis is performed at each point, yielding
multidimensional output vectors.  If {\it analysis} is omitted, an
analysis specification is expected to be found in the circuit deck.

If a sweep analysis is paused, for example by pressing {\kb Ctrl-C},
it can be resumed by entering the {\cb sweep} command again.  No
arguments are required in this case, however if the {\vt -c} option is
given the old analysis is cleared, and a new analysis started if
further parameters are supplied.  The {\vt -c} is ignored if there was
no sweep analysis in progress.  The {\cb resume} command will also
resume a paused sweep analysis.  The {\cb reset} command given with
the {\vt -c} option will also clear any paused sweep analysis.

\subsubsection{Without explicit device parameter setting}

Assume in this section that the {\it pstr1} and {\it pstr2} parameter
specification strings do not appear, then the the shell variables {\vt
value1} and {\vt value2}, which are accessible in the circuit as {\vt
\$value1} and {\vt \$value2}, are incremented, as in operating range
analysis.  This is the behavior of the historic {\cb loop} command. 
As in operating range analysis, there are various related ways of
introducing the variations.

\begin{enumerate}
\item{
Perhaps the most direct method is to include the forms {\vt
\$value1} and {\vt \$value2} (if two dimensional) for substitution in
the current circuit.  The variables will be replaced by the
appropriate numerical values before each trial, as for shell variable
substitution.
}

\item{
If a variable named ``{\vt value1}'' is set to a string token
with the {\cb set} command, then a variable of the same name as the
string token will be incremented, instead of {\vt value1}.  The same
applies to {\vt value2}.  Thus, for example, if the circuit contains
expansion forms of the variables {\vt foo1} and {\vt foo2} (i.e., {\vt
\$foo1} and {\vt \$foo2}), one could perform a sweep analysis using
these variables as

\begin{quote}
{\vt set value1 = foo1 value2 = foo2}\\
{\vt sweep} ...
\end{quote}
}

\item{
The method above allows the SPICE options to be iterated.  These are
the built-in keywords, which can be set with the {\cb set} command or
in a {\vt .options} line in an input file, which control or provide
parameters to the simulation.

The most important example is temperature sweeping, using the
{\vt temp} option.  A temperature sweep would look like

\begin{quote}
{\vt set value1=temp}\\
{\vt sweep -50 50 25} {\it analysis}
\end{quote}

This will run the analysis at -50, -25, 0, 25, and 50 Celsius.
}

\item{
If there are existing vectors named ``{\vt checkN1}'' and (if two
dimensions) ``{\vt checkN2}'' that contain integer values, and the
variable named ``{\vt value}'' is set to the name of an existing
vector (or a vector named ``{\vt value}'' exists), then the vector
components indexed by {\vt checkN1} and {\vt checkN2} will be
iterated, if within the size of the vector.  For example:

\begin{quote}
{\vt let vec[10] = 0}\\
{\vt let checkN1 = 5 checkN2 = 6}\\
{\vt set value = vec}\\
{\vt sweep} ...
\end{quote}

The first line creates a vector named ``{\vt vec}'' of size sufficient
to contain the indices.  The iterated values will be placed in {\vt
vec[5]} and {\vt vec[6]}.  The circuit should reference these values,
either through shell substitution (e.g., {\vt \$\&vec[5]}) or directly
as vectors.

Alternatively, a variable named ``{\vt checkN1}'' can be set.  If the
value of this variable is an integer, that integer will be used as the
index.  If the variable is a name token, then the index will be
supplied by a vector of the given name.  The same applies to {\vt
checkN2}.  The following example illustrates these alternatives:

\begin{quote}
{\vt let vec[10] = 0}\\
{\vt set checkN1 = 5}\\
{\vt let foo = 6}\\
{\vt set checkN2 = foo}\\
{\vt sweep} ...
\end{quote}
}

\item{
Given that it is possible to set a vector as if a variable, by using
the {\cb set} command with the syntax

\begin{quote}
{\vt set \&}{\it vector} {\vt =} {\it value}
\end{quote}

it is possible to iterate vectors with the {\cb sweep} command.  The
form above is equivalent to

\begin{quote}
{\vt let} {\it vector} {\vt =} {\it value}
\end{quote}

Note, however, that the `{\vt \&}' character has special significance
to the {\WRspice} shell, so when this form is given on the command
line the ampersand should be quoted, e.g., by preceding it with a
backslash.

Thus, suppose that the circuit depends on a vector named {\vt delta}. 
One can set up iteration using this vector as

\begin{quote}
{\vt set value1 = '\&delta'}\\
{\vt sweep} ...
\end{quote}
}

\item{
The construct above can be extended to ``special'' vectors, which
enable device and model parameters to be set ahead of the next
analysis.  These special vectors have the form

\begin{quote}
{\vt @}{\it devname\/}{\vt [}{\it param\/}{\vt ]}
\end{quote}

where {\it devname} is the name of a device or model in the circuit,
and {\it param} is one of the parameter keywords for the device or
model.  These keywords can be listed with the {\cb show} command.

For example, if the circuit contains a MOS device {\vt m1} one might
have

\begin{quote}
{\vt set value1 = '\&@m1[w]'}\\
{\vt sweep 1.0u 2.0u 0.25u} {\it analysis}
\end{quote}

This will perform the analysis while varying the {\vt m1} {\vt w}
(device width) parameter from 1.0 to 2.0 microns in 0.25 micron
increments.
}
\end{enumerate}

\subsubsection{Explicit parameter setting}
\label{psetting}

If parameters specifications ({\it pstr1} and {\it pstr2\/}) are
given, there is no variable or vector setting.  Instead, the indicated
device parameters are altered directly, very similar to the {\cb
alter} command.

The syntax for the {\it pstr} strings is very similar to the
device/parameter lists accepted by the {\cb show} command.

\begin{quote}
{\it device\_list\/}{\vt ,}{\it param\_list}
\end{quote}

The {\it device\_list} is a list of one or more device names found in
the circuit.  The names are separated by white space, and the list is
terminated with a comma.  Following the comma is one or more parameter
names.  A parameter name is a device or model keyword, which should be
applicable to all of the names listed in the {\it device\_list}.  The
device keywords can be obtained from the {\cb show} command.

The {\it device\_list} can contain ``globbing'' (wild-card)
characters with similar behavior to globbing (global substitution) in
the {\WRspice} shell.  Briefly, `{\vt ?}' matches any character, '{\vt
*}' matches any set of characters or no characters, ``{\vt [abc]}''
matches the characters `{\vt a}', `{\vt b}', and `{\vt c}', and ``{\vt
a}\{{\vt bc,de}\}'' matches ``{\vt abc}'' and ``{\vt ade}''.

If the string contains white space, it must be quoted.  Since the same
range is applied to all the parameters, if would be unusual to list
more than one parameter name.  However, wildcarding or multiple names
in the device list allows setting the values of arbitrarily many
devices for each trial.

If the {\it device\_list} contains only a single name, and the name is
a voltage or current source, resistor, capacitor, or inductor, then
the comma and parameter name can be omitted.  It will be taken as the
output of a source, or the resistance, capacitance, or inductance of
the component.

%SU-------------------------------------
\subsection{\spcmd{tf}}
\index{tf command}

% spCommands.hlp:tf 011909

The {\cb tf} command will initiate a transfer function analysis of the
current circuit.
\begin{quote}\vt
tf {\it tf\_args} [dc {\it dc\_args\/}]
\end{quote}
The arguments appear as they would in a {\vt .tf} line (see
\ref{tfline}) in the input file.  If a dc sweep specification follows,
the transfer function analysis will be performed at each dc sweep
point (see \ref{dcline}).

%SU-------------------------------------
\subsection{\spcmd{trace}}
\index{trace command}

% spCommands.hlp:trace 051219

The {\cb trace} command will add a ``runop'' which prints the value of
the expression(s) at each user analysis point.
\begin{quote}\vt
trace {\it expr\/} [...]
\end{quote}
At each time point, the expressions on the command line will be
evaluated, and their values printed on the standard output.

If a trace command is given at the prompt in interactive mode, it is
placed in a global list, and activity will persist until deleted (with
the {\cb delete} command).  If the command is given in a file, the
command will be added to a list for the current circuit, and will
apply only to that circuit.  Thus, for example, a {\WRspice} file can
contain lines like
\begin{quote}
{\vt *\# trace v(1)}
\end{quote}
and the trace will be performed as that circuit is run, but the ``{\vt
trace v(1)}'' directive will not apply to other circuits.

The traces in effect can be listed with the {\cb status} command,
deleted with the {\cb delete} command, and otherwise manipulated from
the panel brought up with the {\cb Trace} button in the {\cb Tools}
menu.

%SU-------------------------------------
\subsection{\spcmd{tran}}
\index{tran command}

% spCommands.hlp:tran 011909

The {\cb tran} command initiates transient analysis of the current
circuit.
\begin{quote}\vt
tran {\it tran\_args} [dc {\it dc\_args\/}]
\end{quote}
The arguments are the same as those of a {\vt .tran} line (see
\ref{tranline}).  Output is retained at {\it tstart}, {\it tstop}, and
multiples of {\it tstep} in between, unless the variable {\et
steptype} is set to {\vt nousertp}.  In this case, output is retained
at each internally generated time point in the range.  If a dc sweep
specification follows, the transient analysis is performed at each
sweep point.

%SU-------------------------------------
\subsection{\spcmd{vastep}}
\index{vastep command}

% spCommands.hlp:vastep 032420

This command has application when simulating a circuit containing
Verilog block, and the option {\et vastep} has been set to 0.  In this
case, a call to this command will advance the Verilog simulation to
the next clock tick.  This function can be called from a callback
initiated from a {\vt .stop} line, used when co-simulating Verilog and
SPICE.

%SU-------------------------------------
\subsection{\spcmd{where}}
\index{where command}

% spCommands.hlp:where 012209

The {\cb where} command, which takes no arguments, prints information
about the last nonconvergence, for debugging purposes.


%S-----------------------------------------------------------------------------
\section{Data Manipulation Commands}

% spCommands.hlp:datacmds 091714

The following commands perform various operations on vectors, which
are the basic data structures of {\WRspice}.  Vectors from the current
plot can be referenced by name.  A listing of the vectors for the
current plot is obtained by typing the {\cb let} or {\cb display}
commands without arguments, or pressing the {\cb Vectors} button in
the {\cb Tools} menu.  Vectors for other than the current plot are
referenced by {\it plotname\/}.{\it vecname}, for example, {\vt
tran2.v(1)}.  The current plot can be changed with the {\cb setplot}
command, or from the panel brought up by the {\cb Plots} button in the
{\cb Tools} menu.  See \ref{vectors} for more information about
vectors.

Vectors can be created and manipulated in many ways.  For example, typing
\begin{quote}
{\vt let diff = v(1) - v(2)}
\end{quote}
creates a new vector {\et diff}.  All vectors can be printed, plotted, or
used in expressions.  They can be deleted with the {\cb unlet} command.

\begin{tabular}{|l|l|}\hline
\multicolumn{2}{|c|}{Data Manipulation Commands}\\ \hline
\cb compose & Create vector\\ \hline
\cb cross & Vector cross operation\\ \hline
\cb define & Define a macro function\\ \hline
\cb deftype & Define a data type\\ \hline
\cb diff & Compare plots and vectors\\ \hline
\cb display & Print vector list\\ \hline
\cb fourier & Perform spectral analysis\\ \hline
\cb let & Create or assign vectors\\ \hline
\cb linearize & Linearize vector data\\ \hline
\cb pick & Create vector from reduced data\\ \hline
\cb seed & Seed random number generator\\ \hline
\cb setdim & Set current plot dimensions\\ \hline
\cb setplot & Set current plot\\ \hline
\cb setscale & Assign scale to vector\\ \hline
\cb settype & Assign type to vector\\ \hline
\cb spec & Perform spectral analysis\\ \hline
\cb undefine & Undefine macro function\\ \hline
\cb unlet & Undefine vector\\ \hline
\cb vastep & Advance Verilog simulator\\ \hline
\end{tabular}

%SU-------------------------------------
\subsection{\spcmd{compose}}
\index{compose command}

% spCommands.hlp:compose 032220

The {\cb compose} command is used to create vectors.  It has three forms:
\begin{quote}\vt
compose {\it vecname} values {\it value} [...]\\
{\rm or}\\
compose {\it vecname} pattern [{\it n1} [{\it n2\/}]] {\it bstring}
[{\it modifiers\/}] [...]\\
{\rm or}\\
compose {\it vecname} {\it param = value} [...]
\end{quote}

All forms of this command create a new vector called {\it vecname\/}. 
In the first form, indicated by the keyword ``{\vt values}'', the
given values are concatenated to form the vector.  The second form
creates a binary pattern of 1 and 0 values based on the given pattern
specification to be described.  In the third form, the values in the
vector are determined by the parameters given, as will be described.

The pattern generation syntax given in the second form is as used in
pulse source.  See \ref{pattern} for a complete description of the
{\it bstring} and {\it modifier} syntax and usage.  The optional
preceding numbers {\it n1} and {\it n2} apply only when infinite
repetition ({\vt r=-1}) is given in the {\it modifier\/}:  neither
given will will cause the vector to end at the sequence end, with no
repeat.  If {\it n1} only is given, this number is taken as the
maximum length of the vector.  If both {\it n1} and {\it n2} are
given, the two numbers are taken as {\it step} and {\it stop}, with
the maximum length {\it stop}/{\it step}.  This is convenient for
setting the length to match a transient analysis specification.

In the third form, there are three groups of possible parameter sets. 
The first set facilitates creation of uniform arrays.  This set
contains the following parameters.

\begin{quote}
\begin{tabular}{|l|l|}\hline
\vt start & The value at which the vector should start\\ \hline
\vt stop & The value at which the vector should end\\ \hline
\vt step & The difference between successive elements\\ \hline
\vt lin & The number of points, linearly spaced\\ \hline
\vt log & The number of points, logarithmically spaced\\ \hline
\vt dec & The number of points per decade, logarithmically spaced\\ \hline
\end{tabular}
\end{quote}

The words ``{\vt len}'' and ``{\vt length}'' are synonyms for ``{\vt
lin}''.  A subset of these parameters that provides the information
needed is sufficient.  If all four are given, the point count and step
value must be consistent or the command will fail.  The parameter {\vt
start} defaults to zero, unless implicitly set by other parameters. 
The {\vt stop} and {\vt step} have no defaults and must be supplied
unless implied by other parameters.  If the {\vt lin} parameter is not
given, the other parameters determine the vector length.

The second parameter group generates Gaussian random values.

\begin{quote}
\begin{tabular}{|l|l|}\hline
\vt gauss & The number of points in the gaussian distribution\\ \hline
\vt mean & The mean value for the gaussian distribution\\ \hline
\vt sd & The standard deviation for the gaussian distribution\\ \hline
\end{tabular}
\end{quote}

The {\vt gauss} parameter is required, {\vt sd} defaults to 1.0, and
{\vt mean} defaults to 0.  The random number sequences can be reset by
calling the {\cb seed} command.

The third parameter group generates uniform random values.

\begin{quote}
\begin{tabular}{|l|l|}\hline
\vt random & The number of randomly selected points\\ \hline
\vt center & Where to center the range of points\\ \hline
\vt span & The size of the range of points\\ \hline
\end{tabular}
\end{quote}

The {\vt random} parameter is required, {\vt span} defaults to 2.0,
and {\vt center} defaults to 0.  The random number sequences can be
reset by calling the {\cb seed} command.

%SU-------------------------------------
\subsection{\spcmd{cross}}
\index{cross command}

% spCommands.hlp:cross 012009

The {\cb cross} command creates a new vector.
\begin{quote}\vt
cross {\it vecname number} {\it source} [...]
\end{quote}
The vector is constructed, with name {\it vecname} and values
consisting of the {\it number\/}'th element of each of the source
vectors.  If the index is out of range for a vector, 0 is taken.

%SU-------------------------------------
\subsection{\spcmd{define}}
\index{define command}

% spCommands.hlp:define 092312

The {\cb define} command is used to specify user-defined vector
functions.
\begin{quote}\vt
define [{\it function\/}({\it arg1\/}, {\it arg2\/}, ...)]
[=] [{\it expression\/}]
\end{quote}
This will define the user-definable function with the name {\it
function\/} and arguments {\it arg1\/}, {\it arg2\/}, ...  to be {\it
expression}, which will usually involve the arguments.  When the
function is called, the arguments that are given are substituted
for the formal arguments.

The {\cb define} command and the {\vt .param} line in input files can
be used to define user-defined functions (UDFs).  User-defined
function definitions are modularized and prioritized.  At the base of
the hierarchy (with lowest priority) are the "shell" UDFs which are
defined with the {\cb define} command.

Every circuit has its own set of UDFs, which are obtained from {\vt
.param} lines which are not part of a subcircuit.  When a circuit is
the current circuit, its UDFs will be searched before the shell UDFs
to resolve a function reference.  The current circuit's UDF database
is pushed onto a stack, ahead of the shell UDFs.  Most of the time,
this stack is two levels deep.

During initial circuit processing, when subcircuit expansion is being
performed, when a subcircuit is bing expanded, any functions defined
within the {\vt .subckt} text with {\vt .param} lines are pushed on
the top of the stack.  Since subcircuit definitions may be nested,
functions will be pushed/popped according to the depth in the
hierarchy currently being processed.
  
Thus, a function defined in a subcircuit will have priority over a
function of the same name and argument count defined in the circuit
body, and a function defined in the circuit body will have priority
over a function with the same name and argument count defined from the
shell with the {\cb define} command.

When {\cb define} is given without arguments, all currently defined
functions are listed.  Those definitions from the current circuit will
be shown with an asterisk `{\vt *}' in the first column.  Other
functions listed have been defined with the {\cb define} command.  The
functions defined in subcircuits are invisible, their use is only
transient and they are part of the database only during subcircuit
expansion.

If only a function name is given, any definitions for functions with
the given name are printed.

It is possible to define a function that calls a non-existing
function.  The resolution is done when the function is evaluated. 
Thus, functions of functions can be defined in any order.

Note that one may have different functions defined with the same name
but different argument counts.  Some useful definitions (which are
part of the default environment) are:
\begin{quote} \vt
define max(x,y) x > y ? x : y\\
define min(x,y) x < y ? x : y
\end{quote}

%SU-------------------------------------
\subsection{\spcmd{deftype}}
\index{deftype command}

% spCommands.hlp:deftype 012009

The {\cb deftype} command defines a new data type.
\begin{quote}\vt
    deftype v {\it typename} [{\it abbrev\/}]\\
    deftype p {\it plottype} [{\it pattern} ...]
\end{quote}
This is an obscure command that might be useful for exporting rawfile
data to other programs.  If a vector's value indicates furlongs per
fortnight, its type can be so defined.  However, user-defined types
are not compatible with the internal {\WRspice} type propagation
logic.  Vectors with user-defined types, or results involving
user-defined types, will be treated as untyped in {\WRspice}.

The first form defines a new type for vectors.  The {\it typename} may
then be used as a vector type specification in a rawfile.  If an {\it
abbrev} is given, values of that type can be named {\it abbrev}({\it
something}) where {\it something} is the name given in the rawfile
(and {\it something} doesn't contain parentheses).

The second form defines a plot type.  The (one word) name for a plot
with any of the patterns present in its plot type name as given in the
rawfile will be {\it plottype}N, where N is a positive integer
incremented every time a rawfile is read or a new plot is defined.

%SU-------------------------------------
\subsection{\spcmd{diff}}
\index{diff command}

% spCommands.hlp:diff 012009

The {\cb diff} command compares vectors in different plots.
\begin{quote}\vt
diff {\it plot1 plot2} [{\it vecname} ...]
\end{quote}
The command will compare all the vectors in the specified
plots, or only the named vectors if any are given.  If there are
different vectors in the two plots, or any values in the vectors
differ significantly, the difference is reported.  The variables {\et
diff\_abstol}, {\et diff\_reltol}, and {\et diff\_vntol} are used to
determine if two values are ``significantly'' different.

%SU-------------------------------------
\subsection{\spcmd{display}}
\index{display command}

% spCommands.hlp:display 012009

The {\cb display} command prints information about the named vectors,
or about all vectors in the current plot if no names are given.
\begin{quote}\vt
display [{\it vecname} ...]
\end{quote}
This command will list the names, types and lengths of the vectors,
and whether the vector is real or complex.

Additional information is also given:  if there is a minimum or
maximum value for the vector defined, this is listed (see
\ref{rawfilefmt} for the manner in which this and the rest of the
per-vector parameters are defined), if there is a default grid type or
a default plot type, they are mentioned, and if there is a default
color or a default scale for the vector it is noted.  Additionally,
one vector in the plot will have the notation {\vt [default scale]}
appended --- this vector will be used as the x-scale for the {\cb
plot} command if none is given or if the vectors named have no default
scales of their own.  See the {\cb plot} command (\ref{plotcmd}) for
more information on scales.

The vectors are sorted by name unless the variable {\et nosort} is
set.  The {\cb let} command without arguments is equivalent to the
{\cb display} command without arguments.

%SU-------------------------------------
\subsection{\spcmd{fourier}}
\index{fourier command}

% spCommands.hlp:fourier 012009

The {\cb fourier} command performs Fourier analysis.
\begin{quote}\vt
fourier {\it fundamental\_frequency} [{\it value} ...]
\end{quote}
The command initiates a fourier analysis of each of the given values,
using the first 10 multiples of the fundamental frequency (or the
first {\vt nfreqs}, if that variable is set).  The values may be any
valid expression.  They are interpolated onto a fixed-space grid with
the number of points given by the {\et fourgridsize} variable, or 200
if it is not set.  The interpolation will be of degree {\et
polydegree} if that variable is set, or 1.  If {\et polydegree} is 0,
then no interpolation will be done.  This is likely to give erroneous
results if the time scale is not monotonic.  This command is executed
when a {\vt .four} line is present in the input file and {\WRspice} is
being run in batch mode.

%SU-------------------------------------
\subsection{\spcmd{let}}
\index{let command}

% spCommands.hlp:let 060314

The {\cb let} command is used to assign vectors.
\begin{quote}\vt
let [{\it vecname} [= {\it expr\/}]] [{\it vecname} = {\it expr\/} ...]
\end{quote}
With no arguments, the list of vectors from the current plot is
printed, similar to the {\cb display} command.  If one or more
arguments appear without an assignment, information about the named
vectors is printed, similar to the {\cb display} command.  Otherwise,
for each assignment, if {\it vecname} does not exist, a new vector is
created with name {\it vecname} and value given by the expression {\it
expr\/}.  An existing vector with the given name will be overwritten
with new data.

In {\WRspice} releases prior to 3.0.9, only a single assignment could
appear in a {\cb let} command.  In current releases, any number of
assignemnts can be given in a single command line.  The assignments
are performed left-to-right, so that expressions to the right of an
assignment may make use of that assignment, i.e., forms like
\begin{quote}
\vt let a=1 b=a
\end{quote}
work properly.

None of the vector options such as default scale, color, etc.  that
are read from the rawfile are preserved when a vector is created with
the {\cb let} command.

The {\it vecname} above can actually be in the {\it plotname\/}.{\it
vecname} format, where the {\it plotname} is the name of a plot or one
of the plot aliases as described in \ref{vectors}.  In this case, only
the indicated plot will be searched for a vector named {\it
vecname\/}, and if not found, a new vector of that name will be
created in the indicated plot.

If no plot is specified, a search for {\it vecname} will occur in the
current plot, then the context plot if any, and finally the {\vt
constants} plot.  If a match is found, that vector will be reused,
which may not be what is intended.  When a scrpt is run, the current
plot at the time the script starts is saved as the ``context plot''. 
Vectors created in the script before any change in the current plot
are saved in the context plot.  If the script runs an analysis, the
current plot will change, but the previously defined variables will
still be available by name as the context plot will be searched as
well as the current plot.

If the intention is to use or create a vector in the current plot,
the form
\begin{quote}
{\vt let curplot}.{\it vecname} {\vt =} {\it expr}
\end{quote}
should be used, if there is any chance of ambiguity.

The syntax
\begin{quote}
{\vt let a[N] = {\it vec\/}}
\end{quote}
with {\vt N} a non-negative integer, is valid.  If {\it vec\/} is a
vector, then {\vt a[N] = {\it vec\/}[0]}, {\vt a[N+1] = {\it
vec\/}[1]}, etc., If undefined, {\vt a} is defined, and new entries
that are not explicitly set are zeroed.  The length of {\vt a} is set
or modified to accommodate {\it vec\/}.  The syntax {\vt a[0] = {\it
vec\/}} is also valid, and is equivalent to {\vt a = {\it vec\/}}.  If
{\it vec\/} is a vector, then {\vt a} is a copy of {\it vec\/}.  If
{\it vec\/} is a scalar (unit length vector), then {\vt a} is also a
scalar with the value of {\it vec\/}. 

When assignment is from a scalar value, any SPICE number format may be
used.  That is, if alpha characters appear after a number, the initial
characters are checked as a scale factor.  Recognized sequences are t,
g, k, u, n, p, f, m, meg, mil.  Remaining characters are parsed as a
units string.  This is all case insensitive.

The units suffix of a constant value is used to assign the units
of any vector to which the constant is assigned.  This means, for
example, in
\begin{quote}\vt
    let a = v(1)/15o
\end{quote}
{\vt a} has units of current (A).  Use the {\cb settype} command
without arguments to see a list of recognized types.

The ``let'' is actually optional; the {\cb let} command will be
applied to a line with the second token being ``=''.  This is somewhat
less efficient, however.

%SU-------------------------------------
\subsection{\spcmd{linearize}}
\index{linearize command}

% spCommands.hlp:linearize 012009

This {\cb linearize} command is used to create linearized vectors from
vectors whose scales are not evenly spaced. 
\begin{quote}\vt
linearize [{\it vecname\/} ...]
\end{quote}
The command will force data from a transient analysis to conform to a
linear scale, if the plot has been created using raw timepoints.  This
is the case only when the {\et steptype} variable is set to ``{\vt
nousertp}''. 

The {\cb linearize} command will create a new plot with all of the
vectors in the current plot, or only those mentioned if arguments are
given.  The new vectors will be interpolated onto a linear time scale,
which is determined by the values of {\vt tstep}, {\vt tstart}, and
{\vt tstop} in the currently active transient analysis.  The currently
loaded deck must include a transient analysis, or a {\cb tran} command
may be run interactively, and the current plot must be from this
transient analysis.  The variable {\et polydegree} determines the
degree of interpolation.

%SU-------------------------------------
\subsection{\spcmd{pick}}
\index{pick command}

% spCommands.hlp:pick 012009

The {\cb pick} command creates a new vector from elements of other vectors.
\begin{quote}\vt
pick {\it vecname offset period vector} [{\it vector ...\/}]
\end{quote}
The command creates a vector {\it vecname} and fills it with every
{\it period\/}'th value starting with {\it offset} from the vectors. 
The {\it offset} and {\it period} are integers.  For example, for
\begin{quote}\vt
    pick xx 1 2 v1 v2
\end{quote}
we obtain
\begin{quote}\vt
  xx[0] = v1[1]\\
  xx[1] = v2[1]\\
  xx[2] = v1[3]\\
  xx[3] = v2[3]\\
\end{quote}
and so on.

%SU-------------------------------------
\subsection{\spcmd{seed}}
\index{seed command}

% spCommands.hlp:seed 032715

The {\cb seed} command will reset the internal random number generator.
\begin{quote}\vt
seed [{\it seed\_integer\/}]
\end{quote}
The {\it seed\_integer\/}, if given, will be used to seed the new
random number sequence.  This affects the statistical functions in
\ref{statfuncs} and other functions that generate random values. 
Setting the seed explicitly enables the sequence of ``random'' values
returned from these functions to be repeatable (the default seed is
random).

%SU-------------------------------------
\subsection{\spcmd{setdim}}
\index{setdim command}

% spCommands.hlp:setdim 022015

The {\cb setdim} command allows the dimensions of the current plot to
be changed.

\begin{quote}
{\vt setdim} [{\it numdims} [{\it dim} ...]]
\end{quote}
  
If given without arguments, the length and dimensions of the scale
vector of the current plot are printed.  Otherwise, all arguments are
non-negative numbers.  There should be {\it numdims\/}-1 ``{\it
dims}'' given.  The {\it numdims} is the new dimensionality of the
plot.  Values of 0--8 are allowed.  The sub-dimensions that follow are
integers 2 or larger.

The dimension list must be compatible with the existing plot
dimensions in that the total number of points must remain the same,
and the size of the basic vector (scale period) remains the same.

There is a special case where the {\it numdims} is the same as the
vector length.  The plot will become multidimensional, with each
dimension having one point.  There is no limit to the number of
dimensions in this case.  Such vectors plot as collections of
multi-colored points.  This type of plot is generated normally by, for
example, use of the {\cb loop} command to repeat op analysis. 
Additional argumens to the command are ignored.

Giving {\it numdims} a value of 0 or 1 will set to ``no''
dimensionality, the status of a regualar vector.

%SU-------------------------------------
\subsection{\spcmd{setplot}}
\index{setplot command}

% spCommands.hlp:setplot 040410

The {\cb setplot} command can be used to set the current plot, or to
create a new, empty plot and make it the current plot.
\begin{quote}\vt
setplot [{\it plotname\/}]
\end{quote}
Here, the word ``plot'' refers to a group of vectors that are the
result of one {\WRspice} simulation run.  Plots are created in memory
during a simulation run, or by loading rawfile data.  When more than
one file is loaded in, or more than one plot is present in one file,
{\WRspice} keeps them separate and only shows the vectors in the
current plot.  One generally accesses a given plot by first making it
the current plot. 

The same functionality is available from the {\cb Plots} button in the
{\cb Tools} menu.  The {\cb setplot} command will set the current plot
to the plot with the given {\it plotname\/}, or if no name is given,
prompt the user with a menu.  The plots are named as they are loaded,
by reading in a rawfile, or created by running a simulation, with
names like {\vt tran1} or {\vt ac2}.  These names are shown by the
{\cb setplot} and {\cb display} commands and are used by other
commands.

The {\it plotname} can also be a numerical index.  Plots are saved in
the order created, and as listed by the {\cb setplot} command without
arguments, and in the {\cb Plots} tool.  In addition to the plot name,
the following constructs are recognized.  Below, {\it N} is an
integer.

\begin{description}
\item{\vt -}{\it N}\\
Use the {\it N\/}'th plot back from the current plot.  {\it N} must be
1 or larger.  For example, ``{\vt setplot -1}'' will set the current
plot to the previous plot.  The command will fail if there is no such
plot.

\item{\vt +}{\it N}\\
This goes in the reverse direction, indicating a plot later in the 
list than the current plot.

\item{\it N}\\
An integer without {\vt +} or {\vt -} indicates an absolute index into
the plot list, zero-based.  The value 0 will always indicate the
``constants'' plot, which is the first plot created (on program
startup).
\end{description}

If the {\it plotname} is ``{\vt new}'', a new plot is created, which
becomes the current plot.  This plot has no vectors.

The current plot can also be changed by resetting the {\et curplot}
variable.  There are three read-only variables which are reset
internally whenever the current plot changes.  Each contains a string
describing a feature of the current plot.  These are {\et
curplotdate}, {\et curplotname}, and {\et curplottitle}.

%SU-------------------------------------
\subsection{\spcmd{setscale}}
\index{setscale command}

% spCommands.hlp:setscale 012009

The {\cb setscale} command is used to set the vector used as a scale
when plotting other vectors.
\begin{quote}\vt
setscale [{\it plot\/} or {\it vector\/}] [{\it vectors} ...]
\end{quote}
Each plot has a default scale, which can be set with this command. 
Each vector has a scale variable, which if set will override the
default scale of the plot.  These also can be set with this command. 
This command takes as input the names of a plot and a new scale vector
in that plot, or the names of vectors from the current plot.  The
wildcard forms using ``{\vt all}'' and the plot prefix form {\it
plot\/}.{\it vector} are not allowed in this command.  If only one
argument is given, i.e.
\begin{quote}
{\vt setscale} {\it vector}
\end{quote}
then {\it vector} is assigned as the default scale of the current plot.
The vector must already exist in the current plot.

If two arguments are given, the first argument is initially
interpreted as the name of a plot, and the second argument is the name
of a vector in that plot to use as the scale.  The plot has names like
``{\vt tran1}'' or ``{\vt ac2}'' and the {\it vector} must exist in
that plot.

If the first argument is not a plot name, or there are more than two
arguments, the arguments are expected to be vectors in the current
plot, and the last vector will be assigned as the scale for the other
listed vectors.

The scales assigned to vectors can be removed by assigning the vector
that is the current default scale for the plot, or the scale vector
name given can have the special names ``{\vt none}'' or ``{\vt
default}''.  The scale for plots can't be removed, since a plot must
always have a default scale (if any vectors are defined).

The {\cb let} command without arguments lists the vectors and will
show the scales, if any.

%SU-------------------------------------
\subsection{\spcmd{settype}}
\index{settype command}
\label{settype}

% spCommands.hlp:settype 012009

The {\cb settype} command is used to change the data types
of the vectors in a plot.
\begin{quote}\vt
settype [{\it type\/}] [{\it vector} ...]
\end{quote}
The command will change the type of the named vectors to {\it type\/}. 
With no arguments, the list of recognized types and abbreviations is
printed.  The {\it type} field can consist of a single name, or a
single token containing a list of abbreviations.  The token list can
contain a digit power after an abbreviation, and a single `/' for
denominator units.  Examples are ``F/M2'', ``Wb2/Hz''.  Units of
vectors generated during analysis are set automatically.

The {\WRspice} numerical input format (see \ref{number}) allows the
type to be specified when a value is given to {\WRspice}, either
interactively or in an input file.

Type names can also be found in the description of the rawfile format
in \ref{rawfilefmt}, or they may be defined with the {\cb deftype}
command.  However, only the primitive types listed below propagate
through expressions and are recognized by the {\WRspice}
type-propagation system.

The primitive built-in types and abbreviations are:
\begin{quote}
\begin{tabular}{|l|c|}\hline
\vt time        & S\\ \hline
\vt frequency   & Hz\\ \hline
\vt voltage     & V\\ \hline
\vt current     & A\\ \hline
\vt charge      & Cl\\ \hline
\vt flux        & Wb\\ \hline
\vt capacitance & F\\ \hline
\vt inductance  & H\\ \hline
\vt resistance  & O\\ \hline
\vt conductance & Si\\ \hline
\vt length      & M\\ \hline
\vt area        & M2\\ \hline
\vt temperature & C\\ \hline
\vt power       & W\\ \hline
\end{tabular}
\end{quote}

The codes from the rawfile are:
\begin{quote}
\begin{tabular}{|l|c|c|}\hline
Name         & Description          & SPICE2 Numeric Code\\ \hline\hline
{\vt notype}       & Dimensionless value  & 0\\ \hline
{\vt time}         & Time                 & 1\\ \hline
{\vt frequency}    & Frequency            & 2\\ \hline
{\vt voltage}      & Voltage              & 3\\ \hline
{\vt current}      & Current              & 4\\ \hline
{\vt output-noise} & SPICE2 .noise result & 5\\ \hline
{\vt input-noise}  & SPICE2 .noise result & 6\\ \hline
{\vt HD2}          & SPICE2 .disto result & 7\\ \hline
{\vt HD3}          & SPICE2 .disto result & 8\\ \hline
{\vt DIM2}         & SPICE2 .disto result & 9\\ \hline
{\vt SIM2}         & SPICE2 .disto result & 10\\ \hline
{\vt DIM3}         & SPICE2 .disto result & 11\\ \hline
{\vt pole}         & SPICE3 pz result     & 12\\ \hline
{\vt zero}         & SPICE3 pz result     & 13\\ \hline
\end{tabular}
\end{quote}

%SU-------------------------------------
\subsection{\spcmd{spec}}
\index{spec command}

% spCommands.hlp:spec 012009

The {\cb spec} command will create a new plot consisting of the
Fourier transforms of the vectors given on the command line.
\begin{quote}\vt
spec {\it start\_freq stop\_freq step\_freq vector} [...]
\end{quote}
This is based on a SPICE3 {\cb spec} command by Anthony Parker of
Macquarie University in Sydney Australia, which is available as part
of the patch set from\\
{\vt http://www.elec.mq.edu.au/cnerf/spice/spice.html}.

The command will create a new plot consisting of the Fourier
transforms of the vectors given on the command line.  Each vector
given should be a transient analysis result, i.e., have time as a
scale, and each should have the same time scale.  The Fourier
transform will be computed using the frequency parameters given, and
will use a window function as given with the {\et specwindow}
variable.

The following variables control operation of the {\cb spec} command. 
Each can be set with the {\cb set} command, or equivalently from the
{\cb Fourier} tab of the {\cb Commands} tool.

\begin{description}
\index{spectrace variable}
\item{\et spectrace}\\
This enables messages to be printed during Fourier analysis with the
{\cb spec} command, for debugging purposes.

\index{specwindow variable}
\item{\et specwindow}\\
This variable is set to one of the following strings, which will
determine the type of windowing used for the Fourier transform in the
{\cb spec} command.  If not set, the default is {\vt hanning}.

\begin{tabular}{ll}
\vt bartlet & Bartlet (triangle) window\\
\vt blackman & Blackman order 2 window\\
\vt cosine & Hanning (cosine) window\\
\vt gaussian & Gaussian window\\
\vt hamming & Hamming window\\
\vt hanning & Hanning (cosine) window\\
\vt none & No windowing\\
\vt rectangular & Rectangular window\\
\vt triangle & Bartlet (triangle) window\\
\end{tabular}

\index{specwindoworder variable}
\item{\et specwindoworder}\\
This can be set to an integer in the range 2--8.  This sets the order
when the gaussian window is used in the {\cb spec} command.  If not
set, order 2 is used.
\end{description}

%SU-------------------------------------
\subsection{\spcmd{undefine}}
\index{undefine command}

% spCommands.hlp:undefine 012009

The {\cb undefine} command is used to undefine user-defined functions
that have previously been defined with the {\cb define} command.
\begin{quote}\vt
undefine {\it word} [...]
\end{quote}
The command deletes the definitions of the user-defined functions
passed as arguments.  If the argument is ``*'', then all macro
functions are deleted.  Note that all functions with the given names
are removed, so there is no way to delete a function with a particular
argument count without deleting all functions with that name.

%SU-------------------------------------
\subsection{\spcmd{unlet}}
\index{unlet command}

% spCommands.hlp:unlet 012009

The {\cb unlet} command will delete the vectors listed as arguments. 
\begin{quote}\vt
unlet {\it vecname} [...]
\end{quote}
The current plot is assumed, though the {\it plot\/}.{\it vector}
notation is accepted.  When the default scale vector is deleted,
another random vector will become the default scale.  The names can be
``{\vt all}'', indicating that all matching vectors should be removed. 
If the vector name is ``{\vt all}'', all vectors in the plot are
removed, but the plot itself is not deleted.  Giving ``{\vt all.all}''
will clear the vectors in all plots (not very useful).  To delete a
plot, use the {\cb destroy} or {\cb free} commands.


%S-----------------------------------------------------------------------------
\section{Graphical Output Commands}
\label{grout}

% spPlot.hlp:graphcmds 012209

The following commands display the output of simulations graphically,
either on-screen or on a printing device.  Many take as input a list of
vectors or expressions to plot, and in some cases ambiguities may arise.
An example would be
\begin{quote}
{\vt plot v(1) -v(2)}
\end{quote}
which would be interpreted as a plot of the difference between the 
vectors (1 trace) rather than two traces.  To resolve such ambiguities, 
double quotes may be used, as in 
\begin{quote}
{\vt plot v(1) "-v(2)"}
\end{quote}
which enforses interpretation as separate expressions.  Additional
parentheses may also be used to the same effect.

In the expression list, a ``.'' token is replaced with the vector list
found in a {\vt .plot} line from the file with the same analysis type
as the current plot.  For example, if the input file contained
\begin{quote}\vt
   .tran .1u 10u\\
   .plot tran v(1) v(2)\\
\end{quote}
then one can type ``{\vt run}'', then ``{\vt plot .}'' to plot {\vt
v(1)} and {\vt v(2)}.

\begin{tabular}{|l|l|}\hline
\multicolumn{2}{|c|}{Graphical Output Commands}\\ \hline
\cb asciiplot & Generate line printer plot\\ \hline
\cb combine & Combine plots\\ \hline
\cb hardcopy & Send plot to printer\\ \hline
\cb iplot & Plot during simulation\\ \hline
\cb mplot & Plot range analysis output\\ \hline
\cb plot & Plot simulation results\\ \hline
\cb plotwin & Pop down and destroy plot windows\\ \hline
\cb xgraph & Plot simulation results using {\vt xgraph}\\ \hline
\end{tabular}

%SU-------------------------------------
\subsection{\spcmd{asciiplot}}
\index{asciiplot command}

% spPlot.hlp:asciiplot 012209

The {\cb asciiplot} command generates a crude plot on a character mode
device.  It is not often used in modern environments, but is retained
for compatibility with SPICE2.
\begin{quote}\vt
asciiplot {\it plotargs}
\end{quote}
The {\it plotargs} are vectors or expressions to be plotted, as with
the {\cb plot} command.  The plot is sent to the standard output, so
one can put it into a file by using redirection.  The variables {\et
width}, {\et height}, and {\et nobreak} determine the width and height
of the plot, and whether there are page breaks, respectively, though
if the {\cb asciiplot} is printed on-screen or in a window, the plot
width and height are determined by the window size.

There are problems if one tries to plot something with an X scale that
is not monotonic, because {\cb asciiplot} uses a simple-minded sort of
linear interpolation.  Also, most of the variables that the {\cb plot}
command recognizes aren't used by {\cb asciiplot}.  The scaling and
other variables can be set with the {\cb set} command as for the {\cb
plot} command.  These variables can also be set with the {\cb Plot
Options} tool from the {\cb Tools} menu of the {\cb Tool Control}
window.

The {\et nointerp} variable is used only by the {\cb asciiplot}
command.  Normally {\cb asciiplot} interpolates data onto a linear
scale before plotting it.  If this option is given this won't be done
--- each line will correspond to one data point as generated by the
simulation.  Since data are already linearized unless from a transient
analysis with {\et steptype} set to {\vt nousertp}, setting this
variable will avoid a redundant linearization.

Ordinarily, the first vector plotted has its values also printed in
the first column.  This can be suppressed by setting the variable {\et
noasciiplotvalue}.  When printing, the number of significant digits
used can be set with {\et numdgt} variable.

This command is completely obsolete, but is retained for nostalgia for
those who fondly remember punched cards and line printers.

%SU-------------------------------------
\subsection{\spcmd{combine}}
\index{combine command}

% spPlot.hlp:combine 012209

The {\cb combine} command takes no arguments.  The command will
combine the two most recent plots, if similar, into a single plot, and
expands the dimensionality of the resulting plot.  The two plots must
have identical vector names and compatible lengths.  The purpose of
this command is to create a single multi-dimensional plot from
sequences of runs.  The most recent plot is added to the end of the
previous plot, and is deleted.

Example:
\begin{quote}
{\vt\raggedright
while i < 5\\
\qquad {\rm (set parameters for run)}\\
\qquad run\\
\qquad if i > 0\\
\qquad\qquad combine\\
\qquad end\\
\qquad i = i + 1\\
end}
\end{quote}

This will combine all the data from the five runs into a single
plot.

%SU-------------------------------------
\subsection{\spcmd{hardcopy}}
\index{hardcopy command}

% spPlot.hlp:hardcopy 012209

The {\cb hardcopy} command is used to generate hardcopy plots of
simulation data on a printer or plotter.  This capability is similar
to the {\cb Print} button which appears on each of the on-screen plots
from the {\cb plot} command.
\begin{quote}\vt
hardcopy [{\it setupargs\/}] {\it plotargs\/}\\
setupargs: -d {\it driver} -c {\it command} -f {\it filename}
 -r {\it resolution} -w {\it width} -h {\it height}\\
 -x {\it left\_marg} -y {\it top\_marg} -l
\end{quote}
This command uses the internal hardcopy drivers to generate a hard
copy of the vectors and expressions given in {\it plotargs}.  The {\it
plotargs} are vectors or expressions to be plotted, as with the {\cb
plot} command.  If no {\it plotargs} are provided, the arguments are
taken to be the same as those given to the last plotting command given
(these include {\cb plot}, {\cb asciiplot}, {\cb hardcopy}, and {\cb
xgraph}).  The {\it setupargs} override the current values established
using the {\cb set} command or the {\cb Plot Options} tool in the {\cb
Tools} menu, and default to the driver defaults if not specified
either way.

The {\vt -d} {\it driver} specifies the name of a hardcopy driver
using one of the keywords known to the {\et hcopydriver} variable.  If
the {\vt -d} option is not specified, the {\et hcopydriver} variable
will be used if set.  If no driver is set, or set to an unrecognized
driver name, the hardcopy is aborted.

The {\vt -c} {\it command} option specifies the operating system
command used to send the job to the printer, and overrides the value
of the {\et hcopycommand} variable, which is otherwise used if set. 
The value is a string (which must be quoted it it contains space),
where the characters ``{\vt \%s}'' are replaced by the name of the
(possibly temporary) file used to store the plot data.  If no {\vt
\%s} appears, the file name is appended to the end of the command
string.  In BSD Unix, the command string might be ``{\vt lpr -h
-Pmyprinter}'', for example.  See the man page for the print command
on your machine for more information.  If there is no command string
given using the {\vt -c} option and {\et hcopycommand} is undefined,
the data will be saved in a file, but not printed.

The {\vt -f} {\it filename} option gives a file name to store the plot
data.  There is no analogous ``set'' variable.  If given, the plot
will be saved in the file, and {\it not} sent to the printer.

The {\vt -r} {\it resolution} command will set the printer to use the
specified resolution, if that resolution is supported by the driver
and the printer.  If not given, the value of {\et hcopyresol} is used,
if set, otherwise the driver default is used.  The default is almost
always the best choice.

The {\vt -w} {\it width} and {\vt -h} {\it height} options set the
size of the image as it would appear on a portrait-oriented page.  The
numbers given represent inches, unless followed by ``cm'' which
indicates centimeters.  If these options are not given, the {\et
hcopywidth} and {\et hcopyheight} variables are used if set, otherwise
the driver defaults are used.

The {\vt -x} {\it xoffset} and {\vt -y} {\it yoffset} options control
the position of the image on the page, as defined in portrait
orientation.  The {\it yoffset} may be measured from the top or bottom
of the page, depending upon the driver.  These values default to those
in the variables {\et hcopyxoff} and {\et hcopyyoff} if set, otherwise
driver defaults are used.  The numbers represent inches, unless
followed by ``cm'' indicating centimeters.

If the {\vt -l} option is given, or the {\et hcopylandscape} variable
is set, the image will be rotated and printed in landscape
orientation.

The variables which control plot presentation also control the
presentation of the hardcopy (see \ref{plotcmd}).  The hardcopy
command is suited for use in scripts.  For general plotting, the {\cb
Print} button in the {\cb plot} windows brings up a panel which
provides a superior user interface.
 
%SU-------------------------------------
\subsection{\spcmd{iplot}}
\index{iplot command}

% spPlot.hlp:iplot 022519

The {\cb iplot} command adds an incremental plot to the ``runop'' list.
While a simulation is running, the plots will be generated, allowing
immediate feedback as to whether the simulation is producing the
``right'' results.
\begin{quote}\vt
iplot {\it plotargs}
\end{quote}
The {\it plotargs} are vectors or expressions to be plotted, as with
the {\cb plot} command.  The variables which control plotting also
apply to iplots.  These are set with the {\cb set} command, or with
the {\cb Plot Options} tool in the {\cb Tools} menu of the {\cb Tool
Control} window.

The argument list can not be empty.  Similar to the {\cb plot}
command, if the argument list contains a token consisting of a single
period (``{\vt .}''), this is replaced with the vector list found in
the first {\vt .plot tran} line from the input file.  For example, if
the input file contains
\begin{quote} \vt
.plot tran v(1) v(2)
\end{quote}
then one can type ``{\vt iplot .}'' as a short cut for ``{\vt iplot
v(1) v(2)}''.

The related syntax {\vt .@}{\it N} is also recognized, where {\it N}
is an integer representing the {\it N\/}'th matching {\vt .plot tran}
line.  The count is 1-based, but {\it N\/}=0 is equivalent to {\it
N\/}=1.  The token is effectively replaced by the vector list from the
specified {\vt .plot tran} line found in the circuit deck.

The iplots can be deleted with the {\cb delete} command, and can also
be specified and deleted using the panel brought up by the {\cb Trace}
button in the {\cb Tools} menu.  The {\cb status} command will list
the runops, including iplots.

If an {\cb iplot} command is given at the prompt in interactive mode,
it is placed in a global list, and activity will persist until deleted
(with the {\cb delete} command or with the {\cb Trace} tool).  If the
command is given in a file, the command will be added to a list for
the current circuit, and will apply only to that circuit.  Thus, for
example, a {\WRspice} file can contain lines like
\begin{quote}\vt
*\# iplot v(1)
\end{quote}
and the iplot will be performed as that circuit is run, but the ``{\vt
iplot v(1)}'' directive will not apply to other circuits.

%SU-------------------------------------
\subsection{\spcmd{mplot}}
\label{mplot}
\index{mplot command}

% spPlot.hlp:mplot 041711

The {\cb mplot} command is used to plot the results from margin analysis,
which includes operating range and Monte Carlo analyses.  It is also used
to set and clear interactive margin analysis plotting.
\begin{quote}\vt
mplot [[-on|-off] | [-c][{\it filename} ...] | {\it vector\/}]
\end{quote}
The {\it filenames} are names of files produced by the margin
analysis.  If no file is specified, the file produced by the last
margin analysis run in the current session is assumed.  If no margin
analysis files have been produced in the current session, the file
named ``{\vt check.dat}'' is assumed.  It is also assumed that these
files exist in the current directory.  The name of the most recent
margin analysis output file produced in the current session is saved
in the {\et mplot\_cur} variable.

The results from operating range/Monte Carlo analysis are also hidden
away in the resulting plot structure.  The {\cb mplot} can be displayed by
entering ``{\vt mplot} {\it vector}'' where {\it vector} is any vector
in the plot.

The {\it vector} can actually be any multi-dimensional vector, from
margin analysis or not.  The selections (see below) can then be used
to determine which dimensions are displayed in subsequent plots.

The {\vt -c} option combines the operating range data from the files
on the command line into a single display, if possible.  Thus, if two
or more successive operating range analysis runs are required to
obtain the total operating range, then it is possible to plot all of
the results on a single graph with the {\vt -c} option.  The data must
have identical coordinate spacing and projected origins to be
combinable.

There are two switches, {\vt -on} and {\vt -off}, which control
whether or not operating range analysis results are plotted on the
screen during analysis, similar to the {\cb iplot} command.  Entering
{\vt mplot -on} will cause margin analysis results to be plotted while
simulating, and {\vt mplot -off} will turn this feature off.

The display consists of an array of cells, each of which represent the
results of a single trial.  As the results become available, the cells
indicate a pass or fail.  In operating range analysis, the cells
indicate a particular bias condition according to the axes.  In Monte
Carlo analysis, the position of the cells has no significance.  In
this case the display indicates the number of trials completed.

The panel includes a {\cb Help} button which brings up the appropriate
topic in the help system, a {\cb Redraw} button to redraw the plot if,
for example, the plotting colors are redefined, and a {\cb Print}
button for generating hard copy output of the plot.

Text entered while the pointer is in the {\cb mplot} window will
appear in the plot, and hardcopies.  This text, and other text which
appears in the plot, can be edited in the manner of text in {\cb plot}
windows.

\subsubsection{Selections}

The cells in an {\cb mplot} can be selected/deselected by clicking on
them.  Clicking with button 1 will select/deselect that cell.  Using
button 2, the row containing the cell will be selected or deselected,
and with button 3 the column will be selected or deselected.  A
selected cell will be shown with a colored background, with an index
number printed.

Only one {\cb mplot} window can have selections.  Clicking in a new
window will deselect all selections in other {\cb mplot} windows.

At present, the selections are used to facilitate plotting of
multidimensional plots, such as those obtained from the {\vt -k}
option of the {\cb check} command.  If selections exist, only the data
from the selected cells will be plotted from the associated
multidimensional vectors in the {\cb plot} command.

For example, after running ``{\vt check -k}'', suppose one has a
resulting vector v(1) (which will contain data from all of the
trials).  If not using ``{\vt mplot -on}'' during the run, one can
type ``{\vt mplot}'' after the run to display the pass/fail results. 
In the {\cb mplot} window, select one of the cells.  Then, ``{\vt plot
v(1)}'' will plot the v(1) from that trial only.  If no cells are
selected, or all cells are selected, ``{\vt plot v(1)}'' would show
the superimposed v(1) traces from all trials.  The index number that
appears in the cell is the vector index, so for example if a single
box is selected with index 4, ``{\vt plot v(1)}'' would be equivalent
to ``{\vt plot v(1)[4]}''.  Note that the selection mechanism allows
combinations of traces to be plotted which can't easily be obtained
from indexing.

This capability is carried a step further for general multidimensional
plots.  If one enters ``{\vt mplot} {\it vector}'' where {\it vector}
is the name of a multidimensional vector from whatever source, an {\cb
mplot} will appear.  If the vector originated from operating range or
Monte Carlo analysis, the resulting {\cb mplot} will appear (the
pass/fail results are saved in the plot structure, as well as in the
output file).  Otherwise, the {\cb mplot} has nothing to do with range
analysis, and all cells are marked "fail".  Either case allows the
selection mechanism to be used for displaying the plots.

Suppose for example one has a multidimensional plot from a
loop-transient analysis.  Entering ``{\vt mplot time}'' will bring up
a dummy {\cb mplot} whose cells represent the loop iterations (time is
the scale vector for the plot, but any data vector in the plot would
suffice).  Then, by selecting the cells, one can choose which
iterations will be visible when vectors from the plot are plotted with
the {\cb plot} command.

The plot window will use a ``flat'' dimension map which can
subsequently be used to control which dimensions are visible.  The
mplot selections set the initial state of this map.

%SU-------------------------------------
\subsection{\spcmd{plot}}
\index{plot command}
\label{plotcmd}

% spPlot.hlp:plot 052411

The {\cb plot} command is used to plot simulation output on-screen. 
Each execution of a {\cb plot} command will bring up a window which
displays the plot, along with several command buttons.  Each plot will
remain on-screen until dismissed with the {\cb Dismiss} button.
\begin{quote}\vt
plot [{\it expr} ... ] [vs {\it x-expr}] [{\it attributes\/}]
\end{quote}
The set of expressions can be followed with a ``{\vt vs} {\it x-expr}''
clause, which will produce an x-y plot using the values of {\it
x-expr} as the x scale.

If no arguments are given, the arguments to the last given {\cb plot}
command are used.  If the argument list contains a token consisting of
a single period (``.''), this is replaced with the vector list found
in the first {\vt .plot} line from the input file with the same
analysis type as the current plot.  For example, if the input file
contains
\begin{quote}\vt
    .tran .1u 10u\\
    .plot tran v(1) v(2)\\
\end{quote}
then one can type ``{\vt run}'' followed by ``{\vt plot .}'' to plot
{\vt v(1)} and {\vt v(2)}.

The related syntax {\vt .@}{\it N} is also recognized, where {\it N}
is an integer representing the {\it N\/}'th matching {\vt .plot} line. 
The count is 1-based, but {\it N\/}=0 is equivalent to {\it N\/}=1. 
The token is effectively replaced by the vector list from the
specified {\vt .plot} line found in the circuit deck.

Vectors and expression results will be interpolated to the scale used
for the plot.  This applies when using forms like ``{\vt tran2.v(2)}''
where the {\vt tran2} may have a different scale, for example the
x-increment may be different, or the data may correspond to internal
time points vs.  user time points.

The plot style can be controlled by a number of variables (listed
below), which can be set with the {\cb set} command.  These define
default behavior, as the plot window contains buttons which also
determine presentation.  The {\cb Plot Options} tool from the {\cb
Tools} menu of the {\cb Tool Control} window can also be used to set
these variables.  The {\cb Colors} tool from the {\cb Tools} menu can
be used to change the colors used for plotting.

For each of the variables listed in the table below with an asterisk
in the middle column, if a variable named {\et \_temp\_}{\it varname}
is defined, its value will be used rather than that of {\it varname}. 
This allows temporary overriding of the nominal settings of the
variables, and is used internally for the zoom-in operation.  In
addition, there are certain variables such as {\et gridstyle} which
can be set to one of several keywords.  If the keyword itself is set
as a boolean variable, it will override the string variable.  For
example, one could issue ``{\vt set gridstyle = lingrid}'' to set a
nominally linear grid.  This can be changed by, for example, ``{\vt
set loglog}'' (or ``{\vt set \_temp\_loglog}''), but it is an error to
have two or more such keywords set as booleans at a time.

The variables with an asterisk in the middle column can appear in a
{\vt .options} line in a circuit file.  The option will be in force
when the circuit containing this line is the current circuit.

Many of these attributes can also be set from the {\cb plot} command
line, which will override any corresponding variable, if set.  The
functionality is as described for the variables.  The ``value'' of the
variables (if any) should follow the keyword, separated by space
and/or an optional `{\vt =}' character.  For values consisting of two
numbers, a comma and/or space can delimit the numbers.  The variable
names that are also recognized as command line keywords are shown with
an asterisk in the third column in the table below.

\begin{longtable}{|l|l|l|}\hline
\bf Variable    & \_temp\_{\it name\/}? & Attribute?\\ \hline
\vt color       & & \\ \hline
\vt combplot    & * & *\\ \hline
\vt gridsize    & & \\ \hline
\vt gridstyle   & & \\ \hline
\vt group       & * & *\\ \hline
\vt lingrid     & * & *\\ \hline
\vt linplot     & * & *\\ \hline
\vt loglog      & * & *\\ \hline
\vt multi       & * & *\\ \hline
\vt nogrid      & * & *\\ \hline
\vt nointerp    & * & *\\ \hline
\vt noplotlogo  & * & *\\ \hline
\vt plotposn    & & \\ \hline
\vt plotstyle   & & \\ \hline
\vt pointchars  & & \\ \hline
\vt pointplot   & * & *\\ \hline
\vt polar       & * & *\\ \hline
\vt polydegree  & & \\ \hline
\vt polysteps   & & \\ \hline
\vt scaletype   & & \\ \hline
\vt single      & * & *\\ \hline
\vt smith       & * & *\\ \hline
\vt smithgrid   & * & *\\ \hline
\vt ticmarks    & & \\ \hline
\vt title       & * & *\\ \hline
\vt xcompress   & * & *\\ \hline
\vt xdelta      & * & *\\ \hline
\vt xindices    & * & *\\ \hline
\vt xlabel      & * & *\\ \hline
\vt xlimit      & * & *\\ \hline
\vt xlog        & * & *\\ \hline
\vt ydelta      & * & *\\ \hline
\vt ylabel      & * & *\\ \hline
\vt ylimit      & * & *\\ \hline
\vt ylog        & * & *\\ \hline
\vt ysep        & * & *\\ \hline
\end{longtable}

When a plot is read from a rawfile, defaults for the presentation
attributes are set as specified in the rawfile.  These can be
overridden by reseting the attributes in {\WRspice}, with the exception
of the color specification in the rawfile.  If given, that color will
be used for a particular trace independent of the current setting
within {\WRspice}.  {\WRspice} never sets the color specification, when
writing a rawfile, unless that color was indicated from a previous
rawfile.  If a certain unalterable color is desired for a trace, the
rawfile can be edited with a text editor to specify that color.

Any text typed while the pointer is in the plot window will appear on
the plot (and hardcopies).  This is useful for annotation.  Entered
and existing text can be edited and moved.  In addition, traces in the
plot can be moved to change the order, or moved to other (x-scale
compatible) plot windows.  The description of the plot window
(\ref{plotpanel}) contains more information.


%SU-------------------------------------
\subsection{\spcmd{plotwin}}
\index{plotwin command}

% spPlot.hlp:plotwin 091714

The {\cb plotwin} command provides an interface for destroying plot
windows, most useful in scripts.  It applies in graphical mode only. 
There are two forms:
 
\begin{quote}
{\vt plotwin} [{\vt id}]
\end{quote}
 
Given without an argument, or with literal ``{\vt id}'' (only the
first letter is significant, case insensitive), the identification
number of the most recently created plot window is printed.  Every
plot window has a unique running identification number, which can be
used as a ``handle'' to the window.
 
\begin{quote}
{\vt plotwin} {\vt k}[{\vt ill}] [{\it idarg\/}]
\end{quote}
 
This form is used to destroy plot windows.  The first token is a word
starting with `{\vt k}', case insensitive.  The {\it idarg} is a
number.  If not given or zero, the most recently created plot is
destroyed.  If {\it idarg} is positive, the plot window with that
identification number is destroyed.  If {\it idarg} is negative, the
plot window relative to the most recently created plot window is
destroyed.  For example, -1 destroys the plot before the most recent,
-2 the one before that, etc.  The {\it idarg} can also be a word
starting with `{\vt a}' (for ``{\vt all}'') in which case all plot
windows are destroyed.


%SU-------------------------------------
\subsection{\spcmd{xgraph}}
\index{xgraph command}

% spPlot.hlp:xgraph 012209

The {\cb xgraph} command will produce plots using the UNIX {\vt
xgraph} utility.
\begin{quote}\vt
xgraph {\it file plotargs}
\end{quote}
This command is similar to the {\cb plot} command, however the {\vt
xgraph} program (an obsolete plotting package) actually generates
the plots.  If the given {\it file} is either ``{\vt tmp}'' or ``{\vt
temp}'', then a temporary file is used to hold the data while being
plotted.  Polar and Smith plots are not supported, otherwise the
variables associated with the {\cb plot} command apply.

The {\et xglinewidth} variable specifies the line width in pixels to
use in the plots.  If not set, a minimum line width is used.
 
If {\et xgmarkers} is set, point plots will use cross marks, otherwise
big pixels are used.


%S-----------------------------------------------------------------------------
\section{Miscellaneous Commands}

% spCommands.hlp:misccmds 062515

These commands perform miscellaneous functions.

\begin{tabular}{|l|l|}\hline
\multicolumn{2}{|c|}{Miscellaneous Commands}\\ \hline
\cb bug & Submit bug report\\ \hline
\cb help & Enter help system\\ \hline
\cb helpreset & Clear help system cache\\ \hline
\cb qhelp & Print command summaries\\ \hline
\cb quit & Exit program\\ \hline
\cb rusage & Print resource usage statistics\\ \hline
\cb stats & Print resource usage statistics\\ \hline
\cb version & Print program version\\ \hline
\end{tabular}

%SU-------------------------------------
\subsection{\spcmd{bug}}
\index{bug command}

% spCommands.hlp:bug 012209

The {\cb bug} command facilitates sending bug reports and other
messages to the {\WRspice} administrator.  Issuing the {\cb bug}
command will pop up a mail editing window if graphics is available,
or will allow a message to be entered on the command line if not. 
The environment variable {\et SPICE\_BUGADDR} is used to set the
internet address to which bug reports are sent (this can be changed
in the pop-up mail editor window).  If not set, the report is sent
to the Whiteley Research technical support staff.  This command
takes no arguments.

The mail editor window can also be displayed by pressing the {\cb WR}
button in the {\cb Tool Control} window.

%SU-------------------------------------
\subsection{\spcmd{help}}
\index{help command}
\label{help}

% spCommands.hlp:help 012209

The {\cb help} command brings up a help window describing the topic
keyword passed as an argument to the command, or the top-level entry
if no argument is given.
\begin{quote}\vt
help [-c | {\it topic\/}]
\end{quote}
When graphics is not available, the help text is presented in a
text-only format on the terminal.  The HTML to ASCII text converter
only handles the most common HTML tags, so some descriptions may look
a little strange.  The figures (and all images) are not shown, and
links are not available, except for the ``subtopics'' and
``references'' lists.

The help data files are found in directories specified in the {\et
helppath} variable, or from the\\ {\et SPICE\_HLP\_PATH} environment
variable.  If for some reason the help directory is not found, a very
minimal internal text-mode help system will be provided.  The single
character `{\vt ?}' is internally aliased to ``{\vt help}''.

If the single argument ``{\vt -c}'' is given, the internal topic hash
tables are cleared.  Since the topics are hashed as offsets into the
files, if a topic text changes, the offsets will be incorrect.  After
changes are made to a help file, or new help files are added, if in
{\WRspice} and the help database has already been cached by viewing
any help topic, giving ``{\vt help -c}'' will ensure that new topics
are found and present topics display correctly.  This is the same
effect as giving the {\cb helpreset} command.

The {\et helpinitxpos} variable specifies the distance in pixels from
the left edge of the screen to the left edge of the help window, when
it first appears.  If not set, the value taken is 100 pixels.  The
{\et helpinitypos} variable specifies the distance in pixels from the
top edge of the screen to the top edge of the help window, when it
first appears.  If not set, the value taken is 100 pixels.

See \ref{helpsys} for more information about the {\WRspice} help
system.

%SU-------------------------------------
\subsection{\spcmd{helpreset}}
\index{helpreset command}

% spCommands.hlp:helperset 012209

This will clear the internal topic cache used by the help system.  The
cache saves topic references as offsets into the help ({\vt .hlp})
files, so that if the text of a help file is modified, the offsets are
probably no longer valid.  This function is useful when editing the
text of a help file, while viewing the entry in {\WRspice}.  Use this
function when editing is complete, before reloading the topic into the
viewer.  Although the offset to the present topic does not change when
editing, so that simply reloading would look fine, other topics in the
file that come after the present topic would not display correctly if
the offsets change.

This is the same effect as giving the {\cb help} command with the {\vt
-c} option.

%SU-------------------------------------
\subsection{\spcmd{qhelp}}
\index{qhelp command}

% spCommands.hlp:qhelp 012209

The {\cb qhelp} command prints a brief description of each command
listed as an argument.  If no arguments are given, all commands are
listed.  This is not part of the main help system.

%SU-------------------------------------
\subsection{\spcmd{quit}}
\index{quit command}

% spCommands.hlp:quit 012209

The {\cb quit} command terminates the {\WRspice} session.  If there
are circuits that are in the middle of a simulation, or plots that
have not been saved in a file, the user is reminded of this and
asked to confirm.  The variable {\et noaskquit} disables this. 
{\WRspice} can also be terminated from the {\cb Quit} button in the
{\cb File} menu of the {\cb Tool Control} window.  The command takes
no arguments.

%SU-------------------------------------
\subsection{\spcmd{rusage}}
\index{rusage command}
\label{rusage}

% spRusage.hlp:rusage 032320

The {\cb rusage} command is used to obtain information about the
consumption of system resources and other statistics during the
{\WRspice} session.
\begin{quote}\vt
rusage [all] [{\it resource\/} ...]
\end{quote}
If any resource keywords are given, only those resources are printed. 
All resources are printed if the keyword {\vt all} is given.  With no
arguments, only total time and space usage are printed.  The {\cb
show} command can also be used to obtain resource statistics.  The
recognized keywords are listed below.

The {\cb stats} command is almost identical to {\cb rusage}, and
accepts the same keywords.  The difference is that {\cb stats} given
without arguments will print all run statistics.

In release 4.3.10 and later, statistics accumulate in Monte Carlo,
operating range, and sweep operations.  This was not the case in
earlier releases.

The two tables that follow list the available resource statistics.  An
internal statistical database maintains these values, the {\cb rusage}
and {\cb stats} commands are the user interface to this database.  The
following are a few keywords handled by the {\cb rusage} and {\cb
stats} commands directly.  Other keywords are passed in queries to the
internal statistical database.

\begin{description}
\item{\vt elapsed}\\
\index{rusage command!elapsed}
This keyword prints the total amount of time that has elapsed since
the last call of the {\cb rusage} or {\cb stats} command with the {\vt
elapsed} keyword (explicit or implied with ``{\vt all}''), or to the
program start time.

\item{\vt faults}\\
\index{rusage command!faults}
This keyword prints the number of page faults and context switches
seen by the program thus far.  See also {\et pagefaults}, {\et
involcxswitch}, and {\et volcxswitch} for the values that occurred
during the last anslysis.

\item{\vt space}\\
\index{rusage command!space}
This keyword will print the memory presently in use by {\WRspice}.

\item{\vt totaltime}\\
\index{rusage command!totaltime}
If this keyword is given, the total time used in the present session
will be printed.
\end{description}

subsubsection{Statistical Database Entries}
\index{statistical database}

The statistical database contains the following data items, listed in
the tables below and with a more detailed description of each item
following.

\begin{longtable}{|p{1in}|p{4.75in}l|} \hline
\bf Resource Name & \bf Description&\\ \hline
\multicolumn{3}{|l}{}\\
\multicolumn{3}{|l}{\bf Real-Valued Parameters}\\ \hline
{\vt cvchktime} & \rr Time spent convergence testing.&\\ \hline
{\vt loadtime} & \rr Device model evaluation and matrix load time.&\\ \hline
{\vt lutime} & \rr L-U decomposition time.&\\ \hline
{\vt reordertime} & \rr Matrix reordering time.&\\ \hline
{\vt solvetime} & \rr Matrix solve time.&\\ \hline
{\vt time} & \rr Total analysis time.&\\ \hline
{\vt tranlutime} & \rr Transient L-U decomposition time.&\\ \hline
{\vt tranouttime} & \rr Transient output recording time.&\\ \hline
{\vt transolvetime} & \rr Transient solve time.&\\ \hline
{\vt trantime} & \rr Transient time.&\\ \hline
{\vt trantstime} & \rr Transient timestep computation time.&\\ \hline

\multicolumn{3}{|l}{}\\
\multicolumn{3}{|l}{\bf Integer-Valued Parameters}\\ \hline
{\vt accept} & \rr Accepted timepoints.&\\ \hline
{\vt equations} & \rr Circuit equations.&\\ \hline
{\vt fillin} & \rr Fill-in terms from decomposition.&\\ \hline
{\vt involcxswitch} & \rr Involuntary context switches during analysis.&\\ \hline
{\vt loadthrds} & \rr Number of device loading helper threads.&\\ \hline
{\vt loopthrds} & \rr Number of repetitive analysis helper threads.&\\ \hline
{\vt matsize} & \rr Matrix size.&\\ \hline
{\vt nonzero} & \rr Number of nonzero matrix entries.&\\ \hline
{\vt pagefaults} & \rr Number of page faults during analysis.&\\ \hline
{\vt rejected} & \rr Number of rejected timepoints.&\\ \hline
{\vt runs} & \rr Accumulated core analysis runs.&\\ \hline
{\vt totiter} & \rr Total iterations.&\\ \hline
{\vt trancuriters} & \rr Transient interations at last timepoint.&\\ \hline
{\vt traniter} & \rr Transient interations.&\\ \hline
{\vt tranitercut} & \rr Transient timepoints where iteration limit exceeded.&\\ \hline
{\vt tranpoints} & \rr Transient timepoints.&\\ \hline
{\vt trantrapcut} & \rr Transient timepoints where trapcheck failed.&\\ \hline
{\vt volcxswitch} & \rr Voluntary context switches during analysis.&\\ \hline
\end{longtable}

\subsubsection{Real Valued Database Entries}

\begin{description}
\item{\vt cvchktime}\\
\index{rusage command!cvchktime}
Print the time spent checking for convergence in the most recent dc or
transient analysis (including operating point).

\item{\vt loadtime}\\
\index{rusage command!loadtime}
If given, print the time spent loading the matrix in the last
simulation run.  This includes the time spent in computation of device
characteristics.

\item{\vt lutime}\\
\index{rusage command!lutime}
The {\vt lutime} keyword will print the time spent in LU factorization
of the matrix during the last simulation run.

\item{\vt reordertime}\\
\index{rusage command!reordertime}
Print the time spent reordering the matrix for numerical stability in
the most recent simulation.

\item{\vt solvetime}\\
\index{rusage command!solvetime}
This will print the time spent solving the matrix equations, after
LU decomposition, in the last simulation run.

\item{\vt time}\\
\index{rusage command!time}
This keyword will print the time used by the last simulation run.

\item{\vt tranlutime}\\
\index{rusage command!tranlutime}
The time spent LU factoring the matrix in the most recent transient
analysis, not including the dc operating point calculation.

\item{\vt tranouttime}\\
\index{rusage command!tranouttime}
Print the time spent saving output in the most recent transient
analysis.

\item{\vt transolvetime}\\
\index{rusage command!transolvetime}
This keyword prints the matrix solution time required by the last
transient analysis, not including the operating point calculation.

\item{\vt trantime}\\
\index{rusage command!trantime}
This keyword will print the total time spent in transient analysis
in the last transient analysis, not including the operating point
calculation.

\item{\vt trantstime}\\
\index{rusage command!trantstime}
Report the time spent computing the next timestep in the most recent
transient analysis.
\end{description}

\subsubsection{Integer Valued Database Entries}

\begin{description}
\item{\vt accept}\\
\index{rusage command!accept}
This keyword prints the number of accepted time points from the last
transient analysis.

\item{\vt equations}\\
\index{rusage command!equations}
Print the number of nodes in the current circuit, including internally
generated nodes.  This includes the ground node so is one larger than
the matrix size.

\item{\vt fillin}\\
\index{rusage command!fillin}
Print the number of fillins generated during matrix reordering and
factoring.  This is not available from KLU.

\item{\vt involcxswitch}\\
\index{rusage command!involcxswitch}
This provides the number of involuntary context switches seen during
the last analysis.  If multiple threads are being used, this is the
total for all threads.

\item{\vt loadthrds}\\
\index{rusage command!loadthrds}
Report the number of threads used for device evaluation and matrix
loading during the most recent dc (including operating point) or
transient analysis.  This would be at most the value of the {\et
loadthrds} option variable in effect during the analysis, but is the
number of threads actually used.

\item{\vt loopthrds}\\
\index{rusage command!loopthrds}
Report the number of threads in use for repetitive analysis in the
most recent analysis run.  This would be at most the value of the {\et
loopthrds} option variable in effect during the analysis, but is the
number of threads actually used.

\item{\vt matsize}\\
\index{rusage command!matsize}
Print the size of the circuit matrix.

\item{\vt nonzero}\\
\index{rusage command!nonzero}
Print the number of nonzero matrix elements.

\item{\vt pagefaults}\\
\index{rusage command!pagefaults}
Report the number of page faults seen during the most recent analysis.

\item{\vt rejected}\\
\index{rusage command!rejected}
This keyword prints the number of rejected time points in the last
transient analysis.

\item{runs}\\
In Monte Carlo, operating range, and sweep analysis, this returns    
the number of trial runs over which statistics have accumulated.

\item{\vt totiter}\\
\index{rusage command!totiter}
This keyword prints the total number of Newton iterations used in
the last analysis.

\item{\vt trancuriters}\\
\index{rusage command!trancuriters}
This prints the number of Newton iterations used in the most recent
transient analysis time point evaluation.

\item{\vt traniter}\\
\index{rusage command!traniter}
The {\vt traniter} keyword will print the number of iterations used in
the last transient analysis.  This does not include iterations used in
the operating point calculation, unlike {\vt totiter} which includes
these iterations.

\item{\vt tranitercut}\\
\index{rusage command!tranitercut}
The number if times that the most recent transient analysis had a time
step cut by iteration count.  If the {\et itl4} limit is reached when
attempting convergence at a transient time point, the timestep is cut
and convergence is reattempted.

\item{\vt tranpoints}\\
\index{rusage command!tranpoints}
This keyword prints the number of internal time steps used in the
last transient analysis.

\item{\vt trantrapcut}\\
\index{rusage command!trantrapcut}
This is the number of times in the most recent transient analysis that
a timestep was cut due to the trapcheck algorithm.  This may occur
when the {\et trapcheck} variable is set, which enables a test to
detect numerical problems in trapezoidal integration. 

\item{\vt volcxswitch}\\
\index{rusage command!volcxswitch}
This provides the number of voluntary context switches seen during the
last analysis.  If multiple threads are being used, this is the total
for all threads.
\end{description}


%SU-------------------------------------
\subsection{\spcmd{stats}}
\index{stats command}

% spRusage.hlp:stats 062515

The {\cb stats} command is basically identical to the {\cb rusage}
command, and accepts the same arguments as described for that command.

\begin{quote}\vt
stats [all] [{\it resource\/} ...]
\end{quote}

The difference is that when given without an argument, all run statistics
are printed.  This is the same as ``{\vt rusage all}'' with the {\vt
totaltime}, {\vt elapsed}, {\vt space}, and {\vt faults} fields
omitted.


%SU-------------------------------------
\subsection{\spcmd{version}}
\index{version command}

% spCommands.hlp:version 011909

The {\cb version} command is used to determine the version of {\WRspice}
running.
\begin{quote}\vt
version [{\it version\_name\/}]
\end{quote}
With no arguments, this command prints out the current version of
{\WRspice}.  If there are arguments, it compares the current version
with the given version and prints a warning if they differ.  A version
command is usually included in the rawfile.

% -----------------------------------------------------------------------------
% Xic Manual
% (C) Copyright 2009, Whiteley Research Inc., Sunnyvale CA
% $Id: variables.tex,v 1.153 2017/03/22 07:29:57 stevew Exp $
% -----------------------------------------------------------------------------

% -----------------------------------------------------------------------------
% set:variables 101212
\chapter{{\Xic} Variables}
\label{variables}

{\Xic} maintains an internal list of keyword/value associations. 
Although this list can be used for general purposes, there are a
number of special keywords, or ``variables'', whose value will affect
{\Xic} operation.  Variables are set with the {\cb !set} command, and
can be unset with the {\cb !unset} command.  The script functions {\vt
Set}, {\vt Unset}, {\vt SetExpand}, and {\vt Get} also provide
an interface to this database.  Variables can be set from the
technology file, and a number of the buttons in menus and various
pop-ups really do nothing more than control the state of one of these
variables.

Any variable name can be set with the {\cb !set} command.  The
variables and constructs that have meaning to {\Xic} are summarized in
the table below.  These are described more fully in the sections that
follow.

\begin{longtable}{|l|l|} \hline
\multicolumn{2}{|c|}{\kb Special Constructs}\\ \hline
\vt !set & List variables currently set\\ \hline
\vt !set ? & List these variables\\ \hline
{\vt @}{\it devname}.{\it property} & Set device property\\ \hline

% 012014
\multicolumn{2}{|c|}{\kb Startup}\\ \hline
\et DatabaseResolution & Set internal units\\ \hline
\et NetNamesCaseSens & Net names are case-sensitive\\ \hline
\et Subscripting & Set net name subscripting character\\ \hline
\et DrfDebug & Report undefined layer attribute names\\ \hline

% 102613
\multicolumn{2}{|c|}{\kb Paths and Directories}\\ \hline
\et Path & Design data file search path\\ \hline
\et LibPath & Startup file and library search path\\ \hline
\et HelpPath & Help file search path\\ \hline
\et ScriptPath & Script file search path\\ \hline
\et NoReadExclusive & Don't move stripped path to front of search path\\ \hline
\et AddToBack & Add stripped path to back of search path\\ \hline
\et DocsDir & Directory containing release documentation\\ \hline
\et ProgramRoot & Set to the program's installation directory\\ \hline
\et TeePrompt & Copy messages to given filename or ``stdout''\\ \hline

% 010918
\multicolumn{2}{|c|}{\kb General Visual}\\ \hline
\et MouseWheel & Set mouse wheel rate parameters\\ \hline
\et ListPageEntries & Maximum entries per page in list pop-ups\\ \hline
\et NoInstnameLables & Don't use instance names in unexpanded instances\\ \hline
\et NoLocalImage & Don't compose images locally\\ \hline
\et NoPixmapStore & Don't use screen backing memory\\ \hline
\et NoDisplayCache & Don't use multi-object rendering for boxes\\ \hline
\et LowerWinOffset & Pixel spacing of pop-up windows above prompt line\\ \hline
\et PhysGridOrigin & Set the origin of the grid displayed in physical mode\\
  \hline
\et ScreenCoords & Show window pixel coordinates\\ \hline
\et PixelDelta & Cursor selection proximity is screen pixels\\ \hline
\et NoPhysRedraw & When set, don't redraw physical windows after layer
  visibility change\\ \hline
\et NoToTop & Don't move obscured windows to top\\ \hline

% 102817
\multicolumn{2}{|c|}{\kb `!' Commands}\\ \hline
\et Shell & Path to shell used for external commands\\ \hline

%\ifoa
% 030416
\multicolumn{2}{|c|}{\kb OpenAccess Interface}\\ \hline
\et OaLibraryPath & Set location for hidden libraries\\ \hline
\et OaDefLibrary & Default library name\\ \hline
\et OaDefTechLibrary & Default technology attachment library\\ \hline
\et OaDefLayoutView & Default layout view name\\ \hline
\et OaDefSchematicView & Default schematic view name\\ \hline
\et OaDefSymbolView & Default symbol view name\\ \hline
\et OaDefDevPropView & Default device property view name\\ \hline
\et OaDmSystem & Set design management system\\ \hline
\et OaDumpCdfFiles & Dump CDF data to a file\\ \hline
\et OaUseOnly & Restrict to physical/electrical data\\ \hline
%\fi

% 031815
\multicolumn{2}{|c|}{\kb Parameterized Cells}\\ \hline
\et PCellAbutMode & Control pcell auto-abutment\\ \hline
\et PCellHideGrips & Hide stretch handles if set\\ \hline
\et PCellGripInstSize & Instance size threshold for stretch handles\\ \hline
\et PCellKeepSubMasters & Include pcell sub-masters in file output\\ \hline
\et PCellListSubMasters & Include pcell sub-masters in modified cells list\\ \hline
\et PCellScriptPath & Search path for pcell scripts\\ \hline
\et PCellShowAllWarnings & Show warnings during pcell evaluation\\ \hline

% 031815
\multicolumn{2}{|c|}{\kb Standard Vias}\\ \hline
\et ViaKeepSubMasters & Include standard via sub-masters in file output\\ \hline
\et ViaListSubMasters & Include standard via sub-masters in modified cells list\\ \hline

% 102613
\multicolumn{2}{|c|}{\kb Scripts}\\ \hline
\et LogIsLog10 & The log function returns base-10 when set\\ \hline

% 062715
\multicolumn{2}{|c|}{\kb Selections}\\ \hline
\et MarkInstanceOrigin & Show origin of selected instances\\ \hline
\et MarkObjectCentroid & Show centroids of selected physical objects\\ \hline
\et SelectTime & Set delay (msec) to activate move\\ \hline
\et NoAltSelection & Use legacy click-selection logic\\ \hline
\et MaxBlinkingObjects & Maximum number of objects shown blinking\\ \hline

% 101016
\multicolumn{2}{|c|}{\kb Side Menu Commands}\\ \hline
\et MasterMenuLength & Maximum masters in {\cb Cell Placement Control} menu\\ \hline
\et DevMenuStyle & Set presentation style of device menu\\ \hline
\et LabelDefHeight & Default text label height in microns\\ \hline
\et LabelMaxLen & Max length of displayed label string\\ \hline
\et LabelMaxLines & Max lines of displayed label string\\ \hline
\et LabelHiddenMode & Set scope for hidden labels\\ \hline
\et LogoEndStyle & End style for logos: 0 flush, 1 round, 2 extend\\ \hline
\et LogoPathWidth & Path width for logos, 1 -- 5\\ \hline
\et LogoAltFont & Specify alternate font for logos\\ \hline
\et LogoPrettyFont & Name of system font to use for logos\\ \hline
\et LogoPixelSize & Specify the ``pixel'' size for logos\\ \hline
\et LogoToFile & Create subcell for logos\\ \hline
\et NoConstrainRound & No DRC constraints creating round objects\\ \hline
\et RoundFlashSides & Number of sides to use in physical round objects\\ \hline
\et ElecRoundFlashSides & Number of sides to use in electrical round objects\\
\et SpotSize & Set mask resolution\\ \hline

% 022619
\multicolumn{2}{|c|}{\kb SPICE Interface}\\ \hline
\et SpiceListAll & Include unconnected devices in Spice output\\ \hline
\et SpiceAlias & Device key aliases for Spice output\\ \hline
\et SpiceHost & Name of {\WRspice} server\\ \hline
\et SpiceHostDisplay & X display string to use on remote host\\ \hline
\et SpiceInclude & Add include file to SPICE netlist\\ \hline
\et SpiceProg & Path name of {\WRspice} executable, supersedes below\\ \hline
\et SpiceExecDir & Directory containing {\WRspice} executable\\ \hline
\et SpiceExecName & Name of {\WRspice} executable\\ \hline
\et SpiceSubcCatchar & Character used by {\WRspice} in subcircuit expansion\\ \hline
\et SpiceSubcCatmode & Mode for {\WRspice} subcircuit expansion\\ \hline
\et CheckSolitary & Report unconnected terminals in netlist\\ \hline
\et NoSpiceTools & Do not show {\WRspice} toolbar\\ \hline

% 102613
\multicolumn{2}{|c|}{\kb File Menu -- Printing}\\ \hline
\et NoAskFileAction & Don't ask before file actions in File Selection pop-up\\
  \hline
\et DefaultPrintCmd & Default print command (printer name in Windows)\\ \hline
\et NoDriverLabels & Don't use driver text for hard copy labels\\ \hline
\et RmTempFileMinutes & Set up temporary file removal\\ \hline

% 101212
\multicolumn{2}{|c|}{\kb Cell Menu Commands}\\ \hline
\et ContextDarkPcnt & Control illumination of context in {\cb Push}
  command\\ \hline

% 101515
\multicolumn{2}{|c|}{\kb Editing General}\\ \hline
\et AskSaveNative & Prompt to save modified native cell when editing new cell\\ \hline
\et Constrain45 & Constrain polygon and wire angles to 45-degree multiples\\ \hline
\et NoMergeObjects & Suppress merging new boxes, polygons\\ \hline
\et NoMergePolys & Clip/merge boxes only when merging\\ \hline
\et NoFixRot45 & Don't ``fix'' vertex locations after non-Manhattan rotation\\ \hline

% 100616
\multicolumn{2}{|c|}{\kb Edit/Modify Menu Commands}\\ \hline
\et UndoListLength & Number of operations saved in the undo list\\ \hline
\et MaxGhostDepth & Maximum subcell expansion depth in ghosting\\ \hline
\et MaxGhostObjects & Maximum number of objects shown in ghosting\\ \hline
\et NoWireWidthMag & Don't change the width of magnified wires\\ \hline
\et CrCellOverwrite & Allow Create Cell to overwrite memory cells\\ \hline
\et LayerChangeMode & Specify layer change during move/copy\\ \hline
\et JoinMaxPolyVerts & Upper bound of vertices in polygons from join
  (def. 600)\\ \hline
\et JoinMaxPolyGroup & Limit number trapezoids per poly in join (def. 300)\\
  \hline
\et JoinMaxPolyQueue & Limit number trapezoids to form polys in join (def.
  1000)\\ \hline
\et JoinBreakClean & Manhattan split polygons with too many vertices\\ \hline
\et JoinSplitWires & Include wires in join/split operations\\ \hline
\et PartitionSize & Partition grid size in microns for layer operations\\
  \hline
\et Threads & Number of helper threads to employ\\ \hline

% 012815
\multicolumn{2}{|c|}{\kb View Menu Commands}\\ \hline
\et InfoInternal & Use internal coordinates in info windows\\ \hline
\et PeekSleepMsec & Per-layer delay in peek command, milliseconds\\ \hline
\et LockMode & Don't allow physical/electrical mode change\\ \hline
\et XSectNoAutoY & Disable cross-section automatic Y scaling\\ \hline
\et XSectYScale & Set cross-section Y scale factor\\ \hline

% 041224
\multicolumn{2}{|c|}{\kb Attributes Menu Commands}\\ \hline
\et TechNoPrintPatMap & Use hex format for stipple maps when writing tech
  file\\ \hline
\et TechPrintDefaults & Set printing of default values in tech file update\\ \hline
\et BoxLineStyle & Line style mask for highlighting box\\ \hline
\et EraseBehindProps & Erase behind phys properties in props command\\ \hline
\et PhysPropTextSize & Pixel text height used in props command\\ \hline
\et EraseBehindTerms & Erase behind physical mode terminals marks\\ \hline
\et TermTextSize & Pixel height of text used in terminal marks\\ \hline
\et TermMarkSize & Pixel width of cross used for terminal marks\\ \hline
\et ShowDots & Control electrical connections display\\ \hline
\et FullWinCursor & Enable full-window cursor\\ \hline
\et CellThreshold & Min size in pixels of displayed subcell, integer $>=$ 0\\
  \hline
\et GridNoCoarseOnly & Don't show coarse grid without fine grid\\ \hline
\et GridThreshold & Minimum visible grid spacing pixels\\ \hline

% 020918
\multicolumn{2}{|c|}{\kb Convert Menu -- General}\\ \hline
\et ChdFailOnUnresolved & Halt CHD operation if unresolved cell\\ \hline
\et ChdCmpThreshold & Set CHD compression block size threshold\\ \hline
\et MultiMapOk & Allow non-1--1 mapping of {\Xic} layers and GDSII
  layer/datatypes\\ \hline
\et NoPopUpLog & Don't pop up log file if warnings or errors\\ \hline
\et UnknownGdsLayerBase & Base number for generated GDSII layers\\ \hline
\et UnknownGdsDatatype & Datatype for generated GDSII layers\\ \hline
\et NoStrictCellnames & Allow white space in cell names\\ \hline
\et NoFlattenStdVias & Keep standard via instances when flattening\\ \hline
\et NoFlattenPCells & Keep parameterized cell instances when flattening\\ \hline
\et NoFlattenLabels & Ignore labels in subcells when flattening\\ \hline
\et NoReadLabels & Ignore text labels when reading physical cell data\\ \hline
\et KeepBadArchive & Don't delete failed conversion output archive file\\
  \hline

% 020918
\multicolumn{2}{|c|}{\kb Convert Menu -- Input and ASCII Output}\\ \hline
\et ChdLoadTopOnly & Load requested cell from CHD only, create reference\\
  \hline
\et ChdRandomGzip & Use random-access table for gzipped files\\ \hline
\et AutoRename & Automatically change clashing cell names when reading\\ \hline
\et NoCreateLayer & Don't create new layers when reading\\ \hline
\et NoMapDatatypes & New layers take all datatypes in GDSII read\\ \hline
\et NoAskOverwrite & Suppress prompting for overwrite instructions\\ \hline
\et NoOverwritePhys & Don't overwrite phys memory cells when reading\\ \hline
\et NoOverwriteElec & Don't overwrite elec memory cells when reading\\ \hline
\et NoOverwriteLibCells & Don't overwrite library cells when reading\\ \hline
\et NoCheckEmpties & Skip checking for empty cells while reading\\ \hline
\et NoPolyCheck & Skip polygon reentrancy tests when reading\\ \hline
\et DupCheckMode & Check for duplicate items when reading\\ \hline
%\ifoa
\et EvalOaPCells & Attempt to create sub-master for OpenAccess pcell instances\\ \hline
%\fi
\et NoEvalNativePCells & Don't attempt to create sub-master for native pcell instances\\ \hline
\et MergeInput & Merge boxes and coincident objects when reading\\ \hline
\et LayerList & Layer list for conversion input filtering\\ \hline
\et UseLayerList & How to use layer list, skip or use only\\ \hline
\et LayerAlias & List of name=alias pairs\\ \hline
\et UseLayerAlias & Map layers using layer alias list\\ \hline
\et InToLower & Map lower case cell names to upper in archive read\\ \hline
\et InToUpper & Map upper case cell names to lower in archive read\\ \hline
\et InUseAlias & Use alias file when reading archive\\ \hline
\et InCellNamePrefix & Cell name translation prefix for archive read\\ \hline
\et InCellNameSuffix & Cell name translation suffix for archive read\\ \hline
\et CifLayerMode & CIF layer resolution method, 0--2\\ \hline
\et OasReadNoChecksum & Ignore checksum in OASIS input file\\ \hline
\et OasPrintNoWrap & Use one line per record in OASIS ASCII output\\ \hline
\et OasPrintOffset & Add file offsets to OASIS ASCII output\\ \hline

% 040724
\multicolumn{2}{|c|}{\kb Convert Menu -- Output}\\ \hline
\et StripForExport & Strip all format extensions from output file\\ \hline
\et WriteMacroProps & Include deprecated {\et macro} properties in output\\ \hline
\et KeepLibMasters & Write library cells when creating archive file\\ \hline
\et SkipInvisible & Do not write invisible layers to output\\ \hline
\et NoCompressContext & Don't compress instance lists in archive context\\
  \hline
\et RefCellAutoRename & Use auto-rename when writing reference cell data\\
  \hline
\et UseCellTab & Enable use of the cell override table in CHD access\\ \hline
\et SkipOverrideCells & Skip cells in override table in CHD access\\ \hline
\et OutAllCells & Output all cells in symbol table, not only current\\ \hline
\et Out32nodes & Use old 3.2 node property syntax in output\\ \hline
\et OutToLower & Map lower case cell names to upper in archive write\\ \hline
\et OutToUpper & Map upper case cell names to lower in archive write\\ \hline
\et OutUseAlias & Use alias file when writing archive\\ \hline
\et OutCellNamePrefix & Cell name translation prefix for archive write\\ \hline
\et OutCellNameSuffix & Cell name translation suffix for archive write\\ \hline
\et CifOutStyle & CIF output dialect and extensions specifier\\ \hline
\et CifOutExtensions & CIF output extension flags\\ \hline
\et CifAddBBox & Add bounding box comment to objects in CIF output\\ \hline
\et GdsOutLevel & GDSII release level conformance code (0--2)\\ \hline
\et GdsMunit & Modify M-UNITS value in GDSII output file\\ \hline
\et GdsTruncateLongStrings & Cut strings too long for record\\ \hline
\et NoGdsMapOk & Ignore unmapped layers in GDSII/OASIS output\\ \hline
\et OasWriteCompressed & Compress records in OASIS output\\ \hline
\et OasWriteNameTab & Use string table referencing in OASIS output\\ \hline
\et OasWriteRep & Try to combine similar objects in OASIS output\\ \hline
\et OasWriteChecksum & Compute and add checksum to OASIS output\\ \hline
\et OasWriteNoTrapezoids & Don't convert polys to trapezoids\\ \hline
\et OasWriteWireToBox & Convert wires to boxes when possible\\ \hline
\et OasWriteRndWireToPoly & Convert rounded-end wires to polygons\\ \hline
\et OasWriteNoGCDcheck & Don't look for common divisors in repetitions\\ \hline
\et OasWriteUseFastSort & Use faster but less effective sorting\\ \hline
\et OasWritePrptyMask & Don't write certain properties\\ \hline

% 101212
\multicolumn{2}{|c|}{\kb Custom Property Filtering}\\ \hline
\et PhysPrpFltCell & Physical cell property filter string\\ \hline
\et PhysPrpFltInst & Physical instance property filter string\\ \hline
\et PhysPrpFltObj & Physical object property filter string\\ \hline
\et ElecPrpFltCell & Electrical cell property filter string\\ \hline
\et ElecPrpFltInst & Electrical instance property filter string\\ \hline
\et ElecPrpFltObj & Electrical object property filter string\\ \hline

% 021615
\multicolumn{2}{|c|}{\kb Design Rule Checking}\\ \hline
\et Drc & Enable interactive rule checking\\ \hline
\et DrcNoPopup & Suppress violation reporting pop-up\\ \hline
\et DrcLevel & Set violation reporting level\\ \hline
\et DrcMaxErrors & Quit testing when this many violations found\\ \hline
\et DrcInterMaxObjs & Maximum number of objects to test interactively\\ \hline
\et DrcInterMaxTime & Maximum milliseconds for interactive test\\ \hline
\et DrcInterMaxErrors & Maximum violation count for interactive test\\ \hline
\et DrcInterSkipInst & Skip expensive instance check in interactive test\\ \hline
\et DrcChdName & Name of CHD for batch test\\ \hline
\et DrcChdCell & Name of top cell in CHD to test\\ \hline
\et DrcLayerList & List of layer names for filtering\\ \hline
\et DrcUseLayerList & Use only or skip layers in list\\ \hline
\et DrcRuleList & List of rule names for filtering\\ \hline
\et DrcUseRuleList & Use only or skip rule in list\\ \hline
\et DrcPartitionSize & Partition grid size in microns\\ \hline

% 102114
\multicolumn{2}{|c|}{\kb Extraction Tech}\\ \hline
\et AntennaTotal & Default input for {\cb !antenna} command\\ \hline
\et Db3ZoidLimit & Trapezoid limit for the 3-D database\\ \hline
\et LayerReorderMode & Default layer sequencing option\\ \hline
\et NoPlanarize & When set, no layers are assumed planarizing\\ \hline
\et SubstrateEps & Relative dielectric constant of substrate\\ \hline
\et SubstrateThickness & Assumed thickness of substrate in microns\\ \hline

% 061916
\multicolumn{2}{|c|}{\kb Extraction General}\\ \hline
\et ExtractOpaque & Ignore the OPAQUE flag in extraction\\ \hline
\et FlattenPrefix & Cell name prefix to flatten in extraction\\ \hline
\et GlobalExclude & Layer expression to exclude objects during extraction\\ \hline
\et GroundPlaneGlobal & Ground all pieces of clear-field ground plane\\ \hline
\et GroundPlaneMulti & Handle nets in dark-field ground plane\\ \hline
\et GroundPlaneMethod & Set ground plane inversion method 0--2\\ \hline
\et KeepSortedDevs & Include devices with terminals shorted\\ \hline
\et MaxAssocLoops & Maximum loop count for association\\ \hline
\et MaxAssocIters & Maximum iteration count for association\\ \hline
\et NoMeasure & Suppress measuring parameters of devices\\ \hline
\et UseMeasurePrpty & Read and update cached measurement results property\\ \hline
\et NoReadMeasurePrpty & Don't read cached measurement results from property\\ \hline
\et NoMergeParallel & Never merge parallel devices\\ \hline
\et NoMergeSeries & Never merge series devices\\ \hline
\et NoMergeShorted & Never merge devices with all terminals shorted\\ \hline
\et IgnoreNetLabels & Ignore labels found in nets\\ \hline
\et UpdateNetLabels & Create or update net labels after association\\ \hline
\et FindOldTermLabels & Search for old-style ``term labels''\\ \hline
\et MergeMatchingNamed & Merge nets with the same logical net name\\ \hline
\et MergePhysContacts & Merge contacts for split-net handling\\ \hline
\et NoPermute & Skip permutation search in association\\ \hline
\et PinLayer & Name of layer for net labels\\ \hline
\et PinPurpose & Name of purpose for net labels\\ \hline
\et RLSolverDelta & Overriding grid spacing for resistance/inductance
  extraction\\ \hline
\et RLSolverTryTile & Attempt to use tiling grid for 
  resistance/inductance extraction\\ \hline
\et RLSolverGridPoints & Grid points per device when not tiling\\ \hline
\et RLSolverMaxPoints & Maximum grid points per device when tiling\\ \hline
\et SubcPermutationFix & Apply post-association permutation fix\\ \hline
\et VerbosePromptline & Print info on prompt line during extraction\\ \hline
\et ViaCheckBtwnSubs & Check connectivity between subcircuit nets by via\\
 \hline
\et ViaSearchDepth & Cell hierarchy depth to search for vias\\ \hline
\et ViaConvex & Assume all vias are convex polygons\\ \hline

% 061516
\multicolumn{2}{|c|}{\kb Extract Menu Commands}\\ \hline
\et QpathGroundPlane & {\cb "Quick" Path}, use of inverted ground plane, 0--2\\
  \hline
\et QpathUseConductor & {\cb "Quick" Path}, allow Conductor objects in net\\
  \hline
\et EnetNet & Print net, {\cb enet} command\\ \hline
\et EnetSpice & Do include SPICE listing, {\cb enet} command\\ \hline
\et EnetBottomUp & Use leaf-to-root ordering in electrical netlist\\ \hline
\et PnetNet & Print extracted net list, {\cb pnet} command\\ \hline
\et PnetDevs & Print extracted device list, {\cb pnet} command\\ \hline
\et PnetSpice & Print extracted SPICE list, {\cb pnet} command\\ \hline
\et PnetBottomUp & Use leaf-to-root ordering in physical netlist\\ \hline
\et PnetShowGeometry & Include wire geometry in netlist file, {\cb pnet}
  command\\ \hline
\et PnetIncludeWireCap & Include routing caps in SPICE netlist, {\cb pnet}
  command\\ \hline
\et PnetListAll & List ignored and flattened subcells, {\cb pnet} command\\
  \hline
\et PnetNoLabels & No net names from labels in {\cb pnet} command output\\
  \hline
\et PnetVerbose & Print more information in {\cb pnet} command output\\ \hline
\et SourceAllDevs & Update internal-named devices in {\cb sourc} command\\
  \hline
\et SourceCreate & Create devices in {\cb sourc} command even if not empty\\
  \hline
\et SourceClear & Clear cell before updating with {\cb sourc} command\\ \hline
\et SourceGndDevName & Name of ground device used with {\cb sourc} command\\ \hline
\et SourceTermDevName & Name of terminal device used with {\cb sourc} command\\ \hline
\et NoExsetAllDevs & Don't use internal-named devices in {\cb exset} command\\
  \hline
\et NoExsetCreate & Don't create devices in {\cb exset} command\\ \hline
\et ExsetClear & Clear cells before updating in {\cb exset} command\\ \hline
\et ExsetIncludeWireCap & Include routing capacitance in {\cb exset} command\\
  \hline
\et ExsetNoLabels & No net names from labels in {\cb exset} command output\\
  \hline
\et LvsFailNoConnect & Force LVS failure if unconnected physical instance\\
  \hline
\et PathFileVias & Include vias in wire net files\\ \hline

% 071814
\multicolumn{2}{|c|}{\kb Capacitance Extraction Interface}\\ \hline
\et FcArgs & Capacitance extractor command line arguments\\ \hline
\et FcForeg & Run capacitance extractor in foreground if set\\ \hline
\et FcLayerName & Capacitance extractor masking layer name\\ \hline
\et FcMonitor & Capacitance extractor output appears in console window if set\\
  \hline
\et FcPlaneTarget & Refined element count target\\ \hline
\et FcPath & Path to capacitance extractor executable\\ \hline
\et FcPlaneBloat & Capacitance extractor substrate bloat dimension\\ \hline
\et FcUnits & Capacitance extractor file units: m, cm, mm, um, in, mils\\ \hline

% 011621
\multicolumn{2}{|c|}{\kb Inductance/Resistance Extraction Interface}\\ \hline
\et FhArgs & {\it FastHenry} command line arguments\\ \hline
\et FhDefaults & Text for {\vt .DEFAULT} line in {\it FastHenry} input\\
  \hline
\et FhDefNhinc & Default for {\vt nhinc} in {\it FastHenry} input\\ \hline
\et FhDefRh & Default for {\vt rh} in {\it FastHenry} input\\ \hline
\et FhForeg & {\it FastHenry} run in foreground if set\\ \hline
\et FhFreq & {\it FastHenry} frequency specification\\ \hline
\et FhLayerName & {\it FastHenry} interface masking layer name\\ \hline
\et FhManhGridCnt & Manhattanization grid cell count\\ \hline
\et FhMonitor & {\it FastHenry} output appears in console window if set\\
  \hline
\et FhOverride & Override {\vt nhinc}, {\vt rh} in {\it FastHenry} input\\ \hline
\et FhPath & Path to {\it FastHenry} executable\\ \hline
\et FhUnits & {\it FastHenry} file units: m, cm, mm, um, in, mils\\ \hline
\et FhUseFilament & Use {\it FastHenry} filaments\\ \hline
\et FhVolElMin & {\it FastHenry} volume element minimum size factor\\ \hline
\et FhVolElTarget & {\it FastHenry} volume element count target\\ \hline
\et FhVolEnable & Enable segment refinement.\\ \hline

% 102613
\multicolumn{2}{|c|}{\kb Help System}\\ \hline
\et HelpDefaultTopic & Suppress or set the default help topic\\ \hline
\et HelpMultiWin & Use separate windows for help references\\ \hline
\end{longtable}


% -----------------------------------------------------------------------------
% !set:special 021515
\section{Special Constructs}
\index{variables!special constructs}

These are special {\cb !set} variables and constructs which have
significance to {\Xic}.

\begin{description}
\item{(no arg)}\\
Pop up a list of the currently set variables.  Variables in this list
(with the exception of the path variables) can be removed with the
{\cb !unset} command.

\item{?}\\
Pop up a list of the variables that have meaning to {\Xic}.

\index{parameter setting}
\item{{\vt @}{\it devname\/}.{\it property}}\\
Set the {\it property\/} on device {\it devname} to {\it value\/}. 
This construct enables device properties to be added to devices via
the command line.  The first character of the {\it name} token must be
`{\vt @}', followed by the name of the device, a period, and the name
of the property to set.  Valid property names are ``{\vt name}'',
``{\vt model}'', ``{\vt value}'', ``{\vt param}'', ``{\vt other}'',
and ``{\vt nophys}''.  For backward compatibility, ``{\vt initc}'' is
recognized as an alias for ``{\vt param}''.  An unrecognized property
name will be saved as an ``other'' property.

Examples:
\begin{description}
\item{\vt !set @L2.value 2ph}\\
sets the value of L2 to 2ph.
\item{\vt !set @Moutput.param L=2}\\
sets the length parameter of mosfet {\vt Moutput}.
\end{description}

The {\it devname} field can be the name of a mutual inductor, in which
case the valid properties are ``{\vt name}'' and ``{\vt value}''.
\end{description}


% -----------------------------------------------------------------------------
% !set:startup 012014
\section{Startup}
\index{variables!startup}

The following variables control fundamental behavior of the {\Xic}
program.  These must be specified before reading design or technology
data.  Unlike all other variables, these can be set only from the {\vt
.xicinit} file, which is read before the technology file, or the
technology file.  These can not be set or unset in a {\vt .xicstart}
file, which is read after the technology file, unless no technology
file is read.  They can not be set by any other means.

The {\vt Set} script function can be used in the initialization files
to set this variable.  In the technology file, the {\vt !set} command
should be used, and this must appear at the top of the file, before
layer definitions.

\begin{description}
\index{DatabaseResolution variable}
\item{\et DatabaseResolution}\\
{\bf Value:} string: ``{\vt 1000}'', ``{\vt 2000}'', ``{\vt 5000},
 or ``{\vt 10000}''.\\
By default, {\Xic} uses an internal resolution of 1000 units per
micron.  In releases prior to 3.0.12, this was internally hard-coded. 
As the dimensions used in integrated circuits continue to shrink, an
option for higher resolution was added through use of the {\et
DatabaseResolution} variable.

The internal resolution can be set with this variable, to one of the
listed choices.  If unset, 1000 units is used.  This resolution
applies only to physical data, electrical resolution is fixed at 1000.

Superficially, changing the internal resolution has only subtle
effects from the user's vantage point.  Some of these are:

\begin{enumerate}
\item{If not 1000, four digits following the decimal point are used
when printing coordinates in microns, in many places in {\Xic}.
Otherwise, only three digits are used.}

\item{The ultimate zoom-in and grid spacing sizes are smaller for
higher resolutions.}

\item{The size of ``infinity'', the maximum accessible size for the
design, becomes smaller as resolution is increased, since coordinates
are stored internally as 32-bit integers.  For 1000 units, the field
width is approximately 2 meters, which decreases to 20 centimeters at
10000 units.  This should still be plenty for most purposes.}

\item{Layout files produced by {\Xic} will use the internal
resolution, so that no accuracy is lost.}
\end{enumerate}

\index{NetNamesCaseSens variable}
\item{\et NetNamesCaseSens}\\
{\bf Value:} boolean.\\
By default, net names are case-insensitive in {\Xic}, and saved
internally as upper-case.  If this boolean variable is set, net names
are taken as case-sensitive.  This impacts lookup of nets by name and
comparison of net names for identification and matching purposes, as
used in the electrical schematic and extraction system.

\index{Subscripting variable}
\item{\et Subscripting}\\
{\bf Value:} string.\\
In {\Xic}, net name and cell instance indexing can employ angle,
square, or curly brackets, as in the forms {\vt mynet<1>}, {\vt
mynet[1]}, and {\vt mynet\{1\}}.  These forms are equivalent and can
be freely mixed in {\Xic} input.

However, on occasion {\Xic} will create a vector name for output.  The
default is to use angle brackets, but this can be changed by setting
this variable.  The variable must be set to a word starting with one
of the letters {\vt a}, {\vt s}, or {\vt c}, case insensitive (the
``word'' can be just the letter).  Only the first letter is
significant.  The letters signify angle, square, or curly brackets.

\index{DrfDebug variable}
\item{\et DrfDebug}\\
{\bf Value:} boolean.\\
This obscure flag applies when using the {\vt ReadDRF}, {\vt
ReadCdsTech} and {\vt ReadCniTech} technology file directives.  If
this variable is set, non-serious warnings encountered when reading
these files will be printed.  One such warning is generated by use in
the Virtuoso or Ciranova technology file of color, stipple, or packet
names that have not been defined in the display resourse file (DRF). 
Since there are defaults, these unresolved name references are not a
serious problem.

At least one commercial process design kit had lots of these issues,
and reporting these as warnings on every {\Xic} startup became
irritating, particularly since it is not something that the typical
user can fix, or want to bother with fixing.  Thus, these
not-really-errors are ignored by default, but if the user desires then
setting this variable will make any such errors visible.

This variable must be set before the files are read.  Setting this
variable at the top of the {\Xic} technology file with the {\vt !set}
construct is a convenient way.
\end{description}


% -----------------------------------------------------------------------------
% !set:paths 102008
\section{Paths and Directories}
\label{pathvars}
\index{variables!paths}

These variables set the search paths (see \ref{searchpaths}) and
document directory used in {\Xic}.  These have counterpart environment
variables (see \ref{xicenv}).  The search paths can also be set from
the technology file.

If not set by any means, internal defaults are used for the search
paths and document directory.  Under Windows, the default is set to
point to the actual installation location subdirectories when
necessary.  Under Unix/Linux, the {\et XT\_PREFIX} environment variable
should be set to the installation location prefix that effectively
replaces ``{\vt /usr/local}''.

Below, {\it PREFIX} is obtained from the Windows Registry database
under Windows, which is defined when the program is installed.  Under
Unix/Linux, {\it PREFIX} is obtained from the {\et XT\_PREFIX}
environment variable.  In both cases, the default value for {\it
PREFIX\/}, if another definition is not found, is ``{\vt
/usr/local}''.

\begin{description}
% 062109
\index{Path variable}
\item{\et Path}\\
{\bf Value:} path string, can't be unset.\\
This variable contains the design data search path.  It is always
defined, and can not be unset.  This path is used to find native cell,
archive, and library files.

If not set by any means, a default path is used.\\
Default: ``( {\vt .} )''

% 061308
\index{LibPath variable}
\item{\et LibPath}\\
{\bf Value:} path string, can't be unset.\\
This variable contains the startup library search path.  It is always
defined, and can not be unset.  The library path is used to find the
technology file, device and model libraries, and other initialization
files.

Unlike other search paths, the current directory is {\it always}
searched first, whether or not this is indicated in the search path
string.  If not set by any means, a default library path is used.\\
Default: ``( {\vt .} {\it PREFIX\/}{\vt /xictools/xic/startup} )''

% 061308
\index{HelpPath variable}
\item{\et HelpPath}\\
{\bf Value:} path string, can't be unset.\\
This variable contains the help search path.  It is always defined,
and can not be unset.  This path is used to find files that contain
information for the help system.

If not set by any means, a default help path is used.\\
Default: ``( {\it PREFIX\/}{\vt /xictools/xic/help} )''

% 102208
\index{ScriptPath variable}
\item{\et ScriptPath}\\
{\bf Value:} path string, can't be unset.\\
This variable contains the script search path.  It is always defined,
and can not be unset.  This path is used to find script and menu files
that will appear in the {\cb User Menu}.

If not set by any means, a default script path is used.\\
Default: ``( {\it PREFIX\/}{\vt /xictools/xic/scripts} )''
\end{description}

The treatment of any path which is given with a native cell to open in
the {\cb Open} command can be altered with the next two variables.

\begin{description}
% 061408
\index{NoReadExclusive variable}
\item{\et NoReadExclusive}\\
{\bf Value:} boolean.\\
When a native cell name with a path is opened, the path is
stripped from the cell name.  If the path is not already in the search
path, it is added.  Ordinarily, the path is put in front of the search
path for the duration of the read, so that subcells will be opened
from the same directory.  If this variable is set, the path is not
necessarily moved to the front of the search path.

% 102208
\index{AddToBack variable}
\item{\et AddToBack}\\
{\bf Value:} boolean.\\
A path stripped from a given cell name in the {\cb Open} command is
added to the back of the search path, rather than the front.
\end{description}

The behavior is described below for the various permutations:

\begin{quote}
{\et NoReadExclusive} unset\\
{\et AddToBack} unset\\
(default behavior)\\
\\
The directory is added to the front of the search path during the read.
The ``.'' element of the path, if it exists, is moved to the front
after the read.
\end{quote}

\begin{quote}
{\et NoReadExclusive} unset\\
{\et AddToBack} set\\
\\
The directory is added to the front of the search path during the read.
The ``.'' element of the path, if it exists, is moved to the front,
and the directory is moved to the end after the read.
\end{quote}

\begin{quote}
{\et NoReadExclusive} set\\
{\et AddToBack} unset\\
\\
If the directory exists in the path, nothing is changed, otherwise the
directory is added to the front.  After the read, the ``.'' entry, if it
exists, is moved to the front.
\end{quote}

\begin{quote}
{\et NoReadExclusive} set\\
{\et AddToBack} set\\
\\
If the directory exists in the path, nothing is changed, otherwise the
directory is added to the end.
\end{quote}

\begin{description}
% 061308
\index{DocsDir variable}
\item{\et DocsDir}\\
{\bf Value:} path to directory.\\
The given directory is searched for the release notes, for the {\cb
Release Notes} command in the {\cb Help Menu}.

If not set by any means, a default document directory is used.\\
Default:  ``{\it PREFIX\/}{\vt /xictools/xic/docs}''

% 030413
\index{ProgramRoot variable}
\item{\et ProgramRoot}\\
{\bf Value:} string.\\
This variable is set by the program to the installation location
assumed by the program at program start-up.  For example, for {\Xic}
installed in the standard location, the variable will contain the
string ``{\vt /usr/local/xictools/xic}''.  This variable is not used
by {\Xic}, but is available in scripts so that the user can query the
value when needing to access files in the installation location.  Note
that the user can set or clear this variable arbitrarily.

% 061308
\index{TeePrompt variable}
\item{\et TeePrompt}\\
{\bf Value:} path to file.\\
When set, the prompt line messages are copied to the given file.  If a
file name is not given, or when the variable is unset, redirection
stops.  The value string can be ``{\vt stderr}'' or ``{\vt stdout}''
to redirect output to the terminal window instead of a file.
\end{description}


% -----------------------------------------------------------------------------
% !set:generalvis 100616
\section{General Visual}
\index{variables!general visual}

The following {\cb !set} variables affect general visual attributes of
{\Xic}.

\begin{description}
% 021812
\index{MouseWheel variable}
\item{\et MouseWheel}\\
{\bf Value:} two floating-point numbers.\\
This variable controls the per-click increments for mouse wheel
panning and zooming of drawing windows.  Without a key held, the
mouse wheel scrolls drawing windows up/down.  If {\kb Shift} is held,
scrolling is right/left.  If {\kb Ctrl} is held (overrides {\kb
Shift}), the mouse wheel zooms out or in.

The string provided to this variable consists of two space-separated
floating-point numbers, each in the range of 0 -- 0.5.  The first is
the pan factor, the second is the zoom factor.  The default is {\vt
0.1 0.1}.  Larger numbers increase the effect per mouse wheel click. 
If either number is set to 0, that effect (pan or zoom) is
suppressed.  Thus, to turn off mouse wheel support in drawing windows,
give ``{\vt 0 0}''.

% 061308
\index{ListPageEntries variable}
\item{\et ListPageEntries}\\
{\bf Value:} integer 100--50000\\
This sets the number of entries that appear per page in the pop-ups
that list cells.  If the number of cells to be listed exceeds this
number, a page menu will become visible in the listing panel.  Each
page will contain at most this number of entries.  Only the entries
for the currently selected page will be visible.  If this variable is
not set, the default value is 5000.

% 010918
\index{NoInstnameLabels variable}
\item{\et NoInstnameLabels}\\
{\bf Value:} boolean.\\
Starting in release 4.3.3, the label used in physical display windows
for unexpanded cell instances is the instance name, which consists of
the master cell name followed by a colon separator and a unique
integer index.  When this variable is set, the label shows the master
cell name only, the same as in earlier {\Xic} releases.

% 082008
\index{NoLocalImage variable}
\item{\et NoLocalImage}\\
{\bf Value:} boolean.\\
In {\Xic} generation 3, a ``local image'' may be used to compose
images for screen rendering.  The display image is composed in local
memory, and flushed to the screen when drawing is complete.  When
using X-Windows, this provides much faster rendering of complex
displays, particularly when running remotely over a network, than the
standard method of server-side image manipulation as used exclusively
in previous {\Xic} releases.

The local image method is not used under Windows, since it provides no
benefit in the Windows architecture.  It is also not used if the
hierarchy being shown is not complex, i.e., contains few subcells and
objects, as the conventional drawing mode is quicker in this case.

If this variable is set, the local image feature is disabled, and
rendering is always performed by server-side functions.  This is for
debugging, it is not likely that the user will need to set this
variable.

% 061308
\index{NoPixmapStore variable}
\item{\et NoPixmapStore}\\
{\bf Value:} boolean.\\
In normal operation, the screen refreshes are buffered through an
in-core pixel map.  The geometry is rendered in the map, and when
finished the map is copied to the screen.  This is generally faster
than drawing directly to the screen.  When this variable is set, all
drawing is direct to the screen.  This is intended only for debugging
purposes.

% 061308
\index{NoDisplayCache variable}
\item{\et NoDisplayCache}\\
{\bf Value:} boolean.\\
In normal operation, boxes are cached during rendering, and displayed
with a multiple object rendering call.  This should be faster than
rendering the boxes individually.  When this variable is set, the
caching is disabled.  This is intended only for debugging purposes.

% 100516
\index{LowerWinOffset variable}
\item{\et LowerWinOffset}\\
{\bf Value:} integer -16 to 16.\\
For windows that are automatically placed just above the prompt line,
giving this variable a positive value will position these windows
toward the top of the screen by that many pixels.  This is useful when
using ``plasma'' displays (such as Mac or KDE), where the shadow falls
on the prompt line, which can be distracting.  It might also be
helpful if the window positioning is incorrect, which might occur with
some window managers.  This variable tracks the state of the {\cb
Pixels between pop-ups and prompt line} entry area in the {\cb
General} page of the {\cb Window Attributes} panel from the {\cb
Attributes Menu}.

% 061308
\index{PhysGridOrigin variable}
\item{\et PhysGridOrigin}\\
{\bf Value:} two floating-point numbers.\\
This will set the origin of the displayed grid in physical-mode
windows.  The value consists of two floating-point numbers, which are
taken as the x and y grid origin location in microns.  This applies
only to the displayed grid, and specifically not to the grid/snap used
when creating or locating objects.

When an offset is active, the word {\vt "PhGridOffs"} will be 
displayed in the status area.

% 061308
\index{ScreenCoords variable}
\item{\et ScreenCoords}\\
{\bf Value:} boolean.\\
When set, the coordinate readout area will display the position of the
mouse pointer in the current drawing window in the window's pixel
coordinates.  This is for development/debugging purposes and is not
likely to be useful to the user, and in fact may cause trouble if used
while editing.

% 021912
\index{PixelDelta variable}
\item{\et PixelDelta}\\
{\bf Value:} integer (default 3).\\
This variable determines how close, in screen pixels, a user must
click to a feature for {\Xic} to recognize this as clicking ``on''
that feature.  The value should likely be set larger than the default
for very high-resolution screens, or for inaccurate pointing devices,
or for users with less than the sharpest eyesight. 

% 020715
\index{NoPhysRedraw variable}
\item{\et NoPhysRedraw}\\
{\bf Value:} boolean.\\
When set, physical windows will not be redrawn after a layer
visibility change in the layer table.  This is traditional behavior of
earlier {\Xic} releases, which assumed that screen redraws would take
some time and the user would prefer to force a redraw when desired.

% 030815
\index{NoToTop variable}
\item{\et NoToTop}\\
{\bf Value:} boolean.\\
By default, most if not all {\Xic} sub-windows will automatically rise
to the top if completely covered by the {\Xic} main window.  This
includes plot windows from {WRspice} running under control of {\Xic}
(however most window managers don't support this).  If this variable
is set, the action will be disabled.  This will apply to plot windows
from {\WRspice} that is started after the variable is set. 

Some (probably most) window managers will do this automatically for
sub-windows, in which case setting this variable will have no effect
on the {\Xic} sub-windows, but would still affect {\WRspice} plot
windows if the window manager supports this.  The only window manager
I know of that supports this is Exceed 2008, because it is old.  The
protocol is deemed a security risk and has been disabled in modern
window managers for some time.

\end{description}


% -----------------------------------------------------------------------------
% !set:bangcmds 100616
\section{Keyboard `!' Commands}
\index{variables!`"!' commands}

The {\cb !set} variables below affect the `!' commands available from
the keyboard.  Commands of this form that are not recognized as
internal commands are assumed to be operating system commands, and are
executed in a separate window under a command shell.

\begin{description}
% 061308
\index{Shell variable}
\item{\et Shell}\\
{\bf Value:} string.\\
This variable can be set to the name of a command interpreter which
will be used for the `!' and !{\it shellcmd} inputs.  The interpreter
will be instantiated in its own window.  If not given, the shell
program used will be taken from the {\et SHELL} environment variable,
and if this variable is not found the default is ``{\vt /bin/sh}''. 
{\WRspice} users can set the shell to ``{\vt wrspice}'' for quick
access to the full user interface of that program.

Under Microsoft Windows, the value must be a full path name to the
shell executable, and the {\et COMSPEC} environment variable is also
consulted for the default shell, after the {\et SHELL} variable.
\end{description}


%\ifoa
% -----------------------------------------------------------------------------
% !set:oa 021515
\section{OpenAccess Interface}
\index{variables!OpenAccess}

The following {\cb !set} variables affect the OpenAccess interface. 
These variables have no effect unless the OpenAccess plug-in is
loaded.

\begin{description}
% 021913
\index{OaLibraryPath variable}
\item{\et OaLibraryPath}\\
{\bf Value:} string.\\
This can specify a path to a directory, which will be searched if a
library can not be found.  When opening a library, and the library is
not found in the {\vt lib.defs} (or {\vt cds.lib}) file, the system
will look for the library as a subdirectory of the directory path
specified in this variable, if any.  This allows use of OpenAccess
libraries that are hidden from other tools.

% 022316
\index{OaDefLibrary variable}
\item{\et OaDefLibrary}\\
{\bf Value:} string.\\
This can be set to the name of a library in the OpenAccess {\vt
lib.defs} (or {\vt cds.lib}) file, or a subdirectory of the {\et
OaLibraryPath} if any.  This will be used as the default library in
certain commands, if no other library is given.  Presently, the {\cb
!oabrand}, {\cb !oasave}, and {\cb !oaload} commands use this. 

% 041415
\index{OaDefTechLibrary variable}
\item{\et OaDefTechLibrary}\\
{\bf Value:} string.\\
This can be set to the name of a library in the OpenAccess {\vt
lib.defs} (or {\vt cds.lib}) file, or a subdirectory of the {\et
OaLibraryPath} if any.  When a library is created, it will attach the
technology database associated with the library name found in this
variable, if set.  If the named library has an attached technology,
the same attachment will be applied to the new library.  Otherwise,
the new library will attach the local technology database of the named
library.

% 022316
\index{OaDefLayoutView variable}
\item{\et OaDefLayoutView}\\
{\bf Value:} string.\\
This sets the view name assumed for physical data in OpenAccess.  When
not set, the default is ``{\vt layout}''.  This variable tracks an
entry area in the {\cb OpenAccess Defaults} panel.

% 022316
\index{OaDefSchematicView variable}
\item{\et OaDefSchematicView}\\
{\bf Value:} string.\\
This sets the view name assumed for schematic data in OpenAccess. 
When not set, the default is ``{\vt schematic}''.  This variable
tracks an entry area in the {\cb OpenAccess Defaults} panel.

% 022316
\index{OaDefSymbolView variable}
\item{\et OaDefSymbolView}\\
{\bf Value:} string.\\
This sets the view name assumed for symbol data in OpenAccess.  When
not set, the default is ``{\vt symbol}''.  This variable tracks an
entry area in the {\cb OpenAccess Defaults} panel.

% 022316
\index{OaDefDevPropView variable}
\item{\et OaDefDevPropView}\\
{\bf Value:} string.\\
This provides a default name for a simulator-specific view from which
device properties are obtained.  Often these peoperties are in a
format intended for a specific simulator.  If not set, the default is
``{\vt HspiceD}'', which assumes the Hspice simulator, to which the
{\WRspice} simulator has compatibility.  This variable tracks an entry
area in the {\cb OpenAccess Defaults} panel.

% 021913
\index{OaDmSystem variable}
\item{\et OaDmSystem}\\
{\bf Value:} string.\\
If this variable is set to a string starting with `{\vt t}' or `{\vt
T}', OpenAccess will be set to use the {\et Turbo} design management
system.  Otherwise, OpenAccess will use the default {\et FileSys}
system.  Compatibility with Cadence seems to require use of the {\et
FileSys} system.  The {\et Turbo} system is claimed to have higher
performance.  The format of information stored on disk is very
different in the two approaches.  Supposedly, this should be invisible
to the OpenAccess user.

% 021913
\index{OaDumpCdfFiles variable}
\item{\et OaDefTechLibrary}\\
{\bf Value:} boolean.\\
If this variable is set when a parameterized cell is opened in
OpenAccess, the CDF data for the cell will be dumped to a file in the
current directory.  The file name is the cell name with a ``{\vt
.cdf}'' extension.  This is for development/debugging.

% 030416
\index{OaUseOnly variable}
\item{\et OaUseOnly}\\
{\bf Value:} string.\\
This variable can be used to limit data imported from and exported to
OpenAccess to physical only or electrical only.  The variable tracks
the state of the {\cb Data to use from OA} radio group in the {\cb
OpenAccess Libraries} panel. 

If set to ``1'', or to any text starting with `p' or `P', only
physical layout data will be read from or written to OpenAccess.  If
set to ``2'', or to any text starting with `e' or `E', only electrical
data (schematic and symbol) will be read or written.  If not set, or
set to anything else, both physical and electrical data will be read
or written.

The restriction applies to conversion to and from OpenAccess, by any
method in {\Xic}.

One useful observation is that one can import a schematic from
Virtuoso even if no provision has been made to export pcells.  Unless
the Express PCell feature is enabled in Virtuoso, conversion of
Skill-based pcells will fail as they can not be evaluated outside of
the Cadence environment.  The Express PCell feature makes available a
cache of pre-built sub-masters that can be exported.  Without this,
attempting to import physical data will produce a lot of errors, which
can be avoided if only a schematic is needed by importing electrical
data only.
\end{description}
%\fi

% -----------------------------------------------------------------------------
% !set:pcells 021515
\section{Parameterized Cells}
\index{variables!pcells}

The following {\cb !set} variables affect parameterized cell (pcell)
capabilities.  Most of these track elements of the {\cb PCell Control}
panel obtained from the {\cb PCell Control} button in the {\cb Edit
Menu}.

\begin{description}
% 022513
\index{PCellAbutMode variable}
\item{\et PCellAbutMode}\\
{\bf Value:} integer 0--2, default 1.\\
{\Xic} provides an internal implementation of the Ciranova
auto-abutment protocol (see \ref{pcabut}).  This variable sets the
value of the {\et otherPinsOnNet} parameter mentioned in the protocol
description.  How the pcell uses this variable is up to the pcell
author, there is really no {\it a-priori} interpretation, it is an
integer of value 0, 1, or 2.

The Ciranova {\et Nmos2} example pcell interprets the value to have
the following meanings.  This is likely to be used in other pcells as
well.

\begin{tabular}{ll}\\
0 & Auto-abutment is disabled.\\
1 & Abutment takes place with no contact between the gates.\\
2 & Abutment takes place with a M1 contact between the gates.\\
\end{tabular}

This variable tracks the {\cb Auto-abutment mode} selection menu in
the {\cb PCell Control} panel.

% 022513
\index{PCellHideGrips variable}
\item{\et PCellHideGrips}\\
{\bf Value:} boolean.\\
{\Xic} implements the Ciranova stretch handle protocol (see
\ref{pcstretch}), and by default stretch handles are visible in
selected, expanded cell instances, and when editing the sub-master. 
If this variable is set, all stretch handles will be invisible and
disabled.

This variable tracks the state of the {\cb Hide and disable stretch
handles} check box in the {\cb PCell Control} panel.

% 031113
\index{PCellGripInstSize variable}
\item{\et PCellGripInstSize}\\
{\bf Value:} integer 0--1000.\\
Stretch handles are not shown and inactive if the instance rendering
size is too small.  This is to avoid triggering a stretch
inadvertently.  By default, the smallest of the instance height/width
must be 100 screen pixels or larger to show and activate stretch
handles.  This variable can be set to provide a different threshold.
 
This variable tracks the value of the {\cb Instance min.  pixel size
for stretch handles} entry in the {\cb PCell Control} panel from the
{\cb Edit Menu}.

% 022816
\index{PCellKeepSubMasters variable}
\item{\et PCellKeepSubMasters}\\
{\bf Value:} boolean.\\
When a parameterized cell (pcell) is instantiated, a sub-master cell
is created in memory which represents the instantiation for its given
parameter set.  By default, sub-master cells exist only in memory, and
are created as needed from the original pcell.

When this variable is set, sub-masters that have been created will be
included when writing output.  This will also be true when the {\cb
StripForExport} variable is set.  This applies when writing all
output, {\bf except} when using the {\cb Save} and {\cb Save As}
buttons in the {\cb File Menu}, and the equivalent text accelerators
and including the prompts when exiting the program.  It is also
ignored when using the {\vt Save} script function.

When opening a layout containing pcell instances and the corresponding
cell files are found, the cell files will be read instead of
evaluating the pcell.  This can be faster, and it also allows the
design to be opened if the original pcell is not available or can't be
processed.  However, the cells will behave like normal cells, not
pcells, in this case.

This variable tracks the state of the {\cb Include parameterized cell
sub-masters} check box in the {\cb Export Control} panel.

% 022513
\index{PCellListSubMasters variable}
\item{\et PCellListSubMasters}\\
{\bf Value:} boolean.\\
When a parameterized cell (pcell) is instantiated, a sub-master cell
is created in memory which represents the instantiation for its given
parameter set.  By default, sub-master cells exist only in memory, and
are created as needed from the original pcell.

When this variable is set, sub-masters that have been created will be
included in the list of modified cells contained in the {\cb Modified
Cells} pop-up, which is obtained from the {\cb Save} button in the
{\cb File Menu}.  The sub-masters can be saved as native cell files in
the current directory.

When opening a layout containing pcell instances and the corresponding
cell files are found, the cell files will be read instead of
evaluating the pcell.  This can be faster, and it also allows the
design to be opened if the original pcell is not available or can't be
processed.  However, the cells will behave like normal cells, not
pcells, in this case.

This variable tracks the state of the {\cb List sub-masters as
modified cells} check box in the {\cb PCell Control} panel.

% 021513
\index{PCellScriptPath variable}
\item{\et PCellScriptPath}\\
{\bf Value:} string.\\
This variable provides a search path (see \ref{searchpaths}) to use
when locating parameterized cell (pcell) scripts.  This applies when a
pcell {\et pc\_script} property uses the {\vt @READ} directive to
obtain the corresponding script, and the path provided by the
directive is not rooted.

Unlike the main search path variables described in \ref{pathvars},
this variable is unset by default.

% 022513
\index{PCellShowAllWarnings variable}
\item{\et PCellShowAllWarnings}\\
{\bf Value:} boolean.\\
During pcell script evaluation, certain warning messages are disabled,
including checking for coincident objects.  Some of the Ciranova
example pcells produce such warnings, and it is highly annoying that
the messages pop up after every evaluation.  The warnings may be of
interest to the pcell author, but are generally nothing but a nuisance
to the pcell user.  If this variable is set, then these warnings will
be displayed and not suppressed.

This variable tracks the state of the {\cb Show all evaluation
warnings} check box in the {\cb PCell Control} panel.
\end{description}


% -----------------------------------------------------------------------------
% !set:stdvia 031815
\section{Standard Vias}
\index{variables!strandard vias}

These variables apply to standard vias (see \ref{stdvia}).

\begin{description}
% 022816
\index{ViaKeepSubMasters variable}
\item{\et ViaKeepSubMasters}\\
{\bf Value:} boolean.\\
When a standard via is instantiated, a sub-master cell is created in
memory which represents the instantiation for its given parameter set. 
By default, sub-master cells exist only in memory, and are created as
needed by {\Xic}.

When this variable is set, sub-masters that have been created will be
included when writing output.  This will also be true when the {\cb
StripForExport} variable is set.  This applies when writing all
output, {\bf except} when using the {\cb Save} and {\cb Save As}
buttons in the {\cb File Menu}, and the equivalent text accelerators
and including the prompts when exiting the program.  It is also
ignored when using the {\vt Save} script function.

This variable tracks the state of the {\cb Include standard via cell
sub-masters} check box in the {\cb Export Control} panel.

% 031815
\index{ViaListSubMasters variable}
\item{\et ViaListSubMasters}\\
{\bf Value:} boolean.\\
When a standard via is instantiated, a sub-master cell is created in
memory which represents the instantiation for its given parameter set. 
By default, sub-master cells exist only in memory, and are created as
needed by {\Xic}.

When this variable is set, sub-masters that have been created will be
included in the list of modified cells contained in the {\cb Modified
Cells} pop-up, which is obtained from the {\cb Save} button in the
{\cb File Menu}.  The sub-masters can be saved as native cell files in
the current directory.
\end{description}


% -----------------------------------------------------------------------------
% !set:scripts 021515
\section{Scripts}
\index{variables!scripts}

The following {\cb !set} variables affect the script parser.

\begin{description}
% 021912
\index{LogIsLog10 variable}
\item{\et LogIsLog10}\\
{\bf Value:} boolean.\\
In {\Xic} releases prior to 3.2.23, the {\vt log} function returned
the base-10 logarithm.  This definition was changed in 3.2.23, and
the {\vt log10} function added, for consistency with programming
languages, {\WRspice}, and most other software.  This will require
users to update legacy scripts that use the {\vt log} function to
call {\vt log10} instead.

This variable provides a temporary fix.  When set, the {\vt log}
function will return the base-10 value.  However, it is strongly
recommended that legacy scripts be updated, and this variable not be
used permanently.
\end{description}

See also the {\et ScriptPath} variable in \ref{pathvars}.


% -----------------------------------------------------------------------------
% !set:selections 021515
\section{Selections}
\index{variables!selections}

The following {\cb !set} variables affect object/cell selections using
the pointing device.

\begin{description}
% 062715
\index{MarkInstanceOrigin variable}
\item{\et MarkInstanceOrigin}\\
{\bf Value:} boolean.\\
When set, selected physical instances will have the cell origin marked
with a cross.  This applies to the selection highlighting, as well as
to the ghost rendition which is attached to the mouse pointer during a
move or copy operation.

Showing the origin may seem trivial, but marking the origin requires a
bit of overhead since it requires running a transformation and keeping
track of an additional redisplay area since the origin may be outside
of the cell bounding box.  Thus, the default is to not show the mark.

This variable tracks the state of the {\cb Show origin of selected
physical instances} check box in the {\cb Selections} page of the {\cb
Window Attributes} panel from the {\cb Set Attributes} button in the
{\cb Attributes Menu}.

% 062715
\index{MarkObjectCentroid variable}
\item{\et MarkObjectCentroid}\\
{\bf Value:} boolean.\\
In mathematics, the centroid or geometric center of a two-dimensional
region is the arithmetic mean of all the points in the shape.  When
this check box is set, selected objects will mark the centroid with a
cross.  This applies to the selection highlighting, as well as to the
ghost rendition which is attached to the mouse pointer during a move
or copy operation.

This variable tracks the state of the {\cb Show centroids of selected
physical objects} check box in the {\cb Selections} page of the {\cb
Window Attributes} panel from the {\cb Set Attributes} button in the
{\cb Attributes Menu}.

% 061308
\index{SelectTime variable}
\item{\et SelectTime}\\
{\bf Value:} integer 100--1000.\\
When button 1 is used for object manipulation and editing, there is a
time delay which differentiates a ``click'' from a ``drag''.  This
delay, which defaults to 250 milliseconds, can be adjusted by setting
this variable.  If the user encounters difficulty establishing an area
select, for example, as opposed to a move/copy operation, then setting
a longer time delay may be advantageous.

% 061308
\index{NoAltSelection variable}
\item{\et NoAltSelection}\\
{\bf Value:} boolean.\\
When set, the legacy logic is used for mouse click selection
operations.

% 012411
\index{MaxBlinkingObjects variable}
\item{\et MaxBlinkingObjects}\\
{\bf Value:} integer 500--250000.\\
This can be set to an unsigned integer in the range 500--250000.  If
there are more than this number of objects selected, they won't blink
in true-color display modes.  If {\et NoPixmapStore} is set, this
threshold is divided by 8.  The default if not set is 25000 objects. 
If there are too many objects, the time to redraw for blinking becomes
excessive, this variable can be used to fine-tune this threshold to
the user's graphical system.
\end{description}


% -----------------------------------------------------------------------------
% !set:sidemenu 101016
\section{Side Menu Commands}
\index{variables!side menu commands}

The following {\cb !set} variables affect the functioning of
commands found in the side menu.

\begin{description}
% 100416
\index{MasterMenuLength variable}
\index{master menu length}
\item{\et MasterMenuLength}\\
{\bf Value:} integer 1--75.\\
This integer variable sets the length of the list of master cells
retained in the {\cb Cell Placement Control} panel.  The default is
25, which may not be fully visible for some screen resolutions.

This tracks the setting of the {\cb Maximum menu length} entry in the
{\cb Cell Placement Control} panel from the side menu.

% 102713
\index{DevMenuStyle variable}
\index{device menu}
\item{\et DevMenuStyle}\\
{\bf Value:} integer 0--2.\\
This variable tracks and sets the presentation style of the device
menu (described in \ref{devmenu}) which is used in electrical mode. 
There are three styles, selected by giving this property a value of 0,
1, or 2.  The default menu, style 0, contains a menu bar with entries
for categories, such as {\cb Sources} and {\cb Terminals}.  Style 1 is
similar, however the entries are alphabetic corresponding to the first
letter of the device name.  Style 2 provides buttons marked with the
device schematic symbol.  This style occupies the most screen space,
but may be more convenient for new users.

% 101016
\index{LabelDefHeight variable}
\item{\et LabelDefHeight}\\
{\bf Value:} real 0.01 -- 10.0.\\
This sets the minimum label height, in microns, for new text labels. 
The actual initial height may be larger, depending on the zoom factor
of the window, but it can not be smaller.  The default if this
variable is not set is 1.0 micron.

This variable was named {\et DefLabelHeight} in releases prior to
4.2.14.

This variable tracks the {\cb Default minimum label height} entry area
in the {\cb Labels} page of the {\cb Window Attributes} panel from the
{\cb Set Attributes} button in the {\cb Attributes Menu}.

% 101016
\index{LabelMaxLen variable}
\item{\et LabelMaxLen}\\
{\bf Value:} integer {\vt >=} 6.\\
This variable sets the maximum width, in default-sized character
cells, of a displayed label.  If the label exceeds this width, it is
not shown, and a small box at the text origin is shown instead.  The
default is 256, so this is unlikely to be triggered unless the user
resets the value.

The ``hidden'' status of a label can be toggled by clicking the text
or box with button 1 with the {\kb Shift} key held.  See
\ref{labelbut} for more information.

This variable was named {\et MaxLabelLen} in releases prior to
4.2.14.

This variable tracks the {\cb Maximum displayed label length} entry
area in the {\cb Labels} page of the {\cb Window Attributes} panel
from the {\cb Set Attributes} button in the {\cb Attributes Menu}.

% 101016
\index{LabelMaxLines variable}
\item{\et LabelMaxLines}\\
{\bf Value:} integer {\vt >=} 0.\\
Label text strings may have embedded newline characters which cause
them to be displayed on multiple lines.  This variable, when set to a
positive integer value, provides a limit on the number of lines that
are actually displayed, in labels that respect this limit.  Only the
first {\it N} lines would actually appear in the display, where {\it
N} is given in this property.  If {\it N} is zero, there is no limit.

Labels observe this limit only if an internal flag is set in the
label.  Presently, this is set internally for the labels associated
with {\et value} and {\et param} properties.  The user can apply the
limit to any label by setting the {\vt LIML} flag in the {\et
XprpXform} pseudo-property.

This variable was named {\et MaxLabelLines} in releases prior to
4.2.14.

This variable tracks the state of the {\cb Label optional displayed
line limit} numerical entry in the {\cb Labels} page of the {\cb
Window Attributes} panel from the {\cb Set Attributes} button in the
{\cb Attributes Menu}.

% 101016
\index{LabelHiddenMode variable}
\item{\et LabelHiddenMode}\\
{\bf Value:} integer 0--3.\\
By default, all labels participate in a protocol whereby clicking on
the label with the {\kb Shift} key held will ``hide'' the label,
displaying a small box instead.  {\kb Shift}-clicking on the box
will return to the display of the label text.  This variable limits
the labels which will participate in this protocol. 

\begin{quote}
\begin{tabular}{ll}\\
0 & All labels, the default, same as if not set.\\
1 & Only electrical-mode labels.\\
2 & Only electrical-mode bound property labels.\\
3 & No labels.\\
\end{tabular}
\end{quote}

This variable was named {\et HiddenLabelMode} in releases prior to
4.2.14.

This variable tracks the state of the {\cb Hidden label scope} option
menu in the {\cb Labels} page of the {\cb Window Attributes} panel
from the {\cb Set Attributes} button in the {\cb Attributes Menu}.

In the 3.2 branch of {\Xic} and earlier, the default was effectively
2.

% 061308
\index{LogoEndStyle variable}
\item{\et LogoEndStyle}\\
{\bf Value:} integer 0--2.\\
This sets the path end style used to render vector text in the {\cb
logo} command.  The variable should be set to 0 for flush ends, 1 for
rounded ends or 2 for extended ends.  If unset, extended ends are
used.  This variable tracks the setting in the {\cb Logo Font Setup}
panel in the {\cb logo} command.

% 061308
\index{LogoPathWidth variable}
\item{\et LogoPathWidth}\\
{\bf Value:} integer 1--5.\\
This sets the relative path width used for rendering with the vector
font in the {\cb logo} command.  The variable should be set to an
integer 1--5, where 1 represents the smallest width, and increasing
values makes the rendering appear increasingly bold.  This variable
tracks the setting in the {\cb Logo Font Setup} panel in the {\cb
logo} command.  If not set, a value of 3 is assumed.

% 061308
\index{LogoAltFont variable}
\item{\et LogoAltFont}\\
{\bf Value:} integer 0--1.\\
When set to 0 (zero), the {\cb logo} command will use an internal
bitmap font, and characters will be rendered as Manhattan polygons. 
When set to 1, the {\cb logo} command will use the system font named
in the {\et LogoPrettyFont} variable, or a default if this is not set. 
Characters are rendered as Manhattan polygons derived from the font
bitmaps.  When unset, or the value is not recognized, the {\cb logo}
command will use the vector font, for rendering using wires.  The
status of this variable tracks the check boxes in the {\cb Logo Font
Setup} panel of the {\cb logo} command.

% 061308
\index{LogoPrettyFont variable}
\item{\et LogoPrettyFont}\\
{\bf Value:} font name string.\\
This variable sets the name of the ``pretty'' font to be used for text
rendering in the {\cb logo} command.  It is set by the font selection
panel produced from the {\cb Select Pretty Font} button in the {\cb
Logo Font Setup} panel in the {\cb logo} command.

Under Unix/Linux, in GTK1 releases this variable can be set to the X
font description name of an X font.  In GTK2 releases, a Pango font
description string is expected.  Under Windows, the variable is set to
a string in the form ``{\it face\_name} {\it pixel\_height\/}'' or the
deprecated form ``({\it pixel\_height\/}){\it face\_name\/}''. 
Examples are ``{\vt Lucida Console 24}'' or ``{\vt (24)Lucida
Console}'', which is the default font.

% 052311
\index{LogoPixelSize variable}
\item{\et LogoPixelSize}\\
{\bf Value:} positive real number $<=$ 100.0.\\
When this variable is set to a value, it represents the size in
microns of a ``pixel'' used in the {\cb logo} command for new labels
and images.  With the variable defined, the ``pixel'' size is fixed,
and can not be changed with the arrow keys from the {\cb logo}
command.  This variable is set from and tracks the {\cb Define
``pixel'' size} check box and text entry area in the {\cb Logo Font
Setup} panel.

% 061308
\index{LogoToFile variable}
\index{logo button!use file}
\item{\et LogoToFile}\\
{\bf Value:} boolean.\\
If this variable is set, physical text created with the {\cb logo}
command will be placed in a cell, which is instantiated at the label
locations.  A native cell file containing the cell is written in the
current directory.  If unset, the physical text is placed directly in
the current cell.  The variable tracks the state of the check box in
the {\cb Logo Font Setup} panel.

% 061308
\index{NoConstrainRound variable}
\item{\et NoConstrainRound}\\
{\bf Value:} boolean.\\
When this boolean is set, there is no checking for minimum feature
size of round objects as these objects are being created (they will
still be tested when completed if interactive DRC is enabled).

% 021615
\index{RoundFlashSides variable}
\item{\et RoundFlashSides}\\
{\bf Value:} integer 8--256, default 32.\\
This variable sets the number of sides per 360 degrees to use in round
objects in physical mode, as created with the {\cb round}, {\cb donut}
and {\cb arc} side menu buttons, and corresponding script functions. 
It can be set from the {\cb sides} button in the physical side menu.

% 021615
\index{ElecRoundFlashSides variable}
\item{\et ElecRoundFlashSides}\\
{\bf Value:} integer 8--256, default 32.\\
This variable sets the number of sides per 360 degrees to use in round
objects in electrical mode, as created with the {\cb arc} button in
the menu produced by the {\cb shapes} button in the electrical side
menu.  It can be set from the {\cb sides} button in the same menu.

% 100916
\index{SpotSize variable}
\index{spot size}
\label{spotsize}
\item{\et SpotSize}\\
{\bf Value:} real 0--1.0.\\
When an e-beam mask is written, the layout is rendered using a certain
pixel size.  This implies a mask resolution, usually cited as the
``manufacturing grid'' or ``spot size''.  This size may range from 0.5
microns for the least expensive masks, down to a few nanometers for
the most expensive.

{\Xic} has two parameters which deal directly with mask resolution. 
The {\vt MfgGrid} set in the technology file will force the grid snap
points to be multiples of the value given.  The {\et SpotSize}
variable controls use of a numerical preconditioner for tiny round
objects.  The preconditioning should cause the pixel area to be
constant with respect to positioning and rotation.  This is valuable
to researchers fabricating circular Josephson junctions using
inexpensive mask sets (for example).

In ``rasterizing'' round objects to the e-beam grid, there can be
numerical problems.  Since the round object is rendered as a
collection of spot-pixels, the feature is not particularly round, but
most importantly the number of pixels used may not be well defined,
and therefor the figure area may not be as expected, or consistent.

The internal spot size is used when creating round (disk) objects and
donuts, but not arc objects or general polygons.  It applies to the
{\cb round} and {\cb donut} buttons in the side menu, and the
corresponding script functions, but does not apply to the {\cb arc}
button or general polygons.  The internal spot size is also used as
the default value for the {\cb !tospot} command.

If the {\et SpotSize} variable is given a non-negative value, this
value is used as the internal spot size.  The value is in microns, and
1.0 micron is the largest accepted value.  If this is zero, then no
preconditioning is applied.  If the {\et SpotSize} variable is unset,
the internal spot size will default to the {\vt MfgGrid} given in the
technology file.  Thus, when a manufacturing grid is given, the
default is to use preconditioning when creating round objects.  This
can be suppressed by setting {\et SpotSize} to zero.  Other than this,
there probably is no reason to set the {\et SpotSize} variable, since
it should match the {\vt MfgGrid}>, unless the user has special
requirements.

When the internal spot size has a positive value, objects created with
the {\cb round} and {\cb donut} buttons will be created so that all
vertices are placed at the center of a spot (i.e., in the center of a
manufacturing grid cell), and a minimum number of vertices will be
used.  The {\cb sides} number is ignored.  This applies only to
figures with minimum radius 50 spots or smaller; the regular algorithm
is used otherwise.  An object with this preconditioning applied should
translate exactly to the e-beam grid.  The figures are symmetric with
regard to rotations in multiples of 45 degrees.
\end{description}


% -----------------------------------------------------------------------------
% !set:spice 021515
\section{SPICE Interface}
\label{spicevars}
\index{variables!SPICE interface}

The following {\cb !set} variables affect the interface to the
{\WRspice} simulator, and SPICE output in general.

\begin{description}
% 061308
\index{SpiceListAll variable}
\item{\et SpiceListAll}\\
{\bf Value:} boolean.\\
When set, all devices and subcircuits in the schematic will be
included in SPICE output.  Otherwise, only devices and subcircuits
that are ``connected'' will be included, as explained in the {\cb
deck} and {\cb run} command descriptions.

% 061308
\index{SpiceAlias variable}
\index{spice key mapping}
\item{\et SpiceAlias}\\
{\bf Value:} string.\\
This variable is set to a string which will modify the printing of
device names in SPICE output.  The aliasing operates on the first
token of device lines.  The format of the value string is
\begin{quote}
{\it prefix1\/}={\it newprefix1\/} {\it prefix2\/}={\it newprefix2\/ ...}
\end{quote}
This will cause lines beginning with {\it prefix\/} to have {\it
prefix\/} replaced with {\it newprefix\/}.  If the ``={\it
newprefix\/}'' is omitted, that line will not be printed.  For
example, to map all devices that begin with `B' to `J', and to suppress
all `G' devices, the full command is
\begin{quote}
{\vt !set SpiceAlias B=J G}.
\end{quote}
Note that there can be no space around the `='.  After this command is
given, the indicated mappings will be performed as SPICE text is
produced.

% 012411
\index{SpiceHost variable}
\item{\et SpiceHost}\\
{\bf Value:} host name string.\\
This will set the name of the host which maintains a server for remote
{\WRspice} runs.  If set, this will override the value of the {\et
SPICE\_HOST} environment variable.  The host name specified in the
{\et SPICE\_HOST} environment variable and the {\et SpiceHost} {\cb
!set} variable can have a suffix ``{\vt :}{\it portnum}'', i.e., a
colon followed by a port number.  The port number is the port used by
the {\vt wrspiced} program on the specified server, which defaults to
6114, the IANA registered port for this service.  If the server uses a
non-standard port, and the {\vt wrspice/tcp} service has not been
registered (usually in the {\vt /etc/services} file) on this port, the
port number must be provided.

% 022111
\index{SpiceHostDisplay variable}
\item{\et SpiceHostDisplay}\\
{\bf Value:} X display string.\\
This variable can be set to the X display string to use on a remote
host for running {\WRspice} through a {\vt wrspiced} daemon, from
{\Xic} in electrical mode.  It is intended to facilitate use of {\vt
ssh} X forwarding to take care of setting up permission for the remote
host to draw on the local display.

The variable is set automatically from the {\cb !ssh} command, or can
be set by hand.

When using a remote host, this specifies the X display string to use,
which is needed for running graphics.  If not set, a display name will
be created as follows:  If the local {\et DISPLAY} variable is
something like ``{\vt :0.0}'', the remote display name will be ``{\it
localhostname\/}{\vt :0.0}''.  If the local {\et DISPLAY} variable is
already in the form ``{\it localhostname\/}{\vt :0.0}'', this is
passed verbatim.

One can use {\vt ssh} transport for the X connection on the remote
system as follows.  Use ``{\vt ssh -X}'' to open a shell on the remote
machine.  Type ``{\vt echo \$DISPLAY}'' into this window, it will
print something like ``{\vt localhost:10.0}''.  Use this value for
{\et SpiceHostDisplay}.  The {\cb !ssh} command will set the variable
automatically.  The shell must remain open while running {\WRspice},
exiting the shell will close the X connection.

This will work under Windows, if Cygwin is installed, along with the
OpenSSH package (for the {\vt ssh} command) and the Cygwin X server. 
One weirdness:  use ``{\vt ssh -Y}'' instead of ``{\vt ssh -X}''.  The
{\vt -Y} option, which applies to recent {\vt ssh} versions, is
similar to {\vt -X}, but overcomes stronger security checks included
in recent {\vt ssh} implementations.  This seems to be necessary when
using the Cygwin X server.

{\bf Background}

In legacy X-window systems, the display name would typically be in the
form {\it hostname\/}{\vt :0.0}, where the {\it hostname} could be
(and usually is) missing.  A remote system will draw to the local
display if the local hostname was used in the display name, and the
local X server permissions were set (with {\vt xauth}/{\vt xhost}) to
allow access.  Typically, the user would log in to a remote system
with {\vt telnet} or {\vt ssh}, set the {\et DISPLAY} variable,
perhaps give ``{\vt xhost +}'' on the local machine, then run X
programs.

This method has been largely superseded by use of ``X forwarding'' in
{\vt ssh}.  This is often automatic, or may require the `{\vt -X}' or
`{\vt -Y}'option in the {\vt ssh} command line.  In this case, after
using {\vt ssh} to log in to the remote machine, the {\et DISPLAY}
variable is automatically set to display on the local machine.  X
applications ``just work'', with no need to fool with the {\et
DISPLAY} variable, or permissions.

The present {\Xic} remote access code does not know about the {\vt
ssh} protocol, so we have to fake it in some cases.  In most cases the
older method will still work.

The {\vt ssh} protocol works by setting up a dummy display, with a
name something like ``{\vt localhost:10.0}'', which in actuality
connects back to the local display.  Depending on how many {\vt ssh}
connections are currently in force, the ``{\vt 10}'' could be ``{\vt
11}'', ``{\vt 12}'', etc.

In the present case, if we want to use {\vt ssh} for X transmission,
the display name must match an existing {\vt ssh} display name on the
remote system that maps back to the local display.

If there is an existing {\vt ssh} connection to the remote machine,
the associated {\et DISPLAY} can be used.  If there is no existing
{\vt ssh} connection, one can be established, and that used.  E.g.,
from the {\vt ssh} window, type ``{\vt echo \$DISPLAY}'' and use the
value printed.

The display name provided by the {\et SpiceHostDisplay} variable will
override the assumed display name created internally with the local
host name.

% 022619
\index{SpiceInclude variable}
\item{\et SpiceInclude}\\
{\bf Value:} file path string.\\
This can be set to a file path.  When a SPICE netlist is created with
this variable set, text will be added to the top of the SPICE deck. 
If the file exists and is readable, the text from the file is added to
the deck verbatim.  Othewise, ``{\vt .include} {\it path\/}'' is
added, the {\it path} being the file path from the variable.  This
applies when creating SPICE with the {\cb deck} button, or when
preparing input for the simulator when using the {\cb run} button.

% 061308
\index{SpiceProg variable}
\item{\et SpiceProg}\\
{\bf Value:} program path string.\\
This will set the full path name of the {\WRspice} executable.  This
is useful if there are multiple versions of {\WRspice} available, or
the binary has been renamed.  If given, the value supersedes the
values from environment variables or the {\cb !set} variables
described below.

% 110213
\index{SpiceExecDir variable}
\item{\et SpiceExecDir}\\
{\bf Value:} directory path string.\\
This will set the directory to search for the {\WRspice} executable. 
If given, the value overrides the\\ {\et SPICE\_EXEC\_DIR} environment
variable.  The default search location is ``{\vt /usr/local/xictools/bin}'',
or, if the {\et XT\_PREFIX} environment variable has been set, its
value will replace ``{\vt /usr/local}''.

% 061308
\index{SpiceExecName variable}
\item{\et SpiceExecName}\\
{\bf Value:} program name string.\\
This will set the name of the {\WRspice} binary.  If given, the value
overrides the {\et SPICE\_EXEC\_NAME} environment variable.  The
default name is ``{\vt wrspice}''.

% 021912
\index{SpiceSubcCatchar variable}
\item{\et SpiceSubcCatchar}\\
{\bf Value:} string, single printing character.\\
This sets the concatenation character used in {\WRspice} subcircuit
expansion.  It affects the internally-generated node and other names
within subcircuits.  Please refer to the WRspice-3.2.15 release notes
or documentation for a full description of the {\WRspice} {\et
subc\_catmode} and {\et subc\_catchar} variables and their effects.

% 021912
\index{SpiceSubcCatmode variable}
\item{\et SpiceSubcCatmode}\\
{\bf Value:} string, ``{\vt wrspice}'' or ``{\vt spice3}''.\\
This sets the algorithm used by {\WRspice} for subcircuit expansion. 
It affects the internally-generated node and other names within
subcircuits.  Please refer to the WRspice-3.2.15 release notes or
documentation for a full description of the {\WRspice} {\et
subc\_catmode} and {\et subc\_catchar} variables and their effects.
\end{description}

When running {\WRspice} from {\Xic}, there should not be
compatibility issues, as {\Xic} will automatically recognize the
capabilities of the connected {\WRspice} and compensate accordingly
-- as long as the hypertext facility is used to define node, branch,
and device names.  This is true when point-and-click is used to
generate names.  However, subcircuit reference names that for some
reason are entered by hand may need to be updated, or a {\vt
.options} line added as a spicetext label, or the {\et
SpiceSubcCatchar}, {\et SpiceSubcCatmode} variables may be set to
enforce backward compatibility.

% 061308
\begin{description}
\index{CheckSolitary variable}
\item{\et CheckSolitary}\\
{\bf Value:} boolean.\\
If set, warning messages will be issued when electrical netlists are
generated for nodes having only one connection.  This affects the {\cb
run} and {\cb deck} commands, and the {\cb Dump Elec Netlist} command
in the {\cb Extract Menu}.

% 061308
\index{NoSpiceTools variable}
\item{\et NoSpiceTools}\\
{\bf Value:} boolean.\\
When running {\WRspice} from {\Xic}, by default the {\WRspice} toolbar
is shown, if {\WRspice} is running on the local machine.  This gives
the user much greater flexibility and control over {\WRspice}.  If
this variable is set, {\it before} the connection to {\WRspice} is
established, the toolbar will not be visible.

In releases 3.0.8 and later, this variable will also control toolbar
visibility if the {\vt wrspiced} daemon is used.  However, this
requires {\vt wrspiced} distributed with wrspice-3.0.7 or later.  {\bf
If this variable is set with an earlier {\vt wrspiced} release, the
{\WRspice} connection will not work!}
\end{description}


% -----------------------------------------------------------------------------
% !set:hardcopy 021515
\section{File Menu --- Printing}
\index{variables!printing}

The following {\cb !set} variables affect the commands in the {\cb File
Menu}, mostly the {\cb Print} command.

\begin{description}
% 061308
\index{NoAskFileAction variable}
\item{\et NoAskFileAction}\\
{\bf Value:} boolean.\\
By default, in the {\cb File Selection} and {\cb Path Files Listing}
windows, a confirmation pop-up will appear before move/copy/link
operations on files or directories initiated by drag/drop.  If this
variable is set, this confirmation will not appear.  The confirmation
default is safer, but may be annoying to experienced users.

Note:  in releases prior to 3.0.0, there was no confirmation, as
if this variable were set.

% 061308
\index{DefaultPrintCommand variable}
\item{\et DefaultPrintCommand}\\
{\bf Value:} string.\\
Under Unix/Linux/OS X, this variable overrides the default operating
system command string to print a file.  In Windows, this will be the
printer name instead.

This should probably be set before the {\cb Print} panel is used for
the first time, as some drivers may copy the initial contents so that
changing this variable will have no effect.  It can be set in a
startup file.

If not set, the default print command is ``{\vt lpr}'' (or ``{\vt
default}'' in Windows).  See the man page for {\vt lpr} or {\vt lp}
for the print options which apply on your system, which can be placed
in the default string.  In the printer command string, the characters
``{\vt \%s}'' are replaced with the name of the temporary file to be
printed.  If these characters don't appear, the file name is tacked on
the end of the command string, separated by space.

% 061308
\index{NoDriverLabels variable}
\item{\et NoDriverLabels}\\
{\bf Value:} boolean.\\
The PostScript hard copy drivers use PostScript text for labels by
default, not the vector font used on-screen.  This can be overridden,
and the vector font used, by setting this variable.  Multi-line labels
are always drawn with the vector font, however.

% 061308
\index{RmTempFileMinutes variable}
\item{\et RmTempFileMinutes}\\
{\bf Value:} integer 0--4320.\\
When a layout or page is printed, a temporary file is produced and
saved in one of the system temporary directories.  By default, these
files are not removed.  The temporary directories are generally
cleared when the system is rebooted, or by some other system-level
means.

On some operating systems, the print command can include an option to
delete the temporary source file after the print job is complete.  The
{\et DefaultPrintCmd} variable can be set to include this option.

Otherwise, this variable can be set to delete the temporary file a
number of minutes after the print job is submitted.  On some systems,
the temporary file is copied into the print job queue, so that the
temporary file can be deleted almost immediately.  On other systems,
or for large files, a link into the queue is created instead, so that
the file must not be deleted until the job is complete.  There is no
universal way to determine if a print job has completed, so we need to
wait a reasonable length of time before deleting the file.

This variable can be set to the number of minutes to wait before
deleting the temporary file.  If set to 0, the file will not be
deleted by this system, as is the case if this variable is not set. 
The deletion will occur whether or not the application is still
running.

Currently, this feature is not available on Windows.  It uses the Unix
{\vt at} command (see the manual page for details).  The user must
have permission established for this to work.  A message is printed in
the console when a file is scheduled for deletion, or if an error
(such as lack of permission) occurs.
\end{description}


% -----------------------------------------------------------------------------
% !set:cell 021515
\section{Cell Menu Commands}
\index{variables!Cell Menu commands}

The following {\cb !set} variables affect commands found in the
{\cb Cell Menu}.

\begin{description}
% 042909
\index{ContextDarkPcnt variable}
\item{\et ContextDarkPcnt}\\
{\bf Value:} integer 1--100.\\
While the {\cb Push} command is active, and the surrounding context is
being shown, the context is drawn with reduced illumination intensity
so that objects in the current cell can be visually differentiated. 
The variable allows the context intensity to be adjusted, as a
percentage of the ``normal'' intensity.

If this variable is not set, a value of 65 (percent) will be used.
       
This variable tracks the state of the {\cb Push context display
illumination percent} entry field in the {\cb Window Attributes} panel
from the {\cb Set Attributes} button in the {\cb Attributes Menu}.
\end{description}


% -----------------------------------------------------------------------------
% !set:editgen 021515
\section{Editing General}
\index{variables!Editing General}

The following {\cb !set} variables affect general operations and
parameters that apply during editing.

\begin{description}
% 102913
\index{AskSaveNative variable}
\item{\et AskSaveNative}\\
{\bf Value:} boolean.\\
When set, the user will be prompted to save the current cell if the
cell is modified, and would be saved as a native symbol, and a new
current cell is about to be set.  This was standard behavior in
releases earlier than generation 4.  Although it is always a good idea
to save work periodically, the prompt can be annoying to experienced
users and is now disabled by default.  The user will be given the
chance to save modified cells when exiting {\Xic} in any case.

The {\cb Prompt to save modified native cells} check box in the {\cb
Editing Setup} panel from the {\cb Edit Menu} tracks the state (set or
unset) of this variable.

% 102913
\index{Constrain45 variable}
\item{\et Constrain45}\\
{\bf Value:} boolean.\\
When this boolean variable is set, wire and polygon vertices are
constrained to form angles of multiples of 45 degrees.  By default, a
``smart'' path generator is employed, which will construct a valid
path to the pointer location from the previous point during wire or
polygon construction.  This will often add two vertices:  a 45 degree
extension, followed by a Manhattan extension, in order to connect the
points.  If the {\kb Ctrl} key is held while the new point is defined,
the ``smart'' feature is disabled, and only one new vertex is added. 
If the {\kb Shift} key is held, then the 45 degree constraint is
removed entirely.

When set, rotation angles available in the {\cb spin} command, and
translation angles in the {\cb Stretch} command, and the vertex
editors for polygons and wires, are constrained to multiples of 45
degrees.  However, pressing the {\kb Shift} key will remove the
constraint in these commands while the key is held.  If the {\et
Constrain 45} variable is not set, holding {\kb Shift} will impose the
45 degree angle constraint.  Thus, the {\kb Shift} key inverts the
effective state of the {\et Constrain 45} variable in these commands.

The {\cb Constrain angles to 45 degree multiples} check box in the
{\cb Editing Setup} panel from the {\cb Edit Menu} tracks the state
(set or unset) of this variable.

% 022716
\index{NoMergeObjects variable}
\item{\et NoMergeObjects}\\
{\bf Value:} boolean.\\
This variable tracks the state of the {\cb Merge new boxes and polys
with existing boxes/polys} check box in the {\cb Editing Setup} panel
from in the {\cb Edit Menu} in a logically inverted sense.

By default, when a new box or polygon object is created in the
database from the commands in the side menu, the new object is merged
with existing boxes and polygons on the same layer, if any touch or
overlap, to form a (generally more complex) polygon in the database. 
New wires will link with existing similar wires in the database that
share an endpoint.

If this boolean variable is set, this merging will be disabled. 
Merging can also be disabled on a per-layer basis with the {\vt
NoMerge} technology file keyword, which prevents merging in all cases
on a layer.

The {\et NoMergePolys} variable can be set to prevent merging of
polygons, and will thus revert the merging behavior to that of
releases prior to 3.1.7.

When reading data from a layout file, a different box clipping/merging
capability is controlled by the {\cb Clip and merge overlapping boxes}
setting in the {\cb Setup} page or the {\cb Import Control} panel, and
the corresponding {\et MergeInput} variable.

% 102913
\index{NoMergePolys variable}
\item{\et NoMergePolys}\\
{\bf Value:} boolean.\\
When auto-merging new objects ({\et NoMergeObjects} is not set), only
boxes will be clipped and merged, polygons will be ignored, if this
variable is set.  This reverts to the behavior of releases prior to
3.1.7.

This variable tracks the state of the {\cb Clip and merge new boxes
only, not polys} check box in the {\cb Editing Setup} panel from the
{\cb Edit Menu}.

% 101515
\index{NoFixRot45 variable}
\item{\et NoFixRot45}\\
{\bf Value:} boolean.\\
There are two modes when rotating boxes/polys by non-Manhattan angles. 
The default and legacy method is to use an offset technique referenced
to the lower-left box coordinate, or the first vertex of polygons. 
This ensures that the same figure is generated at any location, and
seems to ensure that all angles are exactly multiples of 45 or 90,
after rotation, in boxes.  However, this has the problem that two
figures that abut before rotation might no longer abut after rotation. 
For example, use the {\cb !split} function to split a disk, then
rotate the collection by 45 degrees.  It is likely that some of the
figures no longer touch.  If merging is enabled, the disk will have
lines through it at these points, where the gaps prevented merging.

If {\et NoFixRot45} is set, the offset fix is not done.  This solves
the problem of gaps appearing between rotated objects, but has its own
problems.  Namely, rectangles aren't preserved, angles can differ from
45s.  Try rotating a small rectangle, say 3x5 internal units, by 45s
in this mode, and one can see it is a mess.  Larger rectangles are not
visually distorted, but there are 1-unit errors in the vertex
placements relative to preservation of 45s or 90s.  This is probably
not acceptable for most work.

Really, rotating by 45 degrees is something best avoided.
\end{description}


% -----------------------------------------------------------------------------
% !set:edit 100616
\section{Edit/Modify Menu Commands}
\index{variables!Edit Menu commands}

The following {\cb !set} variables affect commands found in the
{\cb Edit Menu} and the {\cb Modify Menu}.

\begin{description}
% 102913
\index{UndoListLength variable}
\item{\et UndoListLength}\\
{\bf Value:} integer {\vt >=} 0.\\
This variable sets the number of operations remembered in the {\cb
Undo} command.  If not set, 25 operations are saved.  If set to zero,
the length is unlimited.

This tracks the setting of the {\cb Maximum undo list length} entry
area in the {\cb Editing Setup} panel from the {\cb Edit Menu}.

% 100416
\index{MaxGhostDepth variable}
\item{\et MaxGhostDepth}\\
{\bf Value:} integer 0--8.\\
This variable sets the maximum expansion depth for instance expansion
in ghosting.  If not set, this is the same as the normal expansion
depth.  The actual expansion depth used in ghosting will not be larger
than the normal expansion depth, but can be smaller.  For example,
setting this to 0 (zero) will prevent expansion of ghosted subcells
entirely.

This tracks the setting of the {\cb Maximum subcell depth in ghosting}
menu in the {\cb Editing Setup} panel from the {\cb Edit Menu}.

% 100416
\index{MaxGhostObjects variable}
\item{\et MaxGhostObjects}\\
{\bf Value:} integer 50--50000.\\
This sets the maximum number of objects to render individually as
``ghosts'' attached to the mouse pointer during operations such as
move and copy.  This can be set to an unsigned integer in the range
50--50000.  If there are more than this number, some outlines won't be
shown, the smaller-area objects will be skipped.  If subcells are
being expanded, objects are rendered top-down, so that if the limit is
reached, objects deeper in the hierarchy will not be shown.

The default is 4000 if this variable is not set.  If, when moving a
large number of objects, the pointer motion is too sluggish, the user
can set this variable to compensate, or can limit the subcell
expansion depth by setting {\et MaxGhowtDepth} if expansion causes the
problem.

This tracks the setting of the {\cb Maximum number of ghost-drawn
objects} entry area in the {\cb Editing Setup} panel from the {\cb
Edit Menu}.

% 102913
\index{NoWireWidthMag variable}
\item{\et NoWireWidthMag}\\
{\bf Value:} boolean.\\
When set, the width of wires does not change when the wire undergoes
magnification, in a {\cb Move}, {\cb Copy}, or {\cb Flatten}
operation.

The {\cb No wire width change in magnification} check box in the {\cb
Editing Setup} panel from the {\cb Edit Menu} tracks the state (set or
unset) of this variable.

% 102913
\index{CrCellOverwrite variable}
\item{\et CrCellOverwrite}\\
{\bf Value:} boolean.\\
When set, The {\cb Create Cell} operation in the {\cb Edit Menu} and
the {\vt CreateCell} script function can overwrite cells already in
memory.  This can be dangerous and is prevented by default.

The {\cb Allow Create Cell to overwrite existing cell} check box in
the {\cb Editing Setup} panel from the {\cb Edit Menu} tracks the
state (set or unset) of this variable.

% 101212
\index{LayerChangeMode variable}
\item{\et LayerChangeMode}\\
{\bf Value:} tri-state.\\
This variable applies during all move and copy operations, and during
the {\cb spin} command in the physical side menu and similar.  In
these commands, when objects being moved or copied are ghost drawn as
attached to the mouse pointer, it is possible to change the current
layer.  The operation is then completed by clicking at the new
location in a drawing window.

This is a tri-state variable.  If not set, there will be no layer
change in these commands.  Thus by default any current layer change
made during the command is ignored by the command.  If set to the
string ``{\vt all}'' (case insensitive), then a layer change will
apply to all objects being moved or copied.  All new objects will be
placed on the new layer, regardless of the original layers of the
objects.  If set to anything else, including to nothing (i.e., as a
boolean) then only objects on the previous current layer will be
changed to the new layer.  Other objects will remain on their original
layer.

This variable tracks the state of the radio buttons in the {\cb Layer
Change Mode} pop-up, which appears when the {\cb Set Layer Chg Mode}
button in the {\cb Modify Menu} is pressed.

% 100616
\index{JoinMaxPolyVerts variable}
\item{\et JoinMaxPolyVerts}\\
{\bf Value:} integer 0 or  20--8000.\\
This variable applies to the {\cb Join} and {\cb Join All} buttons in
the {\cb Join or Split Objects} panel (from the {\cb Edit Menu}), the
{\cb !join} command, the join (merging) operation when new objects are
created, and the associated script functions and elsewhere where join
operations occur.

This sets an upper bound on the number of vertices in polygons created
by a join operation.  The default is 600 vertices.  If set to 0, no
limit is applied.  The variable tracks the {\cb Maximum vertices in
joined polygon} entry in the {\cb Join or Split Objects} panel.

There is no internal limit on the vertex count of a polygon in memory. 
Although setting {\et JoinMaxPolyVerts} to 0 allows arbitrarily large
polygons to be created, one should be reasonable.  Huge polygons can
be cumbersome and inefficient.  Oversize polygons and wires will be
broken up, if necessary, when a file is saved to disk.  For the
different formats, the limits are

\begin{tabular}{ll}
native & no limit\\
CIF & no limit\\
CGX & 8000 vertices\\
GDSII & depends on {\et GdsOutLevel}, max is 8000 vertices\\
OASIS & no limit\\
\end{tabular}

For CIF files, {\Xic} can read/write arbitrarily large polygons and
wires, but beware that other tools may have built-in limits.

% 100616
\index{JoinMaxPolyGroup variable}
\item{\et JoinMaxPolyGroup}\\
{\bf Value:} integer {\vt >=} 0.\\
This variable applies to the {\cb Join} and {\cb Join All} buttons in
the {\cb Join or Split Objects} panel (from the {\cb Edit Menu}), the
{\cb !join} command, the join (merging) operation when new objects are
created, and the associated script functions and elsewhere where join
operations occur.

When a collection of trapezoids is being combined into polygons during
a join operation, the collection is first divided into connected
groups, each of which will be converted to one or more polygons.  This
variable limits the number of trapezoids in the groups.  The default
value (when this variable is unset) is 0, meaning that there is no
limit.  Generally, applying a limit (for example, 300) provides faster
join operations, however this will leave as separate objects more
polygons that could have been joined.

This variable tracks the {\cb Maximum trapezoids per poly for join}
entry in the {\cb Join or Split Objects} panel.

% 100616
\index{JoinMaxPolyQueue variable}
\item{\et JoinMaxPolyQueue}\\
{\bf Value:} integer {\vt >=} 0.\\
This variable applies to the {\cb Join} and {\cb Join All} buttons in
the {\cb Join or Split Objects} panel (from the {\cb Edit Menu}), the
{\cb !join} command, the join (merging) operation when new objects are
created, and the associated script functions and elsewhere where join
operations occur.

When objects are being joined, they are first decomposed into
trapezoids.  The trapezoids from the objects are saved in a single
list, and when the list length exceeds a certain value the list is
sent to the function that recombines the trapezoids into polygons. 
This variable is used to set the length threshold.  The default value
(when this variable is unset) is 0, which allows the list to grow
without bound.  Generally, applying a limit (for example, 1000)
provides faster processing, but will produce more polygons.

This variable tracks the {\cb Trapezoid queue size for join} entry in
the {\cb Join or Split Objects} panel.

% 100616
\index{JoinBreakClean variable}
\item{\et JoinBreakClean}\\
{\bf Value:} boolean.\\
This variable applies to the {\cb Join} and {\cb Join All} buttons in
the {\cb Join or Split Objects} panel (from the {\cb Edit Menu}), the
{\cb !join} command, the join (merging) operation when new objects are
created, and the associated script functions and elsewhere where join
operations occur.

In a join operation, when building up the polygons and the vertex
limit ({\et JoinMaxPolyVerts}) is reached, ordinarily the present
polygon is output, and a new one is started immediately.  This
generally produces a set of polygons with complicated and seemingly
arbitrary borders.  If this variable is set, then the polygons are
initially built ignoring the vertex limit, and polygons that exceed
the vertex limit are split into pieces along Manhattan bisectors, so
that no piece exceeds the vertex count.  This gives a much nicer
looking layout, but is more compute intensive.

This variable tracks the {\cb Clean break in join operation limiting}
check box in the {\cb Join or Split Objects} panel.

% 100616
\index{JoinSplitWires variable}
\item{\et JoinSplitWires}\\
{\bf Value:} boolean.\\
This applies to join operations as listed for the variables above, but
not for the joining when new objects are created.  It also applies to
the split operation.

By default, wires do not participate in join/split operations, these
operate on boxes and polygons only.  Wires, however, will be joined
with other wires on the same layer it they share an endpoint and have
the same width.  If this variable is set, then wires will be treated
like polygons in join and split operations, but wires never
participate in the join operation when new objects are created. 

This variable tracks the {\cb Include wires (as polygons) in
join/split} check box in the {\cb Join or Split Objects} panel.

% 100616
\index{PartitionSize variable}
\item{\et PartitionSize}\\
{\bf Value:} floating-point number.\\
This variable applies to layer expression evaluation, including from
the {\cb Evaluate Layer Expression} panel (from the {\cb Edit Menu}),
the {\cb !layer} and {\cb !compare} commands, and the {\vt
AdvanceZref} script function.

In releases prior to 3.0.0, this variable was named ``{\et
LayerPartSize}''.

When geometrical operations are performed over a large area, a logical
square grid is created over the area relative to the lower-left
corner.  The operations are performed for each grid element that
intersects the area, and the results are combined.  This can be more
efficient than performing the operations over the entire area in one
shot.  Performance rapidly degrades as the amount of geometry per grid
area increases.  Best performance is probably obtained with 10000 or
fewer trapezoids per grid.
 
This variable specifies the size of the grid, in microns, set as a
floating-point number.  If not set, the default grid size is 100
microns.  Acceptable values are 1.0 -- 10000.0, or 0.  If set to
0, partitioning is not used.

The variable tracks the {\cb Partition size} set in the {\cb Evaluate
Layer Expression} panel.

% 100516
\index{Threads variable}
\item{\et Threads}\\
{\bf Value:} integer 0--31.\\
{\bf PRELIMINARY, EXPERIMENTAL!}

This will enable new multi-threaded functionality as it becomes
available.  This is set to the number of helper threads that can be
called upon to parallelize certain operations.  The best value is
probably one less than twice the number of available processor cores.  It
should not be set to a larger value, but one might wish to try smaller
values.  If unset, or set to 0, the program is single threaded.

This variable tracks the {\cb Number of helper threads} entry in the
{\cb Evaluate Layer Expression} panel from the {\cb Edit Menu}.

Presently, multi-threading is used when evaluating a layer expression
using a grid.  Evaluation in each of the grid cells can be done in
parallel, so these jobs are submitted to the thread pool.  One can
experiment with the partition size to get fastest results, larger
partitions are more likely to overcome the multi-threading overhead.
\end{description}


% -----------------------------------------------------------------------------
% !set:view 021515
\section{View Menu Commands}
\index{variables!View Menu commands}

The following {\cb !set} variables affect commands found in the
{\cb View Menu}.

\begin{description}
% 121508
\index{InfoInternal variable}
\item{\et InfoInternal}\\
{\bf Value:} boolean.\\
When set, the {\cb Info} command in the {\cb View Menu} and the {\cb
Info} command in the {\cb Cells Listing} panel will print dimensions
using internal database units (default is 1000 per micron) rather than
in microns.

% 061408
\index{PeekSleepMsec variable}
\item{\et PeekSleepMsec}\\
{\bf Value:} integer {\vt >=} 0.\\
This sets the delay time in milliseconds to wait after a layer is
drawn in the {\cb Peek} command.  The default is 400.

% 061408
\index{LockMode variable}
\item{\et LockMode}\\
{\bf Value:} boolean.\\
This variable, when set, locks the current mode (physical or
electrical).  In addition, while reading any type of file, only the
information for the present mode is read.  All features which apply to
the other mode are disabled, and no data are stored for the other
mode.  By not storing stubs for the electrical data, for example, more
memory space is available for a large physical-only file.

As files written from this mode have only one type of data, it is
possible to overwrite files that originally contained both types of
data.  The user should be aware of this possibility.

% 032213
\index{XSectNoAutoY variable}
\item{\et XSectNoAutoY}\\
{\bf Value:} boolean.\\
By default, the cross-section display is shown with a vertical scale
adjusted such that the entire layer stack occupies most of the window. 
This is maintained independent of the window magnification, which
consequently changes only the X-scale.  If this boolean variable is
set, the auto-scaling will not be done.

This variable is set by the {\cb Auto Y-Scale} check box that appears
in the {\cb Set Display Window} pop-up that is called by the {\cb
Zoom} button in the {\cb View} menu of the cross-section display
window.  The setting is done only when the user presses the {\cb
Apply} button.

% 032213
\index{XSectYScale variable}
\item{\et XSectYScale}\\
{\bf Value:} real 1e-3 -- 1e3.\\
This variable supplies a Y-scale to the cross-section display.  If the
auto-scaling is enabled, the scale factor determines how much of the
vertical window dimension is occupied by the layer stack.  Without
auto-scaling, this scale is applied directly to the vertical axis.

The horizontal grid lines and ruler gradations take into account the
scale.  The scaling allows easy visualization when the thickness is
much larger or much smaller than typical line widths.

This variable is set from the {\cb Y-Scale} numerical entry area that
appears in the {\cb Set Display Window} pop-up that is called by the
{\cb Zoom} button in the {\cb View} menu of the cross-section display
window.  The setting is done only when the user presses the {\cb
Apply} button, and the value has been set to something other than
unity.
\end{description}


% -----------------------------------------------------------------------------
% !set:attri 101016
\section{Attribute Menu Commands}
\index{variables!Attributes Menu commands}

The following {\cb !set} variables affect the commands found in the
{\cb Attributes Menu}.

\begin{description}
% 100616
\index{TechNoPrintPatMap variable}
\item{\et TechNoPrintPatMap}\\
{\bf Value:} boolean.\\
When set, {\Xic} will use the hex format when writing stipple patterns
for layers when writing a technology file.  If unset, an ASCII format,
that provides a rendition of the map, is used.  The hex format is
compatible with {\Xic} releases prior to 3.2.25, if the stipple map
sizes are restricted to 8x8, 16x8, 8x16, or 16x16.  Technology files
can be written using the {\cb Save Tech} button in the {\cb Attributes
Menu}.

% 100616
\index{TechPrintDefaults variable}
\item{\et TechPrintDefaults}\\
{\bf Value:} boolean or string.\\
When a technology file is written with the {\cb Save Tech} button, by
default entries that would set a parameter to a program default value
are omitted, as they are redundant and increase the size and
complexity of the file.  This will be the case when this variable is
not set.  If this variable is set to no value, i.e., as a boolean,
then these lines will be added to the technology file as comments.  If
this variable is set to any value, then these lines will be added as
active text.

This variable tracks the radio buttons in the {\cb Write Tech File}
pop-up which appears from the {\cb Save Tech} button in the {\cb
Attributes Menu}.

% 100616
\index{BoxLineStyle variable}
\item{\et BoxLineStyle}\\
{\bf Value:} integer, default e38 (hex).\\
This sets the line style mask of the boxes used in electrical mode,
and in physical mode for some highlighting purposes, such as the
current cell boundary.  The style is an integer whose binary value is
replicated to form the lines used in the box.  The line style editor
in the {\cb Grid Setup} panel can be used to generate line style
masks.

The {\cb Global Attributes} button in the {\cb Tech Parameter Editor}
provides a prompt-line interface for setting this variable.  This is
called from the {\cb Edit Tech Params} button in the {\cb Attributes
Menu}.

% 100616
\index{EraseBehindProps variable}
\item{\et EraseBehindProps}\\
{\bf Value:} boolean.\\
If given, the area inside the bounding box of text generated by the
{\cb Show Phys Properties} command in the {\cb Main Window} sub-menu
of the {\cb Attributes Menu} or the sub-window {\cb Attributes} menu
is erased, to promote visibility of the text.

This tracks the state of the {\cb Erase behind physical properties
text} check box in the {\cb Phys Props} page of the {\cb Window
Attributes} panel from the {\cb Set Attributes} button in the {\cb
Attributes Menu}.

% 100616
\index{PhysPropTextSize variable}
\item{\et PhysPropTextSize}\\
{\bf Value:} integer 6--48.\\
This variable can be used to set the height, in pixels, of the text
used to render physical properties on-screen when physical properties
are being displayed.  If not set, the default is 14.

This tracks the state of the {\cb Physical property text size
(pixels)} entry area in the {\cb Phys Props} page of the {\cb Window
Attributes} panel from the {\cb Set Attributes} button in the {\cb
Attributes Menu}.

% 100616
\index{EraseBehindTerms variable}
\item{\et EraseBehindTerms}\\
{\bf Value:} boolean or ``{\vt all}''.\\
If set, the area inside the bounding box of terminals made visible
by the {\cb Show Terminals} command is erased, to promote visibility
of the text.  If set to ``{\vt all}'', all terminals are erased
behind, otherwise only the cell's formal terminals are erased behind.

This tracks the setting of the {\cb Erase behind physical terminals}
menu in the {\cb Terminals} page of the {\cb Window Attributes} panel
from the {\cb Set Attributes} button in the {\cb Attributes Menu}.

% 100616
\index{TermTextSize variable}
\item{\et TermTextSize}\\
{\bf Value:} integer 6--48.\\
This variable can be used to set the height, in pixels, of the text
used in rendering terminals and cell labels in electrical mode.  If
not set, the default is 14.

This tracks the setting of the {\cb Terminal text pixel size} entry in
the {\cb Terminals} page of the {\cb Window Attributes} panel from the
{\cb Set Attributes} button in the {\cb Attributes Menu}.

% 103113
\index{TermMarkSize variable}
\item{\et TermMarkSize}\\
{\bf Value:} integer 6--48.\\
This variable can be used to reset the pixel size of the cross used as
a terminal mark.  If not set, the default is 10.

This tracks the setting of the {\cb Terminal mark size} entry in the
{\cb Terminals} page of the {\cb Window Attributes} panel from the
{\cb Set Attributes} button in the {\cb Attributes Menu}.

% 041224
\index{ShowDots variable}
\item{\et ShowDots}\\
{\bf Value:} boolean or ``{\vt n}'' of ``{\vt a}''.\\
This variable sets the mode used to display connection indications
(dots) in schematics in electrical mode.  It tracks and sets the state
of the buttons in the {\cb Connection Points} panel available from the
{\cb Connection Dots} button in the {\cb Attributes Menu}.

If not set or set as a boolean, the normal indication is used, whereby
only ``ambiguous'' connection points are marked.  These are wire
vertices common to two or more wires (except for common end vertices
of two wires), non-endpoint wire vertices common with device or
subcircuit terminals, and any point common to three or more terminals
or wire vertices.  Note that the default for release 4.3.13 and
earlier was to not show connection dots.

If set to ``{\vt none}'' or any word starting with `{\vt n}' or `{\vt
n}', no connection point indication is used.  If set to ``{\vt all}''
of any word starting with `{\vt a}' or `{\vt A}', all connections are
marked with a dot.  This is sometimes useful to see if a connection
actually exists at a given location.

The computation of dot locations can be repeated by turning dots off,
then on again.  This may be needed occasionally to correct misplaced
dots.

% 012715
\index{FullWinCursor variable}
\item{\et FullWinCursor}\\
{\bf Value:} boolean.\\
When this variable is set, the default cursor consists of horizontal
and vertical lines that extend completely across the drawing window. 
The lines intersect at the nearest snap point in the current window.

This variable tracks the state of the {\cb Use full-window cursor}
check box in the {\cb General} page of the {\cb Window Attributes}
panel.  The {\cb Set Attributes} button in the {\cb Attributes Menu}
produces this.

% 100616
\index{CellThreshold variable}
\item{\et CellThreshold}\\
{\bf Value:} integer 0--100.\\
This sets the size threshold in pixels for physical mode subcells to
be shown in the display.  If not set, the value is effectively 4. 
Subcells that are smaller than this size in the display are either
shown as a bounding box, or not shown at all, depending on the setting
of the {\cb Subthreshold Boxes} button in the {\cb Main Window}
sub-menu in the {\cb Attributes Menu} or the sub-window {\cb
Attributes} menu.  If set to 0, all detail is drawn, which can
significantly increase rendering time.  This applies to hard copy
output as well as to on-screen rendering.

This variable tracks the {\cb Subcell visibility threshold (pixels)}
entry area in the {\cb General} page of the {\cb Window Attributes}
panel from the {\cb Set Attributes} button in the {\cb Attributes
Menu}.

In electrical mode, the threshold is effectively fixed at one pixel.

% 100616
\index{GridNoCoarseOnly variable}
\item{\et GridNoCoarseOnly}\\
{\bf Value:} boolean.\\
When this boolean variable is set, as one zooms out, when the fine
grid becomes so fine that it is not shown, the coarse grid will also
not be shown.  Otherwise, the coarse grid (only) will be shown.  This
tracks the state of the check box in the {\cb All Windows} group in
the {\cb Style} page of the {\cb Grid Setup} panel of the main window. 
This can be brought up with the {\cb Set Grid} button in the {\cb Main
Window} sub-menu of the {\cb Attributes Menu}.  This applies in
physical mode only, in electrical mode the coarse grid is not shown
without the fine grid.

% 100616
\index{GridThreshold variable}
\item{\et GridThreshold}\\
{\bf Value:} integer 4--40, default 8.\\
This sets the number of pixels that is the minimum grid spacing, in
both physical and electrical modes.  If the grid were to have a
smaller displayed spacing, it will not be shown.  Accepted values are
in the range 4 -- 40, and the value is taken as 8 if this variable is
not set.  This tracks the value of the numerical entry area in the
{\cb All Windows} group in the {\cb Style} page of the {\cb Grid
Setup} panel for the main window.  This can be brought up with the
{\cb Set Grid} button in the {\cb Main Window} sub-menu of the {\cb
Attributes Menu}.
\end{description}


% -----------------------------------------------------------------------------
% !set:cvgen 020918
\section{Convert Menu --- General}

Below are general variables relating to data input/output and format
translation.

\begin{description}
% 111908
\index{ChdFailOnUnresolved variable}
\item{\et ChdFailOnUnresolved}\\
{\bf Value:} boolean.\\
If this variable is set, when doing an operation with a Cell Hierarchy
Digest (CHD) that was created from a file containing unresolved
references (cells that were referenced but not defined in the file),
and the cells can't be referenced through libraries, the operation
will fail.  If not set, processing will continue, with the
non-references either being ignored (e.g., when flattening), or
converted to empty cells (when reading into the database), or
propagated to output (when writing output), depending on the
operation.

% 120110
\index{ChdCmpThreshold variable}
\item{\et ChdCmpThreshold}\\
{\bf Value:} integer {\vt >=} 0.\\
When using a Cell Hierarchy Digest (CHD), by default instance lists
larger than 256 bytes are stored in compressed form in memory.  This
reduces memory use, but there is a small speed penalty.

This variable sets the size threshold for compression.  If set to a
value less than 100, no compression is done.  Otherwise, instance
lists larger than the set size (in bytes) will be compressed. 
Experimentation suggests that the largest blocks dominate the
decompression overhead, so that the value of this variable has little
effect, except when turning off compression entirely.

% 061408
\index{MultiMapOk variable}
\item{\et MultiMapOk}\\
{\bf Value:} boolean.\\
When set, multiple input/output GDSII layer/datatype mapping to {\Xic}
layers is enabled (as was always the case in {\Xic} releases prior to
2.5.67-5).  This allows objects in GDSII/OASIS files to be created on
more than one {\Xic} layer, and objects on {\Xic} layers to be
instantiated more than once in GDSII/OASIS output files (each with a
different layer/datatype).  When not set, each object is created or
written once only, using the first mapping in the internal list that
applies (first matching layer or {\et StreamOut} keyword found).

% 102208
\index{NoPopUpLog variable}
\item{\et NoPopUpLog}\\
{\bf Value:} boolean.\\
When set, the {\cb File Browser} loaded with the log file which
appears if there were errors or warnings when reading an input file or
writing output will {\it not} appear.  This applies to the {\cb Open}
command and equivalent, and the file input/output operations in the
{\cb Convert Menu}.  It is not recommended to set this in general, but
the browser popping up does become annoying at times, so this variable
can be set when the user knows what to expect in the file.

% 061408
\index{UnknownGdsLayerBase variable}
\item{\et UnknownGdsLayerBase}\\
{\bf Value:} integer 0--65535.\\
When translating to GDSII or OASIS from a file format that does not
have layer/datatype numbers, and no mapping can be resolved, new
layer/datatype combinations are created.  The new layer numbers are
generated sequentially, starting with the value of {\et
UnknownGdsLayerBase}, or 128 if this variable is not set.  Each is
given the datatype {\et UnknownGdsDatatype}.

% 061408
\index{UnknownGdsDatatype variable}
\item{\et UnknownGdsDatatype}\\
{\bf Value:} integer 0--65535.\\
This is the datatype assigned to new layers generated using the
{\et UnknownGdsLayerBase}.  if not set, a datatype 128 is used.

% 081908
\index{NoStrictCellnames variable}
\item{\et NoStrictCellnames}\\
{\bf Value:} boolean.\\
If the boolean variable {\et NoStrictCellnames} is set, there will be
no checking of cell names for white space, and the legacy behavior (in
releases prior to 3.0.5) of accepting white space in cell names will
be enabled.  Otherwise, white space is not allowed in cell names, and
if such cells are found in an archive being read, aliasing will be
employed to map white space characters to underscores.

% 020918
\index{NoFlattenStdVias variable}
\item{\et NoFlattenStdVias}\\
{\bf Value:} boolean.\\
When set, and when flattening a physical cell hierarchy, standard via
instances will be retained as such rather than being converted to
geometry.  This variable tracks the state of the {\cb Don't flatten
standard vias, move to top} check box in the {\cb Flatten Hierarchy}
panel, and the {\cb Don't flatten standard vias, keep as instances at
top level} check boxes in the {\cb Setup} pages of the {\cb Import
Control} panel, {\cb Export Control} panel, and the {\cb Format
Conversion} panel.

Presently, when the input data source is an archive file, this
variable applies only when sub-masters are {\bf not} contained in the
source file, and are therefor created in {\Xic}.

% 020918
\index{NoFlattenPCells variable}
\item{\et NoFlattenPCells}\\
{\bf Value:} boolean.\\
When set, and when flattening a physical cell hierarchy, parameterized
cell (pcell) instances will be retained as such rather than being
converted to geometry.  This variable tracks the state of the {\cb
Don't flatten param.  cells, move to top} check box in the {\cb
Flatten Hierarchy} panel, and the {\cb Don't flatten pcells, keep as
instances at top level} check boxes in the {\cb Setup} pages of the
{\cb Import Control} panel, {\cb Export Control} panel, and the {\cb
Format Conversion} panel.

Presently, when the input data source is an archive file, this
variable applies only when sub-masters are <b>not</b> contained in the
source file, and are therefor created in {\Xic}.

% 020918
\index{NoFlattenLabels variable}
\item{\et NoFlattenLabels}\\
{\bf Value:} boolean.\\
When set, and when flattening a cell hierarchy (physical or
electrical), labels found in subcells are ignored (not copied into the
current cell).  Labels found in the current cell are retained.  This
is intended to avoid creating conflicting net labels of wire nets from
(subnet) labels in subcells.  This variable tracks the state of the
{\cb Ignore labels in subcells} check box in the {\cb Flatten
Hierarchy} panel, and the {\cb Ignore labels in subcells} check boxes
in the {\cb Setup} pages of the {\cb Import Control} panel, {\cb
Export Control} panel, and the {\cb Format Conversion} panel.

% 022716
\index{NoReadLabels variable}
\item{\et NoReadLabels}\\
{\bf Value:} boolean.\\
When this variable is set, text label elements will not be read from
archive files in physical mode.  This may improve efficiency if the
user is concerned with physical layout data only.  This variable
tracks the setting of the {\cb Skip reading text labels from physical
archives} check box in the {\cb Setup} page of the {\cb Import
Control} panel from the {\cb Convert Menu}.

% 061408
\index{KeepBadArchive variable}
\item{\et KeepBadArchive}\\
{\bf Value:} boolean.\\
When generating an archive file and an error occurs, the archive file
will normally be deleted.  However, if this variable is set, the
output file will be given a ``{\vt .BAD}'' extension and retained. 
This file should be considered corrupt, but may be useful for
diagnostics.
\end{description}


% -----------------------------------------------------------------------------
% !set:cvimport 020918
\section{Convert Menu --- Input and ASCII Output}

The {\cb !set} variables below affect the format conversion when
importing data from a file.  Many of these variables have counterpart
controls in the {\cb Import Control} panel from the {\cb Convert
Menu}.  The functionality also applies in many cases when input is
being read in the {\cb Open} command and similar.

The following table identifies where the variables in this section are
set, if settable from the graphical interface, and specifies the scope
of the variables.

\begin{tabular}{|l|l|l|} \hline
\bf Variable          & \bf Set From              & \bf Notes\\ \hline
\et ChdLoadTopOnly    & \cb Import Control        & 5\\ \hline
\et ChdRandomGzip     &                           & 6\\ \hline
\et AutoRename        & \cb Import Control        & 1\\ \hline
\et NoCreateLayer     & \cb Import Control        & 1\\ \hline
\et NoAskOverwrite    & \cb Import Control        & 1\\ \hline
\et NoOverwritePhys   & \cb Import Control        & 1\\ \hline
\et NoOverwriteElec   & \cb Import Control        & 1\\ \hline
\et MergeInput        & \cb Import Control        & 1\\ \hline
\et NoPolyCheck       & \cb Import Control        & 1\\ \hline
\et DupCheckMode      & \cb Import Control        & 1\\ \hline
\et EvalOaPCells      & \cb Import Control        & 1\\ \hline
\et NoEvalNativePCells & \cb Import Control       & 1\\ \hline
\et NoCheckEmpties    & \cb Import Control        & 1\\ \hline
\et NoReadLabels      & \cb Import Control        & 1\\ \hline
\et LayerList         & layer change module       & 2\\ \hline
\et UseLayerList      & layer change module       & 2\\ \hline
\et LayerAlias        & layer change module       & 2\\ \hline
\et UseLayerAlias     & layer change module       & 2\\ \hline
\et InToLower         & cell name mapping module  & 3\\ \hline
\et InToUpper         & cell name mapping module  & 3\\ \hline
\et InUseAlias        & cell name mapping module  & 3\\ \hline
\et InCellNamePrefix  & cell name mapping module  & 3\\ \hline
\et InCellNameSuffix  & cell name mapping module  & 3\\ \hline
\et NoMapDatatypes    & \cb Import Control        & 1\\ \hline
\et CifLayerMode      & \cb Import Control        & 1\\ \hline
\et OasReadNoChecksum &                           & 1\\ \hline
\et OasPrintNoWrap    & {\cb Format Conversion}, {\cb ASCII Text} page & 4\\
  \hline
\et OasPrintOffset    & {\cb Format Conversion}, {\cb ASCII Text} page & 4\\
  \hline
\end{tabular}

Notes:
\begin{enumerate}
\item{These variables apply whenever a layout file is being read, in
any mode.}

\item{These variables apply to actions initiated from any panel
containing the layer filtering/aliasing module, and to the
following script functions:
\begin{quote}
{\vt OpenCell}\\
{\vt FromArchive}\\
{\vt OpenCellHierDigest}\\
{\vt ChdEdit}\\
{\vt ChdOpenFlat}\\
{\vt ChdWrite}\\
{\vt ChdWriteSplit}\\
{\vt ChdLoadGeometry}
\end{quote}}

\item{These variables apply to actions initiated from any panel
containing the {\cb Cell Name Mapping} control group, and to the
following script functions:
\begin{quote}
{\vt OpenCell}\\
{\vt FromArchive}\\
{\vt OpenCellHierDigest}
\end{quote}}

\item{These variables apply only when writing ASCII text from OASIS
input.}

\item{These variables apply when reading cells into main memory from
a Cell Hierarchy Digest.}

\item{These variables apply when reading gzipped GDSII or CGX files
through a Cell Hierarchy Digest.}
\end{enumerate}

\begin{description}
% 022716
\index{ChdLoadTopOnly variable}
\item{\et ChdLoadTopOnly}\\
{\bf Value:} boolean.\\
When set, when reading cells into the main database from a Cell
Hierarchy Digest (CHD), only the requested cell is actually read.  Any
subcells of the cell become reference cells in the main database. 
This allows editing of the requested cell, and when written to disk
the complete hierarchy will appear, however loading the whole
hierarchy into memory is avoided.

This variable tracks the state of the {\cb Load top cell only} check
box in the {\cb Cell Hierarchy Digests} panel.

% 052222
\index{ChdRandomGzip variable}
\item{\et ChdRandomGzip}\\
{\bf Value:} boolean or 0--255.\\
This variable enables use of a random-access mapping capability for
Cell Hierarchy Digest (CHD) accesses to gzipped GDSII and CGX files. 
This will speed up CHD operations that must seek randomly in the input
file.

CHDs created while this variable is set will include the mapping
structure if the input file is gzipped.  The mapping structure
provides access points to data within the file, spaced by default by
about 1Mb of uncompressed data.  The map requires about 32Kb per
access point.  When seeking in the file, one can jump to the closest
earlier access point, and read to the desired offset.  Without the
mapping, one can only read forward from the current location to the
desired location, or rewind to the beginning and read to the desired
location.

The integer is the number of Mb between access points.  If 0, it is as
if the variable is not set.  Setting as a boolean, i.e., to no value,
is equivalent to setting to 1.

% 022716
\index{AutoRename variable}
\item{\et AutoRename}\\
{\bf Value:} boolean.\\
When set, when reading archive files and a cell is encountered with
the same name as a cell already in memory, the new cell name is
automatically changed to avoid a clash.  Thus, the {\cb Merge Control}
pop-up never appears when this variable is set.  The new name has an
added suffix ``{\vt \$}{\it N}'' where {\it N} is an integer.  When
this is set, the alias file (if enabled) is never updated.  A warning
is added to the log file when a cell name is changed.  This is part of
a more general cell name mapping capability (see \ref{cellname}). 
This variable is set when the {\cb Auto Rename} entry is selected in
the {\cb Default when new cells conflict} menu in the {\cb Setup} page
of the {\cb Import Control} panel from the {\cb Convert Menu}.

% 022716
\index{NoCreateLayer variable}
\item{\et NoCreateLayer}\\
{\bf Value:} boolean.\\
When set, when reading an input layout file and a layer is found that
can't be mapped to the existing {\Xic} layers, the read will be
aborted.  The behavior otherwise is to create new layers as needed.
 
This variable tracks the state of the {\cb Don't create new layers
when reading, abort instead} check box in the {\cb Setup} page of the
{\cb Import Control} panel from the {\cb Convert menu}.

% 022716
\index{NoMapDatatypes variable}
\item{\et NoMapDatatypes}\\
{\bf Value:} boolean.\\
This variable affects only the creation of new layers when a GDSII or
OASIS file is read.  The default behavior is to create a separate new
{\Xic} layer for each GDSII layer/datatype encountered that is not
mapped in the technology file.  With the variable set, all datatypes
on the new GDSII layer are mapped to the same (new) {\Xic} layer. 
This variable tracks the state of the {\cb Map all unmapped GDSII
datatypes to same Xic layer} check box in the {\cb Setup} page of the
{\cb Import Control} panel from the {\cb Convert Menu}.

% 022716
\index{NoAskOverwrite variable}
\item{\et NoAskOverwrite}\\
{\bf Value:} boolean.\\
If a disk file is opened which contains a cell with the same name as
one already in memory, and {\et AutoRename} is not set, the default
behavior is to produce a {\cb Merge Control} pop-up which gives the
user control over how to proceed.  If this variable is set, then the
pop-up will not appear, and the default action will be taken.  The
default action can be specified with the {\et NoOverwritePhys} and
{\et NoOverwriteElec} variables.  This variable tracks the state of
the {\cb Don't prompt for overwrite instructions} check box in the
{\cb Setup} page of the {\cb Import Control} panel from the {\cb
Convert menu}.

\index{NoOverwritePhys variable}
\index{NoOverwriteElec variable}
\item\parbox[b]{4in}{
{\et NoOverwritePhys}\\
{\et NoOverwriteElec}}\\
{\bf Value:} boolean.\\
These control the default behavior when a cell from a file being read
conflicts with the name of a cell already in memory.  The default
behavior is for the cell from the file to overwrite the cell in
memory.  If {\et NoOverwritePhys} is set, the physical part of the
cell in memory will not be overwritten, and the physical part of the
cell in the file will be ignored.  Similarly, if {\et NoOverwriteElec}
is set, the electrical part of the cell in memory will be preserved,
and the electrical part of the cell from the file will be ignored. 
This variable is set according to the choice in the {\cb Default when
new cells conflict} menu in the {\cb Setup} page of the {\cb Import
Control} panel from the {\cb Convert Menu}.

% 111908
\index{NoOverwriteLibCells variable}
\item{\et NoOverwriteLibCells}\\
{\bf Value:} boolean.\\
By default, existing cells in memory can be overwritten if a cell of
the same name is read when opening cells from an archive file, if the
overwriting mode is enabled.  Setting this variable will prevent
existing cells that were opened through the library mechanism (and
thus has the LIBRARY flag set) from being overwritten.

The {\cb No Overwrite Lib Cells} button in the {\cb Libraries Listing}
pop-up tracks the state of this variable.

% 022716
\index{NoCheckEmpties variable}
\item{\et NoCheckEmpties}\\
{\bf Value:} boolean.\\
When set, there is no checking for empty cells as an input file is
being read, and the pop-up that normally appears when a file is opened
for editing if there are empty cells in the hierarchy is suppressed. 
An ``empty cell'' as listed is a cell that is either absent or has no
content in both electrical and physical modes.  It is possible to
check for empty cells at any time with the {\cb !empties} command. 
This variable tracks the setting of the {\cb Skip testing for empty
cells} check box in the {\cb Setup} page of the {\cb Import Control}
panel from the {\cb Convert Menu}.

% 022716
\index{NoPolyCheck variable}
\item{\et NoPolyCheck}\\
{\bf Value:} boolean.\\
When this boolean variable is set, the tests for problematic
conditions such as self-overlap, normally applied to polygons, is
skipped.  The default behavior is to check each polygon for
potentially troublesome geometry specification while the polygon is
being created.  If a layout is known to have only ``good'' polygons,
then turning off this test may slightly reduce reading time.

This variable tracks the setting of the {\cb Skip testing for badly
formed polygons} check box in the {\cb Setup} page of the {\cb Import
Control} panel from the {\cb Convert Menu}.

% 022716
\index{DupCheckMode variable}
\item{\et DupCheckMode}\\
{\bf Value:} boolean or string.\\
When reading layout data and identical objects or subcells are found
at the same location, the default action is to issue a warning message
and read the duplicates into the database.  This variable can be set
to alter the default behavior.  If set to a word starting with `{\vt
r}' (case insensitive), the duplicate objects or subcells will not be
brought into the database.  As duplicates are almost always layout
errors, it makes sense to filter them, though they generally cause no
harm.  If this variable is set to a word starting with `{\vt w}', only
a warning will be issued, exactly as if the variable were not set.  If
set to anything else, including an empty string (i.e., set as a
boolean), testing for duplicates is disabled.  This may very slightly
reduce the time to read in a file.

This variable tracks the setting of the {\cb Duplicate item handling}
menu in the {\cb Setup} page of the {\cb Import Control} panel from
the {\cb Convert Menu}.

%\ifoa
% 030416
\index{EvalOaPCells variable}
\item{\et EvalOaPCells}\\
{\bf Value:} boolean.\\
When a non-native pcell placement is encountered when reading file
input, the default behavior is to not attempt to evaluate the pcell,
and assume that the sub-master has been exported.  Generally,
evaluation of a Skill-based pcell will fail, unless Virtuoso is
accessible and the pcell caching has been turned on and is up to date.
 
If this variable is set, {\Xic} will attempt to evaluate foreign pcell
placements, which is necessary if the sub-masters have not been
supplied by another means.  The OpenAccess library that supplies the
super-master must be open.
 
If sub-masters are available, it is faster to use them rather than
to evaluate the scripts and recreate the sub-master.
 
This variable tracks the status of the {\cb PCell evaluation:  Eval
OpenAccess} check box in the {\cb Setup} page of the {\cb Import
Control} panel from the {\cb Convert Menu}.
%\fi

% 030416
\index{NoEvalNativePCells variable}
\item{\et NoEvalNativePCells}\\
{\bf Value:} boolean.\\
When a native pcell placement is encountered when reading file input,
the default behavior is to attempt to locate the super-master and
evaluate the script, generating the sub-master.  It is assumed
therefor that the super-master is available.  If the sub-masters have
been included in the archive or otherwise made available, then this
variable should be set.  Otherwise, the super-masters must be
available.

This variable tracks the status of the {\cb PCell evaluation:  Don't
eval native} check box in the {\cb Setup} page of the {\cb Import
Control} panel from the {\cb Convert Menu}.

% 022716
\index{MergeInput variable}
\item{\et MergeInput}\\
{\bf Value:} boolean.\\
When this variable is set, and a layout file is being read into the
database, boxes on the same layer are merged together, if possible, as
files are being read in.  Overlapping boxes are clipped and/or merged,
so that in the database no boxes will overlap.
 
Merging will not occur on a layer with the {\vt NoMerge} technology
file keyword applied.
 
This variable tracks the setting of the {\cb Clip and merge
overlapping boxes} check box in the {\cb Setup} page of the {\cb
Import Control} panel from the {\cb Convert Menu}.

% 022916
\index{LayerList variable}
\item{\et LayerList}\\
{\bf Value:} string.\\
This can be set to a space-separated list of layer names (see
\ref{layername}).  These layers can be used for filtering when an
archive file is being read or translated.  Each name should be in a
format which will match a layer in the file type being processed, with
wildcarding allowed.  This variable is part of the layer mapping and
filtering capability, as used in the {\cb Import Control} and {\cb
Format Conversion} panels, and tracks the entry area.  Actual
utilization of the layer list is controlled by the {\et UseLayerList}
variable.

% 022916
\index{UseLayerList variable}
\item{\et UseLayerList}\\
{\bf Value:} boolean or string.\\
This variable determines how and if the {\et LayerList} string is used
when input is being read from an archive file.  This variable is part
of the layer mapping and filtering capability, as used in the {\cb
Import Control} and {\cb Format Conversion} panels, and tracks the
check boxes.

If {\et UseLayerList} in not set, the {\et LayerList} is ignored, and
any layer found in the input file will be read or converted.  If {\et
UseLayerList} is set to a word starting with `{\vt n}' or `{\vt N}',
layers that are listed in the {\et LayerList} will {\it not} be
converted.  If {\et UseLayerList} is set to a anything else (including
no value) {\it only} the layers listed in the {\et LayerList} will be
converted.

% 022916
\index{LayerAlias variable}
\item{\et LayerAlias}\\
{\bf Value:} string.\\
This variable can be set to a string consisting of space-separated
{\it name\/}={\it value} pairs, where {\it name} is an existing layer
name and {\it value} is a layer name to which {\it name} will be
mapped during conversions, if {\et UseLayerAlias} is set.

This variable can be set from the {\cb Layer Aliases} editor, which is
available from pop-ups that control operations where layer filtering
and modification is available, as in the {\cb Import Control} and {\cb
Format Conversion} panels.  The variable can also be set using script
functions.

% 022916
\index{UseLayerAlias variable}
\item{\et UseLayerAlias}\\
{\bf Value:} boolean.\\
When this variable is set, when reading an archive or native file and
layer aliasing is available, layers encountered are aliased according
to entries in the {\et LayerAlias} variable.

Aliasing occurs on reading only, after the {\et LayerList} is
processed, if this feature is used.  Thus, a {\et LayerList} used for
reading should contain the unaliased layer names.  Layer aliasing
applies to physical data only, under conditions equivalent to those
listed for {\et UseLayerList}.  This variable is part of the layer
mapping and filtering capability, and tracks the {\cb Use Layer
Aliases} check box, as in the {\cb Import Control} and {\cb
Format Conversion} panels.

% 022816
\index{InToLower variable}
\item{\et InToLower}\\
{\bf Value:} boolean.\\
When set, cell names found in archive files being read that are
entirely upper case will be mapped to lower case.  A name that is
mixed-case will not be changed.  This mapping occurs for names which
are not aliased in an enabled alias file.  This is part of a more
general cell name mapping facility (see \ref{cellname}), available in
the {\cb Import Control} panel and elsewhere.

% 022816
\index{InToUpper variable}
\item{\et InToUpper}\\
{\bf Value:} boolean.\\
When set, cell names found in archive files being read that are
entirely lower case will be mapped to upper case.  A name that is
mixed-case will not be changed.  This mapping occurs for names which
are not aliased in an enabled alias file.  This is part of a more
general cell name mapping facility (see \ref{cellname}), available in
the {\cb Import Control} panel and elsewhere.

% 022816
\index{InUseAlias variable}
\item{\et InUseAlias}\\
{\bf Value:} boolean or string.\\
This variable enables utilization of the alias file (see
\ref{aliasfile}) when reading from an archive file.  If simply set as
a boolean, i.e., to no value, the alias file will be read before the
operation, and created or updated if necessary after the operation. 
If the variable is set to a word starting with `{\vt r}' (case
insensitive), then the alias file will be read before the operation
and used during the operation (if it exists), but will not be created
or updated after the operation.  If the variable is set to a word
starting with `{\vt w}' or `{\vt s}' (case insensitive), the alias
file will not be read before an operation, but will be created or
updated after the operation completes.  This is part of a more general
cell name mapping facility (see \ref{cellname}), available in the {\cb
Import Control} panel and elsewhere.

% 022816
\index{InCellNamePrefix variable}
\index{InCellNameSuffix variable}
\item{{\et InCellNamePrefix}, {\et InCellNameSuffix}}\\
{\bf Value:} string.\\
These variables are most simply set to a text token that is added to
the beginning or end of cell name strings as archive files are being
read.  Modifications will not be made to cell names found in an
enabled alias file.  The strings can also be given in the form
\begin{quote}
/{\it str\/}/{\it sub\/}/
\end{quote}
where {\it str} and {\it sub} are text tokens, separated by forward
slash characters as shown.  In this case if the characters at the
beginning/end of the cell name (for prefix/suffix) match the {\it
str}, they are replaced by {\it sub}.  This is the same action as is
used in the {\cb !rename} command.  The string token must match
exactly --- there is no wildcarding.  Either the prefix or suffix, or
both, can be defined.  The suffix substitution occurs after the prefix
substitution.  Either can match the whole cell name if one wants to
change the name of a single cell.  This is part of a more general cell
name mapping facility (see \ref{cellname}), available in the {\cb
Import Control} panel and elsewhere.

% 022816
\index{CifLayerMode variable}
\item{\et CifLayerMode}\\
{\bf Value:} integer 0--2.\\
This variable determines how {\Xic} interprets layer directives while
reading CIF files.  This is the same as the {\cb How to resolve CIF
layers} menu in the {\cb Import Control} panel.  Setting to 0 is the
default {\cb Try Both} option, 1 is the {\cb By Name} option, and 2 is
the {\cb By Index} option.

% 061408
\index{OasReadNoChecksum variable}
\item{\et OasReadNoChecksum}\\
{\bf Value:} boolean.\\
When set, the reader will ignore a checksum found in the OASIS file,
if any.  When not set, if a checksum is found, it will be compared
with a computed checksum, using the method (CRC or summation)
indicated in the file, and the conversion will fail if the checksums
are not equal.

% 022916
\index{OasPrintNoWrap variable}
\item{\et OasPrintNoWrap}\\
{\bf Value:} boolean.\\
This applies when converting OASIS input to ASCII text.  When set, the
text output for a single record will occupy one (arbitrarily long)
line.  When not set, lines are broken and continued with indentation.

This variable has a corresponding check box in the {\cb ASCII Text}
output format page of the {\cb Format Conversion} panel.

% 022916
\index{OasPrintOffset variable}
\item{\et OasPrintOffset}\\
{\bf Value:} boolean.\\
This applies when converting OASIS input to ASCII text.  When set, the
first token for each record output gives the offset in the file or
containing CBLOCK.  When not set, file offsets are not printed.

This variable has a corresponding check box in the {\cb ASCII Text}
output format page of the {\cb Format Conversion} panel.
\end{description}


% -----------------------------------------------------------------------------
% xic:cvexport 081318
\section{Convert Menu --- Output}

The {\cb !set} variables below affect the format conversion when
writing data to a file.  Many of these variables have counterpart
buttons in the {\cb Export Control} panel from the {\cb Convert Menu}. 
The functionality may also apply to files created with the {\cb Save}
command and similar.

The following table identifies where the variables in this section are
set, if settable from the graphical interface, and specifies the scope
of the variables.

\begin{tabular}{|l|l|l|} \hline
\bf Variable          & \bf Set From              & \bf Notes\\ \hline
\et StripForExport & {\cb Format Conversion} and {\cb Export Control} & 4\\
  \hline
\et WriteMacroProps   & \cb Export Control        & 1\\ \hline
\et KeepLibMasters    & \cb Export Control        & 3\\ \hline
\et SkipInvisible     & \cb Export Control        & 3\\ \hline
\et KeepBadArchive    &                           & 1\\ \hline
\et NoCompressContext &                           & 5\\ \hline
\et RefCellAutoRename &                           & 5\\ \hline
\et UseCellTab        &                           & 5\\ \hline
\et SkipOverrideCells &                           & 5\\ \hline
\et OutToLower        & cell name mapping module  & 2\\ \hline
\et OutToUpper        & cell name mapping module  & 2\\ \hline
\et OutUseAlias       & cell name mapping module  & 2\\ \hline
\et OutCellNamePrefix & cell name mapping module  & 2\\ \hline
\et OutCellNameSuffix & cell name mapping module  & 2\\ \hline
\et CifOutStyle       & \cb Export Control & 1\\ \hline
\et CifOutExtensions  & \cb Export Control & 1\\ \hline
\et CifAddBBox        &                           & 1\\ \hline
\et GdsOutLevel       & \cb Export Control & 1\\ \hline
\et GdsMunit          & \cb Export Control & 1\\ \hline
\et NoGdsMapOk        & \cb Export Control & 1\\ \hline
\et OasWriteCompressed & \cb Export Control & 1\\ \hline
\et OasWriteNameTab   & \cb Export Control & 1\\ \hline
\et OasWriteRep       & \cb Export Control & 1\\ \hline
\et OasWriteChecksum  & \cb Export Control & 1\\ \hline
\et OasWriteNoTrapezoids & \cb Advanced OASIS Export Parameters & 1\\ \hline
\et OasWriteWireToBox    & \cb Advanced OASIS Export Parameters & 1\\ \hline
\et OasWriteRndWireToPoly & \cb Advanced OASIS Export Parameters & 1\\ \hline
\et OasWriteNoGCDcheck   & \cb Advanced OASIS Export Parameters & 1\\ \hline
\et OasWriteUseFastSort  & \cb Advanced OASIS Export Parameters & 1\\ \hline
\et OasWritePrptyMask    & \cb Advanced OASIS Export Parameters & 1\\ \hline
\end{tabular}

Notes:
\begin{enumerate}
\item{These variables apply whenever a layout file is being written,
in any mode.}

\item{These variables apply to actions initiated from a panel
containing the {\cb Cell Name Mapping} control group, and to the
following script functions:
\begin{quote}
{\vt ToXIC}\\
{\vt ToCGX}\\
{\vt ToCIF}\\
{\vt ToGDS}\\
{\vt ToGdsLibrary}\\
{\vt ToOASIS}
\end{quote}}

\item{Applies when a file is being written using the {\cb Export
Control} panel, and with the script functions listed above.}

\item{The {\et StripForExport} variable applies as described below.}

\item{These variables apply when using a Cell Hierarchy Digest (CHD)
to access cells for writing.  Reference cells are pointers to CHD
data.}
\end{enumerate}

\begin{description}
% 022418
\index{StripForExport variable}
\item{\et StripForExport}\\
{\bf Value:} boolean.\\
When this variable is set, files produced through the {\cb Export
Control} and {\cb Format Conversion} panels will contain the basic
syntax elements with no extensions.  Thus, they contain physical data
only.  The {\et StripForExport} variable applies when writing all
output, {\bf except} when using the {\cb Save} and {\cb Save As}
buttons in the {\cb File Menu}, and the equivalent text accelerators
and including the prompts when exiting the program.  It is also
ignored when using the {\vt Save} script function, but applies in the
{\vt ToArchive} script function.

Within {\Xic}, archive file representations consist of two sequential
records in each file.  The first record is the physical information,
and the second record contains the electrical information.  These
files should be compatible with other CAD systems, as these files are
generally expected to have only one record, and consequently the
electrical information may be ignored.  However, one should not count
on this.  When in effect, only the physical record is output.  This
produces a file which should be an absolutely conventional physical
layout file.

Additionally, when {\et StripForExport} is set, and when writing out a
hierarchy from the main database, all cells in the hierarchy will be
written, whether or not the {\et KeepLibMasters} variable is set. 
Thus, the file will not contain unsatisfied cell references, as
(physical) library cells will be included.  Further, all referenced
pcell and standard via sub-masters will be written to output, similar
to the case when the {\et PCellKeepSubMasters} and {\et
ViaKeepSubMasters} variables are set. 

This variable tracks the state of the {\cb Strip For Export - (convert
physical data only)} check box which appears in the {\cb Export
Control} and {\cb Format Conversion} panels.  This button should be
active when creating a file to be sent to a vendor for use in
generating photomasks.  Note that the electrical information can never
be recovered from a stripped file.

% 081318
\index{WriteMacroProps variable}
\item{\et WriteMacroProps}\\
{\bf Value:} boolean.\\
When set, output will include {\et macro} properties, which are no
longer in use in 4.3.6 and later.  This variable can be set to force
generation of these properties, thus providing backwards
compatibility.

% 022816
\index{KeepLibMasters variable}
\item{\et KeepLibMasters}\\
{\bf Value:} boolean.\\
When writing an archive file from a hierarchy in the main database,
cells in the hierarchy that were opened through the library mechanism
are by default {\bf not} included in the file.  References to these
cells remain, though no library cell definition records will appear in
output.  The file will not be self-contained, as the library cell
references are unresolved without the corresponding libraries.

When this variable is set, files produced with the {\cb Export
Control} panel will include all cells in the hierarchy, and the file
produced will not have any unsatisfied references (except for
electrical device library cells, which are never included in output). 
The variable also applies to the script functions listed in the notes
to the table at the top of this section.  It does {\it not} apply to
the {\cb Save} and {\cb Save As} commands, which always omit library
cells.

This variable tracks the state of the {\cb Include Library Cells}
check box in the {\cb Export Control} panel.

% 022816
\index{SkipInvisible variable}
\item{\et SkipInvisible}\\
{\bf Value:} boolean or string.\\
When this variable is set, only layers that are currently visible, as
selected with button 2 in the layer table or otherwise, will be
converted when writing output from the {\cb Export Control} panel.  If
set to a word beginning with `{\vt p}' (case insensitive), only
invisible physical layers will be skipped.  If set to a word beginning
with `{\vt e}' (case insensitive) only the invisible electrical layers
will be skipped.  If set to anything else, including the empty string,
both physical and electrical invisible layers will be skipped.  This
variable tracks the state of the {\cb Don't convert invisible layers}
check boxes in the {\cb Export Control} panel.

% 022916
\index{NoCompressContext variable}
\item{\et NoCompressContext}\\
{\bf Value:} boolean.\\
The Cell Hierarchy Digest (CHD) is a data structure which provides a
compact representation of a cell hierarchy found in an archive file. 
This data structure is used in operations where random-access of cells
in the archive file is required.  This is used in some of the
conversion functions provided in the {\cb Format Conversion} panel
from the {\cb Convert Menu}, and elsewhere.

In order to process large files, it is important that the CHD use as
little memory as possible.  In release 2.5.67 and later, a mechanism
is used to compress instance lists by default.  This can shrink the
memory used by the CHD by 50% or more, but requires a little more
computational overhead.

The digest files written by the {\cb Save} button in the {\cb Cell
Hierarchy Digests} panel and the {\vt WriteCellHierDigest} script
function use the compressed instance lists by default, and are
typically more compact than the older format.  These files have a new
magic number and can not be read by {\Xic} releases prior to 2.5.67.

This boolean variable, if set, will prevent use of compression in the
CHD structures, and files written will be backwards compatible.  It is
unlikely that the user will find it necessary to set this variable.

% 022816
\index{RefCellAutoRename variable}
\item{\et RefCellAutoRename}\\
{\bf Value:} boolean.\\
This variable applies when writing hierarchies containing reference
cells, which are cells which point to data obtained through a Cell
Hierarchy Digest but are otherwise empty.  When written to a layout
file, these cells expand into a full cell hierarchy obtained from the
CHD.  The output file can not contain more than one cell definition
for a given name, so by default if a duplicate cell name is
encountered when writing, that cell definition is simply skipped, and
all instances of the cell in output will reference the original
definition.

This is the correct thing to do when duplicate cell names come from
the same (or an equivalent) CHD, as the duplicates really do indicate
the same cell.  However, if the names come from different CHDs, this
could indicate a true name clash.

When this variable is set, names that clash, and that come from
non-equivalent CHDs, will cause an automatic renaming of the cell, and
a cell definition will be generated in output under the new name.  The
subsequent cell instances will be updated to call the new name.  Names
that clash but come from equivalent CHDs will have the cell definition
skipped, as in the default mode.

This variable tracks the {\cb Use auto-rename when writing CHD
reference cells} check box in the {\cb Cell Hierarchy Digests} panel
from the <b>File Menu</b>.

% 022816
\index{UseCellTab variable}
\item{\et UseCellTab}\\
{\bf Value:} boolean.\\
This variable enables cell definition substitution when using a Cell
Hierarchy Digest (CHD) to access cells for purposes other than reading
into main memory.  When set, cell names found in the {\cb Cell Table
Listing}, which also are visible in the main database will replace
cells of the same name when accessing a hierarchy through a CHD.  This
feature can be used to modify cells in a hierarchy without having to
read the entire hierarchy into main memory.

This variable tracks the state of the {\cb Use cell table} check box
in the {\cb Cell Hierarchy Digests} panel.

% 061408
\index{SkipOverrideCells variable}
\item{\et SkipOverrideCells}\\
{\bf Value:} boolean.\\
This variable applies only when {\et UseCellTab} is set.  When this
variable is also set, cell names listed in the {\cb Cell Table
Listing} will be skipped, rather than substituted.  When writing
output, this will produce files that have unresolved references, which
can be satisfied by another source, such as a library.

This variable tracks the state of the {\cb Override} and {\cb Skip}
radio buttons in the {\cb Cell Table Listing} panel.

% 040724
\index{OutAllCells variable}
\item{\et OutAllCells}\\
{\bf Value:} boolean.\\
When set, all cells in the current symbol table, not just the
hierarchy of the current cell, will be output as if they were part of
the hierarchy.  The usual filtering of library and sub-master cells is
retained.  The resulting file may have multiple top-level cells.  This
variable tracks the state of the {\cb Consider ALL cells in current
symbol table for output} check box in the {\cb Export Control} panel
from the {\cb Convert} main menu {\cb Export Cell Data} button.

% 071415
\index{Out32nodes variable}
\item{\et Out32nodes}\\
{\bf Value:} boolean.\\
When set, schematic cell data written to files will use the {\et node}
property syntax of the 3.2 branch of {\Xic}, providing limited
backward compatibility.  This will strip out elements not supported by
the earlier syntax, such as multi-contact points in symbols.

The files will still not really be backward compatible unless all
``new'' features are avoided.  Setting this variable may be useful for
the case where 3.2 compatibility is to be preserved for a design that
originated in 3.2 or earlier, which is read into the current release
of {\Xic}, tweaked, then saved back to disk.

The variable should not be set unless you explicitly need to create
backward-compatible files, as it will prevent features from working in
the resulting files.

% 022816
\index{OutToLower variable}
\item{\et OutToLower}\\
{\bf Value:} boolean.\\
When set, cell names found in archive files being written that are
entirely upper case will be mapped to lower case.  A name that is
mixed-case will not be changed.  This mapping occurs for names which
are not aliased in an enabled alias file.  This is part of a more
general cell name mapping facility (see \ref{cellname}), which applies
in the {\cb Export Control} panel and elsewhere.

% 022816
\index{OutToUpper variable}
\item{\et OutToUpper}\\
{\bf Value:} boolean.\\
When set, cell names found in archive files being written that are
entirely lower case will be mapped to upper case.  A name that is
mixed-case will not be changed.  This mapping occurs for names which
are not aliased in an enabled alias file.  This is part of a more
general cell name mapping facility (see \ref{cellname}), which applies
in the {\cb Export Control} panel and elsewhere.

% 022816
\index{OutUseAlias variable}
\item{\et OutUseAlias}\\
{\bf Value:} boolean or string.\\
This variable enables utilization of the alias file (see
\ref{aliasfile}) when writing to an archive file.  If simply set as a
boolean, i.e., to no value, the alias file will be read before the
operation, and created or updated if necessary after the operation. 
If the variable is set to a word starting with `{\vt r}' (case
insensitive), then the alias file will be read before the operation
and used during the operation (if it exists), but will not be created
or updated after the operation.  If the variable is set to a word
starting with `{\vt w}' or `{\vt s}' (case insensitive), the alias
file will not be read before an operation, but will be created or
updated after the operation completes.  This is part of a more general
cell name mapping facility (see \ref{cellname}), which applies
in the {\cb Export Control} panel and elsewhere.

% 022816
\index{OutCellNamePrefix variable}
\index{OutCellNameSuffix variable}
\item{{\et OutCellNamePrefix}, {\et OutCellNameSuffix}}\\
{\bf Value:} string.\\
These variables are most simply set to a text token that is added to
the beginning or end of cell name strings as archive files are being
written.  Modifications will not be made to cell names found in an
enabled alias file.  The strings can also be given in the form
\begin{quote}
/{\it str\/}/{\it sub\/}/
\end{quote}
where {\it str} and {\it sub} are text tokens, separated by forward
slash characters as shown.  In this case if the characters at the
beginning/end of the cell name (for prefix/suffix) match the {\it
str}, they are replaced by {\it sub}.  This is the same action as is
used in the {\cb !rename} command.  The string token must match
exactly --- there is no wildcarding.  Either the prefix or suffix, or
both, can be defined.  The suffix substitution occurs after the prefix
substitution.  Either can match the whole cell name if one wants to
change the name of a single cell.  This is part of a more general cell
name mapping facility (see \ref{cellname}), which applies
in the {\cb Export Control} panel and elsewhere.

% 022816
\index{CIFoutStyle variable}
\item{\et CIFoutStyle}\\
{\bf Value:} string.\\
When set, this variable will determine the CIF output style.  Changing
the {\cb Cell Name Extension}, {\cb Layer Specification}, or {\cb
Label Extension} option menu choices in the {\cb CIF} page of the {\cb
Export Control} pop-up will update the value of {\et CifOutStyle}.

The {\et CIFoutStyle} variable can be set to the following values,
which will set the CIF output style as indicated.  The syntax
associated with the indices is given in \ref{cifsettings}, describing
the {\cb Export Control} panel.

\begin{tabular}{lllll}
\kb Value &\kb Historical Name &\kb cname\_index &\kb layer\_index &\kb
  label\_index\\
\vt a & Stanford & 1 & 0 & 1\\
\vt b & NCA      & 1 & 1 & 2\\
\vt i & Icarus   & 2 & 0 & 1\\
\vt m & Mextra   & 0 & 0 & 3\\
\vt n & none     & 4 & 0 & 4\\
\vt s & Sif      & 3 & 0 & 1\\
\vt x & Xic      & 0 & 0 & 0\\
{\it cn}:{\it la}:{\it lb} & - & {\it cn} & {\it la} & {\it lb}\\
\end{tabular}

The final form consists of three colon-separated integers which are
interpreted as indices into the option lists as implied above.  If the
style parameters are changed in the {\cb Export Control} pop-up
while {\et CIFoutStyle} is set, the value of {\et CIFoutStyle} will
have this form.

% 022816
\index{CifOutExtensions variable}
\item{\et CifOutExtensions}\\
{\bf Value:} two space-separated integers.\\
The string for this variable consists of two integers that represent
banks of flags.  The first integer represents the extension flags in
use when the {\et StripForExport} variable is not set, the second
integer represents the flags in force when {\et StripForExport} is
set.  The bits of each integer represent the flag state corresponding
to the menu entries of the {\cb CIF Extensions} menu (below the
separator) in the {\cb CIF} page of the {\cb Export Control} panel,
with the top entry corresponding to the least significant bit.  The
extensions are described with the CIF Format Extensions in
/ref{cifext}, and are listed in the table below.

\begin{tabular}{ll}
\bf Extension & \bf Mask\\
scale extension & \vt 0x1\\
cell properties & \vt 0x2\\
inst name comment & \vt 0x4\\
inst name extension & \vt 0x8\\
inst magn extension & \vt 0x10\\
inst array extension & \vt 0x20\\
inst bound extension & \vt 0x40\\
inst properties & \vt 0x80\\
obj properties & \vt 0x100\\
wire extension & \vt 0x200\\
wire extension new & \vt 0x400\\
text extension & \vt 0x800\\
\end{tabular}


% 061408
\index{CifAddBBox variable}
\item{\et CifAddBBox}\\
{\bf Value:} boolean.\\
When set, each object line (boxes, polygons, wires, labels) in CIF
output will be followed by a comment line giving the bounding box of
the object, in the form
\begin{quote}\vt
(BBox {\it left bottom right top\/}{\vt );}
\end{quote}
This may be useful for debugging, but greatly increases file size
so is not recommended for general use.

In {\Xic} releases prior to 3.0.0, the format of the added
comment was 
\begin{quote}
(BBox {\it left\/},{\it top width height}{\vt );}
\end{quote}
and the extension was applied to native cell files as well as CIF
output.

% 022916
\index{GdsOutLevel variable}
\item{\et GdsOutLevel}\\
{\bf Value:} integer 0--2.\\
This variable determines the GDSII release level of GDSII output
files.  The default is release level 7, which was introduced by
Cadence in 2002.  Previous releases specified a limit of 200 or 600
polygon vertices (there seems to be some inconsistency in the
published limit) and 200 vertices for wires.  This applies to format
releases 3, 4, 5, and 6.  The only difference between these formats is
the definition of some Cadence-specific data block types that are
ignored by {\Xic}.  The latest release (7) removed these limits. 
The limits that remain are due to the block size limit (64Kb) of the
format, which implies a maximum of 8000 vertices for polygons and
wires.

When writing GDSII output, it may be necessary to enforce the limits,
if the output is destined for another program which can't handle the
release 7 limits.  The {\Xic} default is to use the release 7 limits.

The {\et GdsOutLevel} variable can be set to an integer 0--2.  The
corresponding GDSII format is as follows:
\begin{description}
\item{level 0: (the default)}\\
max poly vertices: 8000\\
max wire vertices: 8000\\
format level: 7
\item{level 1:}\\
max poly vertices: 600\\
max wire vertices: 200\\
format level: 3
\item{level 2:}\\
max poly vertices: 200\\
max wire vertices: 200\\
format level: 3
\end{description}

By setting {\et GdsOutLevel} to 1 or 2, GDSII files generated with
{\Xic} should not cause difficulty when read by older programs
(including old versions of {\Xic}).

This variable tracks the state of the {\cb GDSII version number,
polygon/wire vertex limit} menu in the {\cb GDSII} page of the {\cb
Export Control} panel from the {\cb Convert Menu}.  This page is also
used in the {\cb Format Conversion} panel, and the {\cb Layout File
Merge Tool} also from the {\cb Convert Menu}.

% 022916
\index{GdsMunit variable}
\item{\et GdsMunit}\\
{\bf Value:} real 0.01--100.0.\\
When writing GDSII, the normal MUNITS (machine units) and UUNITS (user
units) values will be multiplied by this factor, and all coordinates
in the file will be divided by this factor.  The acceptable range is
0.01 -- 100.0.  This will apply to {\it all} GDSII files written. 

This variable tracks the {\cb Unit Scale} entry in the {\cb GDSII}
page of the {\cb Export Control} panel from the {\cb Convert Menu}. 
This page is also used in the {\cb Format Conversion} panel, and the
{\cb Layout File Merge Tool} also from the {\cb Convert Menu}.

The default values for these parameters are
\begin{quote}
\begin{tabular}{ll}
MUNITS: & 1e-6/{\it resolution}\\
UUNITS: & 1.0/{\it resolution}\\
\end{tabular}
\end{quote}
where {\it resolution} is the internal resolution, which defaults to
1000 per-micron, but can be changed with the {\et DatabaseResolution}
variable.

% 022916
\index{GdsTruncateLongStrings variable}
\item{\et GdsTruncateLongStrings}\\
{\bf Value:} boolean.\\
The GDSII and CGX formats use a 16-bit integer to store record size,
limiting the size of records to 64Kb.  This prevents storage of stings
longer than this.  By default, an attempt to write such a string to a
GDSII or CGX file will generate a fatal error, aborting the operation. 
If this variable is set, overrunning strings will be truncated to
maximum possible length, and the operation will continue without
error.  Warnings will appear in the log file, however.

This variable tracks the state of the {\cb Accept but truncate
too-long strings} check box in the {\cb GDSII} and {\cb CGX} pages of
the {\cb Export Control} panel from the {\cb Convert Menu}.  These
pages are also used in the {\cb Format Conversion} panel, and the {\cb
Layout File Merge Tool} also from the {\cb Convert Menu}.

% 022916
\index{NoGdsMapOk variable}
\index{GDSII layer mapping}
\item{\et NoGdsMapOk}\\
{\bf Value:} boolean.\\
When this variable is set, layers without a GDSII output mapping will
be ignored when producing GDSII output, though a warning will appear
in the log file.  Otherwise, this is an error which terminates
conversion.

This tracks the state of the {\cb Skip layers without Xic to GDSII
layer mapping} check box in the {\cb GDSII} and {\cb OASIS} pages of
the {\cb Export Control} panel from the {\cb Convert Menu}.  These
pages are also used in the {\cb Format Conversion} panel, and the {\cb
Layout File Merge Tool} also from the {\cb Convert Menu}.

% 022816
\index{OasWriteCompressed variable}
\item{\et OasWriteCompressed}\\
{\bf Value:} boolean, or the string ``{\vt force}''.\\
When set, created OASIS files will use compression.  The content of
all CELL records and name tables will be placed in CBLOCK records. 
This can significantly reduce file size.  When not set, no compression
will be used.

By default, very short records are not compressed, as more often than
not, compression will {\it increase} the size of these blocks.  If
this variable is set to the word ``{\vt force}'', then all blocks are
compressed.  This can be used for comparison purposes, but is unlikely
to yield the best results.  This tracks the state of the check box in
the {\cb OASIS} page of the {\cb Export Control} panel.

% 022816
\index{OasWriteNameTab variable}
\item{\et OasWriteNameTab}\\
{\bf Value:} boolean.\\
When set, all strings including cell names, properties, and labels are
placed in strict-mode tables.  This will in most cases reduce file
size.  When writing OASIS files with {\et StripForExport} set, i.e.,
writing physical data only, the offset table is placed in the END
record.  With {\et StripForExport} not set, in general we write two
sequential OASIS databases into the file, the first for physical data,
the second for electrical.  In this case, string tables are used in
the physical part only, and the offset table is placed in the START
record.  PAD records are added to avoid overwriting data since this is
a non-sequential operation.  In all cases, strict-mode tables are
used.

The string tables themselves are written just ahead of the END
record in all cases (when tables are used).

This tracks the state of the check box in the {\cb OASIS} page of the
{\cb Export Control} panel.

% 022816
\label{OasWriteRep}
\index{OasWriteRep variable}
\item{\et OasWriteRep}\\
{\bf Value:} string or boolean.\\
When this variable is set, {\Xic} will try to find groups of identical
objects that can be combined into REPETITION records in OASIS output. 
This applies to all OASIS output files.  Although compute intensive,
this can save a lot of space in the output file.

If {\et OasWriteRep} is not set, subcell and object records are
written as encountered when traversing the cell structure.  If set,
objects and subcells will be cached, and similar objects and subcells
are identified and written using repetition records.

When using repetition, the following procedure is used, where
``objects'' can apply to subcells as well as geometrical objects.

\begin{enumerate}
\item{Instead of directly converting each object, the object is
   saved in a cache.}
\item{When a cell traversal is complete or an object count reached,
   the cache is processed, and objects that are identical are
   identified.  The differing objects are sorted to make use of modal
   variables.}
\item{For each group of identical objects, those that form a
   spatially linear, periodic ``run'' are extracted into a new run
   list.}
\item{For each list of runs, the runs that are spatially periodic
   are extracted into a new array list.}
\item{Each array is written using a 2-dimensional repetition.}
\item{Each remaining run is written using a 1-dimensional
   repetition.}
\item{The remaining objects, i.e., those not used in an array or
   run, are written using a random repetition.}
\end{enumerate}

The details of this process, and whether or not it is applied, are
controlled by the {\et OasWriteRep} variable.  This variable can be
set to a string containing several tokens, or set as a boolean (i.e.,
set to nothing).  The tokens can appear in any order.

\begin{quote}
    {\et OasWriteRep}: [{\it word}] [{\vt d}] [{\vt r}]
     [{\vt m}={\it N\/}] [{\vt a}={\it N\/}] [{\vt x}={\it N\/}]
     [{\vt t}={\it N\/}]
\end{quote}

\begin{description}
\item{\it word}\\
This is a token that is not recognized as one of the others.  It
consists of letters that control the type of object that the
replication process is applied to.  If the letter is present, the
corresponding object type will be processed, otherwise the replication
algorithm will not be applied to that type of object, however if this token
is not found (no letters appear), all objects will be processed.
The letters are:

\begin{quote}
\begin{tabular}{ll}
\vt c & subcells\\
\vt b & boxes\\
\vt p & polygons\\
\vt w & wires\\
\vt l & labels\\
\end{tabular}
\end{quote}

For example, ``{\vt cp}'' would indicate use of replications for
subcells and polygons only.  If no token of this type is found, then
{\it all} object types will be processed.
\end{description}

The remaining tokens are identified by the first letter only, and
the remainder of the token (up to `=' in some cases) is ignored.

\begin{description}
\item{\vt d}\\
Some debugging info is printed on the console when processing.

\item{\vt r}\\
No attempt is made to find runs or arrays, and all similar objects are
written using random placement repetitions.

\item{{\vt m}={\it N}}\\
This sets the minimum number of objects in a run.  The default value
is 4, which is also the minimum accepted value.  There can be no space
around the `=', and {\it N} must be an integer.  This is ignored if
{\vt r} is given.

\item{{\vt a}={\it N}}\\
This sets the minimum number of runs in an array.  The default value
is 2.  The value can be set to 0 (zero) in which case two dimensional
repetition finding is skipped.  Otherwise, the value must be 2 or
larger.  There can be no space around the `=', and {\it N} must be an
integer.  This is ignored if {\vt r} is given.

\item{{\vt x}={\it N}}\\
This sets the maximum number of different objects of a given type held
in the cache, before flushing occurs.  This does not include
repetition counts.  The {\it N} is an integer in the range 20 --
50000.  If not set, a default of 5000 is used.  Larger values can
reduce file size, but can greatly increase writing time due to
modality sorting.

\item{{\vt t}={\it N}}\\
This sets the maximum number of similar objects, i.e., those subject
to repetition analysis, that can exist in the cache before flushing. 
Extremely large numbers may require excessive time to scan for
repetitions.  The {\it N} is an integer which can be 0 (zero) in which
case no limit is used, or 100 or larger.  The default value is 1000000
(one million).
\end{description}

If {\et OasWriteRep} is set to an empty string, all objects will be
processed for replication, using the default run and array minimums.

The string for this variable can be composed with the interface found
in the {\cb Advanced OASIS Export Parameters} panel.  The {\cb Find
repetitions} button in the {\cb OASIS} page of the {\cb Export
Control} panel will set the variable to the current string from the
interface, or unset the variable.  It the variable is set by another
method, such as with the {\cb !set} command, the interface will be
updated to the parameters as given.  With default parameters, the
string is empty, so the variable is set as a boolean by default.

% 061408
\index{OasWriteChecksum variable}
\label{OasWriteChecksum}
\item{\et OasWriteChecksum}\\
{\bf Value:} string or boolean.\\
When not set, no checksum is written to the output.  When set as a
boolean (i.e., to no value), or to anything other than ``{\vt 2}'' or
a string beginning with ``{\vt ch}'', a cyclic-redundancy (CRC)
checksum is computed and added to the file.  If set to ``{\vt 2}'' or
a word beginning with ``{\vt ch}'', a byte-sum checksum is added to
the file.  This variable has a corresponding check box in the {\cb
OASIS} page of the {\cb Export Control} panel.  This controls
setting/unsetting as a boolean, thus the check box selects CRC
checksum or none.

% 061408
\index{OasWriteNoTrapezoids variable}
\item{\et OasWriteNoTrapezoids}\\
{\bf Value:} boolean.\\
The normal behavior is to check three and four-sided polygons to see
if they can be written as (more compact) TRAPEZOID or CTRAPEZOID
records.  Setting this variable will suppress this, providing slightly
faster conversion at the cost of larger file size.  This variable
tracks the {\cb Don't write trapezoid records} check box in the {\cb
Advanced OASIS Export Parameters} panel.

% 061408
\index{OasWriteWireToBox variable}
\item{\et OasWriteWireToBox}\\
{\bf Value:} boolean.\\
The normal behavior is to leave wires alone, preserving data-type
integrity.  However, space can be saved by writing two-vertex
rectangular wires as boxes.  Setting this variable will enable this,
which may reduce file size at the expense of slightly more conversion
time.  This variable tracks the {\cb Convert Wire to Box records when
possible} check box in the {\cb Advanced OASIS Export Parameters}
panel.

% 022809
\index{OasWriteRndWireToPoly variable}
\item{\et OasWriteRndWireToPoly}\\
{\bf Value:} boolean.\\
The OASIS format does not have a native ``rounded end'' style for
wires.  These are normally converted to extended-end wires, where the
``rounded'' part becomes Manhattan.  If this variable is set, when
converting rounded-end wires to OASIS, the wire is converted to a
polygon which is shaped the same way as all rounded-end wires in
{\Xic}.  Use of a polygon requires more memory than the wire, but this
preserves exactly the same geometrical coverage, which is valuable in
reducing geometric differences if a layout comparison is performed. 
This variable tracks the {\cb Convert rounded-end Wire records to Poly
records} check box in the {\cb Advanced OASIS Export Parameters}
panel.

% 061408
\index{OasWriteNoGCDcheck variable}
\item{\et OasWriteNoGCDcheck}\\
{\bf Value:} boolean.\\
This applies only when repetitions are being used ({\et OasWriteRep}
is set).  Normally, a greatest common divisor is computed, and if
larger than unity type 10 repetitions are converted to type 11.  This
can reduce file size.  If this variable is set, the GCD is not
computed, probably increasing file size but reducing conversion time. 
This variable tracks the {\cb Skip GCD check} check box in the {\cb
Advanced OASIS Export Parameters} panel.

% 061408
\index{OasWriteUseFastSort variable}
\item{\et OasWriteUseFastSort}\\
{\bf Value:} boolean.\\
When set, writing OASIS may be faster at the expense of file size. 
This was the only mode in releases prior to 2.5.68.  The present
release defaults to using a somewhat slower but more effective
modality sorting algorithm, which will produce smaller files.  This
variable tracks the {\cb Use alternate modal sort algorithm} check box
in the {\cb Advanced OASIS Export Parameters} panel.

% 022913
\index{OasWritePrptyMask variable}
\item{\et OasWritePrptyMask}\\
{\bf Value:} boolean or string.\\
This variable tracks the {\cb Property masking} menu selections in the
{\cb Advanced OASIS Export Parameters} panel.

There are two properties that are added to text labels by default. 
These properties are used by {\Xic} and programs based on {\Xic}
source code, and can be stripped if not needed.  This can lead to
substantial file size reduction if the file contains many text labels.

{\bf Property name}: {\vt XIC\_PROPERTIES}\\
{\bf Property number}: {\vt 7012}

This property is added when reading GDSII source.  It contains values
of attributes of the TEXT element.  These have no analogs in OASIS
format, however if the file is reconverted to GDSII, the attributes
will be restored.  These attributes are found in the following GDSII
record types:

\begin{tabular}{lll}
\bf name & \bf record & \bf description\\
ANGLE & 28 & Rotation angle of text.\\
MAG & 27 & Magnification applied to text.\\
WIDTH & 15 & Width of path used to form characters.\\
PTYPE & 33 & GDSII PATHTYPE used to form characters.\\
\end{tabular}

The property consists of a string containing name/value pairs:  the
names are the text tokens above, the values are numeric.  Tokens are
separated by white space.

{\bf Property name}: {\vt XIC\_LABEL}\\
This is added to all labels to pass the {\Xic} presentation
attributes.  The string consists of two space-separated unsigned
numbers:  {\it width} and {\it flags}.  The {\it width} is the width of
the label bounding box, in containing-cell coordinates.  The {\it
flags} is the label flags word used by {\Xic}, described in
\ref{labelflags}.

If {\et OasWritePrptyMask} is set as a boolean, i.e., to an empty
string, neither of these properties is written.  If the variable is
set to an integer value, the two least-significant bits of the integer
value are flags that mask the creation of these properties, according
to the table below.  If the variable is set to a non-empty and
non-integer value, and during conversions only (as initiated from the
{\cb Format Conversion} panel from the {\cb Convert Menu}) then {\it
all} properties are stripped from output.

\begin{tabular}{ll}
{\bf Bit 0}: & If set, {\vt XIC\_PROPERTIES} \#7012 will not be written.\\
{\bf Bit 1}: & If set, {\vt XIC\_LABEL} will not be written.\\
\end{tabular}

This variable was named ``{\et OasWriteNoXicTextPrps}'' in releases
prior to 3.0.0.
\end{description}


% -----------------------------------------------------------------------------
% !set:prpfilt 021515
\section{Custom Property Filtering}
\index{variables!property filtering}

The {\cb !set} variables below save property filter specification
strings (see \ref{prpfilt}) for use when comparing layout data.  The
{\cb !compare} command and the {\cb Compare Layouts} panel available
from the {\cb Convert} menu provide this comparison function.  The
strings are used when the custom property filtering option is enabled.

\begin{description}
% 080510
\index{PhysPrpFltCell variable}
\item{\et PhysPrpFltCell}\\
{\bf Value:} string.\\
Contains the custom filter string for physical cell properties.

% 080510
\index{PhysPrpFltInst variable}
\item{\et PhysPrpFltInst}\\
{\bf Value:} string.\\
Contains the custom filter string for physical instance properties.

% 080510
\index{PhysPrpFltObj variable}
\item{\et PhysPrpFltObj}\\
{\bf Value:} string.\\
Contains the custom filter string for physical object properties.

% 080510
\index{ElecPrpFltCell variable}
\item{\et ElecPrpFltCell}\\
{\bf Value:} string.\\
Contains the custom filter string for electrical cell properties.

% 080510
\index{ElecPrpFltInst variable}
\item{\et ElecPrpFltInst}\\
{\bf Value:} string.\\
Contains the custom filter string for electrical instance properties.

% 080510
\index{ElecPrpFltObj variable}
\item{\et ElecPrpFltObj}\\
{\bf Value:} string.\\
Contains the custom filter string for electrical object properties.
\end{description}


% -----------------------------------------------------------------------------
% !set:drc 021615
\section{Design Rule Checking}
\label{drcvars}
\index{variables!design rule checking}
\index{variables!drc}

These variables are used by the design rule checking (DRC) system and
are not generated by or recognized in the {\XicII} or {\Xiv} feature
sets.  Unless stated otherwise, these settings can be controlled from
the {\cb DRC Defaults} panel from the {\cb Set Defaults} button in the
{\cb DRC Menu}.

\begin{description}

% 021615
\index{Drc variable}
\item{\et Drc}\\
{\bf Value:} boolean.\\
This sets whether or not the interactive rule checking is applied to
objects being added to the database, tracking the state of the {\cb
Enable Interactive} button in the {\cb DRC Menu}.

% 021615
\index{DrcNoPopup variable}
\item{\et DrcNoPopup}\\
{\bf Value:} boolean.\\
This variable determines whether errors generated in interactive DRC
will be listed in a pop-up window.  If set, the messages will not pop
up automatically.  This initializes the state of the {\cb No Pop Up
Errors} button in the {\cb DRC Menu}.

% 021615
\index{DrcLevel variable}
\item{\et DrcLevel}\\
{\bf Value:} integer 0--2.\\
This sets the error recording level for design rule checking.  If set
to zero (``0'') or not set, only one violation is recorded per object. 
If 1, one violation of each type is recorded per object.  If 2, all
violations found are recorded.

% 021615
\index{DrcMaxErrors variable}
\item{\vt DrcMaxErrors}\\
{\bf Value:} integer 0--100000.\\
This variable sets the maximum number of design rule violations
reported in batch mode, at which point checking terminates.  If set to
zero or not set, no limit is imposed.

% 021615
\index{DrcInterMaxObjs variable}
\item{\et DrcInterMaxObjs}\\
{\bf Value:} integer 0--100000.\\
In interactive design rule checking, this variable provides a limit on
the number of objects checked, to minimize the pause after an
operation.  If set to 0, no limit is imposed.  If not set, a limit of
1000 is taken.

% 021615
\index{DrcInterMaxTime variable}
\item{\et DrcInterMaxTime}\\
{\bf Value:} integer 0--30000.\\
This variable limits the time of the interactive design rule checking
performed after each operation.  The value is given in milliseconds. 
If the value is 0, there is no time limit imposed.  If the variable
is not set a limit of 5000 (five seconds) is assumed.

% 021615
\index{DrcInterMaxErrors variable}
\item{\et DrcInterMaxErrors}\\
{\bf Value:} integer.\\
This variable limits the number of violations to record during
interactive testing.  When the limit is reached, testing stops and
control returns to the user.  If set to 0, there is no limit.  If not
set, a limit of 100 violations is imposed.

% 021615
\index{DrcInterSkipInst variable}
\item{\et DrcInterSkipInst}\\
{\bf Value:} boolean.\\
If a subcell is copied, moved, or placed, by default the subcell is
tested for design rule violations if in interactive mode.  Setting
this variable will cause this checking to be skipped.  The checking
may be redundant and time consuming.

% 010615
\index{DrcChdName variable}
\item{\et DrcChdName}\\
{\bf Value:} string.\\
It is possible to use a Cell Hierarchy Digest (CHD) to specify a
target layout for design rule checking.  This can allow DRC testing of
layouts that are too large to be read into {\Xic} normally.  This
value mirrors the contents of the {\cb CHD reference name} text entry
area in the {\cb DRC Run Control} panel from the {\cb Batch Check}
button in the {\cb DRC Menu}.

% 010615
\index{DrcChdCell variable}
\item{\et DrcChdCell}\\
{\bf Value:} string.\\
This variable stores an optional cell name for use as the top-level
cell when a CHD is used for DRC.  It mirrors the contents of the {\cb
CHD top cell} text entry area in the {\cb DRC Run Control} panel from
the {\cb Batch Check} button in the {\cb DRC Menu}.

% 010615
\index{DrcLayerList variable}
\item{\et DrcLayerList}\\
{\bf Value:} string.\\
It is possible to use only rules on certain layers, or to skip rules
on certain layers, when running DRC.  This variable contains a space
separated list of layer names for use in the layer filtering.  It
mirrors the contents of the {\cb Layer List} text entry area in the
{\cb DRC Parameter Setup} panel from the {\cb Setup} button in the
{\cb DRC Menu}.

% 010615
\index{DrcUseLayerList variable}
\item{\et DrcUseLayerList}\\
{\bf Value:} boolean or string.\\
If this variable is set to a word that starts with `{\vt n}' (case
insensitive) the layers listed in the {\et DrcLayerList} variable will
be skipped during DRC runs, meaning that the rules defined on the
skipped layers will not be evaluated.  If {\et DrcUseLayerList} is set
to anything else, including to an empty string (i.e., as a boolean),
then only rules on layers listed in the {\et DrcLayerList} variable
will be checked during DRC runs.  In this case, if the {\et
DrcLayerList} is not set or empty, the filtering is not done, and
rules on all layers will be checked.  This variable sets, and is set
by, the {\cb Check listed layers only} and {\cb Skip listed layers}
check boxes in the {\cb DRC Parameter Setup} panel from the {\cb
Setup} button in the {\cb DRC Menu}.

% 010615
\index{DrcRuleList variable}
\item{\et DrcRuleList}\\
{\bf Value:} string.\\
It is possible to use only certain rules, or to skip certain rules,
when running DRC.  This variable contains a space separated list of
rule names (technology file rule keywords) for use in this filtering. 
It mirrors the contents of the {\cb Rule List} text entry area in the
{\cb DRC Parameter Setup} panel from the {\cb Setup} button in the
{\cb DRC Menu}.  Rule name matching is case-insensitive.

% 010615
\index{DrcUseRuleList variable}
\item{\et DrcUseRuleList}\\
{\bf Value:} boolean or string.\\
If this variable is set to a word that starts with `{\vt n}' (case
insensitive) the rules listed in the {\et DrcRuleList} variable will
be skipped during DRC runs.  If {\et DrcUseRuleList} is set to
anything else, including to an empty string (i.e., as a boolean), then
only rules listed in the {\et DrcRuleList} variable will be checked
during DRC runs.  In this case, if the {\et DrcRuleList} is not set or
empty, the filtering is not done, and all rules will be checked.  This
variable sets, and is set by, the {\cb Check listed rules only} and
{\cb Skip listed rules} check boxes in the {\cb DRC Parameter Setup}
panel from the {\cb Setup} button in the {\cb DRC Menu}.

% 010615
\index{DrcPartitionSize variable}
\item{\et DrcPartitionSize}\\
{\bf Value:} real number.\\
When this variable is set to a real number larger than {\vt 0.0},
batch mode DRC initiated from the {\cb DRC Run Control} panel will use
a square grid of the indicated size in microns.  The DRC tests will be
performed sequentially in each of the grid areas that overlap the
overall test area.  This variable mirrors the state of the {\cb
Partition grisd size} entry area and {\cb None} button in the {\cb DRC
Run Control} panel.
\end{description}


% -----------------------------------------------------------------------------
% !set:extech 021515
\section{Extraction Tech}
\index{variables!extraction tech}
\label{extech}

These are mostly in support of the extraction system, but the
variables and keywords are handled by the main program, so can be set
or read if the extraction system is not available.

\begin{description}
% 021515
\index{AntennaTotal variable}
\item{\et AntennaTotal}\\
{\bf Value:} real number.\\
This variable applies to the {\cb !antenna} command.  The value is a
threshold total-net antenna ratio, as explained for the {\cb !antenna}
command.  The value is effectively passed to that command as a
default.

The {\cb Global Attributes} button in the {\cb Tech Parameter Editor}
provides a prompt-line interface for setting this variable.

% 102114
\index{Db3ZoidLimit variable}
\item{\et Db3ZoidLimit}\\
{\bf Value:} integer 1000 or larger.\\
This limits the amount of geometry which can be saved in the 3-D
geometry database, which is used in the {\cb Cross Section} command,
and in the interfaces to external capacitance and inductance
extraction programs.  The total trapezoid element count is limited to
10000 by default, i.e., when this variable is not set.  The database
is not designed for large collections, and the limit avoids embarking
on long computations where the program becomes unresponsive.

% 021515
\index{LayerReorderMode variable}
\item{\et LayerReorderMode}\\
{\bf Value:} integer 0--2.\\
This sets the default sequencing assumption used in the
three-dimensional layer sequence generator (see \ref{ldb3d}), which is
used for the cross-section display and the capacitance extraction
interface.  This can be set to an integer in the range 0--2.  The
value 0 is the default, the same as if the variable is not set.  The
other values will internally resequence {\vt Via} layers, as described
for the layer sequence generator.

The {\cb Global Attributes} button in the {\cb Tech Parameter Editor}
provides a prompt-line interface for setting this variable.

% 021515
\index{NoPlanarize variable}
\item{\et NoPlanarize}\\
{\bf Value:} boolean.\\
If set, by default no layers are planarizing, as explained in the
description of the three-dimensional layer geometry database in
\ref{ldb3d}.  Otherwise, the default is that layers with the {\vt
Conductor} keyword given, explicitly or implicitly, or the {\vt Via}
keyword given, will be planarizing by default.  The {\vt Routing},
{\vt GroundPlane}, {\vt GroundPlaneClear}, {\vt Contact} and their
aliases implicitly set the {\vt Conductor} keyword.  Thus, by default
the metal stack is planarized, as in a contemporary semiconductor
process.

The {\cb Global Attributes} button in the {\cb Tech Parameter Editor}
provides a prompt-line interface for setting this variable.

% 021515
\index{SubstrateEps variable}
\item{\et SubstrateEps}\\
{\bf Value:} real number.\\
This variable sets the relative dielectric constant assumed for the
substrate, used by the capacitance extraction interface.  If not set,
the default is 11.9.

The {\cb Global Attributes} button in the {\cb Tech Parameter Editor}
provides a prompt-line interface for setting this variable.

% 021515
\index{SubstrateThickness variable}
\item{\et SubstrateThickness}\\
{\bf Value:} real number.\\
This variable sets the thickness of the substrate assumed by the
program, as a real number in microns.  This is used only by the
capacitance extraction interface.  If not set, a thickness of 75.0
microns will be assumed.

The {\cb Global Attributes} button in the {\cb Tech Parameter Editor}
provides a prompt-line interface for setting this variable.
\end{description}


% -----------------------------------------------------------------------------
% !set:exgen 070214
\section{Extraction General}
\index{variables!extraction general}
The following variables control features of the general extraction and
association process.

\begin{description}
% 103113
\index{ExtractOpaque variable}
\item{\et ExtractOpaque}\\
{\bf Value:} boolean.\\
When set, {\Xic} will ignore the {\vt OPAQUE} flag and perform
extraction normally on cells with this flag set.  The {\vt OPAQUE}
flag would otherwise suppress extraction on the contents of the cell. 
This flag is set in the {\et flags} property of physical cells.

This tracks the setting of the {\cb Extract opaque cells, ignore
OPAQUE flag} check box in the {\cb Net and Cell Config} page of the
{\cb Extraction Setup} panel from the {\cb Setup} button in the {\cb
Extract Menu}.

% 110413
\index{FlattenPrefix variable}
\item{\et FlattenPrefix}\\
{\bf Value:} string.\\
This variable can be set to a string containing a space-separated list
of words.  The words are intended to match cell names or classes of
cell names.  Cells with names that match are {\bf not} treated as
individual cells during extraction, instead they are treated as if
instantiations are part of the containing cell, i.e., they are
logically flattened (see \ref{exthier}).  This applies to physical
cells only, and such cells will have no recognized electrical
counterpart.

{\bf Note:} it is probably more convenient to set the {\et flatten}
property of physical cells that should be flattened into their parent
during extraction.  Setting this property with the {\cb Cell Property
Editor} will have the same effect as including the cell in the
{\et FlattenPrefix} list, but is persistent when the cell is saved.

In the words, the forward slash character (`/') is special, and is
used to indicate the type of matching.  The possibilities are:

\begin{description}
\item{{\it name\/}[/]}\\
This will prefix match cell names, the trailing `/' is optional.  For
example if {\it name} is ``{\vt abc}'', cell names {\vt abc}, {\vt
abc123}, and {\vt abcounter} would match.

\item{/{\it name}}\\
This will suffix match cell names.  For example, if the word is ``{\vt
/bar}'', cell names {\vt bar}, {\vt foobar}, and {\vt crossbar} would
match.

\item{/{\it name\/}/}\\
This will literally match a cell name, for example {\vt /foobar/}
would match only a cell named {\vt foobar}.
\end{description}

This tracks the setting of the {\cb Cell flattening name keys} entry
in the {\cb Net and Cell Config} page of the {\cb Extraction Setup}
panel, which is obtained from the {\cb Setup} button in the {\cb
Extract Menu}.

{\bf Note:}  in {\Xic} releases prior to 3.1.8, this variable could be set
to a single word only, and prefix matching was always employed.  In
releases of {\Xic} prior to 2.5.19, this variable was named
``PnetFlattenPrefix''.

% 103113
\index{GlobalExclude variable}
\item{\et GlobalExclude}\\
{\bf Value:} string (layer expression).\\
This variable can be set to a layer expression (which includes the
case of a layer name).  Any object in the layout which touches a
region where the layer expression evaluates as dark will be ignored by
the extraction system.  This facilitates use of special layers to mask
off parts of a layout to be ignored in extraction.

This tracks the setting of the {\cb Global exclude layer expression}
entry in the {\cb Misc Config} page of the {\cb Extraction Setup}
panel, which is obtained from the {\cb Setup} button in the {\cb
Extract Menu}.

% 103113
\index{GroundPlaneGlobal variable}
\item{\et GroundPlaneGlobal}\\
{\bf Value:} boolean.\\
When set, every object in every cell on a clear-field ground plane
layer is assigned to group 0.  If not set, only the largest area group
on this layer, in the top-level cell, is assigned to group 0.

This tracks the setting of the {\cb Assume clear-field ground plane is
global} check box in the {\cb Net and Cell Config} page of the {\cb
Extraction Setup} panel from the {\cb Setup} button in the {\cb
Extract Menu}.

% 103113
\index{GroundPlaneMulti variable}
\item{\et GroundPlaneMulti}\\
{\bf Value:} boolean.\\
When set, a layer specified as {\et GroundPlaneClear} in the
technology file will be inverted, and the inverted version used for
grouping and extraction.  The {\et MultiNet} keyword which optionally
follows {\et GroundPlaneClear} in the technology file effectively sets
this variable.  If this variable is unset, then no inversion takes
place, and the absence of the {\et GroundPlaneClear} layer is taken to
indicate ground (group 0).  This variable has no effect unless a {\et
GroundPlaneClear} layer exists.\\ {\kb Note}:  This replaces the {\vt
HandleTermDefault} variable which existed in earlier {\Xic} releases. 
It is part of the ground plane support in the extraction system.

This tracks the setting of the {\cb Invert dark-field ground plane for
multi-nets} check box in the {\cb Net and Cell Config} page of the
{\cb Extraction Setup} panel from the {\cb Setup} button in the {\cb
Extract Menu}.

% 103113
\index{GroundPlaneMethod variable}
\item{\et GroundPlaneMethod}\\
{\bf Value:} integer 0--2.\\
This sets the method used to invert the ground plane for grouping and
extraction, if the {\et MultiNet} keyword has been applied to a {\et
GroundPlaneClear} layer in the technology file.  The possible values
are integers 0--2, which have the same meaning as the integer that
optionally follows {\et MultiNet} in the technology file (see
\ref{exkwords}).

This tracks the setting of the inversion method menu in the {\cb Net
and Cell Config} page of the {\cb Extraction Setup} panel from the
{\cb Setup} button in the {\cb Extract Menu}.

% 103113
\index{KeepShortedDevs variable}
\item{\et KeepShortedDevs}\\
{\bf Value:} boolean.\\
By default, if an extracted device is found to have all terminals
shorted together at the time the device is recognized, the device will
be ignored.  This will help reject spurious devices from test
structures, etc.

If the {\et KeepShortedDevs} variable is set, then these devices will
be kept (as in pre-2.5.69 releases).  This flag may be needed for LVS
to pass, if the schematic contains the shorted devices.

This tracks the setting of the {\cb Include devices with terminals
shorted} check box in the {\cb Device Config} page of the {\cb
Extraction Setup} panel, which is obtained from the {\cb Setup} button
in the {\cb Extract Menu}.

% 103113
\index{MaxAssocLoops variable}
\item{\et MaxAssocLoops}\\
{\bf Value:} integer 0--1000000.\\
This variable sets a parameter used by the association algorithm. 
Presently, it is not expected to be useful to the user, and it is
recommended that it not be changed.
 
The variable tracks the setting of the {\cb Maximum association loop
count} entry in the {\cb Misc Config} page of the {\cb Extraction
Setup} panel from the {\cb Setup} button in the {\cb Extraction Menu}.

% 103113
\index{MaxAssocIters variable}
\item{\et MaxAssocIters}\\
{\bf Value:} integer 10--1000000.\\
This variable sets a parameter used by the association algorithm. 
Presently, it is not expected to be useful to the user, and it is
recommended that it not be changed.
 
The variable tracks the setting of the {\cb Maximum association
iterations} entry in the {\cb Misc Config} page of the {\cb Extraction
Setup} panel from the {\cb Setup} button in the {\cb Extraction Menu}.

% 103113
\index{NoMeasure variable}
\item{\et NoMeasure}\\
{\bf Value:} boolean.\\
This turns off the extraction of parametric data for devices in the
extraction system.  This is mainly for debugging, but may save time if
the user is interested in topology only.  The measurements can be time
consuming.

This tracks the setting of the {\cb Skip device parameter measurement}
check box in the {\cb Device Config} page of the {\cb Extraction
Setup} panel from the {\cb Setup} button in the {\cb Extract Menu}.

% 110413
\index{UseMeasurePrpty variable}
\item{\et UseMeasurePrpty}\\
{\bf Value:} boolean.\\
When set, the extraction system will read and update (creating if
necessary) the {\et measures} property (property number 7106) which is
used to cache (see \ref{meascache}) measurement results.  The
measurement of device parameters can be time consuming, and the
caching can speed up the extraction process significantly.  However,
using the measurement cache may require user intervention to maintain
coherency.  If a device layout changes, the user will have to manually
update the cache in order to obtain updated parameters.  With this
variable unset, the default condition will force actual computation of
device parameters, and avoid all use of the caching mechanism.  This
is appropriate while a cell is under development, to avoid cache
coherency issues. 
 
This variable tracks the {\cb Use measurement results cache
property} check box in the {\cb Device Config} page of the {\cb
Extraction Setup} panel from the {\cb Setup} button in the {\cb
Extract Menu}.

% 110413
\index{NoReadMeasurePrpty variable}
\item{\et NoReadMeasurePrpty}\\
{\bf Value:} boolean.\\
This variable is ignored unless {\et UseMeasurePrpty} is set.  When
set, the extraction system will not read the {\et measures} property
(property number 7106) which is used to cache (see \ref{meascache})
measurement results.  When measurement results are required, they will
be computed.  The property will still be updated, after association,
if {\et UseMeasurePrpty} is set.  Thus, by setting this variable and
forcing association, one can get a fresh set of measurement results
into the {\et measures} properties.

This variable tracks the {\cb Don't read measurement results from
property} check box in the {\cb Device Config} page of the {\cb
Extraction Setup} panel from the {\cb Setup} button in the {\cb
Extract Menu}.

% 103113
\index{NoMergeParallel variable}
\item{\et NoMergeParallel}\\
{\bf Value:} boolean.\\
Setting this variable suppresses merging of parallel-connected devices
during extraction.  This applies to all devices, and supersedes the
{\et Merge} directive in the device blocks or the technology file.

This variable tracks the setting of the {\cb Don't merge parallel
devices} check box in the {\cb Device Config} page of the {\cb
Extraction Setup} panel, which is obtained from the {\cb Setup} button
in the {\cb Extract Menu}.

% 103113
\index{NoMergeSeries variable}
\item{\et NoMergeSeries}\\
{\bf Value:} boolean.\\
Setting this variable suppresses merging of series-connected devices
during extraction.  This applies to all devices, and supersedes the
{\et Merge} directive in the device blocks of the technology file. 

This variable tracks the setting of the {\cb Don't merge series
devices} check box in the {\cb Device Config} page of the {\cb
Extraction Setup} panel, which is obtained from the {\cb Setup} button
in the {\cb Extract Menu}.

% 103113
\index{NoMergeShorted variable}
\item{\et NoMergeShorted}\\
{\bf Value:} boolean.\\
When including devices with all terminals shorted (the {\et
KeepShortedDevs} variable is set), setting this variable will prevent
such devices from being merged as parallel devices, if parallel
merging is enabled for the device type.

This variable tracks the setting of the {\cb Don't merge devices with
terminals shorted} check box in the {\cb Device Config} page of the
{\cb Extraction Setup} panel, which is obtained from the {\cb Setup}
button in the {\cb Extract Menu}.

% 103113
\index{IgnoreNetLabels variable}
\item{\et IgnoreNetLabels}\\
{\bf Value:} boolean.\\
If set, net name labels will be ignored by the extraction system. 
This is probably only useful for debugging.  Although this may allow
correct association if a net name label is wrong, the recommended
solution is to correct the offending label.

This variable tracks the setting of the {\cb Ignore net name labels}
check box in the {\cb Net and Cell Config} page of the {\cb Extraction
Setup} panel, which is obtained from the {\cb Setup} button in the
{\cb Extract Menu}.

% 103113 
\index{UpdateNetLabels variable}
\item{\et UpdateNetLabels}\\
{\bf Value:} boolean.\\
When set, net name labels will be updated, and new net name labels
possibly created, after association completes.  The label text is
obtained from corresponding electrical net names.

This is a dangerous operating mode, as if association fails, it is
possible that incorrect net name labels will be created.  These will
subsequently prevent correct association and cause LVS failure, until
removed or corrected by hand.

When creating library cells, running extraction with this variable set
can be a final action before saving the finished cell.  This must only
be done if the cell passes LVS.  The created net name labels should
improve association efficiency, but are not essential.

This variable tracks the state of the {\cb Update net name labels
after association} check box in the {\cb Net and Cell Config} page of
the {\cb Extraction Setup} panel, which is obtained from the {\cb
Setup} button in the {\cb Extract Menu}.

% 112513 
\index{FindOldTermLabels variable}
\item{\et FindOldTermLabels}\\
{\bf Value:} boolean.\\
When this variable is defined, {\Xic} will recognize the ``term
labels'' of earlier releases as net labels.  In {\Xic}-3, term labels
were used (optionally) to specify the conductor groups that were
associated with cell terminals in layouts.  These are labels, created
by the user on conducting layers, placed over an object on the same
layer.

The term labels would also be recognized as net labels if the {\et
PinPurpose} variable is set to an empty string, or the ``{\vt
drawing}'' purpose name.  Setting the {\et FindOldTermLabels} is
redundant in that case.  The label searches are separate, and both
will be done if enabled.

Whether this variable is set or not mirrors the status of the {\cb
Find old-style net (term name) labels} check box in the {\cb Net and
Cell Config} page of the {\cb Extraction Setup} panel from the {\cb
Extract Menu}.

% 110213
\index{MergeMatchingNamed variable}
\item{\et MergeMatchingNamed}\\
{\bf Value:} boolean.\\
If two physically unconnected conductor groups have the same logical
net name (see \ref{netname}), if this variable is set the groups will
be logically merged and treated as a single group.  This allows
successful top-level LVS of cells containing split nets.  Below the
top level, split nets are detected by other means so setting this
variable is not required for successful LVS if the top-level cell
contains no split nets.

The group names that apply are obtained from net name labels, or from
cell terminals that have been placed by the user.  By default, net
name matching is case-insensitive, though this can be changed with the
{\et NetNamesCaseSens} variable.  The name matching also treats as
equivalent various subscripting delimiters, as listed in the
description of the {\et Subscripting} variable.

This variable tracks the state of the {\cb Merge groups with matching
net names} check box in the {\cb Net and Cell Config} page of the {\cb
Extraction Setup} panel, which is obtained from the {\cb Setup} button
in the {\cb Extract Menu}.

% 103113
\index{MergePhysContacts variable}
\item{\et MergePhysContacts}\\
{\bf Value:} boolean.\\
When set, additional association logic is employed to detect and
account for split nets in instance placements.  A ``split net'' is a
logical net consisting of two or more disjoint physical conductor
groups.  The disjoint parts of the net are connected when instances
are placed, through parent cell metalization.  If the schematic shows
the net fully connected in the master, LVS will fail on the parent
unless this variables is set.

This variable tracks the state of the {\cb Logically merge physical
contacts for split net handling} check box in the {\cb Misc Config}
page of the {\cb Extraction Setup} panel, which is obtained from the
{\cb Setup} button in the {\cb Extract Menu}.

% 110613
\index{NoPermute variable}
\item{\et NoPermute}\\
{\bf Value:} boolean.\\
When this variable is set, the association algorithm will not attempt
to use symmetry trials to find a solution.  Symmetry trials are
normally used to iterate through permutations when searching for a
solution.  During a trial, a particular set of associations is
assumed, and the algorithm continues.  If an inconsistency is found
later, the associations made during the trial are reverted, and a new
trial is started.

Many circuits do not require a permutation search.  In some circuits,
though, the permutation search can be a very time-consuming process. 
In circuits where association is known to fail perhaps because the
wiring is incomplete, setting this variable will save time.  This
variable is mostly for debugging, or for cases where association is
not needed.  Of course, if a permutation search is needed and not
performed, LVS will fail.

Permutes are also skipped if a device or subcircuit is found that can
not possibly be associated.

This tracks the setting of the {\cb Don't run symmetry trials in
association} check box in the {\cb Misc Config} page of the {\cb
Extraction Setup} panel, obtained from the {\cb Setup} button in the
{\cb Extract Menu}.

% 061916
\index{PinLayer variable}
\item{\et PinLayer}\\
{\bf Value:} string.\\
If this variable is set to a layer name (or layer-purpose pair name)
all net name labels must appear on the named layer.  The ``{\vt pin}''
purpose, and any setting of the {\et inPurpose} variable, are ignored.

The label will be associated with the conducting object containing the
label origin that is highest (farthest from the substrate) in the
layer table.  Possible ambiguity with the associated layer makes this
scheme not recommended, but support is present for compatibility with
older cell libraries, such as the open-source CMOS libraries from
Oklahoma State University.

This variable tracks the {\cb Net label layer} entry in the
{\cb Net Config} page of the {\cb Extraction Setup} panel, obtained
form the {\cb Setup} button in the {\cb Extract Menu}.

% 061916
\index{PinPurpose variable}
\item{\et PinPurpose}\\
{\bf Value:} string.\\
This applies when the {\et PinLayer} variable is not set.  By default,
net name labels must reside on a layer-purpose pair where the purpose
name is ``{\vt pin}''.  However, if this variable is set to another
valid purpose name, then that name will be required of net labels
instead.
 
If the property is set to an empty string (i.e., as a boolean), the
``{\vt drawing} purpose is assumed.  One could equivalently give the
name explicitly.  This is not really recommended as it can be
inefficient.
 
This variable tracks the {\cb Net label purpose name} entry in the
{\cb Net Config} page of the {\cb Extraction Setup} panel, obtained
form the {\cb Setup} button in the {\cb Extract Menu}.

% 110113
\index{RLSolverDelta variable}
\item{\et RLSolverDelta}\\
{\bf Value:} floating point $>=$ 0.01.\\
It this value is set, the resistance/inductance extractor will assume
this grid spacing, in microns.  The number of grid cells enclosed in
the device will increase for physically larger devices, so that larger
devices will take longer to extract.  If this variable is set, the
other {\et RLSolver} variables are ignored.  Setting this variable may
be appropriate if all resistors are ``small'' and dimensions conform
to a layout grid.

This tracks the setting of the {\cb Set/use fixed grid size} entry in
the {\cb Device Config} page of the {\cb Extraction Setup} panel,
which is obtained from the {\cb Setup} button in the {\cb Extract
Menu}.

% 110113
\index{RLSolverTryTile variable}
\item{\et RLSolverTryTile}\\
{\bf Value:} boolean.\\
If set, the extractor will attempt to use a grid that will fall on
every edge of the device body and contacts.  The device and contact
areas must be Manhattan for this to work.  If such a grid can be
found, and the number of grid cells is a reasonable number, this will
give the most accurate result.

This tracks the setting of the {\cb Try to tile} check box in the {\cb
Device Config} page of the {\cb Extraction Setup} panel, which is
obtained from the {\cb Setup} button in the {\cb Extract Menu}.

% 110113
\index{RLSolverGridPoints variable}
\item{\et RLSolverGridPoints}\\
{\bf Value:} integer 10--100000.\\
When not tiling ({\et RLSolverTryTile} is not set), this sets the
number of grid points used for resistance/inductance extraction.  This
number will be the same for all device structures, so that computation
time per device is nearly constant.  Higher numbers give better
accuracy but take longer.  The value used if not set is 1000.

This tracks the setting of the {\cb Set fixed per-device grid cell
count} entry in the {\cb Device Config} page of the {\cb Extraction
Setup} panel, which is obtained from the {\cb Setup} button in the
{\cb Extract Menu}.

% 110113
\index{RLSolverMaxPoints variable}
\item{\et RLSolverMaxPoints}\\
{\bf Value:} integer 1000--100000.\\
When tiling ({\et RLSolverTryTile} is set), the maximum number of grid
cells is limited to this value.  If the tile is too small, it will be
increased in size to keep the count below this value, in which case
the tiling will not have succeeded so there may be a small loss of
accuracy.  Using a large number of grid points can take a long time. 
The value used if not set is 50,000.

This tracks the setting of the {\cb Maximum tile count per device}
entry in the {\cb Device Config} page in the {\cb Extraction Setup}
panel, which is obtained from the {\cb Setup} button in the {\cb
Extract Menu}.

% 110113
\index{SubcPermutationFix variable}
\item{\et SubcPermutationFix}\\
{\bf Value:} boolean.\\
Setting this variable enables additional association logic.  It
applies when there is perfect topological matching between layout and
schematic, but LVS is failing due to different permutations of
permutable subcell contacts being assumed in the electrical and
physical parts.  Setting the variable will enforce the electrical
permutation on the physical solution, which will allow LVS to pass if
the permutation difference was the only issue.
 
This should no longer be needed, as the two-pass association algorithm
in current use should resolve these cases automatically.  This
variable should therefor not be set in general, but it is possible
that it might allow successful LVS in some obscure case.
 
This variable tracks the {\cb Apply post-association permutation fix}
check box in the {\cb Misc Config} page of the {\cb Extraction Setup}
panel, which is obtained form the {\cb Setup} button in the {\cb
Extract menu}.

% 020213
\index{VerbosePromptline variable}
\item{\et VerbosePromptline}\\
{\bf Value:} boolean.\\
When set, lots of messages will be printed on the prompt line during
extraction.  Otherwise not much is printed, which may speed things up. 
This variable is linked to the {\cb Be very verbose on prompt line
during extraction} check box of the {\cb Misc Config} page of the {\cb
Extraction Setup} panel.

% 091614
\index{ViaCheckBtwnSubs variable}
\item{\et ViaCheckBtwnSubs}\\
{\bf Value:} boolean.\\
By default, it is assumed that connections between subcells will be
made by touching metal only.  This includes the case where the metal
is from a flattened wire-only cell, as would be provided by via cells
as described in \ref{viafind}.  One can easily adapt layout
methodology where this is true.  Otherwise, this variable can be set,
which will cause explicit testing for the presence of vias between
subcircuit nets.  This is a very expensive operation.

Whether this variable is set or not tracks the state of the {\cb Check
for via connections between subcells} check box in the {\cb Net
Config} page of the {\cb Extraction Setup} panel from the {\cb Extract
Menu}.

% 091614
\index{ViaSearchDepth variable}
\item{\et ViaSearchDepth}\\
{\bf Value:} non-negative integer.\\
If we have intersecting areas of top and bottom conductor, and we are
searching for an area of via material that would connect the two metal
objects, this sets the depth in the current cell hierarchy to search
(see \ref{viafind}).  The default is zero, indicating to search the
current cell only.  Generally, layout methodology can easily ensure
that this value can be safely zero, but there may be cases that
require extraction where such methodology was not practiced.  In such
a case, where the methodology is completely unknown, this value should
be set to a large number (internally it is limited to 40, the maximum
cell hierarchy depth) which will ensure that all via-induced
connections are found.  This can dramatically increase extraction
time.

The value of this variable tracks the {\cb Via search depth} entry
area in the {\cb Net Config} page of the {\cb Extraction Setup} panel
from the {\cb Extract Menu}.

% 071820
\index{ViaConvex variable}
\item{\et ViaConvex}\\
{\bf Value:} boolean.\\
This applies when checking for connectivity through a via during
extraction.  When set, all non-rectangular vias are assumed to be
convex polygons.  The test region is taken as a small rectangle
centered on the via bounding box.  This simplifies and should speed
testing.  It is intended specifically for circular vias, as used in
superconductive electronics.  It has no effect on rectangular vias. 
It should {\bf not} be set if any vias are non-convex polygons, as
incorrect results may occur.

Whether or not this variable is set tracks the state of the {\cb
Assume convex vias} check box in the {\cb Net Config} page of the {\cb
Extraction Setup} panel from the {\cb Extract Menu}. 
\end{description}


% -----------------------------------------------------------------------------
% !set:extrc 061408
\section{Extraction Menu Commands}
\index{variables!extraction menu commands}

The {\cb !set} variables below affect the commands found in the
{\cb Extract Menu}.

\begin{description}
% 051809
\index{QpathGroundPlane variable}
\item{\et QpathGroundPlane}\\
{\bf Value:} integer 0--2.\\
This variable controls how the {\cb "Quick" Path} command in the
extraction {\cb Path Selection Control} panel uses the inverted ground
plane.  Normally, during extraction, if the {\et GroundPlaneClear}
keyword has been given, an inverted ground plane is created on a
temporary layer for internal use.  Since the {\cb "Quick" Path} mode
operates outside of the extraction system, the inverted ground plane
may or may not be available.  The choices are:

\begin{description}
\item{\vt 0}\\
Use the inverted ground plane if available.  This is the default.  If
an inverted ground plane has already been created and is current, it
will be used when determining paths.  If the ground plane does not
have a current inversion, the absence of the layer will imply a ground
contact, as in extraction without the {\et MultiNet} keyword.  This
choice avoids the sometimes lengthly inversion computation, but makes
use of the inversion if it has already been done.

\item{\vt 1}\\
Create the inverted ground plane if necessary, and use it.  If the  
extraction system would use an inverted ground plane, it will be  
created if not already present and current.  The path selection  
will include the inverted layer.  

\item{\vt 2}\\
The {\cb "Quick" Path} mode will never use the inverted ground 
plane.
\end{description}

This variable tracks the state of the {\cb "Quick" Path ground plane
handling} menu in the {\cb Path Selection Control} panel.

% 110113
\index{QpathUseConductor variable}
\item{\et QpathUseConductor}\\
{\bf Value:} boolean.\\
By default, when this variable is not set, only objects on layers with
the {\et Routing} attribute applied will be considered for inclusion
in the path extracted with the {\cb "Quick" Path} button in the {\cb
Path Selection Control} panel, which is obtained from the {\cb Net
Selections} button in the {\cb Extract Menu}.  If this variable is
set, objects on layers with the {\et Conductor} attribute will be
allowed.  The {\et Routing} attribute implies {\et Conductor}, but may
be more restrictive.

This variable tracks the state of the {\cb "Quick" Path use Conductor}
check box in the {\cb Path Selection Control} panel.

% 061516
\index{EnetNet variable}
\item{\et EnetNet}\\
{\bf Value:} boolean.\\
If set, the netlist in internal format is incuded when writing output
in the {\cb Dump Elec Netlist} command.  This variable corresponds to
the {\et net} check box available in that command.

% 061408
\index{EnetSpice variable}
\item{\et EnetSpice}\\
{\bf Value:} boolean.\\
If set, SPICE output is included in the file produced from the {\cb
Dump Elec Netlist} command.  This variable corresponds to the {\et
spice} check box available in that command.

% 061408
\index{EnetBottomUp variable}
\item{\et EnetBottomUp}\\
{\bf Value:} boolean.\\
When set, the electrical netlist file (produced by the {\cb Dump Elec
Netlist} command) order will be leaf-to-root, i.e., subcells will be
listed first.  If not set, the reverse order is used.

% 061516
\index{PnetNet variable}
\item{\et PnetNet}\\
{\bf Value:} boolean.\\
If set, the extracted netlist listing in the internal format is
included in output from the {\cb Dump Phys Netlist} command.  This
variable corresponds to the {\et net} check box available in that
command.

% 061516
\index{PnetDevs variable}
\item{\et PnetDevs}\\
{\bf Value:} boolean.\\
If set, the extracted device listing in internal format is included in
output from the {\cb Dump Phys Netlist} command.  This variable
corresponds to the {\et devs} check box available in that command.

% 061516
\index{PnetSpice variable}
\item{\et PnetSpice}\\
{\bf Value:} boolean.\\
If set, the SPICE listing of extracted devices is included in output
from the {\cb Dump Phys Netlist} command.  This variable corresponds
to the {\et spice} check box available in that command.

% 061408
\index{PnetBottomUp variable}
\item{\et PnetBottomUp}\\
{\bf Value:} boolean.\\
When set, the physical netlist file (produced by the {\cb Dump Phys
Netlist} command) order will be leaf-to-root, i.e., subcells will be
listed first.  If not set, the reverse order is used.

% 061408
\index{PnetShowGeometry variable}
\item{\et PnetShowGeometry}\\
{\bf Value:} boolean.\\
If set, the {\et net} field (if activated) in the file produced from
the {\cb Dump Phys Netlist} command will include a listing of the
objects that comprise the wire net.  The listing is in modified CIF
syntax where 1000 units per micron is used.  This variable corresponds
to the {\et show geometry} check box available in that command.

% 061408
\index{PnetIncludeWireCap variable}
\item{\et PnetIncludeWireCap}\\
{\bf Value:} boolean.\\
If set, the {\et spice} field (if activated) in the file produced from
the {\cb Dump Phys Netlist} command will include capacitors
representing the computed wire net capacitance to ground.  The {\et
Routing} layers must have the {\et Capacitance} keyword applied in the
technology file.  The added capacitors have a special prefix ``{\vt
C@NET}'' which allows them to be subsequently recognized as wire net
capacitors by {\Xic}.  This variable corresponds to the {\et include
wire cap} check box available in that command.

% 061408
\index{PnetListAll variable}
\item{\et PnetListAll}\\
{\bf Value:} boolean.\\
In files produced with the {\cb Dump Phys Netlist} command, references
to subcells that are flattened or wire-only are normally not listed. 
If this variable is set, these cells are included in the listing,
which may be useful for debugging.  This variable corresponds to the
{\et include all devs} check box available in that command.

% 121113
\index{PnetNoLabels variable}
\item{\et PnetNoLabels}\\
{\bf Value:} boolean.\\
When set, output from the {\cb Dump Phys Netlist} command will use
group numbers to designate non-global nets.  When not set, output will
use group names as provided by net name labels (see \ref{netname})
where found.  This variable mirrors the state of the {\cb ignore
labels} check box in the {\cb Dump Phys Netlist} panel.

% 061516
\index{PnetVerbose variable}
\item{\et PnetVerbose}\\
{\bf Value:} boolean.\\
This boolean variable is intended to enable additional information
when printing output from the {\cb Dump Phys Netlist} command. 
Presently, it only applies when printing the device table ({\et
PnetDevs} is set).  It will print additional information about
multi-component (merged) devices.  This variable mirrors the state of
the {\cb devs verbose} check box in the {\cb Dump Phys Netlist} panel,
available from the button in the {\cb Extract Menu}.

% 061408
\index{SourceAllDevs variable}
\item{\et SourceAllDevs}\\
{\bf Value:} boolean.\\
In the {\cb Source SPICE} command, ordinarily only devices which have
fixed (user-specified) device names will have properties updated. 
This is to avoid errors, since the internally generated names can
change, and may not match those in the SPICE file.  If this variable
is set, the default action is to update all devices.  This variable
corresponds to the {\et all devs} check box available in that command. 

% 061408
\index{SourceCreate variable}
\item{\et SourceCreate}\\
{\bf Value:} boolean.\\
In the {\cb Source SPICE} command, if this variable is set, the
default action is to create missing devices.  Otherwise, device
parameters may be updated, but no new devices are created.  This
variable corresponds to the {\et create} check box available in that
command.

% 061408
\index{SourceClear variable}
\item{\et SourceClear}\\
{\bf Value:} boolean.\\
In the {\cb Source SPICE} command, if this variable is set the default
action is to discard the existing contents of the electrical part of
the cell before updating.  This variable corresponds to the {\et
clear} check box available in that command.

% 042611
\index{SourceGndDevName variable}
\item{\et SourceGndDevName}\\
{\bf Value:} string.\\
This variable specifies the name of the ground terminal device to use
when devices are created and placed in the {\cb Source SPICE} and
(consequently) the {\cb Source Physical} extraction commands.  If not
set, the name ``{\vt gnd}'' will be assumed.  If this variable is set
to a name, a ground device of that name must appear in the device
library file.

% 042611
\index{SourceTermDevName variable}
\item{\et SourceTermDevName}\\
{\bf Value:} string.\\
This variable specifies the name of the terminal device to use when
devices are created and placed in the {\cb Source SPICE} and
(consequently) the {\cb Source Physical} extraction commands.  If not
set, the name ``{\vt tbar}'' will be assumed, if that name is found
for a terminal device in the device library.  If not found, the name
``{\vt vcc}'' will be assumed.  If this variable is set to a name,
that name must match the name of a terminal device in the device
library file.

% 061408
\index{NoExsetAllDevs variable}
\item{\et NoExsetAllDevs}\\
{\bf Value:} boolean.\\
In the {\cb Source Physical} command, if this variable is set, only
devices that have a permanent (user-supplied) name will be updated. 
If not set, all devices will be updated.  This variable corresponds to
the {\et all devs} check box available in that command, with inverse
logic.

% 061408
\index{NoExsetCreate variable}
\item{\et NoExsetCreate}\\
{\bf Value:} boolean.\\
The default behavior of the {\cb Source Physical} command is to create
missing devices.  Setting this variable will change the default action
to no device creation.  This variable corresponds to the {\et create}
check box available in that command, with inverse logic.

% 061408
\index{ExsetClear variable}
\item{\et ExsetClear}\\
{\bf Value:} boolean.\\
When set, the electrical cells are cleared before updating with the
{\cb Source Physical} command.  This implies {\et create}, i.e., new
devices will be created since the cell is empty.  This variable
corresponds to the {\et clear} check box available in that command. 

% 061408
\index{ExsetIncludeWireCap variable}
\item{\et ExsetIncludeWireCap}\\
{\bf Value:} boolean.\\
When set, computed routing capacitors will be updated or created in
the electrical database when using the {\cb Source Physical} command. 
These capacitors have a name prefix of ``{\vt C@NET}''.  This variable
corresponds to the {\et include wire cap} check box available in that
command.

% 110313
\index{ExsetNoLabels variable}
\item{\et ExsetNoLabels}\\
{\bf Value:} boolean.\\
When set, output from the {\cb Source Physical} command will use group
numbers to designate non-global nets.  When not set, output will use
group names as provided by net name labels (see \ref{netname}) where
found.

% 091009
\index{LvsFailNoConnect variable}
\item{\et LvsFailNoConnect}\\
{\bf Value:} boolean.\\
During LVS analysis, the electrical (schematic) part of the design is
used as the basis for recursion through the hierarchy.  Thus, physical
subcells that have no connection to the circuit will not be detected,
and are basically ignored.  However, an explicit test is performed for
such cells, and those found will be listed in the LVS report.  If this
variable is set, the presence of such cells will force LVS failure,
otherwise they are ignored for comparison purposes.

This variable tracks the state of the {\cb fail if unconnected
physical subcells} check box in the panel brought up by the {\cb Dump
LVS} button in the {\cb Extract Menu}.

% 070909
\index{PathFileVias variable}
\item{\et PathFileVias}\\
{\bf Value:} boolean or string.\\
This variable determines whether and how vias are included in the
files produced with the {\cb Save path to file} button in the {\cb
Path Selection Control} panel from the {\cb Net Selections} button in
the {\cb Extract Menu}.  It tracks (and sets) the state of the {\cb
Path file contains vias} and {\cb Path file contains check layers}
check boxes in the panel.

If not set, via layers will not be included in the file, only the
conductors will appear.  If set as a boolean (i.e., to no value), the
via layers will be included, but not the check layers.  If set to any
text, the check layers will also be included.
\end{description}

% -----------------------------------------------------------------------------
% !set:fc 071814
\section{Capacitance Extraction Interface}
\label{fcvars}
\index{variables!capacitance etraction}
The following variables apply to the capacitance extraction interface
described in \ref{fcinterf}.  Most of these are associated with entry
fields in the {\cb Cap Extraction} panel (see\ref{fcpanel}), which is
brought up with the {\cb Extract C} button in the {\cb Extract Menu}.

\begin{description}
% 071814
\index{FcArgs variable}
\item{\et FcArgs}\\
{\bf Value:} string.\\
This variable can be set to a string, which will be included in the
argument list when capacitance extraction is initiated through the
interface, with the {\cb Run Extraction} button in the {\cb Run} page
of the {\cb Cap Extraction} panel, or through the {\cb !fc} command. 
The variable tracks the {\cb FcArgs} text entry area in the {\cb Run}
page of the {\cb Cap Extraction} panel, from where the variable is
most conveniently set or edited.

If the interface detects that {\it FasterCap} from {\vt
FastFieldSolvers.com} is being used, and this entry is empty, the
default argument string
\begin{quote} \vt
-b -a0.01
\end{quote}
will be imposed.  A ``{\vt -b}'' option will always be added if
missing from the {\it FasterCap} arguments list, as this argument is
necessary for correct {\it FasterFap} operation in this mode.  The
``{\vt -a}'' option is almost always used, as it specifies
auto-refinement, however it is technically not necessary and won't be
imposed if not given, except in the case where no arguments are given
at all.

% 071814
\index{FcForeg variable}
\item{\et FcForeg}\\
{\bf Value:} boolean.\\
If this variable is set, then the {\cb Run Extraction} button in the
{\cb Cap Extraction} panel {\cb Run} page will initiate a process
running in the foreground.  If not set, jobs are run in the
background, so that the user can continue using {\Xic} while the run
is in progress.

It is not clear why there would be any reason to run in the
foreground, except possibly for debugging.

This variable controls, and is controlled by, the setting of the {\cb
Run in foreground} check box in the {\cb Run} page of the {\cb Cap
Extraction} panel from the {\cb Extract Menu}.

% 071814
\index{FcLayerName variable}
\item{\et FcLayerName}\\
{\bf Value:} string.\\
The capacitance extraction interface uses a special layer for masking
of objects to be included in the capacitance extraction run.  By
default, this layer is named ``{\vt FCAP}''.  If any shapes exist on
this layer in the current cell hierarchy, all objects will be clipped
by these shapes before capacitance extraction.  If no shapes are found
on this layer, then all objects in the current cell hierarchy will be
included in capacitance extraction.

If this variable is set to the name of an existing layer name in the
layer table, that layer will do the clipping.

% 071814
\index{FcMonitor variable}
\item{\et FcMonitor}\\
{\bf Value:} boolean.\\
If this variable is set, then the standard output from the running
capacitance extraction program is printed in the console, in addition
to being saved in a file.  The console is the shell window from which
{\Xic} was started.  This allows the user to monitor the run, and
abort if something isn't correct.

This will also apply if the program is being run in the foreground,
however operation is a bit different.  In this case, a ``{\vt | tee}''
is added to the command string ahead of the output file name.  There
are two implications:  the text will be block buffered, and therefor
won't appear in the window immediately, and in Windows, there is no
native {\vt tee} command so that the operation may fail.  However, a
{\vt tee} command is provided with the Cygwin tools, and there are
other sources.  In the normal case of running in the background,
output will again be block buffered under Windows, but there is no
requirement for a {\vt tee} command.

This variable mirrors the state of the {\cb Out to console} check box
in the {\cb Run} page of the {\cb Cap Extraction} panel from the {\cb
Extract Menu}.

!! 071814
\index{FcPanelTarget variable}
\item{\et FcPanelTarget}\\
{\bf Value:} real number 1e3 -- 1e6.\\
When {\bf not} using a capacitance extraction program that provides
automatic refinement, such as {\it FasterCap} from {\vt
FastFieldSolvers.com}, this provides a crude panel refinement
capability.  This variable provides a number, and the interface will
attempt to split all panels into equal area pieces, where the total
number of pieces is the number given.  The refined panels are output
into the list file, which consequently can grow large.

When not set, no such refinement is done.  It should not be set for
normal use of {\it FasterCap}, but is needed if using the Whiteley
Research version of {\it FastCap} or similar.

!! 071814
\index{FcPath variable}
\item{\et FcPath}\\
{\bf Value:} directory path string.\\
This variable can be set to a full path to the capacitance extraction
program executable.

If this is not set, {\Xic} will attempt to use ``{\vt
/usr/local/bin/fastcap}'' as the {\it FastCap} program (or ``{\vt
/usr/local/bin/fastcap.exe}'' in Windows).  If this executable does
not exist, {\Xic} will attempt to find ``{\vt fastcap}'' (or ``{\vt
fastcap.exe}'' in Windows) in the shell search path when running in
the foreground, and background runs will fail.

This tracks the setting of the text entry field in the {\cb Run}
page of the {\cb Cap Extraction} panel.

% 071814
\index{FcPlaneBloat variable}
\item{\et FcPlaneBloat}\\
{\bf Value:} real number 0.0 -- 100.0.\\
If set to a positive value, the substrate is modeled to extend
horizontally outward by this value beyond the bounding box of the
extracted geometry.  See the discussion in the interface description
in \ref{subsaoi} for more information.  If not set, no dimensional
change is assumed.

% 071814
\index{FcUnits variable}
\item{\et FcUnits}\\
{\bf Value:} units string.\\
This variable can be used to specify the length units used in
generated capacitance extraction input files.  The variable can be set
to a string consisting of one or the abbreviations ``{\vt m}''
(meters), ``{\vt cm}'' (centimeters), ``{\vt mm}'' (millimeters),
``{\vt um}''" (microns), ``{\vt in}'' (inches), and ``{\vt mils}''. 
The long form word will also be accepted.  This variable is most
conveniently manipulated with the choice menu found in the {\cb Cap
Extraction} panel {\cb Params} page.
\end{description}


% -----------------------------------------------------------------------------
% !set:fh 111819
\section{Inductance/Resistance Extraction Interface}
\label{fhvars}
\index{variables!fasthenry}
The following variables apply to the inductance/resistance extraction
({\it FastHenry} interface).  Most of these are associated with entry
fields in the {\cb LR Extraction} panel, which is brought up
with the {\cb Extract LR} button in the {\cb Extract Menu}.

\begin{description}
% 111819
\index{FhArgs variable}
\item{\et FhArgs}\\
{\bf Value:} string.\\
This value can be set to a string, which will be included in the
argument list when {\it FastHenry} is initiated with the {\cb Run
FastHenry} button in the {\cb LR Extraction} panel {\cb Run}
page.  The variable is most conveniently manipulated with the text
entry field in the {\cb LR Extraction} panel {\cb Run} page.

% 071820
\index{FhDefaults variable}
\item{\et FhDefaults}\\
{\bf Value:} string.\\
If set to a string, the value will be used in a {\vt .DEFAULT} line in
the {\it FastHenry} input file created by the interface.  The variable
is most conveniently manipulated with the text entry field in the {\cb
LR Extraction} panel {\cb Run} page.  See the {\it FastHenry}
documentation describing the syntax and options for the applicable
text.

% 011621
\index{FhDefNhinc variable}
\item{\et FhDefNhinc}\\
{\bf Value:} integer 1 -- 20.\\
Provide a default value for the {\vt nhinc} parameter as used by {\it
FastHenry\/}.  This is overridden by values specified with the {\vt
FH\_nhinc} technology keyword for layers, unless the {\et FhOverride}
variable is set, in which case the this variable has precedence.  This
tracks the {\cb FhDefNhinc} entry in the {\cb Params} page of the {\cb
LR Extraction} panel.

% 011621
\index{FhDefRh variable}
\item{\et FhDefRh}\\
{\bf Value:} real 0.5 -- 4.0.\\
Provide a default value for the {\vt rh} parameter as used by {\it
FastHenry\/}.  This is overridden by values specified with the {\vt
FH\_rh} technology keyword for layers, unless the {\et FhOverride}
variable is set, in which case the this variable has precedence.  This
tracks the {\cb FhDefRh} entry in the {\cb Params} page of the {\cb LR
Extraction} panel.

% 111819
\index{FhForeg variable}
\item{\et FhForeg}\\
{\bf Value:} boolean.\\
If this variable is set, then the {\cb Run FastHenry} button in the
{\cb LR Extraction} panel {\cb Run} page will initiate a {\it
FastHenry} run in the foreground.  If not set, jobs are run in the
background, so that the user can continue using {\Xic} while the run
is in progress.

It is not clear why there would be any reason to run in the
foreground, except possibly for debugging.

This variable controls, and is controlled by, the setting of the {\cb
Run in foreground} check box in the {\cb Run} page of the {\cb LR
Extraction} panel from the {\cb Extract Menu}.

% 111819
\index{FhFreq variable}
\item{\et FhFreq}\\
{\bf Value:} string.\\
This variable can be used to specify the evaluation frequencies used
for {\it FastHenry}, as included in a generated input file, or when
initiating a run.  The format is the same as is used in the {\it
FastHenry} input format:
\begin{quote}
{\vt fmin=}{\it start\_freq} {\vt fmax=}{\it stop\_freq}
     [{\vt ndec=}{\it num\/}]
\end{quote}
The frequencies are floating point numbers given in hertz, and the
{\vt ndec} parameter, if given, specifies the number of intermediate
frequencies to evaluate.  If the third field is not set, evaluation is
at the start and stop frequencies only, or at the single frequency if
both are the same.  If the variable is not set, the evaluation is at a
single frequency of one kilohertz.  This variable is most conveniently
manipulated with the text entry fields in the {\cb LR Extraction}
panel {\cb Run} page.

% 090414
\index{FhLayerName variable}
\item{\et FhLayerName}\\
{\bf Value:} string.\\
The inductance/resistance extraction interface uses a special layer
for masking of objects to be included in the extraction run.  By
default, this layer is named ``{\vt FHRY}''.  If any shapes exist on
this layer in the current cell hierarchy, all objects will be clipped
by these shapes before inductance/resistance extraction.  If no shapes
are found on this layer, then all objects in the current cell
hierarchy will be included in extraction.

If this variable is set to the name of an existing layer name in the
layer table, that layer will do the clipping.

% 111819
\index{FhManhGridCnt variable}
\item{\et FhManhGridCnt}\\
{\bf Value:} real number 1e2--1e5.\\
When a non-Manhattan polygon is ``Manhattanized'' for {\it FastHenry},
it is converted to an approximating Manhattan polygon.  This variable
can be used to set the minimum rectangle width and height used in the
decomposition.  This value is given by
\begin{quote}
{\vt sqrt(}{\it area\_of\_interest\/}{\vt /FhManhGridCnt)}
\end{quote}
If not set, a value of 1000 is used.  Larger values are more accurate
but slow processing, sometimes dramatically.  The {\it
area\_of\_interest} is the layout area being processed for input to
{\it FastHenry}.

This variable is most conveniently manipulated with the text input
field in the {\cb LR Extraction} panel {\cb Params} page.

% 111819
\index{FhMonitor variable}
\item{\et FhMonitor}\\
{\bf Value:} boolean.\\
If the variable is set, then the standard output from the running {\it
FastHenry} program is printed in the console, in addition to being
saved in a file.  The console is the shell window from which {\Xic}
was started.  This allows the user to monitor the run, and abort if
something isn't correct.

This will also apply if the program is being run in the foreground,
however operation is a bit different.  In this case, a ``{\vt | tee}''
is added to the command string ahead of the output file name.  There
are two implications:  the text will be block buffered, and therefor
won't appear in the window immediately, and in Windows, there is no
native {\vt tee} command so that the operation may fail.  However, a
{\vt tee} command is provided with the Cygwin tools, and there are
other sources.  In the normal case of running in the background,
output will again be block buffered under Windows, but there is no
requirement for a {\vt tee} command.

This variable mirrors the state of the {\cb Out to console} check box
in the {\cb Run} page of the {\cb LR Extraction} panel from the
{\cb Extract Menu}.

% 011621
\index{FhOverride variable}
\item{\et FhOverride}\\
{\bf Value:} boolean.\\
When set, values of the {\et FhDefNhinc} and {\et FhDefRh}
variables will override values provided by {\vt FH\_nhinc} and
{\vt FH\_rh} technology layer parameters to use for {\vt
nhinc} and {\vt rh} parameters in {\it FastHenry} input.  This
tracks the state of the {\cb Override Layer NHINC, RH} button
in the {\cb Params} page of the {\cb LR Extraction} panel. 

% 111819
\index{FhPath variable}
\item{\et FhPath}\\
{\bf Value:} directory path string.\\
This variable can be set to a full path to the {\it FastHenry} executable.

If this is not set, {\Xic} will attempt to use ``{\vt
/usr/local/bin/fasthenry}'' as the {\it FastHenry} program (or ``{\vt
/usr/local/bin/fasthenry.exe}'' in Windows).  If this executable does
not exist, {\Xic} will attempt to find ``{\vt fasthenry}'' (or ``{\vt
fasthenry.exe}'' in Windows) in the shell search path when running in
the foreground, and background runs will fail.

This tracks the setting of the text entry field in the {\cb Run} page
of the {\cb LR Extraction} panel.

% 111819
\index{FhUnits variable}
\item{\et FhUnits}\\
{\bf Value:} units string.\\
This variable can be used to specify the length units used in
generated {\it FastHenry} input files.  The variable can be set to a
string consisting of one or the abbreviations ``{\vt m}'' (meters),
``{\vt cm}'' (centimeters), ``{\vt mm}'' (millimeters), ``{\vt um}''
(microns), ``{\vt in}'' (inches), and ``{\vt mils}''.  The long form
word will also be accepted.  This variable is most conveniently
manipulated with the choice menu found in the {\cb LR Extraction}
panel {\cb Params} page.

% 011621
\index{FhUseFillament variable}
\item{\et FhUseFillament}\\
{\bf Value:} units boolean.\\
If set, {\it FastHenry} will decompose segments into filaments
according to the given {\vt nhinc} and {\vt rh} parameters.  If not
set, the interface will automatically slice segments according to the
same parameters, without further refinement by {\it FastHenry\/}. 
This tracks the state of the {\cb Use FastHenry Internal NHINC, RH}
button in the {\cb Params} page of the {\cb LR Extraction} panel.

% 011621
\index{FhVolElMin variable}
\item{\et FhVolElMin}\\
{\bf Value:} units real 0 -- 1.0, default 0.1.\\
The minimum rectangle edge length is this factor multiplying
the maximum edge length.  This tracks the {\cb FhVolElMin}
entry in the {\cb Params} page of the {\cb LR Extraction}
panel.

% 090314
\index{FhVolElTarget variable}
\item{\et FhVolElTarget}\\
{\bf Value:} real number 1e2 -- 1e5, default 1e3.\\
This controls refinement for {\it FastHenry}.  The total volume of all
conductors is divided by this value, and the cube root taken to
provide a length.  Volume elements are split so that no edge is longer
than this length.  The total number of volume elements is
approximately the value of this variable.  Each volume element
contains six segments, connecting the center node to each face node.

% 011621
\index{FhVolEnable variable}
\item{\et FhVolEnable}\\
{\bf Value:} boolean.\\
Enable segment refinement.  This tracks the state of the {\cb
Enable} button in the {\cb Params} page of the {\cb LR
Extraction} panel.

\end{description}


% -----------------------------------------------------------------------------
% !set:help 021515
\section{Help System}
\index{variables!help}

The following {\cb !set} variables affect the help system.

\begin{description}
% 061408
\index{HelpDefaultTopic variable}
\item{\et HelpDefaultTopic}\\
{\bf Value:} string.\\
If this variable is set to an empty string (i.e., as a boolean) the
default help window which normally appears when the {\cb Help} button
in the {\cb Help Menu} is pressed does not appear.  The help mode is
still set, so help can be obtained in the usual way by pressing
buttons or through other actions, only the initial window is
suppressed.

Otherwise, this variable can be set to a URL or help system keyword,
which will be shown in the initial window when the {\cb Help} menu
button is pressed.

% 061408
\index{HelpMultiWin variable}
\item{\et HelpMultiWin}\\
{\bf Value:} boolean.\\
This variable, when set, causes the help system to use a new window
for each menu item or screen element clicked on in help mode.  If not
set, the original help window is reused.

The state of this variable tracks the {\cb Multi-Window Mode} button
in the {\cb Help Menu}.
\end{description}

See also the {\et HelpPath} and {\et DocsDir} variables in
\ref{pathvars}.



